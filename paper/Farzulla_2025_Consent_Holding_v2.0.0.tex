% ============================================================================
% DISSENSUS AI PREPRINT TEMPLATE v3.0.0
% ============================================================================
% Springer Nature-inspired academic preprint with minimal personal branding.
% Derived from ROM paper (arXiv:2601.06363) — the proven, compiled reference.
%
% Author: Murad Farzulla | Dissensus AI
% Last Updated: February 2026
%
% PAPER: Consensual Sovereignty — Consent-Theoretic Monograph on Quantifying Legitimacy
% VERSION: 2.0.0
% ============================================================================

\documentclass[11pt]{article}

% ============================================================================
% PACKAGE IMPORTS
% ============================================================================

% Page geometry and layout
\usepackage[a4paper, margin=1in]{geometry}
\usepackage{setspace}

% Fonts and typography
\usepackage[T1]{fontenc}
\usepackage{mathptmx}                    % Times (most universal academic font)

% Graphics and figures
\usepackage{graphicx}
\usepackage{float}
\graphicspath{{figures/}{../figures/}}

% Tables
\usepackage{booktabs}
\usepackage{array}
\usepackage{multirow}
\usepackage{longtable}

% Math packages
\usepackage{amsmath}
\usepackage{amssymb}
\usepackage{amsthm}
\renewcommand{\qedsymbol}{$\blacksquare$}  % Filled QED squares (Springer style)
\usepackage{mathtools}

% Theorem environments
\theoremstyle{definition}
\newtheorem{definition}{Definition}[section]
\newtheorem{axiom}{Axiom}

\theoremstyle{plain}
\newtheorem{theorem}{Theorem}[section]
\newtheorem{lemma}[theorem]{Lemma}
\newtheorem{corollary}[theorem]{Corollary}
\newtheorem{proposition}{Proposition}[section]
\newtheorem{hypothesis}{Hypothesis}[section]
\newtheorem{constraint}{Constraint}[section]
\newtheorem{assumption}{Assumption}[section]

\theoremstyle{remark}
\newtheorem{remark}{Remark}[section]
\newtheorem{example}{Example}[section]

% Bibliography
\usepackage[round,authoryear]{natbib}
\bibliographystyle{plainnat}

% Colors — burgundy is the only branding touch
\usepackage{xcolor}
\definecolor{brandburgundy}{RGB}{128,0,32}

% Lists
\usepackage{enumitem}

% Hyperlinks and PDF metadata
\usepackage{url}  % Better URL formatting and line breaking
\usepackage[colorlinks=true,
            linkcolor=brandburgundy,
            citecolor=brandburgundy,
            urlcolor=brandburgundy,
            breaklinks=true,
            pdftitle={Consent-Theoretic Framework for Quantifying Legitimacy},
            pdfauthor={Murad Farzulla},
            pdfkeywords={legitimacy, consent, political stability, social choice, institutional design, friction, stakes-weighting}]{hyperref}

% Allow URLs to break at hyphens and slashes
\def\UrlBreaks{\do\/\do-\do_}
\expandafter\def\expandafter\UrlBreaks\expandafter{\UrlBreaks\do\a\do\b\do\c\do\d\do\e\do\f\do\g\do\h\do\i\do\j\do\k\do\l\do\m\do\n\do\o\do\p\do\q\do\r\do\s\do\t\do\u\do\v\do\w\do\x\do\y\do\z}

% Section formatting — burgundy headings with proper spacing
\usepackage{titlesec}

% Part formatting — centered, large, burgundy
\titleformat{\part}[display]
  {\centering\normalfont\huge\bfseries\color{brandburgundy}}
  {\partname~\thepart}
  {20pt}
  {\Huge}
\titlespacing*{\part}{0pt}{50pt}{40pt}

\titleformat{\section}{\normalfont\large\bfseries\color{brandburgundy}}{\thesection}{0.5em}{}
\titleformat{\subsection}{\normalfont\normalsize\bfseries\color{brandburgundy}}{\thesubsection}{0.5em}{}
\titleformat{\subsubsection}{\normalfont\small\bfseries\color{brandburgundy}}{\thesubsubsection}{0.5em}{}
\titlespacing*{\section}{0pt}{2ex plus 0.8ex minus 0.2ex}{1ex plus 0.3ex}
\titlespacing*{\subsection}{0pt}{1.5ex plus 0.5ex minus 0.2ex}{0.8ex plus 0.2ex}
\titlespacing*{\subsubsection}{0pt}{1.2ex plus 0.4ex minus 0.2ex}{0.6ex plus 0.2ex}

% Footer customization
\usepackage{fancyhdr}

% ============================================================================
% PAPER METADATA
% ============================================================================

\newcommand{\papernum}{DAI-2501}              % Working paper number
\newcommand{\paperver}{2.0.0}
\newcommand{\paperdate}{February 2026}
\newcommand{\paperdoi}{10.5281/zenodo.17684676}
\newcommand{\paperurl}{https://systems.ac/1/DAI-2501}  % ASCRI page

% ============================================================================
% HEADER AND FOOTER CONFIGURATION (minimal)
% ============================================================================

\pagestyle{fancy}
\fancyhf{}
\fancyhead[L]{\small\itshape Farzulla (2026)}
\fancyhead[R]{\small\itshape\nouppercase{\rightmark}}
\fancyfoot[C]{\small\thepage}
\renewcommand{\headrulewidth}{0.4pt}
\renewcommand{\footrulewidth}{0pt}

\fancypagestyle{firstpage}{
  \fancyhf{}
  \fancyfoot[C]{\small\thepage}
  \renewcommand{\headrulewidth}{0pt}
  \renewcommand{\footrulewidth}{0pt}
}

\fancypagestyle{partpage}{
  \fancyhf{}
  \fancyfoot[C]{\small\thepage}
  \renewcommand{\headrulewidth}{0pt}
  \renewcommand{\footrulewidth}{0pt}
}

% ============================================================================
% DOCUMENT BODY
% ============================================================================

\begin{document}

\setstretch{1.15}

\thispagestyle{firstpage}

% Working paper series header
\begin{center}
{\small\textsc{Dissensus AI Working Paper Series}}\\[0.2em]
{\small \href{\paperurl}{\papernum}}
\end{center}

\vspace{1.5em}

% Title block
\begin{center}
{\LARGE\bfseries Consent-Theoretic Framework for Quantifying Legitimacy}\\[0.5em]
{\large\itshape Stakes, Voice, and Friction in Adversarial Governance}\\[1.5em]

% Author
{\large Murad Farzulla}\textsuperscript{1,2,*}\\[0.8em]

% Affiliations — inline, slightly smaller
{\small
  \textsuperscript{1}\href{https://dissensus.ai}{Dissensus AI}, London, UK \quad
  \textsuperscript{2}King's College London, London, UK%
}\\[0.5em]

% Correspondence + ORCID (footnote style)
{\footnotesize
  \textsuperscript{*}Correspondence: \href{mailto:murad@dissensus.ai}{murad@dissensus.ai}
  \quad
  ORCID: \href{https://orcid.org/0009-0002-7164-8704}{0009-0002-7164-8704}%
}\\[0.3em]
{\footnotesize \paperdate}
\end{center}

\vspace{0.5em}

% Abstract
\begin{abstract}
\noindent This paper develops a unified analytical framework for measuring political legitimacy across heterogeneous governance domains. Building on insights from constitutional political economy, social choice theory, and institutional analysis, the framework establishes consent-holding—the mapping from decision domains to those with authority over them—as a structural necessity of collective action. We formalize this intuition through five axioms and five theorems, demonstrating that legitimacy can be operationalized as stakes-weighted consent alignment $\alpha(d,t)$, while friction $F(d,t)$ measures the deviation between outcomes and stakeholder preferences. The framework bridges normative democratic theory and empirical prediction, generating testable hypotheses about institutional stability. Historical validation examines suffrage expansion, abolition movements, labor rights, and contemporary platform governance, demonstrating how misalignment between stakes and voice generates observable instability. Unlike existing approaches that prescribe ideal institutions, this framework provides analytical tools for measuring legitimacy within any governance structure, enabling systematic comparison across democratic, technocratic, and algorithmic systems. Computational mechanism comparison via Bayesian learning dynamics across 1000 Monte Carlo runs demonstrates relative performance under adaptive agents: when preferences update based on observed policy outcomes, stakes-weighted DoCS achieves highest final alignment ($\alpha = 0.872$) with lowest terminal friction ($F = 1.5$, 94.9\% reduction from initial $F = 30.3$). This comparative advantage holds across static baseline ($\alpha = 0.627$), learning dynamics ($\alpha = 0.872$), and alternative temporal mechanisms, suggesting stakes-weighting produces superior initial matches that persist even when agents adapt to institutional performance. The framework's domain-specific approach resolves the apparent tension between consent and competence, showing both as complementary dimensions of institutional legitimacy. This framework is part of the Adversarial Systems Research program, which examines stability, alignment, and friction dynamics in complex systems where competing interests generate structural conflict.

\vspace{0.5em}
\noindent\textbf{Keywords:} legitimacy, consent, political stability, social choice, institutional design, friction, stakes-weighting

\vspace{0.3em}
\noindent\textbf{JEL Codes:} D70 (Social Choice), D71 (Social Choice; Clubs; Committees; Associations), P16 (Political Economy)
\end{abstract}

\vspace{0.8em}

\vspace{1em}

\section*{Research Context}

This work forms part of the Adversarial Systems Research program, which investigates stability, alignment, and friction dynamics in complex systems where competing interests generate structural conflict. The program examines how agents with divergent preferences interact within institutional constraints across multiple domains: political governance (this paper), financial markets (cryptocurrency volatility and regulatory responses), human cognitive development (trauma as maladaptive learning from adversarial training environments), and artificial intelligence alignment (multi-agent systems with competing objectives).

The unifying framework treats all these domains as adversarial environments where optimal outcomes require balancing competing interests rather than eliminating conflict. In political systems, this manifests as the tension between stakeholder consent and technocratic competence. In financial markets, it appears as the conflict between regulatory stability and market innovation. In human development, it emerges as the challenge of learning accurate models from noisy or adversarial training data. In AI systems, it surfaces as the alignment problem when multiple agents optimize for different reward functions.

The Doctrine of Consensual Sovereignty presented here provides the theoretical foundation for analyzing legitimacy in any adversarial environment by formalizing the relationship between stakes, voice, and friction. Future work will extend this framework to algorithmic governance systems, multi-stakeholder climate negotiations, and autonomous agent coordination problems where consent structures remain undefined but friction dynamics are already observable.

\section*{Key Notation}

\begin{table}[h]
\small
\begin{tabular}{ll}
\toprule
\textbf{Symbol} & \textbf{Definition} \\
\midrule
$H_t(d)$ & Consent-holder mapping (who decides in domain $d$ at time $t$) \\
$s_i(d)$ & Stakes of agent $i$ in domain $d$ (material/capability exposure) \\
$C_{i,d}$ & Consent power of agent $i$ in domain $d$ (decision authority) \\
$\alpha(d,t)$ & Consent alignment (stakes-weighted share of voice held by stakeholders) \\
$F(d,t)$ & Friction (stakes-weighted deviation between outcomes and preferences) \\
$L(d,t)$ & Legitimacy ($w_1 \cdot \alpha + w_2 \cdot P$, balancing consent and performance) \\
$P(d,t)$ & Performance/competence metric (domain-specific outcome quality) \\
$x^*_{i,d}$ & Agent $i$'s ideal action in domain $d$ (preference) \\
$x_d(t)$ & Realized action/outcome in domain $d$ at time $t$ \\
\bottomrule
\end{tabular}
\end{table}

\section*{Scope and Limitations}

This paper presents a conceptual framework for analyzing legitimacy in adversarial environments. While we formalize core relationships mathematically—consent alignment $\alpha(d,t)$, friction $F(d,t)$, and legitimacy $L(d,t)$—complete operational measurement remains ongoing empirical work. Our contribution is providing analytical architecture that makes legitimacy comparable across governance domains, enabling systematic analysis previously confined to domain-specific theories.

We demonstrate proof-of-concept through computational validation via agent-based simulation and qualitative validation across seven historical domains (suffrage, abolition, labor rights, civil rights, LGBT inclusion, platform governance, climate policy). \textbf{Methodological note}: The computational models assume agents learn from outcomes (Bayesian updating), which by construction reduces friction as preferences converge—this compares which consent mechanisms produce superior alignment given plausible behavioral assumptions, not whether friction \textit{can} reduce (that follows definitionally). Measurement challenges are acknowledged in Section~\ref{sec:operationalization}.

This v2.0.0 monograph expands the original preprint with quantified historical validation across nine case studies (suffrage, abolition, labor rights, civil rights, LGBT inclusion, corporate governance, platform governance, climate governance), consolidated normative analysis, and systematic cross-case hypothesis testing. Part~III constructs alpha and friction time series from real data where available and ordinal scales where quantitative proxies are sparse, enabling preliminary assessment of the framework's core predictions.

% ============================================================================
% TABLE OF CONTENTS
% ============================================================================

\clearpage
\tableofcontents
\thispagestyle{firstpage}  % No header on TOC page

% ============================================================================
% MAIN BODY
% ============================================================================

\clearpage

% ============================================================================
\part{Foundations}
\label{part:foundations}
\thispagestyle{partpage}
% ============================================================================

\section{Introduction}
\label{sec:introduction}

Political legitimacy presents a fundamental puzzle: how can we measure whether authority is rightfully held across radically different governance domains? A state legislature, corporate board, algorithmic content moderation system, and common-pool resource management regime all make consequential decisions affecting stakeholders, yet existing frameworks struggle to provide unified analytical tools for assessing their legitimacy. Democratic theory emphasizes popular sovereignty, grounding legitimacy in consent of the governed \citep{locke1689,rousseau1762,rawls1971,habermas1984}, public choice highlights constitutional constraints \citep{buchanan1962}, while recent work on algorithmic governance introduces new challenges to consent-based legitimacy \citep{grimmelikhuijsen2022}. What remains elusive is a framework capable of both normative evaluation and empirical prediction that applies consistently across domains.

This paper addresses this gap by developing consent-holding theory, an axiomatic framework that treats legitimacy as a structural property of decision-making systems rather than a binary classification. The central insight is deceptively simple yet powerful: in any domain where collective decisions produce shared consequences, someone must hold the authority to decide. This consent-holder mapping $H_t(d)$—identifying who decides in domain $d$ at time $t$—is not a normative choice but a logical necessity arising from the structure of collective action itself. The framework's contribution lies not in prescribing who should hold consent, but in providing rigorous tools for measuring the consequences of any particular allocation.

The framework makes three distinct contributions to political theory and institutional analysis. First, it establishes a formal connection between consent alignment and observable political friction. While democratic theorists have long argued that excluding affected stakeholders undermines legitimacy \citep{estlund2008}, existing accounts lack operational metrics for testing these claims. We define consent alignment $\alpha(d,t)$ as the stakes-weighted share of decision power held by affected parties, and friction $F(d,t)$ as the stakes-weighted deviation between outcomes and stakeholder preferences. The framework predicts that persistent misalignment generates measurable instability—protests, non-compliance, institutional breakdown—making legitimacy empirically falsifiable rather than purely philosophical.

Second, the framework resolves the apparent tension between consent and competence through a competence-consent trade-off theorem (T4). Epistemic democrats argue that inclusive decision-making produces better outcomes through cognitive diversity \citep{landemore2013,hong2004}, while critics worry that expanding consent sacrifices technical expertise. Our framework shows these concerns reflect different positions on the legitimacy frontier: some domains optimally weight performance highly (nuclear safety, pandemic response), while others prioritize consent alignment (constitutional amendments, community norms). Rather than declaring one approach universally superior, the framework provides tools for identifying domain-appropriate balances.

Third, this approach enables systematic historical and comparative analysis. By operationalizing legitimacy as $\alpha(d,t)$ and friction as $F(d,t)$, we can trace institutional evolution quantitatively. Franchise expansion emerges not as discrete events but as gradual increases in $\alpha(d)$ driven by the accumulating friction $F(d)$ from excluding high-stakes populations. Women's suffrage movements, abolition struggles, labor organizing, and contemporary platform governance rebellions all exhibit the same underlying dynamic: groups with high stakes $s_i(d)$ but zero consent power $C_i$ generate sustained friction until incorporation or suppression occurs. This pattern, predicted by the framework's core theorems, provides empirical validation across centuries and continents.

The monograph is organized in five parts. Part~I establishes foundations: the formal framework proceeds from seven minimal axioms to five core results establishing structural necessities (Section~\ref{sec:framework}), operationalized through empirical measurement specifications (Section~\ref{sec:operationalization}) and situated within twelve research traditions spanning constitutional political economy, social choice theory, political anthropology, social movements, and institutional theory (Section~\ref{sec:literature}).

Part~II develops the normative architecture (Section~\ref{sec:social-contract}), consolidating classical and contemporary social contract theories as competing proposals for consent-power allocation, with formal text-to-model mappings that translate Hobbes, Locke, Rousseau, Rawls, and emerging algorithmic governance frameworks into consent-holding predictions.

Part~III provides the monograph's primary new contribution: quantified historical validation across nine case studies spanning two centuries. Section~\ref{sec:historical-methodology} establishes the methodology for constructing alpha and friction time series from historical data. Subsequent sections trace consent alignment dynamics through suffrage expansion (Section~\ref{sec:suffrage}), abolition and emancipation (Section~\ref{sec:abolition}), labor rights and codetermination (Section~\ref{sec:labor}), civil rights (Section~\ref{sec:civil-rights}), LGBT inclusion (Section~\ref{sec:lgbt}), corporate governance (Section~\ref{sec:corporate-governance}), platform governance (Section~\ref{sec:platform-governance}), and climate governance (Section~\ref{sec:climate}). Each case constructs domain-specific alpha and friction proxies—quantitative where data permits (enfranchised population shares, union density, board composition), ordinal where quantitative proxies are sparse (abolition, LGBT legal recognition)—enabling systematic cross-case comparison (Section~\ref{sec:scope-conditions}). Companion work on structural exclusion in democratic systems \citep{farzulla2025stakes} provides additional theoretical grounding for the formal-effective voice gap documented across cases.

Part~IV presents computational validation through Monte Carlo mechanism comparison under adaptive learning dynamics (Section~\ref{sec:monte-carlo}) and robustness analysis across alternative dynamic specifications (Section~\ref{sec:dynamic-validation}).

Part~V addresses implications: nine objections and replies (Section~\ref{sec:objections}), weight determination as an endogenous constitutional problem (Section~\ref{sec:weight-determination}), and a research agenda spanning quadratic voting, institutional experiments, AI governance applications, and cross-national panel studies (Section~\ref{sec:research-agenda}). The conclusion (Section~\ref{sec:conclusion}) summarizes contributions and limitations.

The framework's title—``consent-holding'' rather than ``consent theory''—reflects its analytical focus. This is not another account of why consent matters normatively, but a systematic investigation of how consent operates structurally. Just as markets emerge from property rights and contracts regardless of normative justifications for capitalism, consent-holding structures emerge from the necessity of collective decision-making regardless of democratic commitments. The framework's power lies in making these structures visible, measurable, and comparable, enabling rigorous analysis of legitimacy claims that have historically remained philosophically contested but empirically elusive.

\section{Literature Review and Theoretical Foundations}
\label{sec:literature}

The consent-holding framework synthesizes and extends insights from nine distinct research traditions. This section reviews each tradition's core contributions, identifies limitations the framework addresses, and demonstrates how operationalizing legitimacy as $\alpha(d,t)$ and friction as $F(d,t)$ enables empirical validation of long-standing theoretical claims. The framework's core claims receive independent convergent support from recent work across disciplines: \citet{sornette2026alignment} independently derive structural friction dynamics from a statistical physics perspective, identifying learned human interaction structures as endogenous sources of alignment failure, providing convergent validation of the friction-based legitimacy framework developed here. \citet{bednar2025jebo} situate these dynamics within the broader complexity economics research programme, arguing that 21st-century governance challenges require precisely the kind of multi-scale, friction-aware analytical tools the consent-holding framework provides.

\subsection{Constitutional Political Economy}

Building on earlier social contract foundations from \citet{hobbes1651} who established consent as prerequisite for legitimate political authority, Buchanan and Tullock's \citeyearpar{buchanan1962} seminal work establishes constitutional choice as a distinct analytical problem requiring different decision rules than ordinary politics. \citet{brennan1985} further develop this constitutional economics framework, distinguishing levels of collective action and establishing how constitutional rules create frameworks for collective choice. Their framework rests on several foundational insights that anticipate the consent-holding approach. First, they distinguish between constitutional rules—rarely changed frameworks establishing decision procedures—and political decisions made within those rules. This maps directly onto our concept of nested consent-holding: $H_t(d_{meta})$ represents the consent-holders for constitutional domains, while $H_t(d)$ operates within constraints established at the meta-level.

Second, Buchanan and Tullock argue that rational agents behind a ``veil of uncertainty'' would unanimously consent to rules benefiting all. Once constitutional structures are established, majority rule becomes acceptable for routine decisions. This anticipates our Theorem 1: consent-holding exists at every level, from object-level policy to constitutional design to amendment procedures. Third, their exchange paradigm treats politics as mutual exchange of consent rather than top-down command. Government achieves legitimacy when citizens ``purchase'' its services consensually through constitutional agreement. The consent-holding framework formalizes this metaphor rigorously through stakes-weighted alignment metrics.

Finally, Buchanan and Tullock model optimal decision rules as minimizing total costs combining external costs (harm from decisions affecting you without your consent) and decision costs (time and effort required to reach agreement). Our friction metric $F(d)$ captures external costs precisely as stakes-weighted deviations from stakeholder ideal points. The framework extends Buchanan and Tullock in four crucial respects. First, we introduce stakes-weighting $s_i(d)$, recognizing that individuals are heterogeneously affected by policies. Second, while Buchanan focuses on one-time constitutional founding moments, we model consent-holding as continuously operating through $H_t(d)$, tracking legitimacy dynamically as institutional configurations evolve. Third, Buchanan discusses ``the'' social contract; we specify that consent-holding varies across domains $d$, with different optimal structures for taxation, criminal justice, environmental regulation, and community norms. Fourth, Buchanan provides normative theory; we operationalize concepts through $\alpha(d,t)$ and $F(d,t)$, enabling empirical validation of constitutional designs rather than purely philosophical justification.

\subsection{Social Choice Theory and Impossibility Results}

Building on \citet{dahl1956} foundational analysis of democratic theory showing how pluralist democracy requires balancing majority rule with minority rights, Arrow's \citeyearpar{arrow1951} impossibility theorem establishes that no ranked voting system can simultaneously satisfy four seemingly minimal desiderata: Pareto efficiency, non-dictatorship, independence of irrelevant alternatives, and unrestricted domain. This result demonstrates that perfect democratic aggregation is mathematically impossible, not merely practically difficult. The Gibbard-Satterthwaite theorem \citep{gibbard1973,satterthwaite1975} extends this impossibility to strategy-proofness: any non-dictatorial voting mechanism over three or more alternatives is manipulable.

Recent quantitative versions show these aren't merely theoretical concerns but quantifiably common, with \citet{keller2012quantitative,mossel2012quantitative,friedgut2011quantitative} quantifying how often Arrow's impossibility manifests in real voting scenarios with finite electorates. Sen's \citeyearpar{sen2017collective} expanded treatment of collective choice integrates economics and ethics, introducing the capability approach that maps directly onto our effective voice concept. \citet{sen1999} argues in \textit{Development as Freedom} that development should be measured not by utility or resources alone but by capabilities—freedoms to achieve valued functionings like health, education, and political participation. This provides theoretical grounding for our $\text{eff\_voice}_i(d)$ term: possessing formal consent power $C_i > 0$ without resources, education, or political freedom represents low capability.

The consent-holding framework relates to social choice theory as meta-analysis rather than competitor. Where Arrow and Gibbard-Satterthwaite ask ``which aggregation rule is best?'', we ask ``how legitimate is any given aggregation rule?'' This shift has three implications. First, our framework doesn't compete with impossibility results; it builds on them by providing tools for measuring consequences of unavoidable trade-offs. Since perfect rules don't exist, we need metrics for comparing imperfect options. Second, stakes-weighting $s_i(d)$ isn't present in classical social choice theory, which typically assumes equal weights. This extension allows domain-specific analysis: simple majority may be optimal for low-stakes routine legislation, while supermajority or even consensus becomes appropriate when stakes concentrate heavily.

Our stakes-weighting approach builds on but diverges from weighted voting power analysis \citep{banzhaf1965,shapley1954}. The Shapley-Shubik and Banzhaf power indices measure \textit{effective} voting power given formal weights in committee systems—recognizing that a voter with 40\% weight may have more than 40\% actual power if they're pivotal in coalitions. This literature addresses measurement of power distribution within given institutional arrangements. Our framework addresses the prior question: how should consent power $C_i,d$ be allocated based on stakes $s_i(d)$? While power index theory takes weights as given and calculates resulting influence, we propose stakes as foundation for determining appropriate weights. Future work integrating these approaches could specify stakes-weighted allocations, then apply Banzhaf or Shapley-Shubik indices to measure resulting effective voice, combining normative allocation principles with positive power analysis.

\subsection{Stakeholder Theory and Corporate Governance}

Building on \citet{pitkin1967} foundational work on representation distinguishing substantive versus descriptive representation and acting for constituents, Freeman's \citeyearpar{freeman1984} stakeholder approach argues that firms should create value for all stakeholders—employees, suppliers, communities, customers, shareholders—not just maximize shareholder returns. This challenges \citet{friedman1970} shareholder primacy doctrine, which treats profit maximization as the sole corporate responsibility and argues that corporate social responsibility beyond shareholder wealth maximization is fundamentally misguided. The 2019 Business Roundtable statement endorsing stakeholder capitalism, signed by 200 CEOs, marks mainstream acceptance of Freeman's stakeholder view \citep{businessroundtable2019}, representing a significant shift from the Friedman doctrine. \citet{phillips2003} further develops this framework by distinguishing stakeholders by the moral obligation owed to them versus their ability to affect the organization, providing a typology of stakeholder legitimacy that maps onto our stakes-consent framework.

The framework operationalizes Freeman's insights by defining stakeholders precisely as agents with $s_i(d) > 0$ in corporate domains. Current governance structures grant consent power almost exclusively to shareholders: they elect boards, approve major transactions, and receive residual claims. Employees, despite high stakes in employment security, working conditions, and workplace norms, hold negligible $C_i$ in most Anglo-American firms. This generates low $\alpha(d_{corporate})$ when stakes are calculated comprehensively. The framework predicts such misalignment produces friction $F(d)$: labor disputes, regulatory pressures, reputation damage, difficulty attracting talent.

Comparative corporate governance research validates these predictions. \citet{vitols2011coordinated} documents how German codetermination—mandatory worker representation on supervisory boards—constrains hostile takeovers and maintains stakeholder orientation. Workers' voice (high $\alpha_{workers}(d)$) prevents zero-sum shareholder maximization strategies. \citet{fauver2011good} show codetermined firms invest more in worker training and career development; higher $\alpha$ produces performance improvements in human capital domains.

\subsection{Common-Pool Resource Governance}
\label{subsec:cpr-governance}

Ostrom's \citeyearpar{ostrom1990} groundbreaking work on common-pool resources challenges both ``tragedy of the commons'' pessimism and top-down state solutions. Through field studies of fisheries, forests, irrigation systems, and groundwater basins across continents, she demonstrates that resource users frequently develop effective self-governance without privatization or centralized authority. Her eight design principles for successful commons management include particularly relevant insights for consent-holding theory. Design Principle 3 requires that ``most individuals affected by the operational rules can participate in modifying the operational rules''---essentially mandating high $\alpha(d_{rules})$ for those with high $s_i(d_{resources})$. Design Principle 8 specifies nested enterprises for larger systems, enabling polycentric governance with consent-holding at multiple scales. This builds on earlier insights from \citet{ostrom1961} on polycentric systems, demonstrating how multiple governing authorities at different scales can achieve better outcomes than monocentric alternatives.

The framework formalizes Ostrom's intuitions. Her ``collective choice arrangements'' represent $H_t(d)$ mappings where users participate in rule modification. Her design principles can be reinterpreted as conditions enabling high $\alpha(d)$: clear boundaries (defining who holds $s_i$), local monitoring (ensuring $C_i$ holders possess information), graduated sanctions (responses to low-$\alpha$ violations), and conflict resolution mechanisms (managing $F(d)$ when it arises). Successful commons maintain high consent alignment; failed commons exhibit persistent misalignment between stakes and voice.

Recent empirical work validates this interpretation quantitatively. \citet{cox2010} conduct a meta-analysis showing that Ostrom's design principles predict commons sustainability across diverse contexts, providing systematic evidence that consent alignment mechanisms enable effective resource governance. \citet{yadav2021institutional} analyze 83 Amazonian communities managing arapaima fisheries, showing that Ostrom's design principles predict ecological outcomes systematically. Communities exhibiting collective choice arrangements (high $\alpha$) maintain sustainable fish stocks; those lacking such arrangements experience depletion.

However, the relationship between community governance and resource outcomes is more complex than Ostrom's initial formulation suggests. \citet{agrawal1999enchantment} caution against romanticizing community-based conservation, arguing that ``community'' itself is a contested category whose boundaries, internal power dynamics, and relationship to external markets require careful specification. In consent-holding terms, defining $S_d = \{i | s_i(d) > 0\}$ for a natural resource domain is itself a political act: who counts as a ``community member'' determines whose stakes register in $\alpha(d)$ calculations. Agrawal and Gibson demonstrate that communities are internally differentiated by wealth, gender, caste, and political connections---differences that produce unequal $C_i$ distributions even within nominally participatory arrangements. A village forestry committee may formally include all households, but if wealthier households dominate agenda-setting while poorer households depend most heavily on forest products, the resulting $\alpha(d_{forest})$ remains low despite institutional inclusiveness.

\citet{dietz2003drama} synthesize the broader challenges facing commons governance, framing environmental problems not as a simple ``tragedy'' but as a complex ``drama'' involving multiple actors with heterogeneous interests, information asymmetries, and cross-scale interactions. Their analysis identifies three fundamental requirements for adaptive commons governance: providing information (enabling informed consent through transparency about resource states), managing conflict (developing institutions that channel $F(d)$ into productive reform rather than destructive competition), and inducing rule compliance (maintaining institutional stability when individual incentives favor defection). Each requirement maps directly onto consent-holding mechanisms. Information provision raises $\text{eff\_voice}_i$ by enabling stakeholders to assess whether $H_t(d)$ serves their interests. Conflict management channels friction $F(d)$ toward institutional adaptation. Rule compliance addresses the gap between formal consent power $C_i > 0$ and the temptation to free-ride on others' governance contributions.

The question of enforcement reveals a crucial insight for legitimacy theory. \citet{gibson2005local} demonstrate that local enforcement produces systematically better forest outcomes than external enforcement, even controlling for rule quality and forest type. Communities that monitor and sanction their own members---where enforcement authority maps onto the affected set $S_d$---maintain forest cover more effectively than those relying on distant government agencies. This finding validates a core prediction of consent-holding theory: governance arrangements where those bearing consequences also hold enforcement authority (high $\alpha(d_{enforcement})$) outperform arrangements where enforcement is externally imposed. Local monitors possess better information about rule violations, stronger social incentives for fair application, and greater legitimacy among those subject to sanctions.

Yet local governance can also fail catastrophically when consent-holding structures are disrupted by external forces. \citet{baland1996halting} document how colonial and post-colonial state interventions systematically undermined indigenous resource management institutions across the developing world. By appropriating forests, fisheries, and rangelands into state ownership, governments simultaneously dismantled existing $H_t(d)$ mappings (removing community consent authority) and failed to replace them with effective alternatives. The result was a legitimacy vacuum: communities retained high $s_i(d_{resources})$ but lost all $C_i$, while distant state agencies held formal authority but lacked local knowledge, monitoring capacity, and social embeddedness. Friction mounted not as organized protest but as quiet defiance---illegal logging, poaching, encroachment---reflecting the theory's prediction that persistent misalignment between stakes and voice produces friction even when formal institutional channels are absent.

\citet{berkes2006globalization} identify an analogous dynamic at the global scale through the concept of ``roving bandits''---mobile actors who exploit resources across jurisdictions without bearing long-term consequences. Industrial fishing fleets operating in developing nations' waters exemplify this pattern: their $s_i(d_{fishery})$ is entirely short-term extractive, they hold effective $C_i$ through economic power and flag-state protection, while local fishing communities possess high long-term $s_i$ but negligible $C_i$ over marine resource decisions. The roving bandit phenomenon thus represents a systematic failure of consent alignment: actors with the lowest legitimate stakes capture decision authority, while those with the highest stakes---communities whose livelihoods, food security, and cultural practices depend on resource health---are excluded from $H_t(d)$. This is domain-specific alpha optimization in reverse: governance arrangements that systematically minimize $\alpha(d)$ by disconnecting stakes from voice.

\citet{weeratunge2014smallscale} deepen this analysis by documenting how small-scale fisheries governance affects multidimensional wellbeing beyond economic income. Their wellbeing framework encompasses material conditions, relational dynamics, and subjective experience---dimensions that map onto our stakes measurement. When $s_i(d_{fishery})$ is measured solely as income share, the case for community governance rests on efficiency grounds alone. But when stakes include nutritional security, cultural identity, intergenerational knowledge transmission, and social cohesion, the affected set $S_d$ broadens substantially, and the stakes-weighted case for high $\alpha(d)$ strengthens correspondingly. Communities' comprehensive stakes in fishery governance exceed their economic stakes, implying that legitimacy assessments based on narrow economic measures systematically understate the consent alignment deficit.

Finally, \citet{ostrom2010} extends the polycentric governance framework to global environmental challenges, arguing that monocentric approaches to climate change and biodiversity loss are both politically infeasible and theoretically suboptimal. Polycentric systems---multiple overlapping governance authorities operating at different scales---allow consent-holding to be domain-specific and scale-appropriate. Municipal recycling programs, regional air quality management, national emissions standards, and international climate treaties each represent distinct $H_t(d)$ mappings for overlapping environmental domains. This nested structure enables what we might call \textit{domain-specific alpha optimization}: each governance level can achieve higher $\alpha(d)$ for the scale at which its affected set $S_d$ is defined than any single-level arrangement could achieve across all scales simultaneously. The insight connects directly to our framework's emphasis on domain-specific analysis: there is no single ``optimal'' governance structure, but rather an ecology of institutional arrangements whose legitimacy must be assessed domain by domain.


\subsection{Deliberative Democracy and Mini-Publics}
\label{subsec:deliberative-democracy}

Building on \citet{dahl1971} polyarchy framework of participation and opposition and \citet{mill1861} considerations on representative government balancing participation and competence, Habermas's \citeyearpar{habermas1984,habermas1990} communicative action theory distinguishes strategic action (oriented toward achieving one's goals) from communicative action (oriented toward mutual understanding through reasoned argument). Legitimate norms are those acceptable to all affected parties through rational discourse free from coercion. His discourse principle holds that ``only those norms can claim validity that could meet with the acceptance of all concerned in their capacity as participants in a practical discourse.'' This maps onto consent-holding directly: ``all concerned'' represents our affected set $S_d = \{i | s_i(d) > 0\}$, while ``acceptance'' requires $C_i > 0$ in decision procedures $H_t(d)$.

Fishkin's \citeyearpar{fishkin2009when,fishkin2018} deliberative polling research operationalizes these theoretical commitments. By convening randomly selected representative samples, providing balanced information, facilitating structured deliberation, and measuring preference changes, deliberative polls demonstrate that informed public judgment shifts significantly through discourse. \citet{fishkin2009when} documents cases across multiple countries where deliberative polls produced measurably different policy preferences compared to raw survey data, with opinion shifts averaging 10--15 percentage points on salient issues. The mechanism is straightforward in consent-holding terms: information provision and structured deliberation raise $\text{eff\_voice}_i$ by transforming uninformed preferences into considered judgments, while random selection approximates equal $C_i$ across the affected population.

Citizens' assemblies extend deliberative innovation to consequential policy domains. The Irish Citizens' Assembly (2016--2018) addressed abortion and climate change through 99 randomly selected citizens deliberating after expert input, demonstrating how sortition combined with deliberation can shift preferences systematically \citep{farrell2019}. \citet{courant2021} analyze the French Citizens' Convention on Climate (2019--2020), which generated 149 policy proposals from 150 randomly selected participants through sortition and deliberation, with many subsequently adopted into legislation.

The framework interprets these innovations as institutional experiments raising $\alpha(d)$ through sortition and deliberation. Random selection approximates equal $C_i$ for participants; demographic stratification can approximate stakes-weighting if groups correlate with $s_i(d)$. Learning phases improve $\text{eff\_voice}_i$ through information provision; deliberation structures enable preference refinement.

Participatory budgeting represents perhaps the most direct institutional mechanism for raising $\alpha(d)$ in municipal governance domains. \citet{wampler2007participatory} analyzes the Brazilian experience, where Porto Alegre pioneered citizen-directed allocation of municipal investment budgets beginning in 1989. The process grants residents direct $C_i$ over spending priorities---infrastructure, health, education, housing---in proportion to neighborhood attendance and need indices. Wampler demonstrates that participatory budgeting systematically redirects resources toward poorer districts, precisely the pattern predicted by consent-holding theory: when $C_i$ is distributed more evenly while $s_i(d_{infrastructure})$ concentrates among underserved populations, the resulting $\alpha(d_{budget})$ better reflects genuine stakes distribution, reducing friction and improving perceived legitimacy. The Brazilian experience has since spread to over 7,000 municipalities worldwide \citep{participatory2023}, constituting a natural experiment in alpha-raising across diverse institutional contexts. Crucially, outcome variation across implementations provides testable predictions: cities where participatory budgeting grants meaningful decision authority (high $C_i$ for participants) should exhibit lower governance friction than those where participation is merely consultative.

\citet{smith2009democratic} provides a systematic comparative analysis of democratic innovations including mini-publics, participatory budgeting, direct legislation, and e-democracy initiatives. Smith evaluates each innovation against criteria that map directly onto our framework: inclusiveness (breadth of the affected set included in $H_t(d)$), popular control (degree of genuine $C_i$ granted to participants), considered judgment (quality of $\text{eff\_voice}_i$ exercised), and transparency (enabling external assessment of $\alpha(d)$). His analysis reveals a persistent tension between depth and breadth: deliberative mini-publics achieve high $\text{eff\_voice}_i$ for participants through intensive information and discussion, but include only a small fraction of the affected population. Referenda include the entire electorate but sacrifice deliberation quality. This depth-breadth trade-off is structural, not merely practical, and consent-holding theory illuminates why: achieving simultaneously high $C_i$ for all members of $S_d$ and high $\text{eff\_voice}_i$ for each participant requires institutional resources that scale with population size.

\citet{bouricius2013democracy} proposes a radical solution to this tension through multi-body sortition, drawing explicitly on Athenian democratic practice. Rather than concentrating deliberative authority in a single citizens' assembly, Bouricius designs a system of six functionally differentiated bodies---agenda council, interest panels, review panels, policy jury, rules council, and oversight council---each populated by sortition with different selection criteria and rotation schedules. The architecture distributes consent-holding across multiple specialized $H_t(d)$ mappings, each optimized for different governance functions. Agenda-setting requires broad representation (maximizing $|S_d|$ coverage); policy deliberation requires depth (maximizing $\text{eff\_voice}_i$ through sustained engagement); oversight requires independence (minimizing capture by interested parties). This multi-body approach represents systematic consent allocation: rather than asking a single institution to optimize all dimensions simultaneously, it decomposes governance into functional domains and assigns consent-holding structures appropriate to each.

The capacity-building dimension of deliberation has received increasing attention. \citet{niemeyer2011emancipatory} provides empirical evidence for what he terms the ``emancipatory effect'' of deliberation: participants in mini-publics develop not only better-informed preferences but enhanced civic capacities that persist after the deliberative event concludes. Through pre- and post-deliberation surveys, Niemeyer demonstrates increases in political efficacy, complexity of political reasoning, and willingness to engage in subsequent civic action. In consent-holding terms, deliberation raises $\text{eff\_voice}_i$ not merely for the issue under discussion but as a durable capacity enhancement. If this emancipatory effect generalizes---and \citet{schneiderhan2008realizing} provide supporting evidence from community organizing contexts, showing that meaningful participatory experiences transform participants' understanding of their own citizenship capacity---then deliberative institutions generate positive externalities beyond their immediate policy domain. Each deliberative episode raises participants' $\text{eff\_voice}_i$ across multiple governance domains, creating spillover effects that improve $\alpha(d)$ systemically rather than domain-specifically.

\citet{schneiderhan2008realizing} examine the relationship between deliberative authority and citizenship formation, documenting how the experience of exercising genuine decision authority---as opposed to merely being consulted---transforms participants' self-understanding as political agents. Their distinction between ``thin'' participation (surveys, public comment periods) and ``thick'' participation (binding deliberation with real consequences) maps onto our framework's emphasis on effective voice: formal inclusion in $H_t(d)$ without meaningful $C_i$ does not constitute genuine consent alignment. The implication is that institutional arrangements granting merely consultative roles---advisory committees, non-binding referenda, stakeholder ``engagement'' processes---may actually reduce $\alpha(d)$ by creating an appearance of consent that masks continued concentration of decision authority.

However, scaling deliberative democracy encounters significant complexity challenges. \citet{hendriks2014emergent} analyzes what she terms ``emergent complexity'' in deliberative systems---the unintended interactions, feedback loops, and power asymmetries that arise when multiple deliberative institutions operate simultaneously within a governance ecosystem. Mini-publics do not function in isolation; they interact with elected legislatures, courts, media, lobbying organizations, and each other. Hendriks demonstrates that these interactions can produce perverse outcomes: a well-designed citizens' assembly may generate high-quality recommendations that are subsequently captured or distorted by existing power structures during implementation. In consent-holding terms, achieving high $\alpha(d_{deliberation})$ within the assembly does not guarantee high $\alpha(d_{policy})$ if the translation mechanism from deliberative output to binding policy is itself characterized by low consent alignment. The deliberative system as a whole---not merely its individual components---must be evaluated for legitimacy.

Our stakes-weighted consent framework confronts democratic equality arguments directly. Building on \citet{mill1859} foundations regarding individual liberty, consent, and limits of state power, \citet{christiano2008} defends equal political voice on dignity grounds: each person possesses equal moral status, entitling them to equal say in collective decisions regardless of stakes or competence. \citet{waldron1999} argues that persistent disagreement about what justice requires makes equal voice procedurally fair even if some possess superior judgment. \citet{brighouse2010} examine whether proportional influence could improve democratic outcomes but conclude that equal voice better respects equality of persons.

We acknowledge this tension while distinguishing \textit{political} domains from \textit{governance} domains generally. In constitutional fundamentals and citizenship rights, equal voice may be intrinsically required by equal moral status---each person gets one vote precisely because they are persons, not because they possess equal stakes. But many governance domains are not \textit{political} in this sense: corporate boards allocating firm resources, technical committees setting safety standards, platform algorithms moderating speech, common-pool resource users managing fisheries. In these contexts, stakes-weighting may be both more efficient (reducing friction, improving outcomes) and more legitimate (those bearing consequences should influence decisions proportionally). The framework enables empirical testing: do equal-voice or stakes-weighted mechanisms generate higher measured legitimacy $L(d,t)$ in different domain types?


\subsection{Algorithmic Governance and Platform Legitimacy}

\citet{grimmelikhuijsen2022} identify three legitimacy dimensions for algorithmic decision-making: input (did citizen preferences inform design?), throughput (does the algorithm follow fair procedures?), and output (do outcomes align with public values?). Current algorithmic governance exhibits severe deficits across all three dimensions. Citizens rarely participate in algorithm design (low input legitimacy), decision-making processes remain opaque black boxes (low throughput legitimacy), and outcomes often replicate historical discrimination (questionable output legitimacy). \citet{kleinberg2017} demonstrate inherent trade-offs in fair determination of risk scores, showing that multiple incompatible definitions of algorithmic fairness exist—making it impossible to satisfy all fairness criteria simultaneously, analogous to Arrow's impossibility theorem in social choice.

\citet{waldman2022} show that high-stakes algorithmic decisions (healthcare allocation, criminal sentencing) are perceived as less legitimate than human decisions even when outcomes are identical. The consent-holding framework diagnoses these challenges structurally. Algorithmic decision-making creates domains $d_{algorithm}$ where algorithms or their designers hold $C \approx 1$ while affected citizens have $C \approx 0$ despite high $s_i(d)$. Credit scoring algorithms determine loan access (high $s_i$ for applicants); hiring algorithms control employment opportunities (high $s_i$ for candidates); content moderation algorithms shape speech norms (high $s_i$ for platform users). In each case, current $\alpha(d_{algorithm}) \approx 0$ because high-stakes populations are excluded from $H_t(d)$.

Platform responses attempting to raise $\alpha$ reveal understanding of legitimacy deficits. Meta's Oversight Board provides independent content moderation appeals, slightly raising $\alpha(d_{moderation})$ by giving users contestation rights, though \citet{douek2022} notes this provides only limited voice expansion while maintaining corporate control over fundamental rules. YouTube Creator Councils consult high-profile creators, extending partial $C_i$ to stakeholders whose $s_i$ is highest.

\subsection{Voting Power Indices and Coalition Analysis}

The power indices literature demonstrates that voting weight $\neq$ voting power. \citet{banzhaf1965} measures critical voter frequency: how often removing your vote changes outcomes from win to loss. \citet{shapley1954} measure pivotal voter frequency in sequential coalition formation. \citet{felsenthal1998} provide comprehensive comparison of these approaches, demonstrating that Banzhaf and Shapley-Shubik indices often diverge substantially and measure different aspects of voting power. These indices often diverge dramatically from nominal weights—Germany holds the most European Council votes but doesn't possess proportional power due to coalition dynamics. Similar phenomena arise in corporate boards (blockholders vs. minority shareholders), legislatures (swing voters vs. party leaders), and qualified majority systems (Security Council veto players).

These insights directly inform consent-holding operationalization. Naive approaches measure $C_i$ as voting weight (shares held, seats controlled). Sophisticated approaches use power indices accounting for coalition structures. In weighted voting contexts (shareholders, federalism), qualified majority rules (constitutional amendments), and veto player systems (UN Security Council), indices capture actual influence more accurately than nominal weights.

The framework integrates power indices into legitimacy measurement: $\alpha(d,t) = \frac{\sum_i s_i(d) \cdot \text{PowerIndex}_i(d,t)}{\sum_i s_i(d)}$, where $\text{PowerIndex}_i$ represents Banzhaf, Shapley-Shubik, or domain-appropriate measures. This refinement matters most when vote concentration enables blocking coalitions. Consider corporate governance: a minority shareholder with 20\% equity plus veto rights over major transactions wields power far exceeding their ownership share. Measuring $C_i = 0.20$ understates influence; calculating Banzhaf index accounting for veto power provides accurate assessment.

Recent extensions analyze endogenous coalition formation \citep{aumann1988endogenous}, showing how equilibrium structures emerge from bargaining. This connects to consent-holding's dynamic aspect: $H_t(d)$ evolves as agents form alliances, shifting power distributions. Nash bargaining solutions \citep{nash1950bargaining} maximize products of utility gains subject to Pareto efficiency—structurally similar to stakes-weighted consent maximization. \citet{kalai1975other} propose alternative axiomatizations highlighting trade-offs between equality (proportional gain-sharing) and efficiency (Pareto optimality), demonstrating solution multiplicity absent unique normative commitments—precisely what Theorem 3 predicts.

\subsection{Relational Autonomy and Consent Capacity}

\citet{mackenzie2014} three-dimensional autonomy framework distinguishes self-determination (choosing one's own life path), self-governance (regulating one's actions), and self-authorization (taking responsibility for choices). Traditional liberal autonomy assumes atomistic individuals; relational approaches \citep{mackenzie2000} recognize that autonomy is socially constituted—relationships and social structures fundamentally enable or constrain autonomous choice rather than merely influencing pre-existing capacities. \citet{nedelsky1989} analyzes how oppressive social structures systematically constrain women's autonomy through relational mechanisms, demonstrating that coercion operates not only through direct force but through systematic limitation of available choices. Oppressive systems constrain capacity for self-governance—gender oppression limits women's educational access, economic opportunities, and freedom from violence, directly undermining autonomous choice.

\citet{koggel2022} extends this to global justice, arguing that respecting autonomy requires enabling threshold capabilities, not merely non-interference. Autonomy necessitates freedom conditions: political liberties (speech, association, conscience) and personal liberties (movement, bodily autonomy, freedom from violence). Agents lacking these conditions cannot exercise meaningful consent even if formally included in $H_t(d)$.

These insights address the framework's handling of $\text{eff\_voice}_i$. Relational autonomy equals effective voice in our legitimacy equation. Simply granting $C_i > 0$ (voting rights) without resources, education, or freedom produces low $\text{eff\_voice}_i$—formal authority without capacity to exercise it. Oppressive structures systematically reduce both stakes recognition (dominant groups deny subordinated groups' $s_i$) and consent power (exclusion from $H_t(d)$ even for high-stakes domains).

This perspective addresses three framework challenges. First, it resolves circularity concerns: ``Who decides who's in $H_t(d)$?'' Answer: those with stakes plus capacity, considering relational constraints that may undermine apparent consent. Second, it handles vulnerable populations ethically. Proxy consent becomes necessary when capacity is impaired, but structures should enable gradual inclusion as capability develops rather than permanent exclusion. Third, it enables justice analysis: systematic exclusion of groups with high $s_i$ but low $\text{eff\_voice}_i$ constitutes legitimacy deficit diagnosable through $\alpha(d)$ measurement.

Application to research ethics illustrates these dynamics. Standard approaches grant legal guardians consent authority over cognitively impaired individuals. Relational approaches recognize impaired persons retain partial capacity and value particular relationships beyond legal guardianship—an older sibling may understand needs better than distant legal guardians. The framework prescription: allocate partial $C_i$ based on measured capacity and expand $H_t(d)$ to include chosen trusted relationships, raising $\alpha(d_{\text{research}})$ for the affected individual.

\subsection{Political Anthropology}
\label{subsec:political-anthropology}

Before modern political philosophy articulated theories of consent, legitimacy, and sovereignty, human societies experimented with radically diverse governance structures across millennia. Recent scholarship in political anthropology challenges linear progress narratives---from band to tribe to chiefdom to state---and reveals that the dynamics formalized by consent-holding theory have been contested for as long as humans have lived in groups.

\citet{graeber2021} present the most comprehensive challenge to conventional narratives of political development. Drawing on archaeological and ethnographic evidence from across the globe, Graeber and Wengrow demonstrate that early human societies were not locked into simple egalitarian bands that inevitably evolved toward hierarchical states. Instead, societies deliberately experimented with different governance forms, often oscillating seasonally between hierarchical and egalitarian arrangements. The Nambikwara of Brazil concentrated authority in a chief during rainy-season village life but dispersed into autonomous bands during the dry season. Pacific Northwest societies maintained rigid hierarchies during winter ceremonial seasons but operated as egalitarian fishing camps in summer. These oscillations were not transitions between developmental stages but conscious institutional choices---societies actively constructing and deconstructing $H_t(d)$ mappings in response to changing ecological, social, and ritual contexts.

This finding carries profound implications for consent-holding theory. If governance structures are \textit{chosen} rather than determined by material conditions or evolutionary stage, then the framework's emphasis on consent as a structural feature rather than a modern invention receives anthropological validation. Societies that oscillate between governance forms demonstrate awareness that different domains and seasons of collective life may require different $H_t(d)$ mappings---precisely the domain-specific analysis our framework insists upon. Graeber and Wengrow further document cases where societies explicitly rejected governance models practiced by their neighbors, suggesting that consent alignment (or its absence) was recognized and acted upon long before political philosophers formalized the concept.

\citet{boehm1999} provides the evolutionary foundation for understanding why consent-holding has always been contested. Studying egalitarian societies across multiple continents, Boehm identifies a universal mechanism he terms ``reverse dominance hierarchy'': coalitions of subordinates actively suppressing would-be dominators through ridicule, ostracism, and in extreme cases, assassination. This is not the absence of hierarchy but its active prevention---a costly institutional arrangement requiring continuous collective action. In consent-holding terms, reverse dominance hierarchies represent friction mechanisms $F(d)$ directed specifically against consent concentration. When any individual attempts to accumulate $C_i \gg \bar{C}$ (disproportionate decision authority), the coalition enforces redistribution. Boehm's evidence suggests that the disposition toward such enforcement is ancient and cross-cultural, implying that resistance to low-$\alpha$ governance is not a modern democratic value but a deep feature of human social organization.

The mechanisms Boehm documents---gossip, public ridicule, refusal to follow, coalition formation against aspirant leaders---constitute a repertoire of friction expression remarkably consistent across unrelated societies. A would-be ``big man'' in a Hadza camp, a domineering chief among the !Kung San, and an overreaching leader in an Amazonian village all face structurally similar collective responses. This cross-cultural convergence suggests that humans possess evolved psychological dispositions that detect consent misalignment (perceived $\alpha$ below some acceptable threshold) and generate friction responses calibrated to restore alignment. The framework captures this dynamic formally: when $\alpha(d) < \tau$ (some culturally variable but psychologically grounded threshold), friction $F(d)$ increases discontinuously, consistent with our Hypothesis H3 on threshold effects.

\citet{scott2017} examines the question from the opposite direction: how did states first impose centralized $H_t(d)$ on populations that actively resisted? Scott argues that early states in Mesopotamia, China, and the Indus Valley succeeded not through legitimacy or superior governance but through the coincidence of grain agriculture (which produces storable, taxable surplus) and population concentration (which enables surveillance and control). Early state formation was, in Scott's analysis, primarily a project of consent extraction rather than consent alignment: concentrating $C_i$ in ruling elites while maximizing $s_i(d)$ for subject populations through sedentarization, walls (which kept people in as much as enemies out), and the replacement of diverse subsistence strategies with legible monocultures.

Scott documents extensive evidence of populations fleeing early states into upland regions, swamps, and frontier zones---what he terms ``escape agriculture'' and ``escape social structures'' designed specifically to resist state incorporation. These populations were not ``primitive'' societies awaiting state formation; they were communities that had experienced state governance and rejected it. In consent-holding terms, they had observed low-$\alpha$ regimes and chosen exit over voice. The ``barbarian'' periphery of every early civilization represented not developmental lag but active friction response: populations voting with their feet when formal voice mechanisms within $H_t(d)$ were absent.

These anthropological findings ground the consent-holding framework in deep historical evidence. Three implications deserve emphasis. First, $H_t(d)$ has always been contested: the question of who decides is not a modern political invention but a perennial structural problem of collective life. Second, friction $F(d)$ is the universal response to consent misalignment---its expression varies from Boehm's reverse dominance hierarchies to Scott's escape agriculture, but the underlying dynamic (perceived misalignment between stakes and voice triggering costly resistance) is constant across millennia and continents, as formalized in the friction dynamics of multi-agent coordination \citep{farzulla2025aoc}. Third, the framework's domain-specific approach is validated by anthropological evidence: societies that separated governance by domain (seasonal oscillation, domain-specific authority structures) demonstrated greater institutional resilience than those imposing uniform $H_t(d)$ across all collective decisions. The consent-holding framework, far from imposing modern liberal categories on pre-modern societies, formalizes dynamics that human communities have navigated for at least twelve thousand years.

\subsection{Social Movements and Collective Action}
\label{subsec:social-movements}

Social movements are the primary mechanism through which friction $F(d)$ manifests at scale and generates pressure for institutional reform. Where Section~\ref{subsec:political-anthropology} established that consent contestation is anthropologically universal, this section examines how organized collective action translates individual-level misalignment perceptions into systemic friction that restructures $H_t(d)$.

\citet{olson1965logic} established the foundational paradox of collective action: rational self-interested individuals will free-ride on others' contributions to public goods, implying that large groups should be unable to organize for collective benefit. This result carries a crucial implication for consent-holding measurement: observed friction $F(d)$ systematically underestimates true consent misalignment. If organizing collective action is costly and individual contributions to large-scale movements are negligible, then many individuals who perceive $\alpha(d) < \tau$ will nevertheless fail to express friction through protest, petition, or political mobilization. The gap between experienced misalignment and expressed friction introduces measurement error into any empirical assessment of $\alpha(d)$ based on observed conflict indicators. Olson's theory predicts that smaller groups with concentrated stakes will mobilize more effectively than larger groups with diffuse stakes---precisely why industry lobbies outperform consumer movements, and why minority rights organizations often generate friction disproportionate to their population share. In consent-holding terms, the covariance between $s_i(d)$ and mobilization capacity introduces systematic bias: groups whose stakes are intense but narrowly distributed produce more observable friction per unit of misalignment than groups whose stakes are broadly distributed but individually dilute.

\citet{tarrow1998power} extends collective action theory by identifying the structural conditions under which social movements overcome Olsonian barriers. His concept of ``political opportunity structures'' describes the institutional openings---elite divisions, electoral realignments, international pressures, regime transitions---that reduce the costs of collective action and increase the expected returns from friction expression. In consent-holding terms, political opportunity structures are moments when the institutional translation function from $F(d)$ to $\Delta H_t(d)$ becomes more favorable: the same level of friction produces larger consent-holding adjustments during periods of elite vulnerability than during periods of consolidated authority.

Tarrow's analysis also highlights the role of framing and cultural resources in movement mobilization. Movements do not simply respond to objective misalignment; they construct interpretive frameworks that make misalignment visible, urgent, and actionable. The civil rights movement's framing of racial segregation as un-American---contradicting national commitments to equality and freedom---was more effective than purely interest-based appeals, precisely because it identified consent misalignment not merely as harmful to Black Americans but as a violation of the polity's own legitimacy principles. This framing dynamic connects to our measurement framework: $\alpha(d)$ is partly socially constructed. The ``objective'' distribution of stakes and voice provides a structural foundation, but collective interpretation of whether that distribution constitutes legitimate governance depends on cultural frames, historical narratives, and political entrepreneurs who articulate misalignment in compelling terms.

\citet{tilly2007} provides the historical foundation for understanding how social movements relate to democratization---the long-term process of raising $\alpha(d)$ across political domains. Tilly argues that democracy is not a fixed institutional form but a moving target, defined by the breadth, equality, binding consultation, and protection of political participation. Democratization occurs when these dimensions expand; de-democratization when they contract. This processual view aligns precisely with the consent-holding framework's dynamic emphasis: $\alpha(d,t)$ is a continuously varying quantity whose trajectory depends on the interaction between institutional design and social mobilization. Tilly's evidence from European and Latin American cases demonstrates that democratization rarely occurs through top-down institutional design alone; it requires sustained bottom-up friction from social movements that force elite concessions, combined with institutional mechanisms that lock in consent-holding gains.

\citet{tilly2008contentious} further specifies the repertoire of friction expression available to social movements. ``Contentious performances''---demonstrations, strikes, petitions, blockades, occupations---are not spontaneous eruptions but culturally learned routines that evolve over time. The petition was invented; the demonstration has a history; the sit-in was consciously innovated. Each performance type represents a technology for converting individual-level misalignment perceptions into collective friction visible to $H_t(d)$ incumbents. Tilly's insight is that the available repertoire constrains friction expression: movements can only deploy performances that are culturally recognized as legitimate protest within their historical context. A medieval grain riot and a modern social media campaign both express friction, but through different culturally available channels. This implies that our friction measure $F(d)$ is not a simple aggregate of individual dissatisfaction but is mediated by the protest technologies and cultural repertoires available in a given historical-institutional context.

At the individual level, \citet{schussman2005process} investigate why some individuals participate in protest while others with similar grievances do not. Their analysis identifies biographical availability (having time and resources), prior activism experience, and social network embeddedness as key predictors of participation, controlling for policy disagreement. These findings connect directly to $\text{eff\_voice}_i$: formal grievance (perceiving low $\alpha(d)$ in a salient domain) is necessary but not sufficient for friction expression. Individuals must also possess the capacity to act---resources, social connections, organizational infrastructure---that transforms dissatisfaction into observable friction. This individual-level analysis reinforces Olson's structural insight: the gap between experienced misalignment and expressed friction is not random but systematically correlated with socioeconomic position. Those with the highest stakes and lowest voice are often precisely those with the least capacity to express friction, creating a measurement paradox where the most severe consent misalignment is the least visible---a dynamic analyzed in detail by \citet{farzulla2025stakes}.

\citet{stekelenburg2010social} synthesize the social psychology of protest participation, identifying perceived injustice, efficacy beliefs, and social identity as the three primary motivational pathways. Perceived injustice maps onto our misalignment detection: individuals who perceive that $\alpha(d)$ falls below acceptable thresholds experience moral outrage that motivates action. Efficacy beliefs correspond to expectations about the translation function from individual friction contributions to institutional change: will protesting actually shift $H_t(d)$? Social identity determines whether individuals conceptualize their stakes as individual or collective---whether misalignment in domain $d$ is experienced as ``my problem'' or ``our problem.'' This three-pathway model explains why identical objective conditions produce different levels of friction across populations and time periods: the same $\alpha(d)$ can generate mass mobilization or quiescent acceptance depending on injustice framing, efficacy perceptions, and identity politicization.

\citet{singh2020globalization} extends the analysis to contemporary globalization, demonstrating that economic integration intensifies cross-domain friction through interconnected stakes. When global supply chains link workers in Bangladesh to consumers in Europe, environmental degradation in the Amazon to commodity prices in Chicago, and financial instability in one country to pension values in another, the affected set $S_d$ for each governance domain expands beyond national boundaries. But $H_t(d)$ remains largely bounded by nation-states, creating a structural gap between transnational stakes and national consent-holding arrangements. Singh's evidence shows that this gap manifests as increased social instability: anti-globalization protests, nationalist backlash, populist movements, and transnational advocacy campaigns all represent friction responses to a global misalignment between who bears consequences and who holds decision authority. The consent-holding framework predicts this pattern: as globalization expands $S_d$ beyond the boundaries of existing $H_t(d)$ arrangements, $\alpha(d)$ declines mechanically, and friction increases as predicted by H1.

These collective action dynamics connect directly to the framework's hypotheses. H1 (higher $\alpha$ predicts lower future $F$) is validated by the inverse: social movements arise precisely in domains where consent misalignment is perceived as severe. H4 (persistent $F$ predicts future $\alpha$ increases) captures the core mechanism of democratic reform through movement pressure. The collective action literature adds crucial nuance: the translation from misalignment to friction is neither automatic nor proportional, but mediated by organizational capacity, political opportunity, cultural repertoires, and individual-level resources. Empirical testing of the framework must account for these mediating factors or risk confounding low friction with high consent alignment when in fact it may reflect high barriers to collective action expression.

\subsection{Institutional Theory and Critical Junctures}
\label{subsec:institutional-theory}

Institutions---the formal rules, informal norms, and enforcement mechanisms that structure collective life---determine how consent-holding evolves over time. While the preceding sections examined how stakeholders contest $H_t(d)$ from below (social movements) and how communities self-govern specific domains (common-pool resources), this section addresses the macro-institutional structures that constrain and enable consent alignment across entire polities.

\citet{acemoglu2012why} provide the most influential contemporary account of how institutional configurations determine long-run development trajectories. Their distinction between \textit{inclusive} institutions (those distributing political and economic power broadly) and \textit{extractive} institutions (those concentrating power among a narrow elite) maps directly onto consent-holding theory's $\alpha(d)$ metric. Inclusive institutions, by definition, extend $C_i > 0$ to broad populations across multiple governance domains---political voice through elections, economic opportunity through property rights and contract enforcement, social mobility through education access. Extractive institutions concentrate $C_i$ among elites while imposing high $s_i(d)$ on excluded populations through taxation, forced labor, and restricted economic participation. In our notation, inclusive institutions maintain high $\alpha(d)$ across multiple domains simultaneously; extractive institutions maintain low $\alpha(d)$ sustained by coercive capacity that suppresses friction expression below the threshold at which institutional change becomes inevitable.

\citet{acemoglu2019narrow} refine this framework through the metaphor of the ``narrow corridor''---the constrained institutional space within which both state capacity and societal mobilization are sufficiently strong to maintain what they call ``Shackled Leviathan'' governance. Too much state capacity without societal counterweight produces despotism ($C_i$ concentrated in state apparatus, $\alpha \approx 0$ for civil domains); too much societal mobilization without state capacity produces anarchy ($H_t(d)$ undefined or contested across all domains, friction $F(d)$ unconstrained). The narrow corridor represents the dynamic equilibrium where state and society check each other: institutional alpha is maintained not by benevolent design but by continuous tension between centralizing and decentralizing forces.

This framework has two important implications for consent-holding theory. First, legitimacy is not a static achievement but a dynamic balance requiring continuous recalibration. Even polities that achieve high $\alpha(d)$ can de-democratize if the balance between state capacity and societal mobilization shifts---a process visible in contemporary democratic backsliding. Second, the narrow corridor metaphor implies that moderate friction is not pathological but structurally necessary: societal mobilization capacity (the potential for friction expression) constrains state overreach, while state enforcement capacity constrains societal fragmentation. A polity with zero observed friction may be despotic (friction suppressed) rather than legitimate (friction absent because $\alpha$ is high). Empirical measurement must distinguish these cases---a distinction the consent-holding framework enables through independent measurement of $\alpha(d)$ and $F(d)$ rather than inferring one from the other.

Critical junctures---moments of radical institutional restructuring---represent the most dramatic changes in $H_t(d)$. \citet{soifer2012causal} clarifies the causal logic of critical junctures, distinguishing between permissive conditions (structural factors that make change possible) and productive conditions (specific events or decisions that determine the direction of change). In consent-holding terms, permissive conditions arise when existing $H_t(d)$ arrangements become unsustainable---accumulated friction exceeds institutional absorptive capacity, external shocks destabilize enforcement mechanisms, or elite coalitions fracture. Productive conditions determine which new $H_t(d)$ emerges from the juncture: revolutionary leadership, constitutional conventions, foreign intervention, or civil war each produce different post-juncture consent configurations.

The critical juncture framework, complemented by \citet{capoccia2007study} on theory, narrative, and counterfactual analysis, implies that $\alpha(d)$ trajectories are path-dependent: early institutional choices create self-reinforcing feedback loops that make subsequent reform either easier or harder. Colonial institutional legacies, for instance, locked in extractive $H_t(d)$ configurations that persisted centuries after colonial rule ended \citep{acemoglu2012why}. The framework's dynamic emphasis---tracking $H_t(d)$ as a time-varying function rather than a static institutional description---captures this path dependence formally: the consent-holding configuration at time $t$ constrains the feasible set of configurations at $t+1$, with critical junctures representing discontinuous jumps outside the normal evolutionary path.

\citet{linz1996problems} bring this theoretical apparatus to bear on the specific challenge of democratic transition and consolidation---the process of raising $\alpha(d)$ from authoritarian to democratic levels and sustaining those gains. Their comparative analysis of transitions in Southern Europe, South America, and post-communist Europe reveals that democratic consolidation requires not merely formal institutional change (new constitutions, elections) but normative commitment: democracy becomes ``the only game in town'' when all significant political actors accept democratic $H_t(d)$ as the exclusive framework for pursuing their interests. In consent-holding terms, consolidation occurs when high $\alpha(d_{political})$ becomes self-reinforcing: the benefits of operating within democratic consent structures exceed the expected gains from extra-institutional power seizure for all relevant actors. Linz and Stepan's conditions for consolidation---an autonomous civil society, a relatively autonomous political society, rule of law, a usable state bureaucracy, and an institutionalized economic society---can be reinterpreted as the infrastructure requirements for sustaining high $\alpha(d)$ across multiple governance domains simultaneously. A democracy that achieves high $\alpha(d_{elections})$ but fails to extend consent alignment to economic governance, judicial independence, or bureaucratic accountability remains unconsolidated and vulnerable to regression.

\citet{sartori1994comparative} examines the constitutional engineering dimension of consent-holding design. Electoral systems, executive-legislative relations, federalism, and judicial review represent institutional choices that structure how $C_i$ is distributed across populations and governance domains. Sartori demonstrates that these choices have systematic consequences: proportional representation raises $\alpha(d_{legislative})$ by granting representation to minority parties but may reduce governability; presidentialism concentrates executive $C_i$ but provides electoral accountability; federalism enables domain-specific $H_t(d)$ optimization but creates coordination challenges. Constitutional design, in this light, is the engineering of consent-holding structures---an optimization problem that the consent-holding framework renders explicit and, in principle, empirically evaluable.

\citet{gerring2015institutional} evaluates one specific institutional arrangement---direct democracy---through a comparative lens. Their institutional theory of direct democracy identifies the conditions under which referenda and initiatives raise or lower governance quality: direct democracy performs well when issues are salient, information is accessible, and outcomes are reversible, but poorly when issues are technical, information asymmetries are severe, and decisions have irreversible consequences. In consent-holding terms, direct democracy maximizes $C_i$ breadth (all citizens vote) but does not guarantee high $\text{eff\_voice}_i$ (informed, considered judgment). The performance conditions Gerring and colleagues identify correspond to domain characteristics where equal-voice mechanisms are likely to produce high or low $L(d,t)$---providing empirical support for our distinction between domains where equal voice and stakes-weighting produce different legitimacy outcomes.

\citet{evans1995embedded} introduces the concept of ``embedded autonomy'' to explain how certain developmental states---South Korea, Taiwan, Singapore---achieved sustained economic growth through technocratic governance that maintained moderate legitimacy despite limited democratic participation. Evans argues that these states succeeded because their bureaucracies were simultaneously autonomous from societal capture (enabling coherent policy) and embedded in dense networks of information exchange with the private sector (enabling responsiveness). In consent-holding terms, embedded autonomous states maintained moderate $\alpha(d_{economic})$ through competence rather than consent: high performance $P(d)$ partially compensated for low $\alpha(d)$ in our legitimacy equation $L(d,t) = w_1 \cdot \alpha(d,t) + w_2 \cdot P(d,t)$. This pattern validates Proposition 1 (competence-consent trade-off) and illuminates its boundary conditions: technocratic legitimacy is sustainable when economic performance remains high but vulnerable to legitimacy crises when performance falters, because no consent reservoir cushions performance shocks.

\citet{slater2010informative} address a methodological challenge facing institutional analysis---and by extension, consent-holding measurement. Their concept of ``informative regress'' highlights the problem of tracing causal chains backward from institutional outcomes to antecedent conditions: at what point does the causal chain become uninformative, and how do analysts distinguish genuinely explanatory critical antecedents from arbitrary starting points? For consent-holding theory, this methodological concern is directly relevant: when we trace $H_t(d)$ backward to explain current consent configurations, we must specify which historical antecedents are genuinely causal versus merely temporally prior. Slater and Simmons propose that informative antecedents are those that \textit{could have been otherwise}---moments of genuine contingency where different choices would have produced different institutional trajectories. This criterion complements our critical juncture analysis: the most informative moments for understanding consent-holding evolution are those where $H_t(d)$ was genuinely contested and could have settled into different configurations. Moreover, as \citet{nolan2015international} demonstrates, critical junctures in consent-holding frequently have international dimensions: wars, alliances, and geopolitical pressures reshape domestic institutional configurations by altering the balance of power between state elites and societal actors. The international system both constrains and enables domestic consent-holding reform, adding an external dimension to the dynamics formalized by our framework.

The institutional theory literature thus provides consent-holding theory with both macro-structural context and analytical tools. Institutions determine the feasible range of $\alpha(d)$ in any given polity; critical junctures explain how that range shifts discontinuously; constitutional engineering describes the design choices available for optimizing consent-holding structures; and embedded autonomy illustrates the competence-consent trade-off that our Proposition 1 formalizes. Together, these insights ground the framework's historical applications (Part III of this monograph) in established institutional analysis while demonstrating that consent-holding theory contributes something the institutional literature lacks: a unified metric ($\alpha(d,t)$) for comparing legitimacy across institutional configurations, governance domains, and historical periods.

\section{Formal Framework: Primitives, Axioms, and Theorems}
\label{sec:framework}

This section establishes the framework's formal foundations through precise definitions, minimal axioms, and structural theorems. The approach proceeds deductively: from spare assumptions about collective decision-making to necessary conclusions about consent-holding's existence, friction's inevitability, and legitimacy's measurement.

\subsection{Primitives and Definitions}

We begin with foundational concepts requiring no prior theoretical commitment. An \textbf{agent} is any entity capable of selecting among actions, indexed $i \in A = \{1, \ldots, N\}$. Agents may be individuals, organizations, algorithms, or collective bodies—the framework remains agnostic about internal composition. A \textbf{domain} represents a decision-relevant sphere—a policy area, firm process, household choice, or any context requiring action selection. The set of domains is $D = \{d_1, \ldots, d_M\}$. Each domain $d$ admits a set of possible actions $X_d$, from which one action $x_d \in X_d$ must be selected.

\textbf{Outcomes} represent realized states resulting from action vectors $\mathbf{x} = (x_{d_1}, \ldots, x_{d_M})$ through an environment mapping $E: \prod_d X_d \to O$, where $O$ denotes the outcome space. An agent $i$'s \textbf{stake} in domain $d$, denoted $s_i(d) \geq 0$, quantifies sensitivity to outcomes in that domain. Stakes may reflect material exposure, legal consequences, capability impacts, or existential threats.

Each agent possesses \textbf{preferences} over outcomes, represented either as complete orderings $\succeq_i$ or utility functions $U_i: O \to \mathbb{R}$. Preferences induce ideal points $x^*_{i,d}$ in each domain—the action agent $i$ most prefers given others' anticipated choices.

\textbf{Consent} represents the normative right to decide in a domain—who may authoritatively say ``yes'' or ``no'' to proposed actions. Following \citet{locke1689} consent theory foundations, political obligation derives from voluntary agreement; actual consent is required for legitimate authority. The \textbf{consent-holder mapping} $H_t(d) \in \Delta(C)$ specifies the distribution of decision authority over possible holders $C$ at time $t$. Individual \textbf{consent power} $C_{i,d} \in [0,1]$ represents agent $i$'s effective share of decision authority in domain $d$, with $\sum_i C_{i,d} = 1$.

\subsection{Axioms}

The framework rests on seven axioms representing minimal commitments about collective decision-making.

\textbf{A1. Action Precedence}: Every non-null outcome in a domain is produced by some action (including ``do nothing'').

\textbf{A2. Decision Requirement}: Every action is selected by some decision procedure (choice, rule, randomization, delegation).

\textbf{A3. Shared Reality}: Outcomes alter a world co-occupied by multiple agents; externalities exist.

\textbf{A4. Finitude}: Agents have finite time, attention, and cognitive capacity; no single agent can decide everything alone.

\textbf{A5. Plurality}: Agents' preference orderings differ on at least some domains.

\textbf{A6. Salience}: For each domain, at least one agent has $s_i(d) > 0$.

\textbf{A7. Fallibility/Subjectivity}: Perception and valuation are frame-dependent; no universal content-level value ordering is logically forced.

\subsection{Theorem 1: Consent-Holding Necessity}

\begin{theorem}[Consent-Holding Necessity]
In any domain $d$ where a non-null outcome occurs, there exists a consent-holder mapping $H_t(d)$.
\end{theorem}

\begin{proof}[Proof Sketch]
By A1-A2, any outcome resulted from an action selected through some procedure. A procedure implies a locus of control—the entity/entities choosing the action, establishing the choice rule, or delegating to randomization. This locus constitutes $H_t(d)$. Therefore, denying $H_t(d)$'s existence contradicts A2. \qed
\end{proof}

\subsection{Theorem 2: Inevitable Friction}

\begin{theorem}[Inevitable Friction]
If there exist agents $i, j$ with divergent preferences on domain $d$ and $s_i(d), s_j(d) > 0$, then unless $H_t(d)$ exactly reproduces stakes-weighted unanimity, at least one agent experiences moral/political friction.
\end{theorem}

We formalize \textbf{friction} in domain $d$ as:
\begin{equation}
F(d,t) = \sum_i s_i(d) \cdot \delta(x_d(t), x^*_{i,d})
\label{eq:friction}
\end{equation}

where $x^*_{i,d}$ represents agent $i$'s ideal action and $\delta$ measures divergence. For discrete choices, $\delta(x, x^*) = 0$ if $x = x^*$, else 1. For continuous policy spaces, $\delta(x, x^*) = |x - x^*|$ captures distance from ideal points.

Introducing tolerance thresholds $\tau_i$ yields:
\begin{equation}
F_\tau(d,t) = \sum_i s_i(d) \cdot \max(0, \delta(x_d, x^*_{i,d}) - \tau_i)
\label{eq:friction-tolerance}
\end{equation}

\subsection{Definition 1: Legitimacy as Consent Alignment}

We operationalize legitimacy through stakes-weighted consent alignment. Define the \textbf{affected set} $S_d = \{i | s_i(d) > 0\}$. \textbf{Consent alignment} is:
\begin{equation}
\alpha(d,t) = \frac{\sum_{i \in S_d} s_i(d) \cdot \text{eff\_voice}_i(d,t)}{\sum_{i \in S_d} s_i(d)}
\label{eq:alignment}
\end{equation}
where $\text{eff\_voice}_i$ represents agent $i$'s effective decision power in $H_t(d)$.

\textbf{Definitional Note}: This is a \textit{measurement framework}, not a derived result. We define legitimacy as the degree to which consent power tracks stakes distribution, making the concept empirically tractable. The framework's predictive power lies in the hypothesis that low $\alpha$ generates observable friction—a claim requiring empirical validation beyond the definition itself.

A minimal procedural legitimacy condition requires $\alpha(d,t) \geq \tau$ for society-specific threshold $\tau$. Persistent $\alpha < \tau$ predicts observable friction through unrest, exit, sabotage, or normative decay.

\subsection{Postulate 1: Competence-Consent Trade-Off}

We model overall legitimacy as combining consent alignment and performance:
\begin{equation}
L(d,t) = w_1 \cdot \alpha(d,t) + w_2 \cdot P(d,t)
\label{eq:legitimacy}
\end{equation}
where $\alpha(d,t)$ represents stakes-weighted consent alignment, $P(d,t)$ denotes performance/competence metrics, and $w_1, w_2 \geq 0$ reflect society-specific weights on voice versus results.

This is a \textbf{postulated relationship} rather than a derived theorem. The linear combination assumes legitimacy trades off between consent and competence, but alternative functional forms (multiplicative, threshold-based) are possible. Empirical work validating this specification against alternatives remains a key research agenda.

\textbf{Remark on Weight Determination}: The weights $w_1, w_2$ are not free parameters requiring external normative specification, but endogenous scope conditions revealed through constitutional-level decisions (see Section~\ref{sec:weight-determination} for full meta-legitimacy resolution). Societies whose weight configurations produce excessive friction face structural pressure to reform. We can characterize admissible weight functions axiomatically: any stable society must satisfy $w_2/w_1 > f(\text{Var}[s_i(d)])$ where $f$ is a function of stakes heterogeneity derived from friction minimization. Future empirical work will estimate weights via:
\begin{equation}
(w_1^*, w_2^*) = \arg\min_{w_1,w_2} \mathbb{E}[F(d,t; w_1, w_2)]
\label{eq:weight-calibration}
\end{equation}
where friction minimization across constitutional reforms provides revealed preference data for weight estimation. Sequential Monte Carlo methods \citep{lux2018} enable parameter estimation for agent-based models through particle filtering, providing techniques applicable to calibrating consent-holding frameworks against empirical institutional data. This transforms weight determination from a normative choice into an empirical optimization problem, sidestepping the meta-legitimacy regress.

This formulation makes explicit that different systems optimize different points on the legitimacy frontier. Technocracies maximize $P$, often sacrificing $\alpha$ by concentrating consent in experts. Direct democracies maximize $\alpha$ through universal suffrage, potentially reducing $P$ on technical domains where distributed knowledge is sparse.

\subsection{Theorem 3: Minimal Absolutism from Relativism}

\begin{theorem}[Relativism $\Rightarrow$ Minimal Absolutism]
Given A7 (value frame-dependence), the claim ``all value judgments are frame-relative'' is coherent only if the \textbf{structure} enabling frames is invariant. Therefore, at least one absolute exists: the necessity of consent-holding over shared outcomes wherever A1-A6 hold.
\end{theorem}

\begin{proof}[Proof Sketch]
Suppose all value claims are frame-dependent (A7). Frame-dependence presupposes frames exist—perspectives from which valuations occur. Frames belong to agents inhabiting shared reality (A3) with plural preferences (A5). These agents make decisions affecting each other (A1-A2). Such decisions require consent-holder mappings $H_t(d)$ (Theorem 1). Therefore, relativism about content-level values doesn't extend to structural necessities. \qed
\end{proof}

\section{Operationalization: Empirical Measurement and Identification}
\label{sec:operationalization}

The theoretical framework provides analytical tools for understanding consent-holding structures. This section bridges theory and empirical application by specifying how the framework's core concepts can be measured, how causal relationships can be identified econometrically, and what testable predictions emerge.

\subsection{Formal Measurement Framework}

We operationalize consent-holding through a consent matrix $\mathbf{C} \in [0,1]^{N \times M}$, where each element $C_{i,d}$ represents agent $i$'s effective decision share in domain $d$, subject to the normalization constraint $\sum_i C_{i,d} = 1$. This matrix captures both de jure authority and de facto power. In simple majority voting systems with equal suffrage, $C_{i,d} = 1/N_{voters}$ for all enfranchised $i$ and $C_{i,d} = 0$ for excluded populations. In shareholder governance, $C_{i,d} = \text{shares}_i / \text{total\_shares}$. In technocratic systems, $C_{i,d} = 1/|E|$ if agent $i$ belongs to the expert set $E$, zero otherwise.

Complementing the consent matrix, the stakes vector $\mathbf{s}(d) \in \mathbb{R}^N_{\geq 0}$ quantifies each agent's exposure to consequences in domain $d$. Stakes measurement presents both conceptual and practical challenges. Conceptually, stakes may reflect material exposure (tax burden relative to income), capability impacts (health outcomes affected), or existential threats (survival risks from climate policy). Different domains may legitimately employ different stakes conceptions.

Combining these elements, consent alignment in domain $d$ at time $t$ is measured as:
\begin{equation}
\alpha(d,t) = \frac{\sum_{i \in S_d} s_i(d) \cdot \text{eff\_voice}_i(d,t)}{\sum_{i \in S_d} s_i(d)}
\label{eq:alignment-operational}
\end{equation}

where $S_d = \{i | s_i(d) > 0\}$ denotes the affected set and $\text{eff\_voice}_i$ represents agent $i$'s effective decision power accounting for both formal authority $C_{i,d}$ and capacity constraints.

\subsection{Friction Metrics and Tolerance-Weighted Extensions}

Political friction represents the stakes-weighted aggregate deviation between realized outcomes and stakeholder preferences. In its basic form:
\begin{equation}
F(d,t) = \sum_i s_i(d) \cdot \delta(x_d(t), x^*_{i,d})
\label{eq:friction-basic}
\end{equation}

For continuous policy spaces, Euclidean distance $\delta(x, x^*) = |x - x^*|$ captures proximity to ideal points. The tolerance-weighted friction measure incorporating agent-specific tolerance parameters $\tau_i \geq 0$ is:
\begin{equation}
F_\tau(d,t) = \sum_i s_i(d) \cdot \max(0, \delta(x_d(t), x^*_{i,d}) - \tau_i)
\label{eq:friction-tolerance-operational}
\end{equation}

This captures that agents tolerate ``good enough'' governance within zones of acceptability, mobilizing only when deviations exceed tolerance thresholds.

\subsection{Empirical Identification Strategies}

The core empirical prediction connecting alignment to friction generates testable hypotheses through panel regression specifications:
\begin{equation}
F_{d,t} = \beta_0 + \beta_1 \cdot \alpha_{d,t} + \beta_2 \cdot P_{d,t} + \gamma \cdot X_{d,t} + \mu_d + \lambda_t + \varepsilon_{d,t}
\label{eq:regression}
\end{equation}

where $F_{d,t}$ represents friction, $\alpha_{d,t}$ denotes consent alignment, $P_{d,t}$ captures performance outcomes, $X_{d,t}$ includes control variables, $\mu_d$ represents domain fixed effects, $\lambda_t$ represents time fixed effects, and $\varepsilon_{d,t}$ is the error term.

The framework's theoretical predictions constrain coefficient signs: $\beta_1 < 0$ (higher alignment reduces friction), $\beta_2 < 0$ (better performance reduces friction). Instrumental variable strategies address endogeneity concerns by exploiting exogenous variation in consent structures. Historical franchise expansions driven by international diffusion provide quasi-experimental variation.

Our operationalization of $\alpha$ through V-Dem data engages an active methodological debate. \citet{little2024measuring} argue that ``objective'' indicators of democracy (electoral incumbency patterns, executive constraints) show little evidence of global democratic backsliding over the past decade, suggesting that expert-coded indices like V-Dem may reflect coder bias rather than genuine institutional change. Their challenge is important for our framework: if $\alpha$ proxies constructed from expert-coded data track coder sentiment rather than actual consent alignment, the historical validation in Part~\ref{part:historical} would be undermined. However, \citet{knutsen2024conceptual} demonstrate through embedded experiments and systematic robustness checks that time-varying expert bias is unlikely to drive V-Dem's backsliding estimates. They show that Little and Meng's ``objective'' alternatives---turnover rates, constitutional term limits---are themselves poor measures of democratic quality, often failing to capture the subtler institutional erosion (media capture, judicial packing, administrative harassment) that expert coders are specifically trained to detect. Our framework actually clarifies this debate: the consent-holding $\alpha$ parameter is conceptually closer to what V-Dem's expert-coded indicators measure (effective stakeholder voice across multiple institutional dimensions) than to simple electoral outcomes. The richness of $\alpha$---requiring information about stakes distribution, effective voice, and capacity constraints---makes expert assessment a feature rather than a bug, provided inter-coder reliability is maintained (see Appendix~\ref{sec:appendix-data}).

\subsection{Testable Predictions and Empirical Hypotheses}

\textbf{Hypothesis 1 (Alignment-Friction Relationship)}: Across domains and time periods, higher consent alignment $\alpha(d,t)$ predicts lower friction $F(d,t+k)$ with lags $k$ reflecting institutional adjustment speeds:
\[
\frac{\partial F(d,t+k)}{\partial \alpha(d,t)} < 0 \quad \text{for } k \geq 0
\]

\textbf{Hypothesis 2 (Stakes-Consent Covariance)}: Institutional reforms increasing the covariance between stakes and consent power—$\text{Cov}(s_i(d), C_{i,d})$—reduce friction through alignment improvement.

\textbf{Hypothesis 3 (Threshold Effects)}: Domains with alignment below societal tolerance thresholds—$\alpha(d) < \tau_{legitimacy}$—exhibit discontinuously higher instability, generating nonlinearity in the alignment-friction relationship.

\textbf{Hypothesis 4 (Temporal Dynamics)}: Persistent friction $F(d,t)$ predicts future alignment increases $\alpha(d,t+k)$ through institutional reform pressure:
\[
\frac{\partial \alpha(d,t+1)}{\partial F(d,t)} > 0
\]

\textbf{Hypothesis 5 (Performance Interactions)}: The alignment-friction relationship weakens in domains with high performance $P(d,t)$, as competent governance partially compensates for voice deficits.

% ============================================================================
\part{Normative Architecture}
\label{part:normative}
\thispagestyle{partpage}
% ============================================================================

\section{Social Contract Theories as Consent-Power Distribution Mechanisms}
\label{sec:social-contract}

Social contract theories can be reinterpreted through the consent-holding framework as different proposals for allocating consent power $C_{i,d}$ across agents in various domains. Rather than treating these theories as competing comprehensive doctrines, we analyze them as institutional design proposals---each optimizing a different legitimacy function subject to domain-specific constraints. This section consolidates the normative architecture: each doctrine is parsed through a uniform interpretive lens, mapped to formal consent-holding variables, and subjected to comparative diagnostic analysis.

\subsection{Interpretive Framework}
\label{subsec:interpretive-framework}

To avoid conflating normative rhetoric with institutional mechanics, we parse each social contract doctrine using four analytic layers:

\begin{enumerate}[leftmargin=1.4em]
  \item \textbf{Allocation rule}: who receives decision authority $C_{i,d}$ in domain $d$, and on what basis.
  \item \textbf{Justification rule}: why that allocation is defended normatively---the moral vocabulary licensing the distribution.
  \item \textbf{Correction rule}: how misalignment between stakeholders and authority-holders is detected and revised.
  \item \textbf{Failure mode}: where friction $F(d,t)$ accumulates under stress, and what form institutional breakdown takes.
\end{enumerate}

This decomposition supports apples-to-apples comparison across doctrines with otherwise incompatible moral vocabularies. A Hobbesian argument for sovereign concentration and a Rousseauian argument for collective self-rule both \textit{propose} specific consent-holder mappings $H_t(d)$; they differ on allocation logic, correction mechanisms, and predicted friction profiles. By holding the analytic frame constant, doctrinal disagreement becomes empirically legible as disagreement over admissible regions in $(\alpha, P, w_1, w_2)$ space.

\subsection{Classical Doctrines}
\label{subsec:classical-doctrines}

\subsubsection{Hobbesian Monopoly and Security-First Legitimacy}

In a Hobbesian template \citep{hobbes1651}, consent is concentrated in a sovereign to suppress violent conflict and coordinate collective defense. The allocation is highly centralized:
\[
C_{\text{sovereign},d} \approx 1, \qquad C_{i \neq \text{sovereign},d} \approx 0.
\]

\textit{Allocation rule}: authority concentrates in a single node capable of enforcing order across broad domains. \textit{Justification rule}: the alternative---war of all against all---is worse for every agent than any stable concentration. \textit{Correction rule}: breakdown and reconstitution after crisis; there is no legitimate internal correction short of collapse. \textit{Failure mode}: when threat conditions normalize but authority remains centralized, consent alignment $\alpha(d,t)$ decays. Friction becomes \textit{repressed} rather than resolved, often reappearing as legitimacy shocks once coercive capacity weakens.

The consent-holding interpretation: high short-run performance weight ($w_2 \gg w_1$) in acute instability contexts can be legitimacy-improving if fragmentation costs are extreme. The model expression for short-run legitimacy under Hobbesian conditions is:
\begin{equation}
L(d,t) = w_1 \alpha(d,t) + w_2 P(d,t), \quad w_2 \gg w_1.
\label{eq:hobbesian-legitimacy}
\end{equation}

The diagnostic prediction follows directly: if threat intensity declines but concentration persists, measured performance may remain acceptable while friction rises in excluded groups:
\[
\frac{\partial F(d,t)}{\partial t} > 0 \quad \text{under persistent low } \alpha(d,t).
\]
This pattern---stable performance masking rising misalignment---characterizes authoritarian decay and connects to the threshold dynamics of Hypothesis H3. The framework's empirical prediction is specific: Hobbesian regimes should show \textit{decoupling} between performance metrics ($P$ remaining stable or high) and friction indicators ($F$ rising among excluded populations) during the period between threat normalization and eventual correction. This decoupling is diagnostically distinctive: in consent-aligned regimes, $P$ and $F$ should co-move (high performance reducing friction), while in Hobbesian regimes, the relationship breaks down as repression substitutes for alignment.

Historical examples abound: the Soviet Union maintained acceptable economic performance through the 1970s while friction accumulated in Eastern Europe, the Baltic states, and among internal dissidents; the performance metrics masked the structural misalignment that produced sudden collapse once coercive capacity weakened in the late 1980s. Singapore's governance model---high performance with limited political consent---represents a currently stable Hobbesian configuration, but the framework predicts rising friction vulnerability as the founding-generation legitimacy claim weakens and affected populations develop higher expectations for consent alignment.

\subsubsection{Lockean Conditional Delegation}

Lockean doctrine \citep{locke1689} treats authority as delegated and revocable. \textit{Allocation rule}: institutions hold $C_{i,d}$ only while preserving basic rights and fiduciary obligations to governed stakeholders. \textit{Justification rule}: consent is the foundation of legitimate authority, but consent is ongoing rather than once-and-for-all---a dynamic contract. \textit{Correction rule}: withdrawal, resistance, and institutional revision when rights violations persist. \textit{Failure mode}: formal revocability without practical capacity (low effective voice $\text{eff\_voice}_i$) yields pseudo-legitimacy; rights language persists while misalignment remains structurally locked in.

This approximates medium-to-high $\alpha$ where property and rights protections are credible and contestation channels are open. The formal representation captures revocability as threshold-triggered correction:
\begin{equation}
\text{if } F(d,t) > \tau_d, \quad H_{t+1}(d) \neq H_t(d),
\label{eq:lockean-correction}
\end{equation}
where the consent-holder mapping updates when misalignment exceeds tolerable limits. The diagnostic prediction: systems with stronger contestation channels should show shorter lag between friction spikes and consent-reallocation reforms. This is directly testable through the historical case studies in Part~III---franchise extensions, for instance, follow sustained periods of elevated friction (petitions, protests, litigation) before institutional correction occurs. The lag between friction onset and institutional response varies with the strength of contestation channels: strong channels (independent judiciary, free press, organized opposition) produce shorter lags; weak channels produce longer accumulation periods followed by more violent correction. The British franchise extensions of 1832, 1867, and 1918 illustrate the pattern: each was preceded by decades of escalating friction (Chartist petitions, Reform League demonstrations, suffragette militancy) that eventually exceeded the tolerance threshold $\tau_d$ and triggered institutional revision.

The Lockean framework also generates a prediction about the \textit{form} of friction under different institutional configurations. Where revocability channels are strong but underutilized, friction should manifest as legal contestation (rights-based litigation, constitutional challenges). Where revocability channels are weak or captured, friction should manifest as extra-institutional action (protests, civil disobedience, rebellion). The consent-holding framework captures this distinction through the effective voice variable: low $\text{eff\_voice}_i$ combined with high stakes generates the conditions under which extra-institutional friction replaces institutional contestation.

\subsubsection{Rousseauian General Will and Collective Self-Rule}

\citet{rousseau1762} seeks legitimacy through collective self-legislation rather than aggregation of private bargaining. \textit{Allocation rule}: high participation in constitutional domains; broad inclusion in rule formation such that $C_{i,d^{\text{const}}} > 0$ for all $i \in S_{d^{\text{const}}}$. \textit{Justification rule}: legitimacy requires citizens to be co-authors of law rather than subjects of an alien will. \textit{Correction rule}: civic deliberation and constitutional revision. \textit{Failure mode}: if institutional mediation is captured, claims of ``general will'' can mask concentrated control. Observed friction then reflects the gap between symbolic inclusion and actual authority distribution.

The model expression targets high alignment in constitutional layers:
\begin{equation}
\alpha(d^{\text{const}},t) = \frac{\sum_{i \in S_{d^{\text{const}}}} s_i(d^{\text{const}}) \cdot \text{eff\_voice}_i(d^{\text{const}},t)}{\sum_{i \in S_{d^{\text{const}}}} s_i(d^{\text{const}})}.
\label{eq:rousseauian-alpha}
\end{equation}

The diagnostic prediction: when institutions claim collective sovereignty but empirical $\text{eff\_voice}_i$ is highly unequal, friction appears as legitimacy contestation over representation authenticity. The Rousseauian failure mode is particularly insidious because it combines high \textit{formal} $\alpha$ (everyone has a vote) with low \textit{effective} $\alpha$ (capture, information asymmetries, agenda control). This gap---between formal and effective consent alignment---is one of the framework's most diagnostically powerful measurements.

The practical relevance is immediate. Contemporary democracies with universal suffrage routinely exhibit the Rousseauian gap: formal $\alpha$ approaches 1 (all adults can vote), but effective $\alpha$ may be substantially lower due to gerrymandering (vote dilution), lobbying (unequal access to decision-makers), media concentration (asymmetric information), and campaign finance structures (consent power correlated with wealth rather than stakes). The consent-holding framework provides the diagnostic tools to measure this gap---and the friction predictions to test whether closing it reduces institutional stress. Citizens' assemblies, as examined in the Irish and French cases (Section~\ref{sec:research-agenda}), represent deliberate institutional experiments in raising effective $\alpha$ within systems where formal $\alpha$ is already high.

\subsection{Modern Variants}
\label{subsec:modern-variants}

\subsubsection{Rawlsian Justice as Maximin Consent}

Rawls's difference principle \citep{rawls1971} can be formalized as maximizing the minimum effective voice:
\begin{equation}
\max_{C_{i,d}} \min_i \{\text{eff\_voice}_i(d)\}
\label{eq:rawlsian-maximin}
\end{equation}
subject to basic liberties constraints ensuring $C_{i,d} > 0$ for all citizens in political domains. This generates concrete institutional predictions: political equality (one person, one vote) in constitutional domains, economic redistribution raising least-advantaged citizens' capability to exercise voice, and priority rules protecting basic liberties even when aggregate welfare would benefit from violation.

\citet{rawls1993} extends this analysis through the idea of an \textit{overlapping consensus}: different comprehensive doctrines can endorse the same political institutions for different reasons. In consent-holding terms, this means multiple justification rules can support the same allocation rule. A Rawlsian society need not require unanimous agreement on \textit{why} maximin consent is justified---only on the institutional structures that implement it. The implication for weight determination is significant: Rawlsian configurations constrain $(w_1, w_2)$ such that $w_1$ dominates in domains affecting basic liberties, while permitting higher $w_2$ in domains of distributive efficiency.

The Rawlsian framework also illuminates the framework's treatment of structural inequality. The veil of ignorance thought experiment corresponds to evaluating consent allocation from the perspective of the \textit{least-advantaged} stakeholder---the agent with the weakest effective voice. The consent-holding framework operationalizes this: $\min_i \{\text{eff\_voice}_i(d)\}$ is directly measurable, and Rawlsian institutional design prescribes raising this minimum. Scandinavian welfare states approximate this configuration: strong labor protections, universal public services, and corporatist bargaining structures all function to raise the floor of effective voice. The consent-holding framework predicts that Rawlsian configurations should show lower variance in effective voice \textit{and} lower average friction, because the maximin constraint prevents the extreme misalignment that generates the sharpest friction. The historical labor and corporate governance case studies (Sections~\ref{sec:labor} and~\ref{sec:corporate-governance}) test this prediction through comparison of codetermination regimes with shareholder-primacy alternatives.

\subsubsection{Utilitarian Consent as Weighted Aggregation}

Classical utilitarianism maximizes stakes-weighted welfare:
\begin{equation}
\max_{x_d} \sum_i s_i(d) \cdot U_i(x_d)
\label{eq:utilitarian-welfare}
\end{equation}

This doesn't directly specify consent allocation, but combined with epistemic assumptions that affected parties possess superior information about their own stakes $s_i(d)$, it motivates giving consent power proportional to stakes---exactly the $\alpha(d)$ alignment measure. The framework reveals utilitarianism's implicit consent structure: let those with stakes decide, weighted by their exposure.

The epistemic justification connects to Condorcet's jury theorem \citep{condorcet1785essay}: if individual voters are more likely right than wrong about their own interests, majority rule among stakeholders converges to optimal outcomes as the affected population grows. \citet{campos2008implementing} show how utilitarian voting can be approximated through delegated agents, providing a mechanism-design pathway from utilitarian theory to implementable consent allocation.

The failure mode is well-known but worth specifying in consent-holding terms: preference intensity is invisible to unweighted aggregation, generating persistent friction among minorities with concentrated stakes who are systematically outvoted by majorities with diffuse interests. The framework captures this precisely: utilitarian aggregation achieves high $\alpha$ when stakes are roughly uniform across the population, but produces systematic misalignment---and thus friction---when stakes are highly concentrated. The tyranny-of-the-majority problem is not a contingent defect but a structural consequence of the allocation rule applied to heterogeneous stakes distributions. This connects the utilitarian failure mode to the minority-protection concerns of Objection~4: both identify cases where aggregate consent alignment ($\alpha$) may be high while subgroup alignment is dangerously low.

\subsubsection{Libertarian Consent as Property Rights}

Nozickean libertarianism allocates consent power through property rights: $C_{i,d} = 1$ if domain $d$ involves only $i$'s property, distributed according to ownership shares otherwise. This generates high $\alpha(d)$ for domains where property rights align with stakes (personal consumption choices) but potentially low $\alpha$ for domains with externalities (pollution, network effects) where those holding property rights differ from those bearing consequences.

The consent-holding framework exposes a structural limitation: libertarian allocation presupposes that property boundaries track stakes boundaries. Where they diverge---climate change being the paradigmatic case---the property-rights allocation systematically misaligns consent power with affected populations. The resulting friction takes characteristic forms: litigation (affected parties seeking judicial correction), regulation (collective attempts to realign authority with stakes), and exit (migration away from externality-generating jurisdictions).

The libertarian doctrine thus occupies a specific region in parameter space: high $\alpha$ in low-externality domains (personal consumption, voluntary exchange) and potentially catastrophic misalignment in high-externality domains (environmental degradation, financial contagion, public health). The framework's prediction is distinctive: libertarian regimes should show \textit{domain-bifurcated} friction profiles---low friction in private-goods domains and high friction in public-goods and externality domains, with the severity of the friction proportional to the magnitude of the stakes-property divergence. This is testable through cross-domain comparison within libertarian-leaning polities.

\subsubsection{Technocratic Delegation as Expertise Concentration}

Technocratic governance concentrates consent power in accredited experts for domains where prediction, safety, or complex coordination dominate. \textit{Allocation rule}: authority goes to those with demonstrated technical competence. \textit{Justification rule}: superior knowledge generates superior outcomes ($P(d,t)$). \textit{Correction rule}: audit, reviewability, and bounded jurisdiction. \textit{Failure mode}: boundary creep from technical to value-laden domains transforms competence advantage into representational deficit, increasing friction despite acceptable narrow performance metrics.

This regime raises expected $P(d,t)$ in high-complexity domains but risks low $\alpha$ for affected populations lacking procedural voice. The consent-holding interpretation connects to \citeauthor{raz1986morality}'s \citeyearpar{raz1986morality} service conception of authority: expertise-based authority is legitimate when, and only when, subjects would better conform with reasons that apply to them by following the expert than by acting on their own judgment. In consent-holding terms, technocratic delegation is legitimacy-sustainable when bounded by domain limits, transparency, and reviewability---but degrades when experts extend authority beyond their epistemic advantage into domains where stakeholder preferences, not technical correctness, are the relevant input.

Central bank independence exemplifies the technocratic allocation. Monetary policy is delegated to experts ($w_2$ dominant) with a specific mandate (price stability) and accountability mechanisms (inflation targets, parliamentary testimony, publication of minutes). The consent-holding framework predicts that this arrangement is stable when the mandate domain is narrow and measurable, but friction rises when central banks influence distributional outcomes (quantitative easing effects on asset prices, interest rate impacts on housing affordability) that extend beyond their technical mandate into domains where affected populations have legitimate stakes requiring consent alignment. The post-2008 criticism of central banking---that monetary policy has massive distributional consequences without democratic input---is precisely the domain-creep failure mode predicted by the framework.

\citet{kiikeri2024epistocracy} reveals that epistocracy and populism---typically positioned as opposing critiques of democracy---share surprising structural features when analyzed at the level of political ontology. Both are second-order ideologies: rather than advancing substantive policy positions, they contest the meta-question of who gets to decide, dividing the citizenry into constitutively different categories and excluding one from political membership. Both assume the existence of political truths, making their conception of authority anti-proceduralist. The consent-holding framework provides the formal apparatus to make this structural parallel precise: epistocracy concentrates $C_{i,d}$ in a credentialed subset while populism concentrates it in a ``true people'' subset---both reduce $\alpha$ by excluding affected stakeholders from governance, differing only in the exclusion criterion (epistemic merit vs.\ elite membership). Kiikeri's observation that both ideologies ``downplay the value of pluralism, deliberation and dissent'' maps directly onto our friction prediction: any regime that structurally excludes stakeholders with genuine stakes will generate friction, regardless of whether the exclusion is justified by expertise or by popular authenticity.

\subsection{Emergent Forms}
\label{subsec:emergent-forms}

\subsubsection{Anarchist and Federal Variants as Domain Fragmentation}

Anarchist and federal traditions distribute authority across local units, associations, and negotiated compacts rather than a single sovereign hierarchy. \textit{Allocation rule}: authority is distributed to the smallest feasible unit with jurisdiction over the affected population. \textit{Justification rule}: proximity maximizes local $\alpha$ through direct participation. \textit{Correction rule}: exit/voice across nodes, federation redesign, renegotiation of compacts. \textit{Failure mode}: inter-domain spillovers and uneven capacity can produce cross-unit externalities where those bearing consequences in one unit lack voice in the unit causing harm. Friction migrates from center-periphery conflict to inter-node conflict.

This maps to polycentric consent allocation: potentially high local $\alpha(d,t)$ through proximity and direct participation, with multi-level coordination reducing single-point legitimacy failures. The diagnostic prediction is that polycentric systems trade \textit{type} of friction rather than eliminating it---reducing center-periphery misalignment at the cost of cross-node externality friction. Whether this trade-off is net beneficial depends on the correlation structure of stakes across domains and the transaction costs of inter-node coordination.

The formal representation captures this trade-off. Let $\mathcal{D} = \{d_1, \ldots, d_k\}$ be the set of governance domains and $\mathcal{N} = \{n_1, \ldots, n_m\}$ be the set of polycentric units. Local $\alpha$ may be high: $\alpha(d_j, n_l, t) \to 1$ for domain $d_j$ within unit $n_l$. But cross-node friction arises wherever the affected set $S_{d_j}$ spans multiple units while authority is localized:
\[
F_{\text{cross}}(d_j, t) = \sum_{l \neq l'} s_{n_l}(d_j) \cdot \delta(x_{d_j}^{n_{l'}}(t), x^*_{n_l, d_j})
\]
where $x_{d_j}^{n_{l'}}$ is the policy outcome chosen by unit $n_{l'}$ and $x^*_{n_l, d_j}$ is the preferred outcome of stakeholders in unit $n_l$ affected by $n_{l'}$'s decision. The net effect depends on whether the between-unit externality friction exceeds the within-unit alignment gains---an empirical question addressable through the proxy methodology. Historical examples range from Swiss cantonal governance (high local $\alpha$ with manageable inter-cantonal coordination) to failed federations (Yugoslavia, the Sonderbund War) where cross-unit friction overwhelmed local alignment.

\subsubsection{Algorithmic Social Contract and Code-Mediated Authority}

In platform and AI-mediated systems, allocation shifts from legal institutions to codebases and model operators. Decision authority is effectively embedded in technical artifacts, policy stacks, and update pipelines. \textit{Allocation rule}: designers and operators hold $C_{i,d}$ by default; users hold residual authority only through exit, complaint, or regulatory intervention. \textit{Justification rule}: often implicit---efficiency, scale, network effects. \textit{Correction rule}: appeals processes, external oversight, governance reform campaigns. \textit{Failure mode}: high output performance in narrow metrics coexists with escalating contestation over legitimacy, because affected populations cannot contest rule formation on equal terms.

Unless design, oversight, and appeal pathways allocate meaningful stakeholder authority, these systems instantiate low-$\alpha$ governance with procedural opacity. The consent-holding framework predicts that algorithmic governance faces the same dynamics as other low-$\alpha$ regimes: friction accumulation, threshold effects, and eventual institutional correction---but with the additional complication that the ``rules'' are embedded in code rather than law, making contestation channels more opaque and correction more difficult. The platform governance case study in Section~\ref{sec:platform-governance} examines this prediction empirically.

What distinguishes the algorithmic case from earlier low-$\alpha$ regimes is the velocity of iteration. Code can be updated overnight; legal institutions evolve over decades. This creates an asymmetry: algorithmic authority can expand faster than correction mechanisms can respond, generating persistent friction gaps where affected populations are perpetually playing catch-up with rule changes they had no voice in designing. The consent-holding framework suggests that effective algorithmic governance requires not just retrospective accountability (auditing past decisions) but prospective consent (voice in system design), bringing the allocation rule into alignment with affected populations \textit{before} deployment rather than correcting misalignment after the fact.

\subsection{Comparative Matrix}
\label{subsec:comparative-matrix}

Table~\ref{tab:social-contract-matrix} synthesizes the doctrinal analysis into a comparative framework. Each doctrine occupies a distinct region in $(\alpha, P, w_1, w_2)$ space, generating different friction profiles and correction dynamics.

\begin{table*}[t]
\centering
\caption{Social Contract Doctrines as Consent Allocation Regimes}
\label{tab:social-contract-matrix}
\begin{tabular}{p{0.12\textwidth}p{0.17\textwidth}p{0.13\textwidth}p{0.14\textwidth}p{0.18\textwidth}p{0.14\textwidth}}
\toprule
\textbf{Doctrine} & \textbf{Primary allocation logic} & \textbf{Weight profile} & \textbf{Correction mechanism} & \textbf{Primary failure mode} & \textbf{Real-world exemplar} \\
\midrule
Hobbesian & Sovereign concentration for order & $w_2 \gg w_1$ & Breakdown/ reconstitution & Repressed friction under prolonged centralization & Emergency powers; wartime coalitions \\
\addlinespace
Lockean & Delegated, revocable authority under rights & Balanced with rights floor & Legal contestation, electoral turnover & Formal revocability without effective capacity & Anglo-American constitutional democracy \\
\addlinespace
Rousseauian & Collective self-rule and civic co-authorship & High $w_1$ in constitutional domains & Civic deliberation, constitutional revision & Symbolic unity masking capture & Swiss cantons; citizens' assemblies \\
\addlinespace
Rawlsian & Maximin effective voice & $w_1$ dominant for basic liberties & Institutional redesign under veil & Institutional rigidity when stakes shift & Scandinavian welfare states \\
\addlinespace
Utilitarian & Stakes-weighted welfare maximization & Domain-variable & Outcome feedback, policy adjustment & Preference intensity invisible to aggregation & Cost-benefit regulatory agencies \\
\addlinespace
Libertarian & Property-rights consent & High $w_1$ where property $=$ stakes & Market exit, litigation & Externalities misalign consent with stakes & Common-law property regimes \\
\addlinespace
Technocratic & Expertise-based delegation & High $w_2$ in technical domains & Audit, bounded jurisdiction & Domain creep into value-laden decisions & Central banks; health agencies \\
\addlinespace
Anarchist/ Federal & Polycentric local authority & Local $w_1$ emphasis & Exit/voice across nodes & Cross-node externalities without cross-node voice & Federated cooperatives; Ostrom CPR \\
\addlinespace
Algorithmic & Code-mediated by designers/ operators & Implicit high $w_2$ & Appeals, external oversight & Opaque low-$\alpha$ with delayed backlash & Platform content moderation \\
\bottomrule
\end{tabular}
\end{table*}

The matrix generates empirical predictions that are testable through the proxy construction methodology developed in Section~\ref{sec:historical-methodology}:

\begin{itemize}[leftmargin=1.4em]
\item \textbf{Hobbesian regimes} should exhibit low measured friction during crises (suppression) followed by spikes when coercive capacity declines. The friction time series should show characteristic U-shapes: initially high (pre-consolidation), suppressed (during stable authoritarian rule), then rising (as legitimacy erodes). The transition from suppressed to rising friction should correlate with declining state capacity or external shocks.

\item \textbf{Lockean regimes} should show friction-reform cycles with lag proportional to contestation channel strength. The stronger the legal system, free press, and organized opposition, the shorter the lag between friction onset and institutional correction. This prediction is testable through cross-polity comparison: common-law systems with independent judiciaries should show shorter correction lags than systems with weaker contestation channels.

\item \textbf{Rousseauian regimes} should show divergence between formal and effective $\alpha$ as a leading indicator of legitimacy crisis. When the gap between formal inclusion (universal suffrage) and effective voice (actual influence on outcomes) widens, friction should increase even if formal $\alpha$ remains constant---a pattern visible in contemporary democracies experiencing populist backlash against perceived elite capture.

\item \textbf{Technocratic regimes} should show domain-specific friction patterns: low friction in narrow technical domains but rising friction at domain boundaries, particularly where technical decisions have distributional consequences that extend beyond the expert mandate.
\end{itemize}

\subsection{Formal Text-to-Model Mapping}
\label{subsec:text-to-model}

The preceding doctrinal analysis can be compressed into a formal translation scheme. For each doctrine, the mapping identifies: (i) the canonical textual proposition, (ii) its allocation implication for $C_{i,d}$, (iii) the model expression in terms of $\alpha$, $F$, and $L$, and (iv) the diagnostic prediction---the friction pattern that should be observed if the doctrine accurately describes the regime.

\textbf{Hobbes} $\to$ \textit{Order-first authorization}: peace requires concentrated authority \citep{hobbes1651}. Allocation: $C_{s,d} \approx 1$. Model: $L$ sustained by high $P$ under $w_2 \gg w_1$. Prediction: $\partial F / \partial t > 0$ under persistent low $\alpha$.

\textbf{Locke} $\to$ \textit{Conditional delegation}: authority legitimate only as continuing trust \citep{locke1689}. Allocation: bounded, revocable. Model: threshold-triggered correction per Equation~\ref{eq:lockean-correction}. Prediction: shorter friction-reform lag where contestation channels are stronger.

\textbf{Rousseau} $\to$ \textit{Collective self-rule}: citizens as co-authors of law \citep{rousseau1762}. Allocation: $C_{i,d^{\text{const}}} > 0$ for all $i \in S_d$. Model: high constitutional $\alpha$ per Equation~\ref{eq:rousseauian-alpha}. Prediction: friction appears as contestation over representation authenticity when formal and effective $\alpha$ diverge.

The modern variants extend this mapping:

\textbf{Rawls} $\to$ \textit{Maximin voice}: institutional design maximizes minimum $\text{eff\_voice}_i$. Allocation: constrained by basic liberties floor. Model: $\max \min_i \{\text{eff\_voice}_i\}$ per Equation~\ref{eq:rawlsian-maximin}. Prediction: lower variance in $\text{eff\_voice}_i$ and lower average friction relative to utilitarian configurations.

\textbf{Utilitarian} $\to$ \textit{Stakes-weighted aggregation}: maximize $\sum s_i \cdot U_i$. Allocation: $C_{i,d}$ proportional to $s_i(d)$ under epistemic assumptions. Model: $\alpha$ measures realized stakes-consent covariance. Prediction: persistent minority friction where stakes concentrate but votes do not.

\textbf{Technocratic} $\to$ \textit{Expertise concentration}: maximize $P(d,t)$ through competent delegation. Allocation: $C_{i,d}$ concentrated among accredited experts. Model: high $w_2$, bounded jurisdiction. Prediction: rising friction when domain boundaries blur from technical to value-laden.

These mappings can be read as different parameterizations of the same structural system. Hobbesian configurations prioritize high $w_2$ under emergency conditions. Lockean configurations prioritize bounded delegation with explicit correction triggers. Rousseauian configurations prioritize high constitutional $\alpha$ through broad co-authorship. Rawlsian configurations constrain the allocation to protect worst-off stakeholders. Under this interpretation, doctrinal disagreement is not only philosophical; it is empirically legible as disagreement over admissible regions in $(w_1, w_2, \alpha, P)$ space and over the dynamics of updating $H_t(d)$ when friction accumulates.

For future empirical work, a doctrine-linked panel can be coded with: (i) domain-level authority concentration index (Hobbesian concentration proxy), (ii) rights-and-revocation channel strength (Lockean conditionality proxy), (iii) constitutional inclusion breadth and effective participation (Rousseauian co-authorship proxy), and (iv) lagged friction-reform elasticity $\partial H_{t+1} / \partial F_t$ across regimes. This allows doctrinal language to generate falsifiable comparative hypotheses without collapsing normative differences into a single scalar.

\subsection{Endogenous Weight Selection Across Doctrines}
\label{subsec:endogenous-weights}

The doctrinal comparison clarifies an empirical strategy for the weight determination problem introduced in Postulate~1. Rather than treating $(w_1, w_2)$ as free parameters requiring \textit{a priori} normative resolution, we can treat doctrine labels as priors over feasible $(w_1, w_2)$ regions and estimate posterior weights from observed friction trajectories and reform timing.

Concretely, doctrine-consistent regimes can be estimated through constrained optimization:
\begin{equation}
\min_{w_1, w_2, \theta} \sum_{d,t} \left[ F_{d,t}^{\text{obs}} - \hat{F}_{d,t}(\alpha_{d,t}(w, \theta), P_{d,t}) \right]^2
\label{eq:doctrine-estimation}
\end{equation}
subject to doctrine-specific constraints on admissible allocations (e.g., rights floors for Lockean configurations, domain bounds for technocratic delegation, veto conditions for minority protection). This converts social-contract debate from pure doctrine adjudication into constrained comparative model selection.

The approach connects to \citeauthor{weber1978economy}'s \citeyearpar{weber1978economy} typology of legitimate domination. Weber's three ideal types---traditional, charismatic, and legal-rational authority---correspond to different weight configurations in consent-holding terms: traditional authority prioritizes historical precedent (high inertia in $H_t(d)$), charismatic authority concentrates $C_{i,d}$ around personal legitimacy claims (high $w_2$ tied to individual performance), and legal-rational authority distributes authority through procedural rules (explicit correction mechanisms, bounded allocation). The consent-holding framework operationalizes these distinctions: Weber's ideal types become empirically distinguishable through their $(\alpha, F, w_1, w_2)$ signatures.

\citet{habermas1987} distinguishes \textit{system} (market and state coordination through media of money and power) from \textit{lifeworld} (communicatively structured domains of meaning and solidarity). In consent-holding terms, system domains tend toward technocratic weight profiles ($w_2$ dominant), while lifeworld domains demand higher $\alpha$ ($w_1$ dominant). The ``colonization of the lifeworld'' that Habermas diagnoses---system logic encroaching on communicative domains---translates directly: friction rises when technocratic weights are imposed on domains where stakeholder consent alignment is the relevant legitimacy criterion. The contemporary manifestation is algorithmic governance of social life---content moderation, recommendation systems, automated hiring---where system-logic ($w_2$: optimize engagement, efficiency, profit) colonizes domains that stakeholders experience as lifeworld ($w_1$: cultural expression, professional identity, community norms). \citet{calhoun1994habermas} further develops the public sphere dimension: the formation of weights $(w_1, w_2)$ is itself a public process, shaped by the quality of deliberation, the openness of discourse, and the distribution of communicative power. Weight formation is thus not merely a technical calibration problem but a political one, subject to the same consent-holding dynamics as any other governance domain.

The framework doesn't adjudicate between these theories normatively but provides tools for comparing their institutional predictions and empirical performance across domains. The comparative matrix (Table~\ref{tab:social-contract-matrix}) serves as a lookup table: given a governance domain and its observed friction profile, which doctrinal configuration best explains the data? And given a desired legitimacy outcome, which configuration's allocation rule is most likely to achieve it?

The endogenous weight approach also connects to historical debates in social choice theory. The eighteenth-century disagreement between \citet{borda1781memoire} and Condorcet over voting rules can be reinterpreted as a disagreement over implicit weight configurations: Borda's count weights preference orderings uniformly, while Condorcet's pairwise method weights majority preferences in each binary comparison. \citet{mclean1994condorcet} document how Condorcet understood the epistemic dimension of voting---its capacity to track truth---while Borda emphasized fairness in aggregation. In consent-holding terms, Condorcet's approach prioritizes $w_2$ (the voting rule's performance in tracking correct outcomes), while Borda's prioritizes $w_1$ (fair representation of each voter's full preference ordering). The consent-holding framework reveals that this historical debate was not merely about voting mechanics but about the relative weight of consent alignment versus epistemic performance in collective decision-making---the same trade-off captured in Postulate~1.

The practical implication is that weight selection need not be resolved philosophically before the framework can be applied empirically. Different societies, different domains, and different historical periods operate under different weight configurations. The framework's contribution is to make these configurations \textit{visible}---to show that what appears to be a disagreement about fundamental values is often a disagreement about parameter values in a shared structural model, amenable to empirical investigation and institutional experimentation.

% ============================================================================
\part{Historical Validation}
\label{part:historical}
\thispagestyle{partpage}
% ============================================================================

\section{Methodology for Historical Case Analysis}
\label{sec:historical-methodology}

The consent-holding framework's empirical viability depends on whether its core constructs---consent alignment $\alpha(d,t)$ and friction $F(d,t)$---can be operationalized across historical domains where governance authority was contested, restructured, or overthrown. This section develops a systematic methodology for constructing time series of $\alpha$ and $F$ from historical data, establishing the inferential foundations for the case studies that follow.

The methodological challenge is substantial. Unlike contemporary governance domains where survey data, election returns, and institutional records provide direct measurement, historical cases require proxy construction from heterogeneous sources: census records, petition archives, parliamentary debates, rebellion chronologies, legal codes, and organizational membership rolls. We do not claim to reconstruct ``true'' consent alignment with precision; rather, we construct ordinal and ratio-scale proxies whose trajectories can be compared across cases and tested against the framework's predictions.

\subsection{Domain Definition Protocol}
\label{subsec:domain-definition}

Each historical case study begins by defining the governance domain $d$ under analysis. A domain is a bounded set of decisions producing shared consequences for an identifiable population. Domain definition requires three specifications:

\begin{enumerate}[label=(\roman*)]
  \item \textbf{Decision scope}: What decisions does this domain encompass? For suffrage, $d_{\text{suffrage}}$ covers decisions about who may participate in collective governance---franchise rules, voter eligibility criteria, electoral procedures. For abolition, $d_{\text{slavery}}$ covers decisions about the legal status, bodily autonomy, and labor conditions of enslaved persons.

  \item \textbf{Affected set}: Who bears consequences from decisions in this domain? The affected set $S_d = \{i \mid s_i(d) > 0\}$ includes all agents with positive stakes. In suffrage, $S_d$ encompasses all adult residents whose interests are shaped by political decisions from which they may be excluded. In abolition, $S_d$ includes enslaved persons (existential stakes), slaveholders (economic stakes), and broader populations affected by the slave economy.

  \item \textbf{Consent-holder mapping}: Who actually decides? The consent-holder mapping $H_t(d)$ identifies, at each time point $t$, which agents hold effective decision power $C_{i,d} > 0$. This mapping is the object whose evolution we track: franchise extensions, emancipation acts, and institutional reforms all constitute changes to $H_t(d)$.
\end{enumerate}

Domain boundaries are inevitably porous. Suffrage and abolition overlapped in both organizational networks and philosophical commitments---Seneca Falls grew directly from abolitionist organizing. We handle this by tracking each domain separately while documenting cross-domain linkages in dedicated subsections. The framework's notation accommodates this: an agent $i$ may hold different consent power levels across domains, with $C_{i,d_1} > 0$ and $C_{i,d_2} = 0$ simultaneously.

A further consideration is the distinction between \textit{nested} and \textit{adjacent} domains. Suffrage is nested within broader political governance: franchise rules determine who holds consent power in all policy domains simultaneously. Abolition is adjacent to but distinct from franchise: enslaved persons' legal status determined their eligibility for political participation, but the slavery domain itself encompassed decisions (labor conditions, physical treatment, family separation) that extended far beyond the franchise question. Nested domains exhibit stronger cross-domain $\alpha$ transmission---raising $\alpha$ in the franchise domain automatically raises it in subordinate policy domains---while adjacent domains require independent $\alpha$ expansion in each domain.

\subsection{Ordinal and Proxy Alpha Construction}
\label{subsec:alpha-construction}

The consent alignment measure $\alpha(d,t)$ is defined theoretically as the stakes-weighted share of decision power held by affected parties (Equation~\ref{eq:alignment-operational}). Direct computation requires cardinal measurement of both stakes $s_i(d)$ and effective voice $\text{eff\_voice}_i(d,t)$---feasible in some contemporary settings but rarely in historical ones. We therefore employ two proxy strategies depending on data availability.

\textbf{Ratio-scale proxies.} Where quantitative data permit, we construct $\alpha$ as a population ratio:
\begin{equation}
  \hat{\alpha}(d,t) = \frac{N_{\text{enfranchised}}(t)}{N_{\text{affected}}(t)}
  \label{eq:alpha-ratio}
\end{equation}

This proxy is available for suffrage expansion (enfranchised adults as a share of total adult population), labor rights (union density, board representation ratios), and corporate governance (employee representation on supervisory boards). The ratio-scale proxy assumes equal stakes across the affected population---a simplification that understates $\alpha$ divergence when high-stakes subpopulations are disproportionately excluded. We discuss this limitation where it binds.

Data sources for ratio-scale proxies include the Varieties of Democracy (V-Dem) dataset \citep{dahl1971}, which provides standardized suffrage coverage indicators across countries and years; census records identifying the total adult population; and institutional records documenting franchise eligibility criteria. V-Dem's methodology---expert coding with Bayesian aggregation---provides the most reliable cross-national suffrage data available, though measurement error increases for pre-1900 periods.

\textbf{Ordinal-scale proxies.} For domains where cardinal measurement is impossible, we construct ordinal $\alpha$ indices mapping legal and institutional configurations to a $[0, 1]$ scale. The abolition case, for instance, employs a six-point ordinal index:

\begin{table}[htbp]
\centering
\caption{Ordinal alpha scale for abolition domain}
\label{tab:alpha-abolition-scale}
\begin{tabular}{@{}cl@{}}
\toprule
$\hat{\alpha}$ & Legal--institutional configuration \\
\midrule
0.00 & Chattel slavery (full legal dehumanization, zero personhood) \\
0.10 & Amelioration laws (restrictions on punishment, limited protections) \\
0.25 & Gradual emancipation statutes (phased freedom, conditional) \\
0.50 & Immediate emancipation (legal freedom, limited citizenship rights) \\
0.75 & Full legal citizenship (constitutional amendments, formal equality) \\
1.00 & Effective citizenship with enforced rights (substantive participation) \\
\bottomrule
\end{tabular}
\end{table}

Ordinal scales sacrifice cardinality for tractability. We cannot claim that the move from 0.00 to 0.25 represents the same ``amount'' of consent expansion as the move from 0.50 to 0.75. What we can claim is monotonicity: higher ordinal values represent unambiguously greater incorporation of affected populations into decision structures. This suffices for testing the framework's directional predictions (H1, H4) and threshold predictions (H3), even if it cannot test precise functional forms.

The validity of ordinal proxies depends on whether the ranked categories correspond to genuine differences in the consent alignment construct. We ground each scale in the framework's operational definition: does this institutional configuration give affected parties greater effective voice over decisions bearing on their stakes? Amelioration laws (0.10) grant more voice than chattel slavery (0.00) because they impose constraints on slaveholders' decision power---but only marginally, as the constrained decisions concern treatment rather than status. Full citizenship (0.75) grants substantially more voice because it extends formal decision rights, even where enforcement remains incomplete.

\textbf{Measurement validity considerations.} Two threats to validity deserve explicit treatment. First, \textit{construct validity}: do our proxies capture the theoretical concept of consent alignment rather than some correlated but distinct quantity? The ratio-scale proxy (Equation~\ref{eq:alpha-ratio}) faces this concern when the franchise is formally broad but effectively narrow---the American case between the 15th Amendment (1870) and the Voting Rights Act (1965) exemplifies this, where formal $\hat{\alpha}$ overstated effective consent alignment. We address this by supplementing formal measures with effective participation indicators where available (voter registration rates, turnout data, office-holding patterns). Second, \textit{inter-coder reliability}: ordinal scales require judgment about category assignment. A slaveholding territory with amelioration laws restricting extreme punishment could be coded 0.10 or 0.25 depending on the scope of restrictions. We report our coding decisions transparently and note where alternative codings would substantively alter the analysis.

\subsection{Friction Proxy Construction}
\label{subsec:friction-construction}

Friction $F(d,t)$ captures the stakes-weighted deviation between realized governance outcomes and stakeholder preferences (Equation~\ref{eq:friction-basic}). Direct measurement is impossible historically: we cannot survey the dead about their preference-outcome gaps. Instead, we operationalize friction through its observable manifestations---the actions agents take when governance deviates from their interests.

We distinguish three categories of friction proxies, ordered by data quality:

\begin{enumerate}[label=(\roman*)]
  \item \textbf{Count-based proxies}: Petition signatures, protest event counts, strike days, rebellion frequency. These provide ratio-scale measurement where archival records are systematic. British petitioning data is particularly rich: \citet{miller2021} documents petition campaigns with signature counts from the 1780s onward, enabling construction of annual friction series. Protest event databases (adapted from \citealt{tilly2008contentious} contentious politics methodology) provide comparable data for labor and civil rights mobilization.

  \item \textbf{Institutional proxies}: Parliamentary votes against the status quo, litigation rates, regulatory complaints, advertiser boycotts. These capture friction channeled through existing institutions rather than extra-institutional mobilization. They are particularly useful for platform governance and corporate domains where institutional channels exist.

  \item \textbf{Intensity-weighted indices}: Where raw counts are available, we weight by intensity measures---petition signature counts rather than binary petition existence, rebellion casualties rather than rebellion counts, strike duration rather than strike frequency. This captures the distinction between low-level persistent friction and mobilization spikes that the framework associates with threshold-crossing events (H3).
\end{enumerate}

A critical methodological issue arises when multiple friction channels operate simultaneously. British abolitionism generated petition friction, consumer boycott friction (the sugar boycott movement of the 1790s), parliamentary friction (annual abolition motions defeated repeatedly), and moral-philosophical friction (pamphlets, sermons, public lectures). Our friction index should, in principle, aggregate across channels, but the heterogeneity of measurement units (petition signatures, boycott participation estimates, parliamentary vote margins) makes simple aggregation impossible. We adopt a pragmatic approach: within each case study, we identify the primary friction channel based on data quality and construct the friction time series from that channel, reporting supplementary channels qualitatively.

The fundamental measurement challenge is that friction proxies capture mobilized friction rather than latent friction. Populations with zero consent power may experience maximal preference-outcome deviation yet lack the organizational capacity to generate observable friction. This creates a systematic downward bias in friction measurement for the most excluded populations---a bias the framework itself predicts, since consent power correlates with organizational capacity. We address this by supplementing mobilization-based proxies with narrative evidence of latent grievance where possible, and by interpreting the absence of observed friction under extreme exclusion as consistent with capacity constraints rather than satisfied preferences.

\subsection{Time Series Construction}
\label{subsec:time-series}

From proxy measures, we construct paired $\hat{\alpha}(d,t)$ and $\hat{F}(d,t)$ trajectories at the finest temporal resolution the data support. The panel structure has the form:

\begin{equation}
  \{(\hat{\alpha}_{d,c,t}, \hat{F}_{d,c,t}) \mid d \in \mathcal{D}, c \in \mathcal{C}, t \in \mathcal{T}_d\}
  \label{eq:panel-structure}
\end{equation}

where $d$ indexes domains (suffrage, abolition, labor, etc.), $c$ indexes countries or polities, and $t$ indexes time periods whose granularity varies by data availability---annual for well-documented domains (British suffrage post-1780), decadal for sparser cases (early colonial abolition trajectories).

Several construction decisions require justification:

\textbf{Interpolation.} Between observed institutional changes, we hold $\hat{\alpha}$ constant (step function). Consent alignment shifts discretely through institutional reform---a franchise act, an emancipation statute, a board representation mandate---rather than continuously. Friction, by contrast, may be interpolated linearly between observed data points when annual data are unavailable, reflecting the assumption that mobilization capacity evolves gradually between observed peaks.

\textbf{Missing periods.} Some trajectories contain gaps where neither $\alpha$ nor $F$ data are available. We mark these explicitly rather than imputing values, restricting analysis to periods with at least one observed proxy. Where $\alpha$ is observed but $F$ is not (e.g., periods of effective suppression where mobilization data are absent), we note the censoring rather than coding $F = 0$.

\textbf{Normalization.} Cross-domain comparison requires normalizing $\hat{F}$ to a common scale. We report friction both in raw units (petition signatures, strike days) and as within-domain z-scores relative to the domain's observed range, enabling comparison of friction trajectories across domains with different absolute scales.

\textbf{Structural breaks.} Major institutional reforms---franchise acts, emancipation statutes, constitutional amendments---produce structural breaks in both $\alpha$ and $F$ time series. We model these as regime changes rather than continuous evolution, consistent with the framework's prediction that consent alignment changes discontinuously through institutional intervention. For statistical testing in future large-$N$ work, Chow tests or Bai--Perron structural break detection can identify the timing and magnitude of regime changes, enabling formal tests of whether breaks in $\alpha$ correspond to predicted breaks in $F$.

\textbf{Source hierarchy.} For each case study, we document our source hierarchy explicitly. Primary sources (census records, parliamentary rolls, petition archives, legal codes) are preferred over secondary compilations. Where we rely on secondary sources---particularly for pre-1800 data---we cross-reference against at least two independent secondary accounts. Quantitative data (petition signature counts, voter registration numbers, rebellion casualties) are preferred over qualitative assessments, but we integrate qualitative evidence where it provides information about mechanisms that quantitative data cannot capture (e.g., the reasons legislators cited for supporting reform, the internal deliberations of abolitionist organizations).

\subsection{Comparative Design}
\label{subsec:comparative-design}

The historical case studies employ a structured, focused comparison design \citep{capoccia2007study}. Cases were selected along two dimensions: (i) variation in alpha trajectories---gradual incorporation (suffrage), delayed incorporation with violent friction (abolition), cross-national divergence (labor), and early-stage emergence (platform governance); and (ii) variation in stakes types---political (suffrage), existential (abolition), economic (labor), and informational (platform). This variation enables testing whether the framework's predictions hold across fundamentally different governance contexts.

What makes these cases comparable despite their manifest differences is precisely the framework's contribution: the common metric of consent alignment $\alpha(d,t)$ and friction $F(d,t)$ provides a structural vocabulary for describing dynamics that manifest through different institutional mechanisms and historical contingencies. A franchise extension and an emancipation act are institutionally distinct but structurally identical---both raise $\alpha$ by expanding $H_t(d)$ to include previously excluded stakeholders.

The comparative logic proceeds as follows. Within each case, we trace the $\alpha$--$F$ co-evolution and assess consistency with the framework's five hypotheses. Across cases, we test whether the predicted relationships hold with sufficient regularity to constitute empirical generalizations rather than post hoc rationalization. The framework's contribution is not explaining each historical episode \textit{ex post}---any sufficiently flexible narrative can do that---but generating predictions about \textit{comparative dynamics}: cases with lower $\alpha$ should exhibit higher $F$; sustained $F$ should predict eventual $\alpha$ increases; and the mode of transition (gradual vs. revolutionary) should correlate with the trajectory of friction escalation relative to institutional accommodation capacity.

The comparative design also enables a form of process tracing. Within each case, we identify the causal mechanisms linking $\alpha$ and $F$: through what specific channels does low $\alpha$ generate friction? How does accumulated friction translate into institutional reform? Process tracing complements the correlational evidence from $\alpha$--$F$ trajectories by documenting the intervening steps---petition campaigns leading to parliamentary debates, rebellion costs shifting elite cost--benefit calculations, international demonstration effects lowering resistance thresholds.

We acknowledge the inferential limitations of this design. Seven case studies cannot establish causal relationships with the confidence of randomized experiments or even well-identified natural experiments. What they can do is demonstrate the framework's descriptive adequacy---its ability to organize diverse historical dynamics within a common analytical structure---and generate precise predictions testable through future large-$N$ cross-national panel analyses. The hypotheses specified in Section~\ref{sec:operationalization} (H1--H5) provide the testable predictions; the case studies provide initial evidence of consistency or inconsistency with those predictions.

\subsection{Causal Identification and Endogeneity Concerns}
\label{subsec:causal-identification}

The framework's core predictions are directional: low $\alpha$ causes high $F$ (H1), and persistent $F$ causes future $\alpha$ increases (H4). Testing these causal claims requires addressing the obvious endogeneity: $\alpha$ and $F$ are jointly determined by unobserved institutional and cultural factors. An authoritarian system may exhibit both low $\alpha$ and low observed $F$---not because alignment is high, but because repression suppresses friction expression. Conversely, a liberalizing system may exhibit rising $\alpha$ alongside rising $F$ if liberalization enables previously suppressed friction to become visible.

For the historical case studies, we rely on three identification strategies, none of which is fully satisfactory but which together provide reasonable confidence in directional claims:

\begin{enumerate}[label=(\roman*)]
  \item \textbf{Temporal sequencing}: Where friction mobilization demonstrably preceded institutional reform (e.g., Chartist petitions preceded the Reform Acts; the Jamaican rebellion preceded the Abolition Act), we can infer that friction contributed to reform rather than the reverse. Temporal precedence is necessary but not sufficient for causal identification.

  \item \textbf{Mechanism documentation}: Process tracing through archival sources can establish the channels through which friction translated into reform---parliamentary debates citing petition numbers, Cabinet discussions referencing rebellion costs, legislative records linking specific mobilization events to specific reform proposals. \citet{clarkson1808} provides precisely this kind of mechanism documentation for British abolition.

  \item \textbf{Cross-case comparison}: Variation in institutional accommodation capacity across polities (British parliamentary system vs. American constitutional veto points vs. Haitian colonial absence of accommodation channels) provides quasi-experimental variation in the $F \rightarrow \alpha$ transmission mechanism, enabling tests of the framework's predictions about how institutional structure moderates this relationship.
\end{enumerate}

Future work should exploit quasi-experimental variation more systematically---for instance, using franchise extensions driven by international diffusion \citep{ramirez1997} or wartime mobilization needs as instruments for $\alpha$ changes, and examining their effects on subsequent friction levels.

\section{Suffrage Expansion (1790s--1970s)}
\label{sec:suffrage}

The expansion of political suffrage provides the consent-holding framework's cleanest historical test case. The domain is well-defined, the alpha proxy is directly measurable as a population ratio, friction proxies are abundant in the archival record, and the dynamics span nearly two centuries across multiple polities. If the framework cannot organize suffrage history coherently, it cannot organize anything.

\subsection{Domain Definition}
\label{subsec:suffrage-domain}

Define $d_{\text{suffrage}}$ as the governance domain covering decisions about political franchise---who may vote, stand for office, and participate in collective self-governance. The affected set $S_d$ encompasses all adult residents whose lives are shaped by political decisions: taxation, conscription, property law, family law, criminal justice, welfare provision, and public goods allocation. The consent-holder mapping $H_t(d_{\text{suffrage}})$ identifies, at each time $t$, the subset of the adult population with effective electoral participation rights.

The stakes distribution in $d_{\text{suffrage}}$ is approximately uniform across the affected population: every adult resident has substantial stakes in political outcomes, though stakes vary by domain-specific exposure. Women had disproportionate stakes in family law (coverture laws extinguished married women's legal personhood), property rights (married women could not own property independently until the Married Women's Property Acts of the 1870s--1880s), and employment regulation (no legal protections for women workers until the late 19th century). Workers had disproportionate stakes in labor regulation, poor relief, and criminal law enforcement. These differential stakes distributions mean that excluding women or workers from the franchise produced particularly severe misalignment in the domains where their stakes were highest---a pattern the framework captures through the stakes-weighting in the $\alpha$ measure.

For the suffrage domain itself, we treat $s_i(d_{\text{suffrage}}) \approx s_j(d_{\text{suffrage}})$ for all $i, j \in S_d$, enabling the ratio-scale alpha proxy:

\begin{equation}
  \hat{\alpha}(d_{\text{suffrage}}, t) = \frac{N_{\text{enfranchised}}(t)}{N_{\text{adult population}}(t)}
  \label{eq:alpha-suffrage}
\end{equation}

This proxy has the virtue of directness: franchise expansion is literally the expansion of consent power across the affected population. Its limitation is that it treats all enfranchised citizens as holding equal effective voice, ignoring capacity constraints (literacy requirements, registration barriers, intimidation) that may depress effective $\alpha$ below formal $\alpha$. We address this discrepancy between de jure and de facto franchise in the American case (Section~\ref{subsec:suffrage-dynamics}).

\subsection{Alpha Proxy: Enfranchised Population Share}
\label{subsec:suffrage-alpha}

Table~\ref{tab:suffrage-alpha} traces $\hat{\alpha}(d_{\text{suffrage}}, t)$ across five countries, documenting the stepwise expansion from narrow property-based franchise to universal adult suffrage. The data reveal several patterns relevant to the framework.

\begin{table}[htbp]
\centering
\caption{Consent alignment proxy $\hat{\alpha}(d_{\text{suffrage}}, t)$: enfranchised share of adult population}
\label{tab:suffrage-alpha}
\begin{tabular}{@{}llrl@{}}
\toprule
Country & Year & $\hat{\alpha}$ (\%) & Institutional change \\
\midrule
\multirow{5}{*}{United Kingdom}
  & 1831 & $\sim$5 & Pre-Reform Act (propertied males only) \\
  & 1832 & $\sim$7 & First Reform Act \\
  & 1867 & $\sim$13 & Second Reform Act (urban working men) \\
  & 1884 & $\sim$28 & Third Reform Act (rural working men) \\
  & 1918 & $\sim$47 & Representation of the People Act (women $>$30) \\
  & 1928 & $\sim$97 & Equal Franchise Act (all adults $>$21) \\
\addlinespace
\multirow{4}{*}{United States}
  & 1790 & $\sim$6 & Constitutional franchise (white male freeholders) \\
  & 1870 & $\sim$19 & 15th Amendment (race-blind, nominal) \\
  & 1920 & $\sim$52 & 19th Amendment (women's suffrage) \\
  & 1965 & $\sim$95 & Voting Rights Act (effective Black suffrage) \\
\addlinespace
\multirow{3}{*}{France}
  & 1791 & $\sim$15 & Active citizens (tax-paying males) \\
  & 1848 & $\sim$48 & Universal male suffrage (Second Republic) \\
  & 1944 & $\sim$96 & Women's suffrage (provisional government) \\
\addlinespace
\multirow{2}{*}{Switzerland}
  & 1848 & $\sim$48 & Universal male suffrage (federal constitution) \\
  & 1971 & $\sim$96 & Women's suffrage (federal level) \\
\addlinespace
New Zealand & 1893 & $\sim$95 & Electoral Act (first full female suffrage) \\
\bottomrule
\end{tabular}
\end{table}

The trajectories reveal three distinct patterns. First, \textit{stepwise rather than continuous} expansion: $\alpha$ jumps at discrete institutional moments separated by periods of stasis. The framework explains this through threshold dynamics (H3)---friction must accumulate to levels exceeding institutional accommodation capacity before discrete reform occurs. Second, \textit{substantial cross-national variation} in timing: Switzerland delayed women's suffrage until 1971, a full 78 years after New Zealand. The framework predicts that this variation correlates with differences in friction intensity, repression costs, and elite interest alignment. Third, \textit{acceleration over time}: later extensions came faster, consistent with international diffusion effects documented by \citet{ramirez1997}.

The American case reveals the critical distinction between formal and effective $\alpha$. The 15th Amendment (1870) nominally raised $\hat{\alpha}$ to include Black men, but poll taxes, literacy tests, grandfather clauses, and violent intimidation held effective Black political participation near zero across the South until the Voting Rights Act of 1965 \citep{acemoglu2000why}. The framework handles this through the effective voice concept: $\text{eff\_voice}_i$ can diverge sharply from formal $C_{i,d}$ when capacity constraints---including deliberate suppression---prevent formal rights from translating into actual decision power. Mississippi's Black voter registration rate was approximately 6.7\% in 1964 despite formal constitutional eligibility since 1870---a 94-year gap between formal and effective $\alpha$ that reveals the limitations of de jure measurement alone.

\citet{acemoglu2000why} provide an influential formal model of franchise extension as elite preemption of revolutionary threat, arguing that the West extended the franchise when the threat of revolution made concession cheaper than repression. The consent-holding framework encompasses this mechanism but generalizes it: franchise extension raises $\alpha$ not only to preempt revolutionary friction (the Acemoglu--Robinson channel) but also to reduce ongoing governance costs from non-revolutionary friction (petitioning, protest, non-cooperation) and to capture the performance benefits of broader inclusion (the Chapman and Batinti channels). The Acemoglu--Robinson model maps onto H3 (threshold effects) and H4 (reform pressure from persistent friction), while the consent-holding framework additionally captures H1 (the continuous alignment--friction relationship below the revolutionary threshold) and H5 (performance interactions).

The Swiss case merits specific attention as an outlier. Switzerland---one of Europe's oldest democracies---denied women the federal franchise until 1971, a full 123 years after introducing universal male suffrage in 1848. The canton of Appenzell Innerrhoden resisted women's suffrage until forced by federal court order in 1990. The framework interprets this as a case where the institutional structure (cantonal autonomy, direct democracy requiring male-majority referenda to extend the franchise to women) created a structural barrier: the population holding consent power had to vote to dilute its own power, a collective action problem that the framework's friction dynamics eventually resolved but only with extreme temporal lag. The Swiss case thus represents a natural experiment in institutional accommodation capacity: when franchise extension requires approval from the currently enfranchised, friction must overcome not just elite resistance but majoritarian self-interest.

\subsection{Friction Proxy: Petition Data and Protest Events}
\label{subsec:suffrage-friction}

The suffrage case offers unusually rich friction data, particularly for Britain where petitioning constituted a primary mechanism for expressing political demands. Petitioning was not merely a symbolic act; it was the primary formal channel through which the unenfranchised population could communicate demands to Parliament. The right to petition, enshrined in the 1689 Bill of Rights, provided an institutional channel for friction expression even when the franchise itself was denied---making Britain a natural laboratory for studying how friction operates within institutional constraints. \citet{miller2021} documents the British women's suffrage petition campaigns in systematic detail, enabling construction of a friction time series spanning the 1830s through the 1910s.

\textbf{British Chartist and suffrage petitions.} The Chartist movement provides the earliest large-scale friction data: the People's Charter petitions of 1839 (1.3 million signatures), 1842 (3.3 million signatures), and 1848 (estimated 2--6 million signatures, though contested) represent massive mobilization by working men excluded from the post-1832 franchise. These petition counts, normalized by adult population, provide a friction index:

\begin{equation}
  \hat{F}_{\text{petition}}(t) = \frac{N_{\text{signatures}}(t)}{N_{\text{adult population}}(t)}
  \label{eq:friction-petition}
\end{equation}

The Chartist petition trajectory---escalating from 1.3 million to 3.3 million signatures over three years---illustrates the framework's prediction that sustained low $\alpha$ generates escalating friction (H4). Yet the Chartist movement's failure to secure immediate reform also illustrates scope conditions: friction alone does not guarantee incorporation when elite interests oppose expansion and repression costs are manageable.

\textbf{Women's suffrage friction.} The women's suffrage movement generated friction through multiple channels: petition campaigns (1866 petition with 1,499 signatures initiating the parliamentary suffrage campaign; subsequent petitions escalating to hundreds of thousands of signatures by the 1900s), public demonstrations (the ``Mud March'' of 1907 drawing 3,000 participants; the 1908 Hyde Park rally attracting an estimated 250,000--500,000), and militancy. \citet{pankhurst1914} documents the escalation of the Women's Social and Political Union (WSPU) from constitutional methods to window-smashing, arson, and hunger strikes---a friction trajectory consistent with H3's prediction that sustained exclusion produces escalation when institutional channels prove inadequate.

The suffragette militancy represents an important theoretical case: friction escalation beyond petition and protest into property destruction and bodily sacrifice (hunger strikes, forcible feeding). The framework interprets this as threshold-crossing behavior: when $\alpha = 0$ persists despite mounting constitutionalist friction, some fraction of the excluded population escalates to higher-cost, higher-visibility friction forms. \citet{mason1912} provides contemporary documentation of this escalation logic from within the movement itself.

\textbf{American suffrage friction.} The American suffrage movement generated friction through state-by-state campaigns (over 480 campaigns to get suffrage referenda on ballots between 1868 and 1920), demonstrations (the 1913 Washington suffrage parade involving 5,000--8,000 marchers), and civil disobedience (the 1917 White House pickets and the ``Night of Terror'' at Occoquan Workhouse). The suffrage movement also deployed \textit{proxy friction}---enlisting enfranchised male allies to pressure legislators---demonstrating that friction can be channeled through existing consent-holders when direct institutional access is blocked.

\citet{higginson1859} represents an early articulation of the exclusion logic underlying suffrage friction: the argument that denying women education and political participation was self-reinforcing, as the resulting demonstrated incapacity was then used to justify continued exclusion. In framework terms, this identifies a feedback loop where low $\alpha$ suppresses the capacity required to generate the friction necessary to raise $\alpha$.

Table~\ref{tab:suffrage-friction} summarizes the major friction events across the suffrage cases, documenting the escalation pattern the framework predicts.

\begin{table}[htbp]
\centering
\caption{Selected friction events in the suffrage domain}
\label{tab:suffrage-friction}
\begin{tabular}{@{}lllr@{}}
\toprule
Country & Year & Event & Scale indicator \\
\midrule
UK & 1839 & Chartist petition (1st) & 1.3M signatures \\
UK & 1842 & Chartist petition (2nd) & 3.3M signatures \\
UK & 1866 & Women's suffrage petition & 1,499 signatures \\
UK & 1907 & ``Mud March'' demonstration & 3,000 marchers \\
UK & 1908 & Hyde Park rally & 250,000--500,000 \\
UK & 1909--14 & WSPU militancy campaign & 1,000+ arrests \\
US & 1848 & Seneca Falls Convention & 300 attendees \\
US & 1913 & Washington suffrage parade & 5,000--8,000 \\
US & 1917 & White House pickets/Night of Terror & 218 arrests \\
France & 1848 & Revolutionary demands & Mass mobilization \\
NZ & 1891--93 & Petition campaigns & 31,872 signatures (1893) \\
\bottomrule
\end{tabular}
\end{table}

The table reveals the escalation dynamic across both time and mode. British friction began with constitutionalist petitioning (millions of signatures in the Chartist period), continued through mass demonstration (the 1908 rally remains one of the largest political gatherings in British history), and escalated to militancy when constitutional methods proved insufficient. The WSPU's campaign of property destruction---1,500 windows smashed in a single coordinated action in March 1912, letter-box arson, cutting of telegraph wires, and bombing of the Chancellor's unoccupied residence---represents the framework's threshold-crossing prediction in its most dramatic form.

\subsection{Alpha--Friction Dynamics}
\label{subsec:suffrage-dynamics}

The co-evolution of $\alpha$ and $F$ in the suffrage domain provides systematic evidence for four of the framework's five hypotheses.

\textbf{H1 (Alignment--Friction Relationship).} The British trajectory demonstrates the predicted inverse relationship. The pre-1832 period combined $\hat{\alpha} \approx 0.05$ with sustained friction (radicalism, petitioning, occasional riot). Each franchise extension---1832, 1867, 1884---was followed by declining friction from the newly incorporated group, though friction from still-excluded populations continued. The 1867 Reform Act's incorporation of urban working men reduced Chartist-style class friction but left women's suffrage friction untouched. After 1928, when $\hat{\alpha}$ reached near-universal levels, suffrage-specific friction effectively disappeared. \citet{berlinski2010extension} provides econometric evidence that the Second Reform Act shifted political behavior in newly enfranchised constituencies, consistent with $\alpha$ expansion reducing friction through incorporation.

\textbf{H3 (Threshold Effects).} The dynamics exhibit clear threshold behavior. Long periods of low $\alpha$ with rising friction (UK 1790s--1832, US 1850s--1870 for Black suffrage, UK 1860s--1918 for women's suffrage) alternate with rapid $\alpha$ jumps when accumulated friction exceeds institutional tolerance. The UK's 1918 Representation of the People Act exemplifies this: the combination of women's wartime contributions (demonstrating capacity), suffragette militancy (imposing friction costs), and the need to re-enfranchise soldiers returning from the trenches created a political window where the costs of continued exclusion exceeded the costs of incorporation.

\textbf{H4 (Temporal Dynamics---Reform Pressure).} Persistent friction predicts future $\alpha$ increases with lags reflecting institutional accommodation speeds. British suffrage exhibits lags of approximately 15--40 years between the onset of sustained friction and franchise extension: Chartist mobilization (1838--1848) preceded the Second Reform Act (1867) by nearly two decades; organized women's suffrage campaigns (1860s--1910s) preceded the 1918 Act by approximately 50 years. The framework predicts these lags should shorten as the costs of repression rise and international demonstration effects lower elite resistance thresholds---a prediction broadly consistent with the acceleration of suffrage adoption globally documented by \citet{ramirez1997}.

\textbf{H5 (Performance Interactions).} \citet{chapman2020extension} demonstrates that franchise extension in 19th-century Britain was followed by increased government expenditure on public goods---sanitation, education, infrastructure---benefiting newly enfranchised populations. \citet{batinti2022voting} find that suffrage extension across 15 European countries (1870--2010) improved population health outcomes, an effect operating through expanded public goods provision. These findings support H5's prediction indirectly: franchise expansion ($\alpha$ increase) improved governance performance ($P$ increase), which in turn reduced friction ($F$ decrease) below what $\alpha$ expansion alone would predict. The consent--performance nexus creates a virtuous cycle where incorporation improves both alignment and outcomes.

\textbf{H2 (Stakes--Consent Covariance).} The suffrage case provides indirect support. Prior to franchise extension, the covariance between stakes and consent power was negative for excluded populations: women held high stakes in family law, property law, and employment regulation but zero consent power in these domains. Each franchise extension increased $\text{Cov}(s_i(d), C_{i,d})$ by bringing consent power into alignment with stakes distribution. The subsequent reduction in suffrage-specific friction---the dissolution of suffrage organizations following enfranchisement---is consistent with H2's prediction that increasing stakes--consent covariance reduces friction. The mechanism is straightforward: once a population can vote, it gains institutional channels for expressing preferences, reducing the need for extra-institutional friction.

\textbf{International diffusion.} \citet{ramirez1997} document that women's suffrage adoption followed a diffusion pattern consistent with the framework's threshold dynamics: early adopters (New Zealand 1893, Australia 1902, Finland 1906) demonstrated feasibility, lowering the perceived costs of incorporation for later adopters. In framework terms, international demonstration effects reduced the tolerance threshold $\tau$ across countries: the fact that female suffrage did not produce the governance catastrophes opponents predicted lowered elite resistance to domestic reform. This creates a cross-national version of H4: friction in one country, combined with successful reform in another, accelerates reform adoption.

\textbf{Counterfactual analysis.} The framework generates a specific counterfactual: had franchise extensions not occurred when they did, friction would have continued escalating, potentially producing revolutionary outcomes. The British case provides limited evidence for this counterfactual. The 1832 Reform Act was passed in the context of widespread unrest---the Bristol riots of 1831, agricultural laborer uprisings (the Swing riots), and explicit revolutionary rhetoric from reform societies. Contemporary observers, including the Duke of Wellington, assessed the risk of revolution as genuine. The framework interprets this as a case where friction approached the threshold at which institutional accommodation became urgent, and reform represented the less costly alternative to suppression. Whether actual revolution would have occurred absent reform is inherently unknowable, but the framework's prediction that the cost calculus favored accommodation over repression is supported by the legislative outcome.

Figure~\ref{fig:alpha-suffrage} traces $\hat{\alpha}(d_{\text{suffrage}}, t)$ for five nations, documenting the stepwise expansion from narrow propertied-male franchise to near-universal adult suffrage. The trajectories' common shape---long plateaus punctuated by discrete jumps---illustrates the framework's prediction that consent alignment changes discontinuously through institutional reform rather than continuously through gradual social evolution.

\begin{figure*}[htbp]
\centering
\includegraphics[width=0.95\textwidth]{alpha_suffrage.pdf}
\caption{Consent alignment trajectories $\hat{\alpha}(d_{\text{suffrage}}, t)$ for five nations, plotted from V-Dem v15 annual data (\texttt{v2x\_suffr}: share of population with suffrage). Key institutional moments from Table~\ref{tab:suffrage-alpha} annotated. The trajectories exhibit the framework's predicted pattern of long plateaus punctuated by discrete jumps, while the annual resolution reveals dynamics invisible at the institutional-milestone level: France's Restoration-era collapse (1814), the gradual Jacksonian expansion of white male suffrage in the United States (1790s--1840s), and Switzerland's 123-year plateau at approximately 50\% (male-only suffrage, 1848--1971). All five trajectories converge toward universal suffrage but through markedly different temporal paths.}
\label{fig:alpha-suffrage}
\end{figure*}

\subsection{Cross-Case Connections}
\label{subsec:suffrage-connections}

Suffrage connects to the other historical cases through three mechanisms. First, \textit{organizational overlap}: the American suffrage movement grew directly from abolitionist networks. The 1848 Seneca Falls Convention, where \citet{stanton1848} Declaration of Sentiments articulated women's political exclusion in the language of the Declaration of Independence, was organized by women radicalized through anti-slavery activism. Frederick Douglass attended and supported the suffrage resolution. This organizational linkage illustrates how friction mobilization in one domain ($d_{\text{slavery}}$) can catalyze mobilization in adjacent domains ($d_{\text{suffrage}}$) through shared organizational infrastructure and frame extension.

Second, \textit{class vs. gender franchise}: suffrage expansion proceeded along two partially independent axes---economic (property requirements) and gender (sex restrictions). The British trajectory reveals that these axes interacted: the Third Reform Act (1884) extended the vote to agricultural laborers, incorporating the class dimension, while leaving the gender dimension unchanged for another 34 years. The framework accommodates this through multi-dimensional $\alpha$ decomposition: $\hat{\alpha}$ can be disaggregated by exclusion dimension to track which populations are incorporated at which rates.

Third, \textit{formal vs. effective suffrage}: the American case---where the 15th Amendment (1870) and 19th Amendment (1920) nominally achieved near-universal franchise but Jim Crow suppression maintained effective Black exclusion until 1965---connects directly to the civil rights case study. This 95-year gap between formal and effective $\alpha$ expansion represents the framework's most important empirical challenge: accounting for the divergence between institutional consent structures and realized consent power. The consent-holding framework handles this through the effective voice construct, but the American case demonstrates that effective voice measurement requires attention to enforcement, intimidation, and institutional barriers that formal legal analysis alone cannot capture.

Fourth, \textit{the franchise--public goods nexus}: suffrage expansion did not merely redistribute existing consent power; it changed what governance produced. \citet{chapman2020extension} demonstrates that 19th-century British franchise extensions were followed by substantial increases in public goods provision---sanitation, education, poor relief---targeted at the newly enfranchised populations. \citet{batinti2022voting} find parallel effects on health outcomes across 15 European countries. This nexus illustrates a feedback mechanism the framework captures through the performance variable $P(d,t)$: raising $\alpha$ improves $P$ (because governance becomes more responsive to a broader set of stakes), which in turn further reduces friction (because even those who did not gain voice benefit from improved public goods). The virtuous cycle of incorporation---higher $\alpha \rightarrow$ better $P \rightarrow$ lower $F \rightarrow$ reduced pressure for further reform---explains why franchise extensions, once achieved, proved remarkably stable: no Western democracy reversed universal suffrage once it was established (unlike abolition, where Reconstruction-era gains were reversed for nearly a century).


\section{Abolition and Emancipation (1780s--1870s)}
\label{sec:abolition}

If suffrage provides the framework's cleanest test case, abolition provides its most extreme. The domain combines existential stakes ($s_{\text{enslaved}}(d) = \text{maximal}$) with zero consent power ($C_{\text{enslaved}} = 0$ by legal definition), producing $\hat{\alpha}(d_{\text{slavery}}) \approx 0$---the theoretical floor. The framework predicts that this configuration generates unsustainable friction, and indeed the abolition of Atlantic slavery involved the most intense and prolonged friction dynamics of any case we examine: centuries of slave resistance, decades of organized abolitionism, and, in the American case, civil war.

The abolition case also introduces a construct absent from simpler cases: \textit{proxy consent}. Enslaved persons, defined out of political personhood by the very system from which they sought liberation, could not directly exercise consent power in the domains governing their status. The friction that eventually raised $\alpha$ was generated partly through direct slave resistance and partly through proxy agents---abolitionists, sympathetic legislators, humanitarian organizations---who channeled moral friction on behalf of those without voice. This mechanism, where enfranchised agents represent the stakes of disenfranchised populations, is central to the framework's account of how $\alpha$ increases when the excluded population lacks the institutional capacity for self-representation.

\subsection{Domain Definition}
\label{subsec:abolition-domain}

Define $d_{\text{slavery}}$ as the governance domain encompassing decisions over enslaved persons' legal status, bodily autonomy, labor conditions, family integrity, and physical security. The decision scope is comprehensive: slaveholders exercised near-total authority over enslaved persons' lives, and legislative bodies determined the legal framework permitting, regulating, or prohibiting enslavement.

The affected set $S_d$ is stratified by stakes magnitude:

\begin{itemize}
  \item \textbf{Enslaved persons}: $s_{\text{enslaved}}(d) = \text{existential}$. Life, liberty, bodily autonomy, family integrity, and physical security were all at stake. By any reasonable stakes metric, enslaved persons held maximal stakes in the slavery domain.

  \item \textbf{Slaveholders}: $s_{\text{slaveholders}}(d) = \text{economic}$. Enslaved labor constituted the primary capital asset in plantation economies. In the American South, the capital value of enslaved persons exceeded the combined value of all manufacturing, railroads, and other productive capital.

  \item \textbf{Broader population}: $s_{\text{public}}(d) > 0$ through economic integration (slave-produced commodities, supply chains), moral implication (complicity in the slave system), and political consequences (the slavery question dominated Anglo-American politics for decades).
\end{itemize}

The consent-holder mapping $H_t(d_{\text{slavery}})$ placed decision power entirely with slaveholders (at the individual level) and with legislatures dominated by slaveholding interests (at the institutional level). Enslaved persons held $C_{\text{enslaved}} = 0$ not as an empirical approximation but as a definitional feature of the legal system: enslaved persons were classified as property, not persons, and thus had no legal standing to participate in decisions about their own status.

This produces $\hat{\alpha}(d_{\text{slavery}}) \approx 0$ despite the affected population holding the highest conceivable stakes. The framework identifies this as the maximally misaligned configuration---the governance equivalent of a system under maximum structural stress.

\subsection{Alpha Proxy: Legal Status Index}
\label{subsec:abolition-alpha}

We construct an ordinal $\hat{\alpha}$ index tracking the legal status of enslaved and formerly enslaved persons across four polities (Table~\ref{tab:alpha-abolition-scale}). The index captures the cumulative expansion of legal personhood, from complete denial to effective citizenship.

\textbf{British Empire trajectory.} The British path from chattel slavery to emancipation proceeded through identifiable institutional stages: the Somerset ruling (1772, $\hat{\alpha} \approx 0.05$---slavery unenforceable in England proper, no effect on colonial slavery); the Slave Trade Act (1807, $\hat{\alpha} \approx 0.10$---prohibition of the trade, not the institution); the Slavery Abolition Act (1833, $\hat{\alpha} \approx 0.50$---emancipation with a six-year ``apprenticeship'' period); and full freedom (1838, $\hat{\alpha} \approx 0.75$---apprenticeship ended, though effective citizenship remained limited by economic marginalization and colonial governance structures).

\textbf{American trajectory.} The American path was slower and more violent: the Northwest Ordinance (1787, $\hat{\alpha} \approx 0.05$---slavery prohibited in new northern territories, entrenched in the South); the ban on the international slave trade (1808, $\hat{\alpha} \approx 0.10$---limited impact given domestic slave breeding); the Emancipation Proclamation (1863, $\hat{\alpha} \approx 0.25$---limited to Confederate states, a war measure); the 13th Amendment (1865, $\hat{\alpha} \approx 0.50$---universal abolition); the 14th and 15th Amendments (1868--1870, $\hat{\alpha} \approx 0.75$---formal citizenship and voting rights); and the effective enforcement beginning with the Civil Rights Act and Voting Rights Act (1964--1965, $\hat{\alpha} \approx 0.90$---substantive citizenship protections, though full equality remains contested).

The American trajectory illustrates a critical pattern: the gap between the 13th Amendment ($\hat{\alpha} \approx 0.50$ in 1865) and effective citizenship ($\hat{\alpha} \approx 0.90$ in 1965) spanned a full century. Reconstruction's brief $\alpha$ expansion (1865--1877) was reversed by Jim Crow, demonstrating that consent alignment gains can be rolled back when the institutional structures supporting them are dismantled. The framework predicts that such reversals generate renewed friction---a prediction confirmed by the subsequent century of Black resistance, from the Niagara Movement (1905) through the Civil Rights Movement (1950s--1960s).

\textbf{Haitian trajectory.} Haiti represents the revolutionary path: $\hat{\alpha}$ jumped from approximately 0.00 to approximately 0.75 through armed revolution (1791--1804). The Haitian Revolution is the only case of enslaved persons directly overthrowing the slave system through military force, rather than relying on proxy consent mechanisms operating through the slaveholding polity's institutions. In framework terms, this represents the case where institutional accommodation capacity was zero---the colonial system provided no mechanisms for raising $\alpha$ incrementally---forcing extra-institutional resolution.

\textbf{French trajectory.} France oscillated: the National Convention abolished slavery in 1794 ($\hat{\alpha}: 0.00 \rightarrow 0.50$), Napoleon re-established it in 1802 ($\hat{\alpha}: 0.50 \rightarrow 0.00$), and final abolition came in 1848 ($\hat{\alpha}: 0.00 \rightarrow 0.75$). The reversal under Napoleon demonstrates a distinctive failure mode: $\alpha$ expansion achieved through elite decision (metropolitan legislative action) without local institutional consolidation proved fragile when elite preferences shifted. The framework predicts that $\alpha$ gains driven by proxy consent rather than direct stakeholder mobilization are less durable, as they lack the friction-generating capacity to resist reversal.

Table~\ref{tab:abolition-trajectories} summarizes the ordinal $\hat{\alpha}$ trajectories across the four polities, documenting both the diverse paths and the common destination.

\begin{table}[htbp]
\centering
\caption{Ordinal $\hat{\alpha}(d_{\text{slavery}}, t)$ trajectories across four polities}
\label{tab:abolition-trajectories}
\begin{tabular}{@{}lcccccc@{}}
\toprule
 & \multicolumn{6}{c}{$\hat{\alpha}$ at key institutional moments} \\
\cmidrule(l){2-7}
Polity & Chattel & Amelioration & Trade ban & Emancipation & Citizenship & Effective \\
\midrule
British Empire & 0.00 & 0.05 (1772) & 0.10 (1807) & 0.50 (1833) & 0.75 (1838) & --- \\
United States & 0.00 & 0.05 (1787) & 0.10 (1808) & 0.50 (1865) & 0.75 (1870) & 0.90 (1965) \\
Haiti & 0.00 & --- & --- & 0.75 (1804) & 0.75 (1804) & --- \\
France & 0.00 & --- & 0.50 (1794) & 0.00 (1802) & 0.75 (1848) & --- \\
\bottomrule
\end{tabular}
\end{table}

The table reveals several patterns. First, gradual trajectories (British, American) passed through intermediate $\alpha$ states while revolutionary trajectories (Haitian) jumped directly from 0.00 to 0.75. Second, the French reversal ($0.50 \rightarrow 0.00$ under Napoleon) is unique among the four cases and supports the framework's prediction about the fragility of proxy-driven $\alpha$ gains. Third, the American trajectory exhibits the longest gap between formal citizenship (1870) and effective citizenship (1965), reflecting the most sustained institutional resistance to $\alpha$ consolidation.

\subsection{Friction Proxy: Resistance and Abolition Mobilization}
\label{subsec:abolition-friction}

The abolition domain generated friction through two distinct channels: direct resistance by enslaved persons and organized mobilization by abolitionist networks. We construct separate friction proxies for each.

\textbf{Direct resistance.} Slave rebellions provide count-based friction data, though the historical record is incomplete (many small-scale acts of resistance---work slowdowns, sabotage, flight---went unrecorded or were suppressed from official accounts). Major rebellions constitute friction spikes visible in the historical record:

\begin{itemize}
  \item \textit{Haiti} (1791--1804): The largest and most successful slave rebellion in history. Beginning with the August 1791 uprising involving an estimated 100,000 enslaved persons, the revolution destroyed the most profitable colony in the Caribbean and established the first Black republic. In framework terms, this represents friction at a scale exceeding the system's accommodation capacity, resulting in complete $\alpha$ restructuring through revolutionary means.

  \item \textit{Jamaica} (1831--1832): The ``Baptist War'' or Christmas Rebellion, involving an estimated 60,000 enslaved persons, destroyed property worth approximately \pounds1.1 million. \citet{blackburn1988} documents how this rebellion directly accelerated parliamentary abolition: the destruction demonstrated that maintaining slavery imposed escalating costs---exactly the friction mechanism the framework predicts.

  \item \textit{Nat Turner's Rebellion} (1831): Though smaller in scale (approximately 70 enslaved persons), Turner's rebellion in Virginia demonstrated the vulnerability of slaveholding society to organized resistance and generated intense fear-based friction among slaveholders, paradoxically leading to both ameliorative discussion and repressive backlash.
\end{itemize}

Everyday resistance constituted a lower-intensity but continuous friction source: malingering, tool-breaking, feigned illness, running away, and maintaining cultural practices prohibited by slaveholders. The scale of flight alone was substantial: an estimated 100,000 enslaved persons escaped via the Underground Railroad between 1810 and 1860, and the Fugitive Slave Act of 1850---requiring Northern states to assist in recapturing escaped slaves---generated its own friction cycle, as Northern populations confronted the slavery system's operational demands directly. This ``infrapolitics'' of resistance \citep{drescher1987} represents the chronic friction the framework predicts under sustained low $\alpha$---too diffuse to generate institutional reform directly but contributing to the slave system's operational costs and moral delegitimation.

The economic dimension of slave resistance as friction deserves emphasis. Each act of resistance---from work slowdowns to rebellion---imposed costs on slaveholders: lost labor, property damage, suppression expenditures, insurance costs, militia maintenance. These costs constituted an ongoing ``friction tax'' on the slave system, eroding its economic efficiency relative to free labor alternatives. \citet{blackburn1988} argues that this friction tax, combined with the growing recognition that free labor was more productive in industrial settings, contributed to the political calculations enabling abolition. In framework terms, rising friction costs shifted the elite cost--benefit analysis: when the costs of maintaining $\alpha \approx 0$ (rebellion suppression, moral opprobrium, economic inefficiency) exceeded the costs of $\alpha$ expansion (lost slave labor, compensation payments), institutional accommodation became rational.

\textbf{Abolition mobilization.} The organized abolitionist movement, operating primarily within the metropolitan polities of slaveholding empires, generated friction through institutional channels. The \citet{abolitionsociety1787} Society for the Abolition of the Slave Trade, founded in London with twelve members, grew into a mass movement generating petition campaigns of unprecedented scale:

\begin{itemize}
  \item 1788: Over 100 petitions to Parliament against the slave trade
  \item 1792: 519 petitions with an estimated 390,000 signatures (approximately 13\% of the adult male population)
  \item 1814: 800 petitions following the Congress of Vienna
  \item 1833: Massive petition campaign accompanying the Abolition Act, with over 1.3 million signatures on a single petition (the largest petition in British history to that date)
\end{itemize}

\citet{clarkson1808} documents the systematic evidence-gathering that undergirded abolitionist friction: collecting testimony from sailors, surgeons, and merchants; procuring physical artifacts (shackles, branding irons, the Brooks slave ship diagram); and conducting investigations at slave-trading ports. This constitutes what we might call \textit{informational friction}---making visible the consequences of zero-$\alpha$ governance to populations with the institutional capacity to demand change.

\citet{wilberforce1807} speeches in Parliament transformed grassroots friction into legislative pressure through sustained advocacy spanning nearly two decades of parliamentary defeats before achieving the Slave Trade Act (1807) and contributing to the momentum that produced abolition in 1833. The parliamentary record is instructive: Wilberforce introduced abolition motions annually from 1789 to 1807, losing repeatedly before succeeding. Each defeat, however, narrowed the margin and expanded the coalition. In 1791, the motion to abolish the slave trade was defeated 163--88; by 1796, the margin had narrowed to 74--70; and in 1807, the Slave Trade Act passed 283--16. This trajectory---repeated friction inputs producing diminishing institutional resistance until a threshold is crossed---exemplifies H4's temporal dynamics at the legislative level.

\citet{equiano1789} narrative---the first widely read autobiography by a formerly enslaved person---operated as a friction transmission mechanism, converting the experiential reality of enslavement into moral and political pressure on the metropolitan public. Equiano's \textit{Interesting Narrative} went through nine editions in his lifetime, was translated into multiple languages, and directly influenced parliamentary debates. In framework terms, the narrative served as a technology for transmitting the lived reality of $\alpha \approx 0$ governance to populations with the institutional capacity to generate effective friction. This represents a generalizable mechanism: when directly affected populations lack institutional access, their experiences must be transmitted to proxy agents through narrative, testimony, and documentation---what we might formalize as a friction transmission function mapping experiential stakes into observable proxy friction.

The sugar boycott of the 1790s represents a distinctive friction channel: consumer activism. An estimated 300,000--400,000 British households boycotted slave-produced sugar, reducing consumption by approximately one-third. This constitutes economic friction---directly reducing the profitability of slave labor---channeled through consumer decisions rather than political institutions. The boycott demonstrates that friction can operate through market mechanisms as well as political ones, a pattern that recurs in the platform governance case (advertiser boycotts) and corporate governance case (divestment campaigns).

\subsection{Alpha--Friction Dynamics}
\label{subsec:abolition-dynamics}

The abolition case illuminates the framework's predictions under extreme conditions---what happens when $\alpha \approx 0$ for the population with maximal stakes.

\textbf{H1 (Alignment--Friction Relationship).} The case provides strong support: $\alpha \approx 0$ over centuries generated sustained friction in every slaveholding society. No slave system achieved durable stability; all faced continuous resistance and mounting opposition. The framework predicts this: existential stakes combined with zero consent power produce the maximum possible preference--outcome deviation, generating correspondingly intense mobilization pressure. The comparative evidence is unambiguous: no polity maintained $\alpha \approx 0$ in a domain with existential stakes indefinitely. The Atlantic slave system lasted approximately 350 years (1500s--1880s in its fullest extent, counting Brazil's abolition in 1888), but its entire duration was marked by resistance, rebellion, and mounting moral and political opposition---the framework would characterize this as three and a half centuries of unsustainable friction under an $\alpha \approx 0$ regime.

The comparative intensity of friction across cases supports H1's directional prediction. Abolition generated higher-intensity friction than suffrage (rebellions, civil war vs. petitions, demonstrations), consistent with the prediction that higher stakes amplify friction: $s_{\text{enslaved}}(d) = \text{existential}$ versus $s_{\text{disenfranchised women}}(d) = \text{political}$. This stakes-friction proportionality is a second-order prediction of H1 that the historical record broadly supports.

\textbf{H3 (Threshold Effects).} The transition dynamics exhibit threshold behavior. In the British case, petition campaigns, sugar boycotts, and parliamentary advocacy accumulated over decades without producing institutional change, then generated rapid $\alpha$ expansion in the 1807--1838 window. The Jamaican rebellion of 1831--1832 appears to have been the threshold-crossing event: it demonstrated that the costs of maintaining slavery (property destruction, military deployment, moral opprobrium) exceeded the costs of compensated emancipation. In framework terms, cumulative friction crossed the tolerance threshold $\tau$, triggering institutional accommodation.

The chronology supports this interpretation precisely. The Jamaica rebellion occurred in December 1831--January 1832. The parliamentary Select Committee on the Extinction of Slavery was established in 1832. The Abolition Act passed in August 1833. The causal proximity is tight: a major rebellion followed by institutional reform within 20 months, after decades of petition-based friction that produced only incremental legislative responses (the 1807 trade ban, the 1823 amelioration resolution). The rebellion did not introduce a new argument---the moral case had been made for decades---but it altered the cost calculus by demonstrating that the slave system's maintenance costs were escalating beyond what compensated emancipation would cost.

The American case shows what happens when institutional accommodation mechanisms are blocked. The constitutional structure---specifically the three-fifths compromise, the Senate's equal representation of slave and free states, and the fugitive slave clause---prevented the kind of gradual $\alpha$ expansion that occurred in Britain. Friction accumulated without institutional outlet until it produced the most destructive conflict in American history. The framework predicts that systems lacking institutional channels for $\alpha$ adjustment face binary outcomes: either friction is suppressed indefinitely (requiring escalating repression costs) or it produces extra-institutional rupture. The American Civil War is the paradigmatic case of the latter.

The American trajectory also reveals a compound failure: not only were institutional channels blocked for the enslaved population ($C_{\text{enslaved}} = 0$), but the enfranchised population was divided between slaveholding and non-slaveholding interests, preventing even proxy consent from operating effectively within the constitutional framework. The Missouri Compromise (1820), the Compromise of 1850, and the Kansas-Nebraska Act (1854) represent successive failed attempts to manage friction through territorial accommodation rather than $\alpha$ expansion. Each compromise generated new friction (``Bleeding Kansas,'' the Dred Scott backlash, John Brown's raid) rather than resolving it, consistent with the framework's prediction that accommodations short of genuine $\alpha$ expansion produce temporary friction reduction followed by renewed escalation.

\textbf{H4 (Temporal Dynamics).} Persistent friction predicted future $\alpha$ increases with varying lags: British abolition required approximately 46 years from organized abolitionist mobilization (1787) to emancipation (1833); American abolition required approximately 78 years from the founding debates (1787) to the 13th Amendment (1865). The Haitian case compressed this to 13 years (1791--1804) through revolutionary intensity that bypassed gradual institutional accommodation entirely.

The comparative lags are theoretically informative. British institutional channels (Parliament, petitioning, the free press) permitted friction to be expressed and accumulated peacefully, enabling gradual accommodation. American institutional channels were blocked by slaveholder veto points in the constitutional structure, producing a longer lag and violent resolution. The framework predicts that institutional channel availability---what \citet{tilly2007} terms the ``opportunity structure'' for contentious politics---moderates the lag between friction accumulation and $\alpha$ expansion.

The lag structure also reveals an important asymmetry between H1 and H4. H1 (low $\alpha$ generates friction) operates relatively quickly---enslaved populations resisted from the beginning of their enslavement, not after some delay. H4 (friction generates $\alpha$ increases) operates with substantial and variable lags, because the translation of friction into institutional reform requires overcoming collective action problems, building coalition capacity, and waiting for political windows. This asymmetry---fast H1, slow H4---produces the characteristic pattern of long periods of low $\alpha$ with rising friction punctuated by rapid $\alpha$ transitions, visible in both the suffrage and abolition cases. The asymmetry is not a deficiency of the framework but a prediction about the differential speeds of preference expression versus institutional change.

\textbf{The proxy consent mechanism.} The abolition case introduces a theoretical construct that recurs in subsequent case studies: proxy consent. Enslaved persons, denied all institutional standing, could not represent themselves in the governance decisions affecting their status. The friction that ultimately raised $\alpha$ was generated substantially through proxy agents---abolitionists, humanitarian organizations, sympathetic legislators---who channeled the moral weight of enslaved persons' existential stakes into institutional friction. \citet{drescher1987} documents how British abolition succeeded through a combination of parliamentary lobbying, mass petition campaigns, and sustained moral pressure---effectively raising $\alpha$ through proxy consent mechanisms before legal emancipation occurred.

The proxy consent mechanism has both strengths and limitations. Its strength is that it enables $\alpha$ expansion even when the directly affected population lacks institutional access. Its limitation is that proxy agents' preferences may diverge from those they claim to represent. British abolitionists advocated compensated emancipation---compensating slaveholders, not the enslaved---a ``liberation'' that left formerly enslaved persons landless and economically marginalized. In framework terms, proxy consent raises $\alpha$ imperfectly: it incorporates the stakes of excluded populations into decision processes, but the mediation through proxy agents introduces distortion that keeps effective $\alpha$ below what direct representation would achieve.

\textbf{The fiction of consent problem.} The abolition case raises the framework's deepest challenge: how does $\alpha$ expand for populations that cannot participate in the process of their own liberation? The answer---through proxy consent, through direct resistance generating friction that others translate into institutional pressure, through narratives like \citet{equiano1789} that transmit the experiential reality of exclusion to enfranchised audiences---is empirically adequate but theoretically uncomfortable. It means that the most extreme cases of low $\alpha$, where the framework's predictions about friction are strongest, are also the cases where $\alpha$ expansion depends on the very populations whose exclusion defines the problem.

This paradox has a structural resolution within the framework. The consent-holding necessity theorem (T1) establishes that someone must hold decision authority; it does not require that the decision-maker be the affected party. Proxy consent mechanisms are second-best: they raise $\alpha$ above zero by incorporating affected parties' stakes indirectly, through representatives who have institutional access. The abolition case demonstrates both the power and the limitations of this mechanism. Proxy consent enabled peaceful British abolition, but the proxies' preferences (compensated emancipation, maintaining colonial governance structures) diverged from the enslaved population's preferences (immediate freedom, land redistribution, self-governance). Effective $\alpha$ under proxy consent remained well below what direct consent would produce---a systematic distortion that the framework should theorize more precisely in future work.

\textbf{Self-citation connection.} The proxy consent mechanism connects to the broader research program's analysis of structural exclusion. \citet{farzulla2025stakes} develops the concept of ``stakes without voice''---populations bearing consequences of decisions in which they have no effective participation---as a general theoretical construct. The abolition case provides the extreme historical instance: maximum stakes, zero voice, and the resulting reliance on proxy mechanisms that inevitably distort the consent they claim to represent. The formal treatment in \citet{farzulla2025aoc} models the friction dynamics of such exclusion through axioms governing multi-agent coordination under consent constraints, providing the mathematical foundations for the empirical patterns documented here.

\subsection{Cross-Case Connections}
\label{subsec:abolition-connections}

The abolition case connects to other historical domains through three mechanisms with significant implications for the framework's general applicability.

First, \textit{movement overlap and organizational inheritance}. The abolition and suffrage movements were deeply intertwined in both Britain and America. Abolitionist networks provided organizational infrastructure, rhetorical templates, and activist cadres for the women's suffrage movement. The \citet{stanton1848} Declaration of Sentiments at Seneca Falls consciously modeled its structure and language on the Declaration of Independence, extending the exclusion logic from race to gender. In framework terms, friction mobilization in $d_{\text{slavery}}$ generated organizational spillovers that reduced the mobilization costs in $d_{\text{suffrage}}$---a cross-domain friction transmission mechanism not captured by single-domain analysis.

This cross-domain linkage also created tensions. After the Civil War, the question of whether to prioritize Black male suffrage or universal suffrage (including women) split the movement: Stanton and Anthony opposed the 15th Amendment because it enfranchised Black men but not women, while Douglass and others argued that this was ``the Negro's hour.'' The framework interprets this as a domain-priority conflict: with institutional accommodation capacity limited, actors disagreed about whether $\alpha$ expansion in $d_{\text{racial suffrage}}$ or $d_{\text{gender suffrage}}$ should take precedence. The resulting organizational split---the American Equal Rights Association fractured into the National Woman Suffrage Association and the American Woman Suffrage Association---illustrates how coalition fragmentation under constrained accommodation capacity can delay $\alpha$ expansion in both domains.

Second, \textit{the incomplete $\alpha$ expansion problem}. American emancipation (1865) did not resolve the underlying consent misalignment---it transformed it. The abolition of slavery raised $\hat{\alpha}(d_{\text{slavery}})$ from approximately 0.00 to approximately 0.50 (legal freedom without full citizenship), but the broader domain of racial governance---civil rights, political participation, economic opportunity---maintained $\alpha$ levels well below threshold for another century. The framework predicts that partial $\alpha$ expansion in one domain generates friction in adjacent domains where exclusion persists, as newly incorporated populations discover the limits of their incorporation. This is precisely what occurred: emancipation was followed by the Reconstruction Amendments, their systematic nullification, and ultimately the Civil Rights Movement---a century-long effort to extend the $\alpha$ expansion from formal freedom to effective citizenship.

Third, \textit{the mode of transition}. The British (peaceful), American (civil war), and Haitian (revolution) paths to abolition represent three distinct modes of $\alpha$ expansion that the framework should account for. The key differentiating variable appears to be the availability and capacity of institutional accommodation mechanisms. Britain's parliamentary system, combined with the geographic separation between metropole and colonies, permitted gradual $\alpha$ expansion through proxy consent mechanisms operating within existing institutions. America's constitutional structure, which gave slaveholders effective veto power, blocked gradual accommodation, producing violent rupture. Haiti's colonial system provided no accommodation mechanisms at all, making revolution the only available path. The framework's research agenda should develop formal models predicting transition mode as a function of institutional accommodation capacity and friction intensity---a project with direct implications for understanding contemporary governance transitions.

Fourth, \textit{compensated versus uncompensated transition}. Britain's compensated emancipation (1833) allocated \pounds20 million---approximately 40\% of annual government expenditure---to slaveholders, while the American transition involved no compensation (the 13th Amendment simply abolished the institution). The framework does not directly predict the compensation mechanism, but it illuminates the political economy: British slaveholders, facing mounting friction costs and declining relative economic returns, accepted compensation as a face-saving exit. American slaveholders, facing the same friction pressures, refused compromise because the constitutional structure gave them sufficient veto power to block legislative accommodation---until the war itself eliminated their institutional capacity to resist. The compensation question reveals how the distribution of consent power among the currently enfranchised (slaveholders' veto power) determines whether $\alpha$ expansion proceeds through negotiated settlement or violent rupture.

Figure~\ref{fig:alpha-abolition} shows the ordinal $\hat{\alpha}$ trajectory for four polities, documenting the diverse paths from $\alpha \approx 0$ (chattel slavery) to $\alpha \geq 0.75$ (legal citizenship). The trajectories' divergence---particularly the Haitian revolutionary jump versus the British gradualist path versus the American violent rupture---illustrates that the framework's prediction of eventual $\alpha$ expansion under unsustainable friction is robust, while the mode and timing of expansion depend on institutional parameters the framework identifies but does not yet formally model.

\begin{figure*}[htbp]
\centering
\includegraphics[width=0.95\textwidth]{alpha_abolition.pdf}
\caption{Ordinal consent alignment trajectories $\hat{\alpha}(d_{\text{slavery}}, t)$ for four polities, constructed from the legal status index (Table~\ref{tab:alpha-abolition-scale}). The right axis maps ordinal values to institutional configurations. Three distinct transition modes are visible: British gradualism (incremental reform through parliamentary channels), American violent rupture (Civil War followed by a 95-year gap between formal and effective citizenship), and Haitian revolutionary discontinuity (direct jump from chattel slavery to independence). The French oscillation---abolition (1794), reversal under Napoleon (1802), and re-abolition (1848)---uniquely demonstrates the fragility of proxy-driven $\alpha$ gains without local institutional consolidation.}
\label{fig:alpha-abolition}
\end{figure*}

Taken together, the suffrage and abolition cases establish the framework's descriptive adequacy for the most fundamental consent realignments in modern political history. Both cases confirm the core prediction that sustained low $\alpha$ with high stakes generates escalating friction (H1, H4), and both reveal the mechanism through which friction translates into institutional reform: by raising the costs of exclusion above the costs of incorporation. The cases diverge in their friction intensities (proportional to stakes, as the framework predicts), their transition modes (gradual vs. violent, determined by institutional accommodation capacity), and their completeness ($\alpha$ expansion in suffrage was more complete and durable than in abolition, where the gap between formal and effective citizenship persisted for generations). These patterns carry forward into the subsequent cases---labor rights, civil rights, and corporate governance---where the same dynamics operate at different scales and through different institutional channels.


\section{Labor Rights and Codetermination (1850s--Present)}
\label{sec:labor}

The workplace constitutes one of the most consequential governance domains in modern economies. Decisions regarding employment security, wages, working conditions, occupational safety, and workplace dignity affect workers' material welfare, psychological well-being, and life prospects with an intensity that few other governance domains approach outside wartime. Yet for most of industrial history, the consent-holder mapping $H_t(d_{\text{workplace}})$ concentrated authority almost entirely in capital owners and their managerial agents, generating a persistent structural gap between stakes and voice that the consent-holding framework predicts would produce sustained, high-intensity friction.

This section traces the labor movement's two-century struggle to raise $\alpha(d_{\text{workplace}})$ across four national contexts, constructing quantitative alpha and friction proxies from OECD and ILO data and demonstrating how variation in institutional design produces variation in alignment-friction dynamics that closely track the framework's predictions.

\subsection{Domain Definition}
\label{subsec:labor-domain}

We define $d_{\text{workplace}}$ as the set of decisions affecting employment security, wages, working conditions, occupational safety, working hours, and workplace dignity. The affected set $S_{d_{\text{workplace}}}$ includes:

\begin{itemize}[nosep]
\item \textbf{Workers} (high $s_i$): livelihood, bodily safety, and daily autonomy depend directly on workplace decisions. For workers in hazardous industries---coal mining, construction, chemical manufacturing---stakes approach existential levels.
\item \textbf{Employers and shareholders} (high $s_i$): returns on capital, firm viability, and strategic direction are contingent on workplace governance.
\item \textbf{Consumers} (moderate $s_i$): product quality, price, and availability are indirectly affected by workplace decisions.
\item \textbf{Communities} (moderate $s_i$): local employment, tax base, and environmental externalities create spatially concentrated stakes.
\end{itemize}

Prior to labor organization, consent power was concentrated almost exclusively in capital owners: $C_{\text{workers}} \approx 0$, $C_{\text{capital}} \approx 1$. This extreme misalignment---high stakes with near-zero voice---generated the conditions for what became the most sustained friction campaign in modern political history: the labor movement. Early labor organizations like \citet{knightsoflabor1878} explicitly articulated this stakes-voice gap in their foundational principles, demanding ``the right of those who create [wealth] to share in its blessings.''

\subsection{Alpha Proxy: Union Density and Collective Bargaining Coverage}
\label{subsec:labor-alpha}

We construct $\alpha(d_{\text{workplace}})$ from two empirical measures: trade union density (the proportion of wage earners who are union members) and collective bargaining coverage (the share of employees whose terms of employment are set through negotiated agreements). These proxy effective worker voice---the capacity to influence workplace decisions---rather than merely formal entitlements.

Union density traces a distinctive arc across national contexts. In the United Kingdom, density rose from approximately 11\% in 1890 to 25\% by 1910, climbing sharply during both world wars and peaking at roughly 56\% in 1979 before declining to approximately 23\% by 2020. The United States followed a compressed trajectory: negligible density before the 1930s, a sharp rise to approximately 35\% by 1954 following the Wagner Act (1935) and wartime labor agreements, then a long secular decline to approximately 10\% by 2020. Germany's path differed qualitatively: moderate density (approximately 30--35\% through most of the postwar period, declining to approximately 17\% by 2020) was supplemented by institutional codetermination that raised effective voice independently of union membership.

The critical distinction is between union density and collective bargaining coverage. In Anglo-American systems, these track closely---only union members benefit from collective agreements. In continental European systems, coverage dramatically exceeds density through extension mechanisms (erga omnes principles that extend negotiated terms to all workers in a sector). Germany maintains approximately 54\% collective bargaining coverage despite only 17\% union density. France achieves over 90\% coverage with under 10\% density. Sweden and the Nordic states sustained both high density (exceeding 80\% at peak) and near-universal coverage (above 90\%). This divergence is analytically consequential: density measures organizational capacity for friction (strikes require organized workers), while coverage measures the breadth of voice in setting employment terms.

We operationalize $\alpha(d_{\text{workplace}})$ as a composite measure:

\begin{equation}
\alpha(d_{\text{workplace}}, t) = \omega_1 \cdot \frac{\text{worker-elected board seats}}{\text{total board seats}} + \omega_2 \cdot \text{CBC}_t + \omega_3 \cdot \text{density}_t
\label{eq:alpha-workplace}
\end{equation}

\noindent where $\text{CBC}_t$ is collective bargaining coverage as a proportion, $\text{density}_t$ is union density, worker-elected board seats capture formal institutional voice, and $\omega_1 + \omega_2 + \omega_3 = 1$ with weights reflecting relative importance of each channel. Using equal weights ($\omega_i = 1/3$) as a baseline:

\begin{itemize}[nosep]
\item \textbf{Germany c.\ 2020}: $\alpha \approx 0.33 \cdot 0.50 + 0.33 \cdot 0.54 + 0.33 \cdot 0.17 \approx 0.40$
\item \textbf{United States c.\ 2020}: $\alpha \approx 0.33 \cdot 0 + 0.33 \cdot 0.12 + 0.33 \cdot 0.10 \approx 0.07$
\item \textbf{Sweden at peak (c.\ 1980)}: $\alpha \approx 0.33 \cdot 0.33 + 0.33 \cdot 0.90 + 0.33 \cdot 0.80 \approx 0.67$
\item \textbf{UK at peak (1979)}: $\alpha \approx 0.33 \cdot 0 + 0.33 \cdot 0.70 + 0.33 \cdot 0.56 \approx 0.42$
\end{itemize}

The equal-weight specification is deliberately transparent; alternative weightings (privileging institutional representation or coverage) shift absolute values but preserve the cross-national ranking: Nordic $>$ Germany $>$ UK $>$ US.

\begin{table}[htbp]
\centering
\caption{Workplace Consent Alignment Across Industrial Democracies}
\label{tab:labor-alpha}
\begin{tabular}{lccccc}
\toprule
Country & Board Seats (\%) & CBC (\%) & Density (\%) & $\hat{\alpha}$ & Strike Days/1000 \\
\midrule
\multicolumn{6}{l}{\textit{Peak alignment period}} \\
Sweden (1980) & 33 & 90 & 80 & 0.67 & 44 \\
Germany (1980) & 50 & 75 & 35 & 0.53 & 5 \\
UK (1979) & 0 & 70 & 56 & 0.42 & 1,273 \\
US (1954) & 0 & 35 & 35 & 0.23 & 680 \\
\midrule
\multicolumn{6}{l}{\textit{Contemporary (c.\ 2020)}} \\
Sweden & 33 & 90 & 65 & 0.62 & 2 \\
Germany & 50 & 54 & 17 & 0.40 & 4 \\
UK & 0 & 26 & 23 & 0.16 & 6 \\
US & 0 & 12 & 10 & 0.07 & 3 \\
\bottomrule
\end{tabular}
\end{table}

Table~\ref{tab:labor-alpha} reveals several patterns consistent with the framework's predictions. Germany's institutional codetermination produces a higher $\alpha$ than its union density alone would suggest, while the UK's high density in 1979 coexisted with zero board-level representation---a configuration that generated voice through bargaining pressure rather than institutional inclusion. The anomalous UK pattern (moderate $\alpha$ but extremely high friction) is explained by the absence of institutional voice channels: without board representation or works councils, British unions could influence outcomes only through adversarial bargaining and the credible threat of work stoppages, making strikes the \textit{primary} rather than \textit{residual} voice mechanism.

\subsection{Friction Proxy: Strike Frequency and Intensity}
\label{subsec:labor-friction}

We measure $F(d_{\text{workplace}}, t)$ through strike activity: days lost per 1,000 workers, supplemented by qualitative assessment of labor unrest intensity. Strike data provide a natural friction metric because strikes represent the explicit withdrawal of cooperative labor when outcomes deviate sufficiently from worker preferences---precisely the tolerance-weighted friction measure of Equation~\ref{eq:friction-tolerance-operational}. A worker strikes when $\delta(x_d(t), x^*_{i,d}) > \tau_i$: the gap between actual and preferred workplace conditions exceeds their tolerance threshold.

The historical record reveals dramatic variation in strike intensity across time and space. The United States experienced extreme friction during the Gilded Age and Progressive Era: the Great Railroad Strike of 1877 paralyzed two-thirds of the nation's rail network and involved over 100,000 workers across eleven states; the Homestead Strike of 1892 produced armed conflict between strikers and 300 Pinkerton agents, killing ten; and the Pullman Strike of 1894 required federal military intervention with 12,000 troops. These episodes represented friction under conditions of near-zero $\alpha$: workers had no institutional voice, no legal protection for organizing, and no bargaining rights. The framework predicts that such extreme misalignment produces extreme friction---and the American record confirms this with some of the most violent labor conflicts in any industrial democracy.

\citet{fine1969} documents the 1936--1937 General Motors sit-down strike as the pivotal conflict that forced UAW recognition and fundamentally reoriented American labor relations. The sit-down---44 days of factory occupation across multiple Flint plants, resisting both police attacks and court injunctions---demonstrated that direct action could compel institutional change when all other channels were foreclosed. The strike's success raised $\alpha$ discontinuously: GM recognized the UAW, establishing collective bargaining in the American auto industry and catalyzing unionization across heavy manufacturing. Between 1936 and 1945, union membership tripled from approximately 4 million to 14 million---the fastest alpha expansion in American labor history.

British labor friction followed a different temporal pattern. \citet{cole1923} documents how labor in coal mining represented an extreme case of high stakes and near-zero consent: miners faced lethal working conditions (over 1,000 deaths annually in British mines during the early 20th century) with no voice in safety decisions, shift scheduling, or wage determination. The resulting friction was intense and sustained. \citet{hinton1973} traces the shop stewards' movement of 1916--1922 as a grassroots challenge to both employer authority and union bureaucracy---effectively a friction event directed at raising $\alpha$ from below when formal channels proved inadequate. The shop stewards emerged from the Clydeside munitions factories during World War I, where skilled engineering workers faced dilution of craft privileges without consultation, generating friction that threatened wartime production.

The General Strike of 1926---1.7 million workers striking for nine days across transport, printing, heavy industry, and docks---remains the largest coordinated friction event in British labor history. The UK Miners' Strike of 1984--85 represented the last major episode of high-intensity labor friction, lasting 362 days with approximately 142,000 miners striking against pit closures. Its defeat marked the decisive shift from organized to latent friction in the British labor movement: post-1985, strike rates collapsed not because alignment improved but because organizational capacity for collective friction was destroyed.

In Germany, by contrast, strike rates remained remarkably low throughout the postwar period---typically under 10 days lost per 1,000 workers annually, compared to triple-digit figures in the UK and US during their peak-friction decades. This was not because German workers lacked grievances, but because institutional channels (works councils at the plant level, supervisory board representation at the firm level, sectoral bargaining at the industry level) provided effective voice mechanisms that resolved disputes before they escalated to work stoppages. Works councils (\textit{Betriebsr\"ate}), mandated by the 1952 Works Constitution Act and strengthened in 1972, gave employee-elected representatives consultation and co-decision rights on working conditions, scheduling, and social matters---creating a continuous low-friction voice channel that supplemented formal bargaining.

\subsection{Alpha--Friction Dynamics: Germany vs United States vs Nordic}
\label{subsec:labor-dynamics}

Three national trajectories provide natural experiments for testing the framework's core predictions. Each represents a distinct institutional strategy for managing the stakes-voice gap in workplace governance, with correspondingly different friction outcomes.

\paragraph{Germany: High $\alpha$, Low $F$.} The German codetermination system represents the most institutionally complete $\alpha$-raising intervention in any industrial democracy. The 1951 \textit{Montanmitbestimmungsgesetz} (Coal and Steel Codetermination Act) mandated 50\% employee representation on supervisory boards of mining and steel companies---a direct response to the role of unreformed industrial elites in enabling National Socialism. The 1976 \textit{Mitbestimmungsgesetz} extended quasi-parity representation (workers elect half the supervisory board, though the chair---selected by shareholders---holds a tie-breaking vote) to all firms with over 2,000 employees. Smaller firms (500--2,000 employees) received one-third employee representation under the 1952/2004 One-Third Participation Act.

\citet{mcgaughey2016} documents how German codetermination emerged not from revolutionary imposition but from negotiated incorporation during reconstruction periods (1918--1922 and 1945--1951). The Stinnes-Legien Agreement of 1918---struck during revolutionary upheaval---established works councils and the eight-hour day as concessions from capital to labor in exchange for political stability. The postwar settlement extended this logic: the Western Allies and moderate German politicians embedded worker voice to prevent the recurrence of industrial authoritarianism. The framework interprets both episodes as H4-consistent: accumulated friction (revolution, fascism) generated institutional reforms that raised $\alpha$ through negotiated incorporation.

The empirical results strongly support H1. \citet{jaeger2022} provide the most comprehensive causal analysis, exploiting size thresholds in codetermination requirements to identify effects. They demonstrate that codetermined firms exhibit significantly lower strike rates, approximately 10\% longer employee tenure, and wage levels roughly 2\% higher than comparable non-codetermined firms---with no significant negative effect on shareholder returns. \citet{fauver2011good} show that codetermined firms invest more in human capital development, consistent with the framework's prediction that higher $\alpha$ produces outcomes closer to stakeholder preferences: when workers have voice in firm decisions, firms allocate more resources to worker-valued outcomes (training, career development, safety) than shareholder-primacy firms would. \citet{vitols2011coordinated} documents how Germany's coordinated market economy sustained stakeholder orientation even during the financialization pressures of the 1990s and 2000s that transformed Anglo-American corporate governance toward shareholder value maximization.

\citet{bosch2013activating} provides an instructive counter-example within the German system. The Hartz IV welfare reforms of 2003--2005 reduced workers' fallback positions by tightening unemployment benefits and imposing stricter conditionality on the long-term unemployed. This effectively lowered the stakes-weighted voice of displaced and precarious workers without formally altering codetermination structures---reducing $\text{eff\_voice}_i$ through capacity constraints rather than institutional exclusion. The framework predicts that reducing effective voice (even through indirect channels like benefit conditionality) should generate friction, and indeed the reforms produced the largest protest mobilizations in postwar German history: the \textit{Montagsdemonstrationen} (Monday demonstrations) of 2004 drew over 200,000 participants across eastern German cities. That friction within a high-$\alpha$ system could be triggered by \textit{indirect} voice reduction confirms the framework's emphasis on effective rather than formal alignment.

\paragraph{United States: Declining $\alpha$, Episodic $F$.} The American trajectory illustrates the framework's predictions about declining alignment. Union density peaked at approximately 35\% in 1954, then declined steadily through a combination of employer opposition (aggressive anti-union campaigns, permanent replacement of strikers after PATCO 1981), legislative erosion (Taft-Hartley 1947, right-to-work laws spreading from 14 states in 1960 to 27 by 2020), deindustrialization (manufacturing employment declining from 28\% of the workforce in 1960 to under 9\% by 2020), and the rise of service-sector employment structurally resistant to traditional organizing. The absence of institutional codetermination meant that declining density translated directly into declining $\alpha(d_{\text{workplace}})$: from approximately 0.23 at peak to approximately 0.07 by 2020.

The friction pattern is analytically revealing. High-intensity strikes characterized the 1930s and 1940s (sit-down strikes, wildcat stoppages, industry-wide shutdowns: over 4,700 work stoppages in 1946 alone, the peak year). Strike activity declined through the 1950s and 1960s as moderate alignment under the postwar labor-management accord produced tolerable outcomes. It declined further from the 1980s onward---but not because alignment improved. Rather, declining union power meant declining \textit{capacity} for organized friction. The framework distinguishes crucially between friction reduction through alignment improvement (the German path) and friction reduction through voice suppression (the American path). The latter generates latent misalignment that manifests through alternative channels:

\begin{itemize}[nosep]
\item \textit{Individual exit}: The US exhibits the highest labor turnover among advanced economies, with annual voluntary quit rates exceeding 25\% in service sectors.
\item \textit{Passive resistance}: Presenteeism, quality deterioration, and workplace disengagement represent friction expressed individually rather than collectively.
\item \textit{Political mobilization}: Populist movements channeling workplace grievances into electoral politics---from the Tea Party to Trumpism to the Sanders campaigns---reflect friction displaced from foreclosed institutional channels.
\item \textit{Health consequences}: Workplace stress, declining life expectancy in deindustrialized regions, and the ``deaths of despair'' documented by Case and Deaton reflect the human costs of unresolved misalignment.
\end{itemize}

The American case thus illustrates a sobering implication: low measured friction is not equivalent to high alignment. When organizational capacity for collective friction is destroyed, misalignment persists but becomes invisible to strike-based metrics---emerging instead through diffuse, individually borne costs that are harder to measure but no less consequential.

\paragraph{Nordic: Maximum $\alpha$, Minimal $F$.} The Scandinavian model achieved the highest sustained $\alpha(d_{\text{workplace}})$ of any industrial democracy. Swedish union density exceeded 80\% through much of the late 20th century (peaking at approximately 86\% in 1995), collective bargaining coverage reached above 90\%, and board-level employee representation (one-third of seats) was established by the 1972 Board Representation Act. The resulting friction was minimal: Sweden averaged under 50 days lost per 1,000 workers annually during its peak-density period, falling to under 5 by the 2010s.

The Nordic case tests H2 (increasing $\text{Cov}(s_i, C_i)$ reduces friction) most directly. Corporatist bargaining structures---centralized negotiations between peak employer and union federations (SAF and LO in Sweden, NHO and LO in Norway), mediated by state institutions---explicitly linked consent power to stakeholder status. The Rehn-Meidner model (1951) institutionalized solidaristic wage policy: centralized bargaining compressed wages across firms and sectors, ensuring that those with the highest workplace stakes (production workers in export industries) received wages set through their unions' central role in economy-wide negotiations. This produced high $\text{Cov}(s_i, C_i)$ by design---exactly the condition the framework predicts should minimize friction.

Recent pressures from globalization, European integration, and the decline of traditional manufacturing have reduced both density and coverage across the Nordic states. Swedish density fell from 86\% in 1995 to approximately 65\% by 2020; Norwegian density declined from 57\% to 49\% over the same period. The resulting modest increases in labor friction (Sweden experienced notable public sector strikes in 2003 and 2008) test whether the model's friction-reducing properties survive institutional erosion---or whether, as the framework predicts, declining $\alpha$ eventually produces rising $F$.

\subsection{Cross-Case Connections}
\label{subsec:labor-connections}

Labor rights connect to multiple other domains examined in this monograph. The franchise expansion analyzed in Section~\ref{sec:suffrage} was driven partly by working-class demands: the Chartist movement (1838--1857) explicitly linked political voice to economic justice, and the Second and Third Reform Acts (1867, 1884) that extended the British franchise were precipitated in part by labor mobilization. The civil rights movement (Section~\ref{sec:civil-rights}) strategically integrated labor demands---the 1963 March on Washington was formally the ``March on Washington for Jobs and Freedom,'' embedding economic stakes within the political franchise campaign, and A.\ Philip Randolph (president of the Brotherhood of Sleeping Car Porters) served as the march's lead organizer.

Corporate governance structures (Section~\ref{sec:corporate-governance}) represent the institutional complement to labor mobilization: where this section examines how workers raised $\alpha$ through collective action and the resulting friction dynamics, Section~\ref{sec:corporate-governance} examines how institutional design can embed voice structurally within firm governance. The German codetermination system bridges both---it is simultaneously a labor achievement (won through decades of mobilization and negotiation) and a corporate governance institution (mandated by statute and embedded in company law).

The labor case also demonstrates the framework's distinction between formal and effective voice. Even where collective bargaining rights are legally established, effective voice requires organizational capacity, information access, and credible exit threats. \citet{farzulla2025stakes} develops this distinction theoretically, showing how populations can hold formal consent power $C_i > 0$ yet exercise near-zero effective voice due to structural constraints on capacity. The decline of American unions illustrates this dynamic: workers retain formal rights to organize under the NLRA (1935), but employer opposition, regulatory erosion, and structural economic change have reduced effective voice far below what formal entitlements suggest---producing a formal-effective gap analogous to (though less extreme than) the Jim Crow gap analyzed in Section~\ref{sec:civil-rights}.


\section{Civil Rights (1950s--Present)}
\label{sec:civil-rights}

The American civil rights movement provides the framework's most powerful illustration of the gap between formal consent power and effective voice. For nearly a century after the Fourteenth and Fifteenth Amendments nominally established equal citizenship and voting rights (1868--1870), Black Americans in the South possessed constitutional $C_i > 0$ yet exercised near-zero $\text{eff\_voice}_i$ due to systematic suppression through violence, legal manipulation, and economic coercion. This case demonstrates that $\alpha(d)$ must be measured through effective participation rather than formal entitlements---and that the resulting misalignment produces exactly the sustained, escalating friction the framework predicts.

\subsection{Domain Definition}
\label{subsec:civil-rights-domain}

We define $d_{\text{civil\_rights}}$ as the set of decisions affecting political participation, legal protection, economic opportunity, and social inclusion for racial minorities. While the analysis focuses on the American case, parallel dynamics operated in South Africa (apartheid regime, 1948--1994), Northern Ireland (sectarian exclusion, 1920s--1998), and colonial contexts globally. The affected set $S_{d_{\text{civil\_rights}}}$ includes:

\begin{itemize}[nosep]
\item \textbf{Racial minorities} (existential $s_i$): legal status, physical safety, economic access, educational opportunity, housing, and fundamental dignity all contingent on civil rights policy. The stakes are existential in the literal sense: Black life expectancy in Mississippi in 1950 was approximately 20 years shorter than white life expectancy.
\item \textbf{Majority population} (moderate $s_i$): stakes in social order, moral standing, and economic costs of exclusion. The moral stakes are understated by purely material measures---as abolitionists demonstrated (Section~\ref{sec:abolition}), sympathetic observers can hold genuine moral stakes in others' exclusion.
\item \textbf{State institutions} (institutional $s_i$): stakes in domestic legitimacy, international standing (Cold War competition made racial exclusion a foreign policy liability), and constitutional coherence.
\end{itemize}

Under Jim Crow, the consent-holder mapping concentrated authority overwhelmingly in white political elites: $C_{\text{Black}} \approx 0$ across political, legal, economic, and social domains despite $s_{\text{Black}}(d) \approx \text{existential}$. This represented one of the most extreme and sustained instances of stakes-voice misalignment in modern democratic history---comparable in structural terms to the abolition case analyzed in Section~\ref{sec:abolition}, though operating through legal manipulation rather than explicit chattel ownership.

\subsection{Alpha Proxy: Voter Registration and Effective Participation}
\label{subsec:civil-rights-alpha}

The civil rights case demands a distinction between formal and effective alpha that no other case study illustrates as starkly. The Fifteenth Amendment (1870) nominally granted Black men voting rights: $C_{\text{Black, formal}} > 0$. Yet Jim Crow mechanisms---literacy tests (often applied selectively: white applicants passed by reciting their names, Black applicants failed for misplacing a comma), poll taxes (the equivalent of one to two days' wages for sharecroppers), grandfather clauses, white-only primaries, and lethal violence against those who attempted to register---reduced effective voice to near zero. Between 1890 and 1910, Mississippi, South Carolina, Louisiana, Alabama, Virginia, Georgia, and North Carolina all rewrote their state constitutions specifically to disenfranchise Black voters while technically complying with the Fifteenth Amendment.

Voter registration data quantify this formal-effective gap with precision. In Mississippi, Black voter registration stood at approximately 5.2\% in 1960---not because 95\% of eligible Black citizens chose not to vote, but because the act of registration invited economic retaliation (eviction from tenant farms, firing from employment), physical violence (beatings, bombings, murder), and institutional obstruction (registrars who opened offices irregularly, required impossible constitutional interpretation tests, or simply refused applications). Across the Deep South, Black registration rates in 1940 were approximately 3\%. Even in the Upper South (Virginia, North Carolina, Tennessee), rates remained below 20\%.

We construct $\alpha(d_{\text{civil\_rights}}, t)$ using effective participation measures:

\begin{equation}
\alpha(d_{\text{civil\_rights}}, t) = \frac{s_{\text{Black}} \cdot \text{eff\_voice}_{\text{Black}}(t) + s_{\text{other}} \cdot \text{eff\_voice}_{\text{other}}(t)}{s_{\text{Black}} + s_{\text{other}}}
\label{eq:alpha-civil-rights}
\end{equation}

\noindent where $\text{eff\_voice}_{\text{Black}}(t)$ reflects actual voter registration rates, the presence of elected Black officials, and access to legal protections---not constitutional text. Under this measure, $\alpha(d_{\text{civil\_rights}})$ in the Deep South circa 1960 was close to zero despite nearly a century of formal constitutional guarantees. This is the framework's formal-effective gap in its starkest empirical form.

The Voting Rights Act (VRA) of 1965 produced the most dramatic single-year increase in $\alpha$ documented anywhere in this monograph. Federal registrars were dispatched to counties where less than 50\% of eligible voters were registered, literacy tests were suspended in covered jurisdictions, and Section 5 required preclearance of any changes to voting procedures. The results were immediate, quantifiable, and transformative:

\begin{table}[htbp]
\centering
\caption{Black Voter Registration in the Deep South, Pre- and Post-VRA}
\label{tab:vra-registration}
\begin{tabular}{lcccc}
\toprule
State & 1960 (\%) & 1966 (\%) & 1970 (\%) & $\Delta$ (pp) \\
\midrule
Mississippi & 5.2 & 32.9 & 71.0 & +65.8 \\
Alabama & 13.7 & 51.2 & 66.0 & +52.3 \\
South Carolina & 15.6 & 51.4 & 56.1 & +40.5 \\
Georgia & 29.3 & 47.2 & 57.5 & +28.2 \\
Louisiana & 31.6 & 47.1 & 57.4 & +25.8 \\
Virginia & 23.0 & 46.9 & 57.0 & +34.0 \\
\midrule
South (total) & $\sim$29 & $\sim$52 & $\sim$62 & +33 \\
\bottomrule
\end{tabular}
\end{table}

Black elected officials provide a complementary alpha measure. Between 1901 and 1972, zero Black members of Congress represented Southern states---a seven-decade absence reflecting the complete exclusion of Black political voice from representative institutions. The number of Black elected officials nationally rose from approximately 1,500 in 1970 to over 4,900 by 1980 and over 10,500 by 2002, though this remained substantially below proportional representation (approximately 2\% of elected officials for 13\% of the population). The election of Black mayors in major Southern cities---Atlanta (Maynard Jackson, 1973), New Orleans (Ernest Morial, 1978), Birmingham (Richard Arrington, 1979)---represented qualitative threshold crossings in local $\alpha$.

This case is \textit{the} demonstration of why $\text{eff\_voice}$ matters in the consent-holding framework. Nominal consent power without capacity to exercise it---without physical safety, economic independence, educational access, or institutional support---produces a measurement artifact that conceals ongoing misalignment. As \citet{farzulla2025stakes} argues, populations with stakes but no effective voice exist in a condition of structural exclusion that formal rights alone cannot resolve. The civil rights case forces any legitimacy framework to distinguish between de jure and de facto consent power---a distinction the $\text{eff\_voice}_i$ specification is designed to capture.

\subsection{Friction Proxy: Protest Events and Litigation}
\label{subsec:civil-rights-friction}

Friction in the civil rights domain manifested through multiple channels, each representing the stakes-weighted deviation between realized outcomes (segregation, disenfranchisement, economic exclusion) and stakeholder preferences (equal citizenship, political participation, economic opportunity). The movement deployed an escalating repertoire of friction tactics---from litigation to direct action to mass mobilization---that the framework interprets as progressive intensification in response to institutional resistance to alpha improvement.

\textit{Legal friction} came first. The NAACP's litigation strategy, developed under Charles Hamilton Houston and Thurgood Marshall, systematically challenged segregation through the courts. \textit{Brown v.\ Board of Education} (1954) ruled school segregation unconstitutional---a formal alpha intervention in the educational domain. However, ``massive resistance'' by Southern states (school closures, interposition resolutions, the Southern Manifesto signed by 101 congressmen) demonstrated that legal alpha increases without enforcement capacity produce phantom alignment. By 1964, only 1.2\% of Black students in the Deep South attended desegregated schools---a decade after \textit{Brown}.

\textit{Direct action} escalated friction. \citet{parks1992} provides a firsthand account of the Montgomery Bus Boycott (1955--1956)---381 days of sustained economic friction targeting segregated public transit. The boycott reduced Montgomery bus revenue by approximately 65\% and demonstrated that coordinated withdrawal of cooperation could impose costs sufficient to force institutional change. \citet{branch1988} situates Montgomery within the broader arc of the King years, documenting how the movement deployed escalating friction tactics calibrated to expose the violence required to maintain low-$\alpha$ arrangements: the strategic genius of the movement was to make the \textit{costs of suppression} visible rather than merely the \textit{grievances of the excluded}.

The sit-in movement of 1960 expanded friction geographically and generationally: within two months of the Greensboro sit-in (four Black college students occupying a segregated Woolworth's lunch counter), over 70,000 participants conducted sit-ins across more than 100 cities in the South, targeting segregated lunch counters, libraries, theaters, and public facilities. The Freedom Rides of 1961 involved over 450 riders testing desegregation of interstate bus facilities, provoking violent reactions (buses firebombed in Anniston, riders beaten in Birmingham and Montgomery) that generated national and international media attention. The 1963 March on Washington drew approximately 250,000 participants---the largest political demonstration in American history at that time---where King's ``I Have a Dream'' speech articulated the stakes-voice gap in language that transcended political science: ``America has given the Negro people a bad check, a check which has come back marked `insufficient funds.'\,''

\citet{king1963} \textit{Letter from Birmingham Jail} provides the most rigorous theoretical articulation of friction as a political strategy within the movement itself. King's argument---that ``nonviolent direct action seeks to create such a crisis and foster such a tension that a community which has constantly refused to negotiate is forced to confront the issue''---maps directly onto H4's prediction that sustained friction generates institutional reform pressure. His distinction between ``negative peace which is the absence of tension'' and ``positive peace which is the presence of justice'' anticipates the framework's distinction between friction reduction through voice suppression and friction reduction through alignment improvement. King explicitly identified low $\alpha$ as the problem: ``We know through painful experience that freedom is never voluntarily given by the oppressor; it must be demanded by the oppressed.''

\citet{levy1998} documents the broader movement history, tracing how friction escalated from legal strategies (NAACP litigation) through direct action (sit-ins, Freedom Rides, marches) to urban uprisings (Watts 1965: 34 dead, \$40 million in property damage; Detroit 1967: 43 dead, 7,231 arrested; post-assassination riots in over 100 cities in April 1968). This escalation follows the framework's prediction precisely: when lower-intensity friction fails to produce alpha increases (Southern resistance to \textit{Brown} was sustained for a decade), friction intensifies through higher-cost repertoires until either alignment improves or the system reaches a breaking point. The urban uprisings represented the highest-cost friction repertoire, emerging in Northern and Western cities where \textit{de facto} segregation, police brutality, and economic exclusion persisted despite formal legal equality.

\subsection{Alpha--Friction Dynamics}
\label{subsec:civil-rights-dynamics}

The civil rights case illuminates four key framework predictions with unusual clarity.

\paragraph{H1: The Formal-Effective Alpha Gap.} The century between the Fifteenth Amendment (1870) and the VRA (1965) demonstrates that formal $\alpha > 0$ is insufficient to reduce friction when effective voice remains near zero. Constitutional promises without enforcement mechanisms, institutional support, or protection from retaliation generate what we term \textit{phantom alignment}---the appearance of consent without its substance. The framework's $\text{eff\_voice}$ specification captures this: $\alpha$ must be computed from effective participation, not paper entitlements. During this period, formal $\alpha$ was moderate (Black men had constitutional voting rights), but effective $\alpha \approx 0$, and friction was high and sustained---exactly as H1 predicts for low effective alignment. The implication is methodological: any empirical operationalization of the framework must distinguish between formal and effective measures, or risk systematic measurement error that understates misalignment and overpredicts stability.

\paragraph{H3: Threshold Effects and Critical Junctures.} \textit{Brown v.\ Board of Education} (1954) represents a critical juncture \citep{capoccia2007study} where judicial intervention crossed a threshold in the legal domain. The decision did not immediately raise $\alpha$ (implementation was resisted through ``massive resistance'' for over a decade), but it shifted the legitimacy baseline: segregation moved from legally sanctioned to constitutionally proscribed. This threshold crossing catalyzed friction escalation---the Montgomery Boycott began the following year, the sit-in movement erupted six years later, and the movement built momentum continuously through 1965. The framework interprets this as a threshold effect: once the legal system signaled that exclusion was illegitimate, the tolerance threshold $\tau_i$ for accepting low $\alpha$ dropped sharply among affected populations, producing discontinuously higher mobilization even though \textit{actual} alignment had not yet changed. Formal legitimacy signals can thus lower friction thresholds and \textit{increase} observed friction in the short run---a counterintuitive prediction that the civil rights timeline confirms.

\paragraph{H4: Friction Generates Alpha.} The sustained friction of 1955--1965 produced the most significant alpha increases in American history: the Civil Rights Act (1964, prohibiting discrimination in employment and public accommodations), the Voting Rights Act (1965, enforcing the Fifteenth Amendment through federal oversight), and the Fair Housing Act (1968, prohibiting discrimination in housing). Each represented an institutional response to accumulated friction that raised effective Black voice in specific domains. The VRA in particular demonstrates H4's prediction that persistent high $F$ generates future alpha increases: a decade of direct action, media attention, televised police violence (Birmingham 1963, Selma 1965), and political pressure culminated in federal enforcement of voting rights that transformed Southern politics within three years (Table~\ref{tab:vra-registration}). The Selma-to-Montgomery marches of March 1965---and particularly the ``Bloody Sunday'' attack on marchers at the Edmund Pettus Bridge, broadcast nationally---provided the proximate friction event that generated the political will for the VRA's passage.

\paragraph{Domain-Specific Alpha Persistence.} The civil rights case also reveals a feature the framework handles naturally: alpha increases are domain-specific and do not automatically transfer across domains. The VRA raised $\alpha$ dramatically in the \textit{political} domain, but left $\alpha$ low in:

\begin{itemize}[nosep]
\item \textit{Criminal justice}: Mass incarceration (Black incarceration rates 5--7 times white rates through the 2000s), policing practices, and sentencing disparities reflect governance domains where Black Americans have minimal effective voice despite formal political rights.
\item \textit{Wealth accumulation}: The racial wealth gap remained approximately 10:1 (median white household wealth to median Black household wealth) through 2020, reflecting centuries of exclusion from wealth-building mechanisms (homeownership, education, inheritance) that political voice alone did not remedy.
\item \textit{Educational quality}: Despite \textit{Brown}, school segregation by race and class persisted through residential patterns and school funding mechanisms tied to local property taxes.
\end{itemize}

Persistent friction in these domains---the Black Lives Matter movement (2013--present), protests against police violence (Ferguson 2014, Baltimore 2015, Minneapolis 2020), educational equity campaigns---reflects continued low $\alpha$ in governance domains where formal political rights did not translate into effective voice over policy outcomes. The framework predicts exactly this: raising $\alpha$ in one domain does not automatically raise it in others, and domains with persistently low effective alignment will continue to generate friction until domain-specific alpha improvements occur.

\subsection{Cross-Case Connections}
\label{subsec:civil-rights-connections}

The civil rights movement built directly on abolition's unfinished alpha expansion (Section~\ref{sec:abolition}). The Reconstruction amendments (1865--1870) nominally resolved abolition's zero-consent condition, but the withdrawal of federal enforcement after the Compromise of 1877 allowed Southern states to reimpose near-zero effective voice through Jim Crow---a case of formal alpha expansion followed by effective alpha collapse. The civil rights movement can thus be understood as completing the alpha expansion that abolition began but Reconstruction failed to sustain.

The connection to labor rights (Section~\ref{sec:labor}) was explicit and strategic. The 1963 March on Washington was organized jointly by civil rights and labor leaders, and its full title---``March on Washington for Jobs and Freedom''---embedded economic stakes within the political franchise campaign. A.\ Philip Randolph, the march's lead organizer, was president of the Brotherhood of Sleeping Car Porters, and the AFL-CIO provided substantial organizational and financial support. This cross-domain mobilization illustrates how friction in one domain (workplace exclusion) can amplify friction in another (political disenfranchisement) when affected populations overlap---a cross-domain friction resonance effect that the framework's multi-domain structure is designed to capture.

The civil rights movement also served as a strategic template for subsequent campaigns. The LGBT rights movement (Section~\ref{sec:lgbt}) adapted civil rights litigation strategies (constitutional equal protection challenges), direct action tactics (pride marches modeled on civil rights demonstrations), and narrative frameworks (coming-out as analogous to visibility politics). Suffrage movements (Section~\ref{sec:suffrage}) and civil rights movements influenced each other bidirectionally: women of color experienced compounded exclusion across both gender and racial domains, and the intersectional character of their stakes drove distinctive organizing strategies that neither movement alone fully addressed.

\section{LGBT Inclusion (1969--Present)}
\label{sec:lgbt}

The struggle for LGBT legal recognition constitutes one of the most compressed consent alignment trajectories in modern history. Within a single lifetime, multiple Western democracies moved from criminalizing homosexuality to constitutionally protecting same-sex marriage---a shift from $\alpha(d_{\text{lgbt}}) \approx 0$ to $\alpha \approx 1.0$ across an expanding set of legal recognition domains. This velocity of consent restructuring, and the distinctive dynamics that enabled it, provides rich material for testing the framework's predictions about threshold effects, friction accumulation, and international norm diffusion.

\subsection{Domain Definition}
\label{subsec:lgbt-domain}

The governance domain $d_{\text{lgbt}}$ encompasses decisions affecting the legal recognition, anti-discrimination protection, relationship status, and social inclusion of lesbian, gay, bisexual, and transgender persons. Formally:
\[
d_{\text{lgbt}} = \{d_{\text{crim}}, d_{\text{discrim}}, d_{\text{relationship}}, d_{\text{identity}}\}
\]
where $d_{\text{crim}}$ covers criminal law regarding consensual sexual conduct, $d_{\text{discrim}}$ covers employment, housing, and public accommodation protections, $d_{\text{relationship}}$ covers marriage and civil union recognition, and $d_{\text{identity}}$ covers gender identity recognition and transition-related healthcare access.

Stakeholders divide along a clear stakes asymmetry. LGBT individuals hold existential stakes across all sub-domains: $s_{\text{lgbt}}(d_{\text{crim}})$ includes physical liberty and safety from prosecution; $s_{\text{lgbt}}(d_{\text{relationship}})$ includes legal recognition of intimate partnerships affecting inheritance, medical decision-making, immigration, and parental rights. These are not policy preferences but conditions of basic social existence. The broader population holds moderate stakes in social norm stability and legal framework coherence---real but categorically lower than the existential exposure of the directly affected population.

Prior to the mid-twentieth century, LGBT persons held effectively zero consent power across all sub-domains. Criminalization meant that visibility itself carried legal risk, creating a structural suppression of friction: the population most affected could not safely organize to contest its exclusion. This feature---where zero alpha suppresses friction visibility rather than merely generating it---distinguishes the LGBT case from suffrage or labor movements where the excluded population could at least visibly mobilize.

\subsection{Alpha Proxy: Legal Recognition Index}
\label{subsec:lgbt-alpha}

We construct an ordinal alpha scale mapping legal status to consent alignment, ranging from complete exclusion to comprehensive inclusion:

\begin{table}[htbp]
\centering
\caption{Ordinal Alpha Scale for LGBT Legal Recognition}
\label{tab:lgbt-alpha}
\begin{tabular}{@{}lll@{}}
\toprule
$\alpha$ Value & Legal Status & Proxy Indicators \\
\midrule
0.00 & Active criminalization & Sodomy laws, police enforcement, persecution \\
0.25 & Decriminalization & Removal of criminal penalties for consensual conduct \\
0.50 & Anti-discrimination protections & Employment, housing, public accommodation \\
0.75 & Civil unions / domestic partnerships & Legal recognition short of marriage \\
1.00 & Full marriage equality + protections & Marriage equality, comprehensive anti-discrimination \\
\bottomrule
\end{tabular}
\end{table}

Tracking this index across jurisdictions reveals markedly different trajectories:

\textbf{United States.} Alpha remained at 0.00 throughout most of the twentieth century, with sodomy laws enforced in all states as late as 1961. Illinois became the first state to decriminalize in 1962 ($\alpha \to 0.25$ locally). The trajectory stalled for decades. \textit{Lawrence v.\ Texas} (2003) decriminalized nationally ($\alpha \to 0.25$). Massachusetts legalized same-sex marriage in 2004 ($\alpha \to 1.0$ locally), and \textit{Obergefell v.\ Hodges} (2015) established national marriage equality ($\alpha \to 1.0$). The jump from $\alpha = 0.25$ to $\alpha = 1.0$ took just twelve years at the national level---an extraordinary compression.

\textbf{United Kingdom.} Partial decriminalization in England and Wales in 1967 ($\alpha \to 0.25$), with Scotland following in 1980 and Northern Ireland in 1982. The Equality Act 2010 consolidated anti-discrimination protections ($\alpha \to 0.50$). Civil partnerships were established in 2004 ($\alpha \to 0.75$), and the Marriage (Same Sex Couples) Act 2013 brought marriage equality ($\alpha \to 1.0$). The trajectory from decriminalization to full recognition took 46 years---roughly one generation slower than the US compression.

\textbf{Netherlands.} The earliest achiever of full recognition: decriminalization in 1811 (Napoleonic Code adoption), anti-discrimination law in 1992, registered partnerships in 1998, and full marriage equality in 2001---the world's first. The Dutch trajectory demonstrates that early decriminalization does not automatically accelerate later stages; the 190-year gap between decriminalization and marriage equality dwarfs the American compression.

\textbf{Global variation.} As of 2025, approximately 70 countries still criminalize homosexuality ($\alpha = 0.00$), while 35 recognize same-sex marriage ($\alpha \approx 1.0$). Pew Research Center surveys document that public acceptance of homosexuality correlates with but does not determine legal alpha---South Africa constitutionally protects same-sex marriage despite lower public acceptance than several countries lacking legal recognition, demonstrating that institutional pathways mediate between social attitudes and consent alignment.

The global distribution of alpha values reveals regional clustering that supports the international diffusion hypothesis. Western Europe and the Americas show the highest average alpha (most countries at 0.75--1.0 by 2025), followed by parts of East Asia (Taiwan recognizing marriage equality in 2019), with Sub-Saharan Africa, the Middle East, and Central Asia remaining predominantly at $\alpha = 0.00$--$0.25$. This regional pattern mirrors suffrage diffusion, where adoption cascaded through culturally proximate states before crossing regional boundaries. The persistence of criminalization in former British colonies---a colonial legal legacy---demonstrates how initial institutional conditions shape long-run alpha trajectories even when underlying social attitudes evolve, consistent with \citet{acemoglu2012why} on the persistence of colonial institutional legacies.

\textbf{Alpha trajectory visualization.} The ordinal alpha trajectories for the United States, United Kingdom, and Netherlands can be represented as step functions with identifiable transition dates:

\begin{table}[htbp]
\centering
\caption{LGBT Alpha Trajectories: Three National Cases}
\label{tab:lgbt-trajectories}
\small
\begin{tabular}{@{}llll@{}}
\toprule
Period & United States & United Kingdom & Netherlands \\
\midrule
Pre-1960 & 0.00 & 0.00 & 0.25 (since 1811) \\
1960s & 0.00 $\to$ 0.25 (IL 1962) & 0.25 (E\&W 1967) & 0.25 \\
1970s--1980s & 0.25 (local only) & 0.25 & 0.25 \\
1990s & 0.25 & 0.25 & 0.25 $\to$ 0.50 (1992) \\
2000--2005 & 0.25 (Lawrence 2003) & 0.50 $\to$ 0.75 (2004) & 0.75 $\to$ 1.0 (2001) \\
2005--2010 & 0.25 $\to$ 0.50 (local) & 0.75 & 1.0 \\
2010--2015 & 0.50 $\to$ 1.0 (2015) & 0.75 $\to$ 1.0 (2013) & 1.0 \\
2015--present & 1.0 (with backlash in $d_{\text{identity}}$) & 1.0 & 1.0 \\
\bottomrule
\end{tabular}
\end{table}

The table reveals that alpha trajectories are not monotonically increasing even within successful cases. The US experienced local reversals (Proposition 8 in 2008, DOMA in 1996) that temporarily reduced alpha in specific jurisdictions or sub-domains before being judicially overturned. These reversals represent counter-friction from populations perceiving their own stakes as threatened by alpha advances---a dynamic the framework must accommodate alongside the primary friction-to-alpha pathway.

\subsection{Friction Proxy: Pride, Litigation, and Political Mobilization}
\label{subsec:lgbt-friction}

Friction in the LGBT domain manifests through three principal channels: direct action and protest, litigation, and counter-mobilization.

\textbf{Stonewall and direct action.} The Stonewall Riots of June 1969 represent the paradigmatic friction eruption: a population with existential stakes ($s \to \max$) and zero consent power ($C = 0$) responding to routine police raids with spontaneous resistance. Stonewall did not create the homophile movement---the Mattachine Society (1950) and Daughters of Bilitis (1955) preceded it---but it transformed friction from individualized to collective. \citet{stekelenburg2010social} identify precisely this transition as critical: the shift from grievance (which can persist indefinitely without collective expression) to collective identity and group efficacy, which convert latent into manifest friction.

The subsequent growth of Pride marches provides a quantifiable friction series: the first Christopher Street Liberation Day (1970) drew a few thousand participants in New York. By the 1990s, annual Pride events mobilized millions globally. Participation growth serves as a proxy for friction intensity---not because Pride events are primarily protest (they evolved into celebration), but because their scale reflects the organized visibility of a previously invisible population.

\textbf{Litigation as institutional friction.} Lambda Legal (founded 1973) and the ACLU LGBT Rights Project systematically channeled friction through the judicial system.
Major cases constitute friction events with measurable institutional impact:
\begin{itemize}
    \item \textit{Bowers v.\ Hardwick} (1986): Friction absorbed---the Supreme Court upheld Georgia's sodomy law, ruling 5--4 that the Constitution did not protect homosexual conduct. In consent-holding terms, the institutional system absorbed the friction input without alpha change.
    \item \textit{Romer v.\ Evans} (1996): Friction partially converted---the Court struck down Colorado's Amendment 2, which had prohibited anti-discrimination protections for LGBT people. This raised alpha by removing a structural barrier to local alpha improvements.
    \item \textit{Lawrence v.\ Texas} (2003): Friction converted---the Court struck down remaining sodomy laws nationwide, explicitly overruling \textit{Bowers}. National alpha jumped from 0.00 to 0.25 in a single decision.
    \item \textit{United States v.\ Windsor} (2013): Friction converted---the Court struck down Section 3 of the Defense of Marriage Act, requiring federal recognition of state-level same-sex marriages.
    \item \textit{Obergefell v.\ Hodges} (2015): Friction fully converted---the Court established marriage equality as a constitutional right. National alpha jumped to 1.0 in $d_{\text{relationship}}$.
\end{itemize}
This litigation trajectory demonstrates Hypothesis 4 in action: persistent friction generating incremental alpha increases through institutional reform pressure.
The sequence also illustrates a ratchet mechanism:
each successful case created legal precedent that made subsequent challenges more likely to succeed,
generating a positive feedback loop between friction expression and alpha expansion.
Failed cases (\textit{Bowers}) did not permanently block the pathway but rather
delayed it while shifting legal strategy---Lambda Legal's response to \textit{Bowers}
was to pursue state-level litigation and build a stronger evidentiary record,
producing the factual foundation for \textit{Lawrence} seventeen years later.

\textbf{Counter-mobilization and backlash.} Alpha advances trigger counter-friction from populations perceiving stake losses from inclusion. California's Proposition 8 (2008)---a ballot measure overturning judicially established marriage equality---represents friction generated by alpha advancement. The measure passed with 52\% support, temporarily reducing $\alpha(d_{\text{relationship}})$ from 1.0 to 0.75 in California before federal courts struck it down. Anti-trans legislation in the 2020s (bathroom bills, sports participation restrictions, healthcare access limitations) demonstrates that backlash friction can target sub-domains ($d_{\text{identity}}$) even as others ($d_{\text{relationship}}$) stabilize at high alpha. \citet{tarrow1998power} identifies this pattern as characteristic of contentious politics: movements and counter-movements escalate in response to each other's gains, creating friction spirals that test institutional capacity to absorb competing claims.

\subsection{Alpha-Friction Dynamics}
\label{subsec:lgbt-dynamics}

The LGBT case illuminates several framework predictions with particular clarity:

\textbf{Suppressed friction under extreme exclusion.} Pre-Stonewall, $\alpha \approx 0$ coexisted with apparently low friction---but this reflected suppression, not satisfaction. Criminalization raised the personal cost of friction expression to the point where $F_{\text{observed}} \ll F_{\text{latent}}$. The framework must therefore distinguish between observed friction (protest events, litigation, mobilization) and latent friction (suppressed grievance weighted by stakes). When suppression mechanisms weaken---in this case, urbanization creating anonymous spaces, the sexual revolution reducing general sexual taboo, police overreach provoking a tipping point---latent friction converts to manifest friction suddenly. This is Hypothesis 3's threshold effect operating through a suppression mechanism: the threshold is not in alpha itself but in the cost-of-friction-expression, which, once crossed, releases accumulated pressure.

The distinction between observed and latent friction has broad implications for the framework's empirical application. In any domain where expressing friction is penalized (whistleblower retaliation, protest criminalization, social ostracism for dissent), observed friction will underestimate the true alignment deficit. The pre-Stonewall period suggests a diagnostic: when observed friction is low but structural alpha is also low, the gap likely reflects suppression rather than satisfaction. Policy implications follow: reducing the cost of friction expression (anti-retaliation protections, protest rights, anonymous complaint mechanisms) reveals rather than creates misalignment.

\textbf{Political opportunity structures and movement timing.} \citet{tarrow1998power} argues that social movement success depends on political opportunity structures---configurations of resources, institutional access, and elite alignments that create openings for collective action. The LGBT rights trajectory confirms this analysis. The post-Stonewall movement emerged during a period of expanding civil liberties jurisprudence, declining religious institutional authority, and increasing cultural pluralism in Western democracies. These structural conditions lowered the cost of friction expression and raised the probability that friction would convert to alpha increases. The framework can incorporate political opportunity structures as mediating variables between friction and alpha response: $\frac{\partial \alpha}{\partial F}$ is not constant but depends on institutional context.

The timing of breakthrough moments further supports this analysis. The Netherlands' early adoption of marriage equality (2001) occurred in a political context of coalition governance, strong LGBTQ advocacy organizations, and a cultural tradition of pragmatic tolerance (\textit{gedoogbeleid}). The US breakthrough came through judicial rather than legislative channels---reflecting the structure of American federalism and the strategic litigation pathway pioneered by civil rights movements. These institutional variations explain why the same level of friction produced different alpha trajectories in different political contexts.

\textbf{The S-curve pattern.} Alpha trajectories in the LGBT domain follow a logistic curve: long dormancy at $\alpha \approx 0$ (pre-1960s), slow initial increase (decriminalization 1960s--1980s), rapid acceleration (1990s--2010s), and stabilization near $\alpha = 1.0$. This S-curve maps onto \citet{granovetter1978threshold} threshold models of collective behavior: as each jurisdiction extends recognition, it lowers the threshold for others by providing demonstration effects and legal precedent. International diffusion accelerates the steep portion of the S-curve, just as \citet{ramirez1997} documented for women's suffrage.

\textbf{The formal-effective alpha gap.} Marriage equality ($\alpha_{\text{formal}} = 1.0$) does not eliminate friction because formal legal recognition differs from effective social inclusion. Workplace discrimination persists in jurisdictions with legal protections (enforcement gaps), hate crimes continue (the gap between law and lived experience), and social stigma creates capability constraints on effective voice even where formal rights exist. This gap between formal and effective alpha---visible also in post-abolition racial inequality and post-suffrage gender gaps---suggests that the framework should track both dimensions, with $\alpha_{\text{effective}} < \alpha_{\text{formal}}$ as a general condition and the gap itself as a predictor of residual friction.

Quantifying the formal-effective gap reveals its persistence. In the United States, despite marriage equality since 2015, the Williams Institute estimates that 46\% of LGBT workers remain closeted at work, and 29 states lacked comprehensive employment non-discrimination protections until the \textit{Bostock v.\ Clayton County} (2020) ruling extended Title VII coverage. Anti-transgender violence has increased even as legal recognition expanded, with the Human Rights Campaign documenting record fatalities in several recent years. The framework interprets these patterns through the effective voice function: $\text{eff\_voice}_i = f(C_{i,d}, \text{capacity}_i)$, where capacity includes not only formal rights but safety, social acceptance, and institutional access. When capacity constraints depress effective voice, $\alpha_{\text{effective}}$ remains below $\alpha_{\text{formal}}$, and the residual gap predicts continued friction---which is precisely what is observed in the form of ongoing advocacy, litigation, and protest focused on implementation rather than formal rights.

\textbf{Domain fragmentation and backlash dynamics.} The LGBT case reveals that alpha can advance unevenly across sub-domains, creating complex friction patterns. Marriage equality ($d_{\text{relationship}}$) reached $\alpha \approx 1.0$ in much of the West by 2020, but gender identity recognition ($d_{\text{identity}}$) has faced intensifying backlash. Anti-trans legislation in the United States accelerated from a handful of bills annually before 2020 to over 500 introduced in 2023 alone, targeting healthcare access, sports participation, and bathroom use. This backlash represents counter-friction generated by alpha advances in adjacent domains: opponents who lost the marriage equality debate redirected friction toward a sub-domain where alpha was lower and the affected population smaller and more vulnerable.

The framework predicts this pattern: when alpha advances in one sub-domain trigger reallocation of opposition resources to another, the aggregate alpha trajectory becomes non-monotonic. Total domain alpha $\alpha(d_{\text{lgbt}})$ may increase on net while $\alpha(d_{\text{identity}})$ decreases, creating a compositional effect that masks localized regression. Tracking alpha at the sub-domain level is therefore essential for accurate friction prediction.

\textbf{Hypothesis testing.} The LGBT case provides evidence for:
\begin{itemize}
    \item \textbf{H1} (alpha-friction inverse): Strongly supported. Jurisdictions with higher legal recognition indices exhibit lower friction (fewer protests, less litigation), controlling for other factors.
    \item \textbf{H3} (threshold effects): Stonewall operates as a critical juncture \citep{capoccia2007study}---a moment where contingent events trigger path-dependent trajectory change. Pre-Stonewall friction was suppressed; post-Stonewall, it became self-reinforcing.
    \item \textbf{H4} (friction predicts future alpha): The strongest case in the portfolio. Decades of sustained friction (1969--2015) generated incremental and then accelerating alpha increases through a combination of litigation, legislation, and judicial interpretation.
\end{itemize}

\subsection{Cross-Case Connections}
\label{subsec:lgbt-cross}

The LGBT rights trajectory builds directly on infrastructure developed during the civil rights movement. Legal strategies pioneered by the NAACP Legal Defense Fund---test case selection, strategic litigation sequences, building precedent incrementally---were adopted by Lambda Legal and the ACLU for LGBT rights. The organizational repertoire of protest (marches, sit-ins, boycotts) transferred from civil rights to gay liberation movements in the 1970s, illustrating how friction repertoires diffuse across social movements \citep{tilly2008contentious}. This cross-movement learning accelerated the LGBT trajectory: rather than developing novel friction strategies from scratch, the movement inherited a proven playbook and adapted it to a new domain. The framework predicts that later movements will generally achieve faster alpha expansion, controlling for other factors, because they inherit accumulated organizational knowledge.

The LGBT case also parallels the suffrage trajectory in a structural sense: both involved long campaigns of friction accumulation followed by rapid international diffusion once a critical mass of adopters was reached. New Zealand's early adoption of women's suffrage (1893) and the Netherlands' early adoption of marriage equality (2001) both served as demonstration cases that accelerated subsequent adoption elsewhere. The diffusion pattern---initial outlier adoption, followed by a cascade among culturally proximate states, followed by broader international adoption---appears to be a general feature of consent expansion in domains where international comparison is salient.

However, the LGBT trajectory also reveals a feature absent from earlier movements: the role of visibility as an alpha-raising mechanism in itself. The act of coming out---making one's stakes visible to family, friends, and colleagues---functioned as a distributed friction strategy that operated interpersonally rather than institutionally. When a legislator's child, colleague, or constituent came out, it personalized abstract policy debates and shifted individual-level cost-benefit calculations about accommodation versus exclusion. No previous movement had an equivalent mechanism: women's status was already visible, enslaved persons' status was already visible, workers' status was already visible. Only the LGBT case involved a population whose initial invisibility was both a source of oppression (closeting as survival strategy) and a barrier to friction expression (invisible grievances cannot generate collective action). The coming-out movement systematically dismantled this barrier, converting latent into manifest friction at the interpersonal level before it aggregated into collective action.

This connects to \citet{farzulla2025stakes} analysis of structural exclusion patterns: once a population's voicelessness becomes visible through comparative institutional analysis, the legitimacy costs of maintaining exclusion increase. The LGBT case demonstrates that visibility operates at multiple scales---individual (coming out), organizational (Pride marches), and international (cross-national legal comparison)---each raising the cost of continued exclusion at its respective level.

\section{Corporate Governance (1950s--Present)}
\label{sec:corporate-governance}

While Section~\ref{sec:labor} examined how workers raised $\alpha(d_{\text{workplace}})$ through collective mobilization, this section examines the institutional architecture of corporate governance itself---who holds formal decision authority within firms, how that authority is allocated, and what happens when governance structures systematically exclude high-stakes populations from voice. The corporate governance domain provides unusually clean natural experiments because codetermination mandates were imposed by statute at specific dates, creating before-and-after comparisons and regression discontinuity designs that approximate the framework's causal predictions more closely than most historical case studies allow.

\subsection{Domain Definition}
\label{subsec:corporate-domain}

We define $d_{\text{corporate}}$ as the set of decisions affecting firm strategy, capital allocation, employment levels, operational practices, and externalities imposed on non-shareholder constituencies. The affected set includes:

\begin{itemize}[nosep]
\item \textbf{Shareholders} (financial $s_i$): returns on invested capital, portfolio risk, and firm survival. Stakes are significant but typically diversifiable---shareholders can exit through sale.
\item \textbf{Employees} (livelihood $s_i$): employment security, wages, career development, working conditions, and workplace dignity. Stakes are high and largely non-diversifiable---workers cannot hedge their human capital exposure to a single employer.
\item \textbf{Communities} (externality $s_i$): local employment, environmental impact, tax base, infrastructure demands, and social fabric. Plant closures can devastate entire municipalities.
\item \textbf{Suppliers} (contractual $s_i$): payment terms, relationship continuity, and supply chain stability.
\item \textbf{Customers} (product $s_i$): quality, safety, pricing, availability, and post-sale support.
\end{itemize}

Under Anglo-American shareholder primacy, the consent-holder mapping concentrates authority overwhelmingly in shareholders and their board-appointed agents: $C_{\text{shareholders}} \approx 1$, $C_{\text{employees}} \approx 0$, $C_{\text{communities}} \approx 0$. \citet{friedman1970} provided the canonical articulation: ``the social responsibility of business is to increase its profits.'' This doctrine generates low $\alpha(d_{\text{corporate}})$ when stakes are calculated comprehensively. The asymmetry is stark: employees hold non-diversifiable livelihood stakes (they cannot spread their dependence across multiple employers the way shareholders spread capital across a portfolio), yet possess negligible formal voice in the governance structures that determine their working lives.

\citet{aguilera2015connecting} demonstrate that this internal governance structure cannot be understood in isolation from external institutional environments. National legal traditions, labor market institutions, financial systems, and political cultures shape the set of feasible consent-holder mappings. The ``varieties of capitalism'' literature \citep{vitols2011coordinated} distinguishes between liberal market economies (LMEs: US, UK) where shareholder primacy dominates, and coordinated market economies (CMEs: Germany, Nordic states) where stakeholder voice is institutionally embedded. The consent-holding framework reinterprets this distinction as variation in $\alpha(d_{\text{corporate}})$ across institutional environments---not as culturally contingent preferences but as different institutional solutions to the universal problem of allocating corporate consent power.

\subsection{Alpha Proxy: Board Composition and Voice Channels}
\label{subsec:corporate-alpha}

We construct $\alpha(d_{\text{corporate}})$ from governance data capturing the distribution of formal decision authority:

\begin{equation}
\alpha(d_{\text{corporate}}, t) = \sum_{k \in \text{stakeholders}} \frac{s_k \cdot V_k(t)}{\sum_j s_j}
\label{eq:alpha-corporate}
\end{equation}

\noindent where $V_k(t)$ represents the effective governance voice of stakeholder category $k$, measured through board representation, voting rights, consultation requirements, and legal standing. Under pure shareholder primacy ($V_{\text{shareholders}} = 1$, $V_{\text{others}} = 0$), this yields low $\alpha$ because shareholders' financial stakes---while genuine---represent only a fraction of the total stakes affected by corporate decisions. Workers' non-diversifiable livelihood stakes, communities' spatially concentrated externality stakes, and consumers' product safety stakes are all governed without their consent.

Board composition provides the most tractable measure. In shareholder-only boards (the Anglo-American default), governance voice maps exclusively to capital: $V_{\text{shareholders}} = 1$. In codetermined boards (Germany post-1976), workers elect half the supervisory board, yielding $V_{\text{workers}} \approx 0.45$--$0.50$ (slightly below parity due to the shareholder-elected chair's tie-breaking vote). In Nordic systems, employee representation is typically one-third of board seats, supplemented by broader corporatist voice channels. In France, the 2013 Rebsamen law introduced employee board representation (1--2 members) in firms with over 1,000 employees---a modest alpha increase from a low baseline.

\citet{goranova2014shareholder} document how shareholder activism---proxy fights, shareholder proposals, institutional investor engagement---constitutes a voice channel within the shareholder constituency. This activism has intensified dramatically: shareholder proposals on governance and social issues rose from under 300 annually in the 1990s to over 900 by 2020. However, this activism operates \textit{within} the shareholder primacy model, redistributing voice among capital providers rather than extending it to non-shareholder stakeholders. From the framework's perspective, intra-shareholder activism raises alpha within the narrow financial domain but leaves the broader $\alpha(d_{\text{corporate}})$ unchanged because non-financial stakeholders remain excluded. It is voice redistribution within a privileged class, not voice extension to affected parties.

\citet{edmans2017equity} show that equity vesting structures create temporal misalignment even within shareholder governance: short-vesting equity incentivizes managerial decisions that boost near-term stock prices at the expense of long-term value (underinvestment in R\&D, deferred maintenance, excessive cost-cutting). The framework interprets this as a within-class alpha problem: managers' decision horizons (set by vesting schedules of 1--3 years) are misaligned with long-term shareholders' actual temporal stakes (investment horizons of decades for pension funds and endowments), producing friction (value destruction) even within the nominally privileged constituency. Myopia-induced friction manifests as shareholder lawsuits, activist campaigns targeting short-termism, and regulatory proposals for longer-term governance structures.

\subsection{Friction Proxy: Shareholder Activism and Stakeholder Pressure}
\label{subsec:corporate-friction}

Friction in the corporate domain manifests through multiple channels reflecting the diverse constituencies excluded from governance voice.

\textit{Labor disputes} represent the most direct friction channel: strikes, work-to-rule actions, and collective bargaining impasses reflect workers' stakes-weighted deviation from preferred outcomes. As analyzed in Section~\ref{sec:labor}, countries with higher $\alpha(d_{\text{workplace}})$ exhibit systematically lower strike rates. The cross-national correlation between board-level employee representation and strike frequency is strongly negative---precisely H1's prediction.

\textit{Shareholder activism} constitutes friction within the governance structure: \citet{goranova2014shareholder} review proxy fights, shareholder proposals, hostile takeover attempts, and ``say on pay'' votes as mechanisms through which excluded or dissatisfied shareholder factions challenge incumbent management. The rise of hedge fund activism since the 2000s---with activists like Carl Icahn, Nelson Peltz, and Elliott Management targeting firms for governance changes---reflects growing friction between dispersed shareholders' return expectations and managerial entrenchment.

\textit{Regulatory pressure} represents political friction from constituencies that lack direct corporate voice. The Sarbanes-Oxley Act (2002, responding to Enron/WorldCom scandals), Dodd-Frank (2010, responding to the financial crisis), the EU's Corporate Sustainability Due Diligence Directive (2024), and national corporate governance codes reflect legislative responses to accumulated stakeholder grievances---exactly the H4 prediction that persistent friction generates institutional reform. Each major corporate scandal or crisis produced a regulatory alpha intervention: an exogenous increase in stakeholder voice through mandatory disclosure, board independence requirements, or stakeholder consultation obligations.

\textit{Consumer and community friction} manifests through boycotts, reputation damage, social media campaigns, and legal challenges. ESG (Environmental, Social, Governance) activism represents a sustained attempt by excluded stakeholders to impose voice constraints on corporate decisions through market and reputational pressure rather than governance mechanisms. The growth of ESG-mandated assets from under \$5 trillion in 2010 to over \$35 trillion by 2020 represents the financialization of stakeholder friction: investor demand for ESG compliance channels non-shareholder grievances through capital markets.

\textit{Employee exit} constitutes passive friction: high turnover, difficulty attracting talent, and ``quiet quitting'' reflect workers' response to governance exclusion through individual rather than collective action. The framework predicts that declining collective voice capacity (Section~\ref{sec:labor}) shifts friction from organized (strikes) to diffuse (turnover, disengagement) channels---a prediction consistent with contemporary labor market dynamics where US firms spend approximately \$1 trillion annually on employee turnover costs.

\subsection{Alpha--Friction Dynamics: Natural Experiments}
\label{subsec:corporate-dynamics}

Three institutional configurations provide quasi-experimental variation in corporate $\alpha$.

\paragraph{German Codetermination: Institutional Alpha.} The German codetermination system provides the cleanest natural experiment in corporate governance. The 1951 Coal and Steel Codetermination Act imposed full parity representation (50\% employee-elected supervisory board seats) on mining and steel firms---sectors where labor friction had historically been most intense (Section~\ref{sec:labor}). The 1976 Codetermination Act extended quasi-parity to all firms with over 2,000 employees. The One-Third Participation Act (1952, amended 2004) mandated one-third employee representation in firms with 500--2,000 employees. These statutory mandates produced exogenous variation in board composition at specific employment thresholds that enables regression discontinuity analysis.

\citet{fauver2011good} find that codetermined firms invest more in human capital development (higher training expenditure per employee), maintain longer employee tenure (approximately 15\% longer than non-codetermined firms), and exhibit lower labor turnover---consistent with the framework's prediction that higher $\alpha$ produces outcomes closer to employee preferences. When workers have voice in firm decisions, firms allocate resources differently: more to training, workplace safety, and job security; less to short-term shareholder distributions and executive compensation. This is alignment working as predicted: higher $\text{Cov}(s_i, C_i)$ produces outcomes that better reflect the stakeholder distribution of stakes.

\citet{jaeger2022} provide the most comprehensive causal analysis, exploiting the 2,000-employee threshold in the 1976 Act to identify codetermination effects through regression discontinuity. Their findings are striking: codetermined firms exhibit significantly lower strike rates, approximately 10\% longer employee tenure, wages roughly 2\% higher, and---crucially---no significant negative effect on shareholder returns or firm valuation. This last finding is important for H5 (performance interactions): the common objection that stakeholder voice reduces efficiency finds no support in the most rigorous available evidence. Codetermination appears to reduce friction costs (fewer strikes, lower turnover, better labor relations) by roughly the same amount it raises labor costs, producing a net wash for shareholders while substantially improving outcomes for employees.

\citet{vitols2011coordinated} documents the institutional resilience of codetermination during financialization. Where Anglo-American firms restructured toward shareholder value maximization during the 1990s and 2000s---through leveraged buyouts, hostile takeovers, mass layoffs, and stock buybacks---German codetermined firms maintained more stable employment, continued investing in apprenticeship programs, and resisted pressure to prioritize quarterly earnings over long-term stakeholder value. The framework interprets this as structurally embedded $\alpha$: codetermination made it institutionally difficult to reduce worker voice even when capital market pressures incentivized shareholder primacy. This structural resistance to alpha reduction---what we might call \textit{institutional stickiness} of voice---is a feature the framework should capture theoretically.

\paragraph{Nordic Model: Corporatist Alpha.} The Nordic countries achieved high corporate $\alpha$ through a different institutional pathway. Rather than board-level codetermination as the primary mechanism, the Scandinavian model embedded stakeholder voice through corporatist bargaining structures: centralized wage negotiations between peak employer and union federations, active labor market policies that maintained full employment (reducing the threat of involuntary exit from the labor market), and comprehensive welfare states that raised workers' fallback positions and thereby their effective bargaining power even outside formal governance channels.

\citet{doyle2020nordic} provides evidence on the Nordic model's effects at the firm level, documenting how companies in coordinated Nordic economies maintain stakeholder orientation through institutional complementarities rather than single governance mandates. Board-level employee representation (typically one-third of seats in Swedish, Norwegian, and Danish firms) supplements rather than replaces the broader institutional framework. The framework interprets the Nordic case as demonstrating that $\alpha(d_{\text{corporate}})$ can be raised through multiple institutional channels operating simultaneously---the key variable is the aggregate effective voice of affected stakeholders, not the specific mechanism through which it is achieved. Whether voice is embedded through board seats (Germany), through corporatist bargaining (Sweden), or through both (Norway), the friction-reducing effects predicted by H1 appear consistent.

\paragraph{Anglo-American Shareholder Primacy: Low $\alpha$, Episodic $F$.} The \citet{friedman1970} doctrine---adopted as orthodoxy in American business schools from the 1970s onward and exported to the UK through Thatcherite reforms---produced the lowest sustained $\alpha(d_{\text{corporate}})$ among advanced industrial democracies. The resulting friction followed the pattern the framework predicts for persistent low alignment: episodic high-intensity conflicts interspersed with periods of latent friction.

The episodic pattern includes: hostile takeover waves of the 1980s (LBO-funded raiders like KKR extracting shareholder value through leveraged restructuring that imposed massive costs on employees and communities---the RJR Nabisco buyout alone produced over 5,000 layoffs); corporate governance scandals of the early 2000s (Enron, WorldCom, Tyco---representing friction from \textit{within} the shareholder class when managers' misaligned incentives destroyed value); the 2008 financial crisis (systemic friction from risk-taking by financial firms whose governance structures excluded the populations bearing the consequences); and the ESG backlash of the 2020s (friction from environmental and social stakeholders demanding voice through capital markets).

The 2019 Business Roundtable statement---signed by 200 CEOs endorsing ``stakeholder capitalism'' and explicitly repudiating the Friedman doctrine---represents an elite response to accumulated friction \citep{businessroundtable2019}. The framework interprets this as H4-consistent: persistent friction from excluded stakeholders (employee activism, consumer boycotts, regulatory pressure, ESG campaigns, political backlash against inequality, and the reputational costs of visibly prioritizing shareholders during the COVID-19 pandemic) generated sufficient pressure to produce at least rhetorical commitment to alpha improvement. Whether this represents genuine institutional change (structural alpha increase) or cheap talk (friction-reduction through signaling without governance reform) remains an empirically testable question: the framework predicts that if the statement is not followed by actual governance changes (board composition, stakeholder consultation mechanisms, accountability structures), friction will persist or intensify.

\citet{jackson2010corporate} analyze corporate social responsibility (CSR) as a friction-reduction strategy within the shareholder primacy model. Their comparative analysis reveals a striking pattern: in CMEs with high institutional $\alpha$ (Germany, Nordic states), explicit CSR is less developed because stakeholder voice is already embedded in governance structures. In LMEs with low $\alpha$ (US, UK), CSR serves as a voluntary substitute for institutional voice---firms address stakeholder concerns through discretionary programs rather than governance reform. The framework generates a testable prediction from this observation: voluntary CSR should be less effective than institutional codetermination at sustainably reducing friction, because discretionary commitments lack the credibility and enforceability of structural voice. Firms can withdraw CSR commitments during downturns; codetermination mandates persist through economic cycles.

\citet{ferrarini2012corporate} examine the tension between shareholder and creditor governance, documenting how corporate governance structures create distributional conflicts among financial claimants that standard shareholder primacy models overlook. The framework extends this analysis: creditors hold substantial stakes in firm solvency and risk management ($s_{\text{creditor}}(d_{\text{risk}})$ can be existential for concentrated lenders), yet possess limited governance voice compared to equity holders. The resulting misalignment produces friction through credit market channels: covenant restrictions, lending conditions, credit rationing, and---in extreme cases---creditor-initiated bankruptcy proceedings represent friction responses to governance exclusion. This within-capital-class friction demonstrates that even the privileged constituency under shareholder primacy is itself fragmented, with different financial claimants holding different stakes and different levels of effective voice.

\subsection{Cross-Case Connections}
\label{subsec:corporate-connections}

Corporate governance connects to multiple domains analyzed in this monograph. Section~\ref{sec:labor} examines how workers raised $\alpha$ through collective mobilization---strikes, unionization, bargaining pressure. This section examines how institutional design can embed voice structurally within firm governance. The two are complementary: German codetermination bridges labor rights and corporate governance by institutionalizing what unions achieved through adversarial bargaining. Where Section~\ref{sec:labor}'s analysis focuses on friction as a mechanism for raising alpha, this section focuses on the institutional outcome: what corporate governance looks like when alpha has been structurally raised, and how the resulting friction reduction validates the framework's predictions.

Platform governance (Section~\ref{sec:platform-governance}) represents the digital extension of corporate governance challenges. Digital platforms are firms whose governance decisions---content moderation, algorithmic recommendation, data privacy, marketplace rules---affect billions of users, yet maintain shareholder-primacy structures that exclude user voice entirely. The consent-holding framework predicts that platform governance should exhibit the same friction dynamics as traditional corporate governance (user boycotts, regulatory backlash, migration to alternatives), with additional intensity due to the unprecedented scale of affected populations and the velocity of decision cycles.

The social contract literature (Section~\ref{sec:social-contract}) provides normative foundations. The Friedman-Freeman debate---whether firms exist solely for shareholder returns or for stakeholder value creation---maps directly onto the consent-holding framework's question of how $C_{i,d}$ should be allocated in corporate domains. The framework resolves this debate empirically rather than normatively: the governance structure that minimizes friction while maintaining performance represents the legitimacy-maximizing allocation, regardless of which philosophical tradition endorses it.

The German codetermination case also provides the strongest available evidence for H5 (performance interactions). Critics from the Friedman tradition predicted that codetermination would reduce competitiveness by constraining managerial discretion, slowing decision-making, and misallocating resources toward employee preferences at shareholders' expense. The empirical evidence from \citet{jaeger2022} and \citet{fauver2011good}---stable returns, lower friction costs, higher human capital investment, sustained export competitiveness despite higher unit labor costs---suggests that the performance cost of higher $\alpha$ is offset by friction reduction. German firms' position as the world's third-largest exporter, sustained through decades of codetermination, provides evidence that H5's performance interaction is not uniformly negative: in domains where affected stakeholders hold decision-relevant information (employees know production processes, safety risks, and market conditions that shareholders and managers may not), higher $\alpha$ can \textit{improve} rather than degrade performance through better information aggregation, reduced turnover costs, greater employee commitment, and lower adversarial friction.

\section{Platform Governance (2010s--Present)}
\label{sec:platform-governance}

Digital platforms represent the earliest-stage case in our portfolio of consent alignment dynamics. Unlike the historical cases examined above---where decades or centuries of friction eventually generated substantial alpha increases---platform governance remains in its pre-institutional phase. Users with high stakes and near-zero consent power have begun generating friction, but the friction has not yet produced structural consent restructuring. The case is therefore predictive rather than retrospective: the framework anticipates trajectories that empirical observation can confirm or disconfirm in real time. The consent-holding analysis extends naturally to digital minds and AI governance more broadly; \citet{mogensen2026digital} survey the ethical landscape of digital minds, raising questions about standing and governance obligations that the consent-holding framework is designed to address through its substrate-neutral treatment of stakes and voice.

\subsection{Domain Definition}
\label{subsec:platform-domain}

Platform governance spans multiple decision domains, each affecting billions of users:
\begin{align*}
d_{\text{moderation}} &: \text{content moderation (speech, community standards, removal)} \\
d_{\text{algorithm}} &: \text{recommendation and ranking (information diet, attention)} \\
d_{\text{data}} &: \text{data governance (collection, retention, sharing, monetization)} \\
d_{\text{policy}} &: \text{platform rules (terms of service, API access, monetization policies)}
\end{align*}

Users hold high stakes across all domains. Content moderation decisions ($d_{\text{moderation}}$) affect speech rights, community norms, and information access. Algorithmic recommendation ($d_{\text{algorithm}}$) shapes information diets, attention allocation, and---through amplification effects---political discourse and mental health outcomes. Data governance ($d_{\text{data}}$) affects privacy, identity control, and economic value extraction from user-generated content. Platform policy ($d_{\text{policy}}$) determines creator livelihoods, developer ecosystem viability, and third-party business models.

Yet consent power concentrates almost entirely in platform executives and engineers: $C_{\text{users}}(d) \approx 0$ across all domains. Platform decisions about content moderation, algorithm design, data handling, and policy changes are made unilaterally by corporate actors with no formal obligation to consult affected users. \citet{gillespie2018} documents how this governance structure emerged not from principled institutional design but from the path-dependent evolution of platform business models---platforms began as technology companies, acquired governance responsibilities through scale, and never developed the consent mechanisms that political institutions evolved over centuries. \citet{gorwa2020algorithmic} extends this analysis, arguing that algorithmic governance creates accountability deficits that traditional regulatory frameworks were not designed to address: the governed cannot see, understand, or contest the rules being applied to them.

The platform domain is distinctive in two respects. First, the consent-holder $H_t(d)$ is a private corporation rather than a state, complicating traditional legitimacy analysis which assumes public governance structures. Second, the primary exit mechanism---leaving the platform---is constrained by network effects, data lock-in, and the absence of interoperable alternatives. This limits exit as a friction channel and concentrates friction expression in voice (protest, boycott) and external regulation.

\subsection{Alpha Proxy: User Representation Indices}
\label{subsec:platform-alpha}

Measuring alpha in platform governance requires identifying institutional mechanisms that give users formal decision power. The results are stark:

\textbf{Board representation.} No major platform allocates board seats to user representatives. Meta's board (12 members) includes no user representatives. Alphabet, Amazon, Apple, and X/Twitter follow the same pattern. Board composition alpha for users: $\alpha_{\text{board}} = 0$. This is remarkable when considered comparatively: even under the shareholder primacy model that preceded stakeholder governance reforms, shareholders---a subset of stakeholders---held formal voting rights on board composition and major corporate decisions. Platform users have \textit{less} governance power than nineteenth-century shareholders, despite bearing stakes that arguably exceed shareholder financial exposure in domains like information access, political discourse, and mental health.

\textbf{Oversight and advisory bodies.} Meta's Oversight Board (established 2020) represents the most significant experiment in raising platform alpha. The Board reviews content moderation decisions and issues binding rulings on specific cases and advisory opinions on policy. However, its jurisdiction is limited (only content moderation, not algorithmic design or data policy), its case selection is narrow (roughly 200 cases in its first four years from millions of user appeals), and its policy recommendations are advisory only. We estimate this raises effective alpha in $d_{\text{moderation}}$ by approximately 0.05---meaningful as precedent, negligible as consent restructuring. YouTube's Creator Advisory Board and similar bodies at other platforms provide even less formal authority.

\textbf{Community moderation.} Reddit's volunteer moderator system represents a structurally different model. Community moderators hold genuine decision power within their subreddits: $C_{\text{mod}}(d_{\text{moderation, local}}) > 0$. However, this authority is delegated (revocable by Reddit administrators), unpaid (creating labor extraction dynamics), and subject to platform-wide policy overrides. Reddit's 2023 API pricing changes---which eliminated third-party moderation tools without moderator consultation---demonstrated the fragility of delegated consent power when the platform retains ultimate authority. Wikipedia's collaborative editing model achieves higher alpha for content decisions through its consensus-based governance, though administrative hierarchies and paid staff retain structural veto power.

\textbf{Composite platform governance index.} We define:
\[
\alpha_{\text{platform}}(d) = \frac{\text{decisions with meaningful user input}}{\text{total governance decisions}}
\]

For major platforms, this ratio approaches zero. Even generous estimates---counting Oversight Board cases, community moderation decisions, and user feedback integration---yield $\alpha_{\text{platform}} < 0.05$ across Meta, Alphabet, and X/Twitter. This is lower than pre-reform corporate governance under shareholder primacy, where at least shareholders (a subset of stakeholders) held formal voting rights. \citet{costanza2020design} argues that this deficit reflects deeper design choices: platforms are built to extract value from users, not to represent their interests, and governance mechanisms would threaten business models predicated on unilateral control.

\begin{table}[htbp]
\centering
\caption{Platform Governance Alpha Estimates by Mechanism}
\label{tab:platform-alpha}
\small
\begin{tabular}{@{}llll@{}}
\toprule
Platform & Mechanism & Domain & Est.\ $\alpha$ \\
\midrule
Meta & Oversight Board & $d_{\text{moderation}}$ (partial) & $\sim$0.05 \\
Meta & Board of Directors & All domains & 0.00 \\
Reddit & Community moderators & $d_{\text{moderation}}$ (local) & $\sim$0.15 \\
Reddit & Platform-wide policy & All domains & 0.00 \\
Wikipedia & Consensus editing & $d_{\text{content}}$ & $\sim$0.40 \\
X/Twitter & None & All domains & 0.00 \\
YouTube & Creator Advisory Board & $d_{\text{policy}}$ (advisory) & $\sim$0.02 \\
\midrule
\multicolumn{3}{@{}l}{Average across major commercial platforms} & $<$0.05 \\
\bottomrule
\end{tabular}
\end{table}

Table~\ref{tab:platform-alpha} reveals the stark alpha deficit across major platforms. Only Wikipedia---a non-profit with a collaborative governance model---achieves alpha values comparable to historical cases' pre-reform baselines. Commercial platforms uniformly maintain near-zero alpha, confirming the framework's prediction that profit-maximizing governance structures resist consent expansion when the cost of accommodation (shared governance) exceeds the cost of managing friction (PR responses, minimal concessions).

\subsection{Friction Proxy: Boycotts, Migration, and Regulatory Action}
\label{subsec:platform-friction}

Platform friction manifests through exit (user migration), voice (boycotts and protest), and external regulatory intervention:

\textbf{User exit and migration.} The \#DeleteFacebook movement (2018), triggered by the Cambridge Analytica scandal, represented the first mass user exit event driven by governance grievances. However, network effects limited its impact: Facebook's monthly active users declined briefly but recovered within a quarter. The Twitter/X migration (2022--2024) following Elon Musk's acquisition was more sustained, with significant user flows to Mastodon, Bluesky, and Threads. Unlike \#DeleteFacebook, the X migration was driven not by a single scandal but by ongoing governance changes (content moderation policy reversals, verified account restructuring, API restrictions) that progressively alienated user segments. Exit as friction is structurally limited by network lock-in, but the emergence of federated alternatives (Mastodon, ActivityPub protocol) is reducing switching costs---potentially transforming exit from an individual choice into a collective action mechanism.

\textbf{Advertiser boycotts.} The Stop Hate for Profit campaign (2020) organized over 1,000 advertisers to pause Facebook spending, costing the platform an estimated \$7.2 billion in market capitalization. The X/Twitter advertiser exodus (2022--2024) was more severe: over half of top-100 advertisers suspended spending, with X's advertising revenue reportedly declining by approximately 50\%. Advertiser boycotts operate as indirect friction: advertisers are not the affected population (users are), but their economic leverage translates user grievances into financial pressure on platform governance. This is analogous to consumer boycotts in the abolition movement, where sympathetic third parties converted moral friction into economic pressure.

\textbf{Regulatory intervention.} The GDPR (2018) represents the first major external alpha-raising intervention: by granting users rights over their data (access, portability, erasure), it shifted consent power in $d_{\text{data}}$ from platforms to users---albeit with enforcement gaps. The Digital Services Act (DSA, 2022) mandated transparency in content moderation and algorithmic recommendation, addressing $d_{\text{moderation}}$ and $d_{\text{algorithm}}$. The Digital Markets Act (DMA, 2023) targeted gatekeeping power, requiring interoperability and data portability. These regulations constitute external alpha-raising: regulatory bodies acting as proxy consent agents for users who lack direct governance power. \citet{resseguier2020ai} argues that such regulatory interventions are necessary precisely because platform self-governance mechanisms lack enforcement teeth---voluntary ethics commitments without institutional backing remain toothless. \citet{mittelstadt2017individual} extends this analysis to high-stakes domains (healthcare, criminal justice) where algorithmic governance without consent mechanisms creates systemic risks that individual-level protections cannot address.

\subsection{Alpha-Friction Dynamics}
\label{subsec:platform-dynamics}

Platform governance illuminates the framework's predictions about early-stage consent dynamics:

\textbf{The pre-codetermination analogy.} Contemporary platform governance resembles early industrial labor relations before codetermination. Workers in nineteenth-century factories had high stakes and zero voice; platform users occupy a structurally analogous position. Early management responses to labor friction---company unions, welfare capitalism, paternalistic reforms---parallel platform responses to user friction: Oversight Boards as company unions (management-created bodies with limited authority), Creator Funds as welfare capitalism (financial concessions without governance power), and transparency reports as paternalistic disclosure (information provision without decision-sharing).

The analogy suggests a predictive trajectory: if platform friction follows the labor pattern, we should expect (a) escalating friction as users develop collective action capacity, (b) institutional innovation in user representation (platform equivalents of works councils), and (c) eventual regulatory codetermination mandates. The EU's regulatory trajectory already mirrors this sequence, with the DSA and DMA representing the early stages of mandatory governance restructuring.

The analogy also highlights structural differences that may alter the trajectory. Labor movements succeeded partly because workers could withhold their labor---a form of friction with direct economic consequences for employers. Platform users cannot easily ``withhold their usage'' because the platform extracts value from their attention and data regardless of their satisfaction level, and network effects make individual withdrawal costly without collective coordination. The closest platform equivalent to a strike is coordinated mass migration---which requires a viable alternative destination. The emergence of federated social media (Mastodon, Bluesky) and interoperability mandates (DMA) may be creating the structural conditions for effective user collective action, analogous to how the right to organize and picket created structural conditions for effective labor action.

\textbf{Network effects as consent lock-in.} Platform governance involves a distinctive mechanism absent from other cases: network effects create consent lock-in that dampens the friction-to-alpha conversion. In the labor case, workers could form unions and strike without losing their workplace entirely. In the platform case, users who exit lose access to their social graph, content history, and community connections---a cost that rises with platform tenure and network centrality. \citet{olson1965logic} collective action theory predicts that this cost structure will suppress friction expression: rational users will free-ride on others' protest rather than bear the personal costs of exit. The implication is that platform alpha may require external intervention (regulation) rather than organic friction-driven reform, making the platform case structurally different from historical cases where affected populations could organize within the system they sought to reform.

\textbf{External vs.\ internal alpha-raising.} Platform alpha is being raised primarily through external regulatory intervention (GDPR, DSA, DMA) rather than internal governance reform. This contrasts with the labor case, where codetermination emerged from negotiated settlements between labor and capital \citep{mcgaughey2016}. The difference may reflect the absence of organized user power: labor unions provided bargaining partners for codetermination negotiations, but no equivalent user organization exists for platforms. The implication is that platform alpha may remain externally imposed---regulatory rather than participatory---unless users develop collective bargaining capacity.

\textbf{Content moderation as consent crisis.} \citet{douek2022} frames content moderation as a systems-level governance challenge rather than a series of individual decisions. In consent-holding terms, this means $d_{\text{moderation}}$ involves billions of micro-decisions daily, each affecting specific users' stakes, with no scalable mechanism for user input. The Oversight Board model attempts to address this through case-by-case review, but the scale mismatch---200 reviewed cases against billions of moderation actions---renders it structurally inadequate as a consent mechanism.

The scale problem distinguishes platform governance from all historical cases. Suffrage decisions affected millions but were made periodically (elections). Labor decisions affected thousands per firm and were negotiated through collective bargaining. Platform moderation affects billions of users through millions of daily automated decisions---a governance volume that exceeds any institutional mechanism's capacity for meaningful consent. This suggests that platform alpha may require not human-scale governance mechanisms but algorithmic governance systems that embed consent principles into automated processes---a fundamentally new form of consent-holding that the framework must eventually accommodate.

\textbf{The AI governance frontier.} Platform governance increasingly intersects with artificial intelligence governance as algorithmic systems take over decision-making in content moderation, recommendation, and risk assessment. \citet{grimmelikhuijsen2022} document how algorithmic decision-making compounds the consent deficit: not only are users excluded from governance decisions, but the decision-making process itself is opaque, non-deliberative, and resistant to the forms of contestation that historical movements relied on. The framework predicts that as AI systems assume more governance functions, the alpha deficit will deepen unless countervailing institutions are developed. This connects platform governance to broader AI alignment concerns: the question of how to ensure AI systems serve affected populations' interests is structurally identical to the question of how to ensure governance systems align decision power with stakes.

\textbf{Hypothesis testing.} The platform case provides evidence for:
\begin{itemize}
    \item \textbf{H1} (alpha-friction inverse): Supported. Near-zero alpha coexists with rising friction (boycotts, migration, regulatory pressure), as predicted. The relationship is particularly visible in the X/Twitter case, where alpha \textit{decreased} after the 2022 ownership change (dissolution of Trust and Safety Council, reversal of content moderation policies) and friction \textit{increased} correspondingly (mass migration, advertiser exodus, regulatory scrutiny). This natural experiment---where alpha moved in the wrong direction---provides within-platform evidence for H1 that cross-platform comparisons cannot.
    \item \textbf{H4} (friction predicts future alpha): Partially supported. Regulatory responses (GDPR, DSA) represent friction-driven alpha increases, though the lag between friction events and regulatory response is measured in years. The Cambridge Analytica scandal (2018) triggered both immediate friction (\#DeleteFacebook) and delayed alpha response (regulatory action), with the GDPR implementation accelerated by the scandal's political salience. The DSA (2022) represents a further response to accumulated platform governance friction. If H4 holds, we should expect additional regulatory alpha-raising in the coming decade as friction continues to accumulate.
    \item \textbf{H5} (performance interactions): The strongest H5 case in the portfolio. Platforms maintain user bases despite near-zero alpha partly because performance ($P$) remains high---search works, social connection functions, content recommendation satisfies immediate desires. The framework predicts that performance degradation (declining content quality, increased spam, worse recommendations) would accelerate friction by removing the performance buffer. The X/Twitter trajectory partially confirms this: as platform performance declined following the ownership change (reduced moderation quality, increased spam, degraded verification systems), friction escalated more rapidly than governance changes alone would predict---suggesting that performance decline amplifies the alpha-friction relationship as H5 predicts.
\end{itemize}

\textbf{Predictive implications.} If the framework's predictions hold, platform governance will follow one of three trajectories over the next decade: (a) the labor trajectory, where sustained friction and regulatory intervention produce structural consent mechanisms (user representation on governance bodies, binding transparency requirements, algorithmic accountability); (b) the corporate governance trajectory, where modest reforms (enhanced oversight boards, community governance experiments) raise alpha incrementally without fundamentally restructuring decision authority; or (c) the indigenous trajectory, where friction persists indefinitely because the four incorporation factors remain unfavorable and platforms retain the capacity to absorb user dissatisfaction without structural reform. The EU's regulatory trajectory suggests outcome (a) is most likely in Europe, while the US's lighter regulatory approach may produce outcome (b) or (c). This cross-jurisdictional variation will itself provide an empirical test of the framework: if EU platforms develop higher alpha through regulatory mandates and exhibit lower user friction as a result, H1 receives additional support from a quasi-experimental design.

\subsection{Cross-Case Connections}
\label{subsec:platform-cross}

Platform governance recapitulates corporate governance dynamics in digital space. The stakeholder model debate---whether corporations should serve shareholders exclusively \citep{friedman1970} or all stakeholders \citep{freeman1984}---maps directly onto the platform governance debate: should platforms serve advertisers and shareholders, or users and the public? The Business Roundtable's \citeyearpar{businessroundtable2019} redefinition of corporate purpose to include all stakeholders has a platform analogue in calls for multi-stakeholder governance models.

The platform case also connects to algorithmic governance literature reviewed in Section~\ref{sec:literature}. \citet{grimmelikhuijsen2022} identify three value-based strategies for algorithmic legitimacy---accuracy, transparency, and participation---mapping onto performance ($P$), information provision, and consent alignment ($\alpha$) respectively. The framework suggests that the third strategy (participation) is structurally necessary when the first two are insufficient, echoing the historical pattern where performance alone could not sustain legitimacy without consent.

Finally, the platform case connects forward to climate governance (Section~\ref{sec:climate}) through the shared structure of diffuse-stakeholder governance. Both domains involve billions of affected parties with limited coordination capacity facing concentrated decision-makers. Both generate friction through proxy mechanisms (advertisers for platform users; youth activists for future generations) rather than through direct collective action by the most affected population. And both face structural barriers to the friction-to-alpha conversion: network effects in the platform case, temporal distance in the climate case. These parallels suggest that twenty-first-century governance challenges share a common structural feature---the scale mismatch between affected populations and governance institutions---that distinguishes them from the more concentrated conflicts of earlier centuries and may require novel institutional forms rather than adaptations of existing ones.

\begin{figure*}[htbp]
\centering
\includegraphics[width=0.95\textwidth]{alpha_historical_trajectories.png}
\caption{Historical consent alignment trajectories across four domains (suffrage, abolition, labor rights, platform governance) showing predicted dynamics: persistent low $\alpha$ generates rising friction $F$ until incorporation or suppression. Suffrage demonstrates gradual incorporation; abolition shows delayed incorporation with violent friction; labor rights exhibits cross-national variation; platform governance shows early-stage friction emergence.}
\label{fig:alpha-trajectories}
\end{figure*}

\begin{figure*}[htbp]
\centering
\includegraphics[width=0.95\textwidth]{friction_trajectories.png}
\caption{Friction trajectories corresponding to Figure~\ref{fig:alpha-trajectories}. Friction $F(d,t)$ rises when alignment remains low, spikes during mobilization peaks (suffrage protests 1910s, Civil War 1860s, labor strikes 1930s, platform revolts 2020s), then declines following institutional reform raising $\alpha$. Dotted lines indicate counterfactual friction under maintained exclusion.}
\label{fig:friction-trajectories}
\end{figure*}

\begin{figure*}[htbp]
\centering
\includegraphics[width=0.95\textwidth]{alpha_p_frontier.png}
\caption{Legitimacy frontier showing trade-offs between consent alignment $\alpha$ and performance $P$ across governance systems. Points represent empirical observations: Scandinavian social democracies achieve high $\alpha$ with moderate-high $P$; technocratic Singapore shows high $P$ with moderate $\alpha$; failed states exhibit low on both dimensions. The efficient frontier (solid curve) shows maximum achievable $P$ at each $\alpha$ level; institutional innovations shift the frontier outward.}
\label{fig:legitimacy-frontier}
\end{figure*}


\section{Climate Governance (2015--Present)}
\label{sec:climate}

Climate governance presents the consent-holding framework with its most theoretically demanding case: a domain where the population bearing the greatest stakes---future generations---possesses zero consent power by temporal exclusion. Unlike the historical cases above, where affected populations could at least in principle mobilize, organize, and articulate claims, future generations are literally absent from present governance processes. This creates a structural zero-alpha condition that cannot be remedied through the standard friction-to-incorporation pathway, forcing the framework to address proxy consent mechanisms and their adequacy.

\subsection{Domain Definition}
\label{subsec:climate-domain}

The climate governance domain $d_{\text{climate}}$ encompasses decisions affecting greenhouse gas emissions, adaptation policy, climate finance, and ecological system integrity. The domain is defined by three distinctive features:

\textbf{Temporal stakes distribution.} Climate decisions made today produce consequences extending centuries into the future. The present generation bears moderate adaptation costs; future generations bear potentially existential costs from cumulative emissions. In consent-holding terms, $s_{\text{future}}(d_{\text{climate}}) \gg s_{\text{present}}(d_{\text{climate}})$, yet $C_{\text{future}} = 0$ by definition. No institutional mechanism can give unborn citizens direct voice in present decisions. This is not a failure of specific institutions but a structural impossibility: temporal distance precludes direct participation.

\textbf{Spatial stakes asymmetry.} The Global South bears disproportionate climate impacts (sea-level rise, agricultural disruption, extreme weather) relative to its historical emissions and governance influence. Small island developing states face existential threats from decisions made predominantly by major emitters. This spatial asymmetry---where stakes and consent power are inversely correlated across nations---mirrors the domestic alpha deficits examined in earlier sections but operates at the international level where enforcement mechanisms are weaker.

\textbf{Youth stakes premium.} Within present generations, younger cohorts bear higher stakes than older cohorts because they will live longer with the consequences of current policy. A 20-year-old in 2025 has roughly 60 years of climate exposure; a 70-year-old has roughly 10. Yet voting power and political influence correlate positively with age in most democracies, creating an intragenerational alpha inversion: those with the highest stakes have the least consent power.

\subsection{Alpha Proxy: Youth Representation and Climate Voice}
\label{subsec:climate-alpha}

Measuring alpha in climate governance requires tracking both direct representation mechanisms and proxy institutions:

\textbf{National climate policy.} In most democracies, climate policy is determined through standard legislative processes where alpha reflects general electoral representation. Since climate is one of many domains covered by elected representatives, specific climate alpha depends on issue salience, party positioning, and lobbying dynamics. For future generations, alpha in national climate policy is definitionally zero.

\textbf{Citizens' climate assemblies.} Several jurisdictions have experimented with deliberative bodies specifically addressing climate governance. The French Convention Citoyenne pour le Climat (2019--2020) assembled 150 randomly selected citizens who produced 149 proposals for reducing greenhouse gas emissions. \citet{courant2021} documents both the promise and limitations: the Convention achieved high deliberative quality and produced ambitious proposals, but the government implemented only a fraction, and the Convention had no binding authority. The UK Climate Assembly \citep{ukclimate2020,ukclimate2022} followed a similar model, producing recommendations on the path to net zero. \citet{wells2021citizen} assesses whether such assemblies are driving democratic climate policymaking, finding that they raise public awareness and political salience but have limited direct policy impact.

These assemblies represent modest alpha increases for present citizens in climate governance, but they do not address the core problem: future generations remain unrepresented. Even randomly selected present citizens cannot genuinely represent future interests because they do not bear the full costs of current inaction. The assemblies also face the ``implementation gap'' common to deliberative innovations: the quality of deliberation may be high, but the binding authority is low. In consent-holding terms, assemblies raise the \textit{quality} of voice (informed, deliberative, representative) without raising its \textit{power} ($C_{\text{assembly}} \approx 0$ for binding decisions). This distinction between voice quality and voice power has broader implications for the framework: merely creating institutions where affected populations can speak does not constitute alpha improvement unless those institutions have genuine decision authority.

\textbf{Youth climate litigation.} Legal challenges filed by young people claiming violations of their rights through inadequate climate policy represent an alternative alpha channel. \textit{Juliana v.\ United States} (filed 2015) alleged that government failure to address climate change violated constitutional rights to life, liberty, and property. \textit{Urgenda Foundation v.\ State of the Netherlands} (2019) successfully required the Dutch government to reduce emissions by 25\% by 2020. \textit{Held v.\ State of Montana} (2023) found that Montana's prohibition on considering climate impacts in fossil fuel permitting violated the state constitution's right to a clean environment. These cases constitute friction events that, when successful, raise alpha by judicially mandating consideration of youth and future stakes in present policy. \citet{lorenzoni2025review} documents the global expansion of climate litigation, with over 2,000 cases filed worldwide as of 2024---a rapidly growing friction series.

\textbf{Constitutional provisions and guardianship institutions.} Several jurisdictions have created institutional proxies for future generations. Ecuador's 2008 constitution grants rights to nature (Pachamama). Bolivia's Law of Mother Earth (2010) establishes similar provisions. Hungary's Ombudsman for Future Generations (established 2008) reviews legislation for intergenerational impact. Wales's Well-being of Future Generations Act (2015) requires public bodies to consider long-term impacts. These institutions raise alpha by creating proxy consent mechanisms---present actors authorized to represent future interests---but their effectiveness depends on enforcement capacity and political insulation from short-term pressures.

\textbf{Voting age and youth quotas.} Austria lowered its voting age to 16 in 2007, directly raising alpha for a younger cohort in all governance domains including climate. Malta (2018), Scotland (2015 for local elections), and several German \textit{L\"ander} have followed suit. Several climate assemblies have included youth quotas. The Intergenerational Foundation \citeyearpar{intergenerational2024} has proposed weighted voting systems giving younger voters additional influence in domains with long-term consequences, though no jurisdiction has implemented such a mechanism.

The aggregate alpha picture in climate governance can be summarized as follows: for present adult citizens, alpha in climate policy is moderate (comparable to alpha in any policy domain mediated through electoral representation, $\alpha \approx 0.3$--$0.5$ depending on democratic quality). For youth (under voting age), alpha is near zero except in the handful of jurisdictions with lowered voting ages. For future generations, alpha is structurally zero, with proxy mechanisms providing at best a thin institutional representation that lacks binding authority. The stakes-weighted alpha---which is what the framework measures---is therefore extremely low because the populations with the highest stakes (future generations, youth) have the lowest consent power. This makes climate governance the domain with the largest stakes-alpha gap in our portfolio, exceeding even pre-abolition slavery when measured by the number of affected future persons multiplied by their per-capita stakes.

\subsection{Friction Proxy: Climate Litigation and Youth Mobilization}
\label{subsec:climate-friction}

Climate friction manifests through three channels of escalating intensity:

\textbf{Youth mobilization.} Fridays for Future, initiated by Greta Thunberg in August 2018,
mobilized millions of young people globally in school strikes demanding climate action.
The movement's growth trajectory illustrates classic friction escalation dynamics:
\begin{itemize}
    \item August 2018: Single protester outside Swedish parliament (individual friction expression)
    \item November 2018: Approximately 17,000 students strike across Australia (geographic diffusion)
    \item March 2019: First Global Climate Strike---1.4 million participants in 128 countries
    \item September 2019: Largest Global Climate Strike---7.6 million participants across 185 countries
    \item 2020--2021: Pandemic suppression of physical protest (friction channel blocked; shift to digital)
    \item 2022--present: Resumed physical protests with reduced scale but sustained institutional engagement
\end{itemize}
This trajectory maps onto \citet{stekelenburg2010social} social-psychological framework:
collective identity formation (``climate generation''),
perceived injustice (intergenerational inequity),
and group efficacy (belief that collective action can influence policy) combine to produce sustained mobilization.
The pandemic interruption provides a natural experiment in friction suppression: when the dominant friction channel (physical protest) was blocked, the movement shifted to digital advocacy and institutional engagement (litigation, political candidacy), demonstrating the substitutability of friction channels predicted by the framework.

\textbf{Direct action.} Extinction Rebellion (XR, founded 2018) escalated friction through disruptive direct action:
road blockades, building occupations, and other civil disobedience tactics
designed to impose visible costs on governance systems that maintain low alpha.
XR's ``Rebellion'' events blocked major London roads in April 2019 (over 1,100 arrests),
October 2019 (over 1,800 arrests), and subsequent actions across multiple countries.
Just Stop Oil and similar movements continued this escalation through 2023--2025
with increasingly confrontational tactics targeting cultural institutions (museum protests),
infrastructure (motorway blockades), and sporting events.

The escalation pattern follows the framework's prediction:
when institutional channels (voting, petitioning) fail to raise alpha,
affected populations shift to higher-cost friction mechanisms.
The progression from marches (Fridays for Future) to civil disobedience (XR)
to infrastructure disruption (Just Stop Oil) mirrors the historical escalation
in suffrage movements (petitions to marches to property destruction to hunger strikes)
and labor movements (petitions to strikes to factory occupations).
\citet{chenoweth2011why} document that nonviolent resistance campaigns
succeed when they achieve sustained participation above approximately 3.5\% of the population---a
threshold that climate movements have approached in some jurisdictions
(estimated 7.6 million participants in September 2019 against a global population of 7.7 billion
represents approximately 0.1\%, well below the threshold).
The framework would predict that unless participation rates increase substantially
or institutional friction channels (litigation, regulation) prove more effective,
direct-action friction alone is unlikely to generate sufficient pressure for structural alpha change.

\textbf{Climate litigation.} As documented above,
the global expansion of climate litigation from a handful of cases in the 2000s
to over 2,000 by 2024 represents a rapidly growing institutional friction series.
Unlike protest, litigation operates within institutional channels
but challenges existing governance structures by invoking rights claims
that override standard legislative processes.
The litigation strategy parallels the LGBT rights trajectory:
strategic case selection, precedent building, and jurisdictional forum shopping
are tactics directly inherited from the civil rights and LGBT movements'
legal playbooks.
Successful cases (\textit{Urgenda}, \textit{Held v.\ Montana}) create precedent
that lowers the barrier for subsequent litigation,
generating the same positive feedback loop observed in the LGBT litigation trajectory.
The key difference is that climate litigation seeks to raise proxy alpha
for future generations, whereas LGBT litigation sought to raise direct alpha
for the litigating community itself.
This makes climate litigation outcomes structurally dependent on courts'
willingness to act as intergenerational guardians---a role for which
most judicial systems were not designed.

\textbf{COP protests and civil society pressure.} Annual COP meetings have become focal points for climate friction expression. COP15 (Copenhagen, 2009) drew approximately 100,000 protesters and was widely perceived as a governance failure, catalyzing the shift toward the bottom-up Paris Agreement architecture. COP26 (Glasgow, 2021) drew an estimated 100,000 marchers. The concentration of friction around COP events creates a distinctive temporal pattern: friction spikes annually around conference dates and declines between them, unlike the more continuous friction patterns observed in domestic political movements. This periodicity may reduce friction's effectiveness by allowing governance systems to absorb short-duration pressure without structural reform.

\textbf{Barriers to friction expression.} \citet{lorenzoni2007barriers} identify psychological, social, and structural barriers to climate engagement: temporal distance (future impacts feel abstract), spatial distance (impacts concentrated elsewhere), collective action problems (individual actions seem futile), and competing priorities. These barriers suppress friction expression relative to the actual stakes involved, creating a gap between latent and manifest friction analogous to the pre-Stonewall suppression in the LGBT case---but driven by cognitive rather than legal mechanisms. The barrier structure explains a puzzle in the framework: climate stakes are arguably the highest of any domain examined (existential, global, irreversible), yet friction levels remain modest relative to stakes. The barriers identified by \citet{lorenzoni2007barriers} operate as friction suppressors, maintaining $F_{\text{observed}} \ll F_{\text{latent}}$ even without deliberate repression.

\subsection{Alpha-Friction Dynamics}
\label{subsec:climate-dynamics}

Climate governance dynamics test the framework's limits in several respects:

\textbf{Structural zero-alpha for future generations.} Unlike every other case in our portfolio, the zero-alpha condition for future generations cannot be remedied through the standard friction-to-incorporation pathway. Future people cannot organize, protest, litigate, or vote. Their interests can only be represented through proxy mechanisms: constitutional provisions, guardianship institutions, youth advocacy, and intergenerational ethics frameworks. The framework must therefore distinguish between \textit{direct alpha} (stakeholders gaining their own voice) and \textit{proxy alpha} (present actors authorized to represent absent stakeholders). All historical cases involved direct alpha expansion; climate governance requires proxy alpha as a structural necessity.

\textbf{The discount rate problem.} Even when proxy mechanisms exist, present decision-makers systematically discount future stakes. Economic discount rates of 3--7\% render impacts beyond 50 years nearly negligible in cost-benefit analysis. In consent-holding terms, this means that even technically adequate proxy alpha is undermined by temporal discounting of the very stakes it is meant to represent. Future generations' stakes are existentially high ($s_{\text{future}} \to \max$) but temporally discounted to near-zero in present governance calculations, creating a structural mismatch between actual and operationalized stakes.

\textbf{The abolition parallel.} Climate governance shares a structural feature with abolition: in both cases, the most affected population cannot represent itself, requiring proxy consent mechanisms. Abolitionist societies represented enslaved persons' interests without enslaved persons' direct participation in governance (though slave narratives like \citealt{equiano1789} provided essential first-person testimony). Climate assemblies and youth advocates play an analogous role for future generations. The abolition case suggests that proxy consent can eventually generate sufficient friction to overcome resistance---but the timeline was measured in decades to centuries.

The analogy has limits. Enslaved persons were contemporaneous---they could resist, testify, and eventually participate in their own liberation. Future generations are temporally absent and cannot contribute to their own representation. This makes climate governance's proxy alpha permanently proxy, never convertible to direct alpha for the most affected population. The framework must therefore evaluate proxy mechanisms not by their convergence toward direct representation (which is impossible) but by their fidelity to the interests they claim to represent---a normative standard that requires institutional design principles beyond the empirical alpha-friction dynamics examined here.

\textbf{The COP architecture and international alpha.} International climate governance through the UNFCCC and annual Conferences of the Parties (COP) adds a layer of alpha complexity. COP operates on a one-country-one-vote basis, giving Tuvalu (population 11,000) equal formal voice to China (population 1.4 billion). In stakes terms, this overweights small island states (existential stakes, minimal emissions) relative to major emitters (high emissions, lower relative vulnerability). The framework interprets COP as a partial alpha correction: by giving vulnerable states voice disproportionate to their population, the architecture partially compensates for the spatial stakes asymmetry. However, consensus requirements and the non-binding character of most COP outcomes limit effective alpha: formal voice does not translate to binding authority when powerful states can defect without consequence. \citet{gardiner2011perfect} characterizes climate change as a ``perfect moral storm'' combining intergenerational, international, and epistemic challenges---each of which maps onto a distinct alpha deficit in the consent-holding framework.

\textbf{Intergenerational justice and discount rate formalization.} The discount rate problem can be formalized within the consent-holding framework. Let $s_i^{\text{actual}}(d)$ represent agent $i$'s actual stakes in climate policy and $s_i^{\text{operationalized}}(d)$ represent the stakes as incorporated into governance calculations. For present citizens, $s_i^{\text{operationalized}} \approx s_i^{\text{actual}}$. For future citizens born at time $t + \tau$:
\[
s_i^{\text{operationalized}}(d) = s_i^{\text{actual}}(d) \cdot e^{-r\tau}
\]
where $r$ is the discount rate. At $r = 0.05$ and $\tau = 50$ years, $s_i^{\text{operationalized}} \approx 0.08 \cdot s_i^{\text{actual}}$---a 92\% reduction in effective stakes. Since $\alpha(d,t)$ is stakes-weighted, temporal discounting mechanically depresses the alpha measure even when proxy institutions exist. This formalization makes explicit what the climate justice literature has long argued: discounting future generations' welfare is equivalent to devaluing their consent claims. The framework provides a quantitative language for what is usually argued in purely normative terms.

\textbf{Hypothesis testing.} The climate case provides evidence for:
\begin{itemize}
    \item \textbf{H1} (alpha-friction inverse): Supported. Near-zero alpha for future generations coexists with rising friction from present proxies (youth strikes, litigation, direct action).
    \item \textbf{H4} (friction predicts future alpha): Weakly supported. Youth mobilization and litigation have generated some institutional alpha increases (climate assemblies, constitutional provisions, litigation victories), but the pace of alpha expansion lags far behind the urgency implied by stakes. The temporal structure of climate governance creates a unique H4 challenge: even if friction generates alpha increases on the historical timescale (decades), the physical climate system may reach irreversible tipping points before consent alignment improves sufficiently to produce adequate policy responses. The framework thus reveals a structural mismatch between institutional and physical timescales that is absent from other cases.
    \item \textbf{H5} (performance interactions): The critical test case. High performance on emissions reduction ($P$) could partially compensate for low alpha---if governments delivered rapid decarbonization, the legitimacy deficit from excluding future generations might be tolerable. But performance has been poor: global emissions continue rising, Paris Agreement targets are being missed, and adaptation investments remain inadequate. Low alpha combined with low performance is the framework's predicted worst case for friction escalation.

    The climate case also provides the clearest counterfactual test for H5. Consider Singapore's technocratic governance: relatively low alpha ($\alpha \approx 0.4$ on V-Dem measures) combined with high policy performance ($P$ high on infrastructure, education, economic growth). Singapore maintains low friction partly because performance compensates for consent deficits. Climate governance is the anti-Singapore case: low alpha and low performance, which the framework predicts should produce the highest friction. That climate friction remains lower than this prediction suggests is explained by the friction barriers identified by \citet{lorenzoni2007barriers}---temporal and spatial distance suppress manifest friction below the level that stakes-alpha calculations would predict.
\end{itemize}

\subsection{Cross-Case Connections}
\label{subsec:climate-cross}

Climate governance shares features with several earlier cases, making it a theoretically rich node in the cross-case network:

\textbf{Abolition parallel.} Both climate governance and abolition require proxy consent for populations that cannot represent themselves. Abolitionist societies articulated enslaved persons' interests through moral persuasion; climate assemblies and youth advocates articulate future generations' interests through deliberative processes. The structural parallel extends to the opposition dynamics: slaveholders defended their economic interests against abolition just as fossil fuel industries defend theirs against decarbonization. The abolition timeline (decades of agitation before breakthrough) may or may not be a useful predictor for climate governance, given the urgency of physical tipping points that create a hard deadline absent from the abolition case.

\textbf{Platform governance parallel.} Both climate and platform governance involve diffuse stakeholders versus concentrated decision-makers. Billions of people bear climate impacts but have minimal influence over emissions policy, just as billions of platform users bear governance impacts but have minimal input into platform decisions. Both cases exhibit the \citet{olson1965logic} collective action problem: the per-capita cost of organization exceeds the per-capita benefit of marginal policy improvement, suppressing friction expression below the level warranted by aggregate stakes.

\textbf{Corporate governance parallel.} Environmental externalities in corporate governance parallel climate externalities in national governance. In both cases, those bearing costs (communities affected by pollution, future generations affected by emissions) are excluded from decisions producing those costs. The corporate governance literature's distinction between shareholder primacy and stakeholder governance \citep{freeman1984} maps directly onto the climate debate between present-generation interests and intergenerational justice. The framework suggests that stakeholder governance in the corporate domain and intergenerational governance in the climate domain are structurally identical problems: both require expanding the affected set $S_d$ to include populations currently excluded from decision authority.

\textbf{Suffrage parallel.} International diffusion operates in climate governance as it did in suffrage extension. Early adopters of ambitious climate policy (Denmark's renewable energy transition, Costa Rica's carbon neutrality commitment, Germany's \textit{Energiewende}) serve as demonstration cases that raise the normative cost of inaction for laggards---just as New Zealand's 1893 women's suffrage raised normative costs for non-adopting democracies. The diffusion mechanism operates through international comparison, treaty obligations, and norm entrepreneurship. However, climate governance faces a diffusion obstacle absent from suffrage: the costs of ambitious climate policy fall disproportionately on energy-intensive economies, creating material resistance to norm adoption that the suffrage case---where extending the franchise was relatively cheap---did not face.

\section{Scope Conditions and Comparative Analysis}
\label{sec:scope-conditions}

The preceding eight case studies---suffrage (Section~\ref{sec:suffrage}), abolition (Section~\ref{sec:abolition}), labor rights (Section~\ref{sec:labor}), civil rights (Section~\ref{sec:civil-rights}), LGBT inclusion (Section~\ref{sec:lgbt}), corporate governance (Section~\ref{sec:corporate-governance}), platform governance (Section~\ref{sec:platform-governance}), and climate governance (Section~\ref{sec:climate})---demonstrate the framework's analytical power across diverse institutional contexts. But they also reveal boundary conditions: the cases were selected partly \textit{because} they exhibit the predicted alpha-friction dynamics. To test the framework rigorously, we must examine both the conditions under which friction successfully generates incorporation and the cases where it does not.

\subsection{When Does Friction Generate Incorporation?}
\label{subsec:friction-incorporation}

Our case studies reveal four factors that determine whether sustained friction converts into alpha expansion:

\textbf{Factor 1: Cost of repression versus accommodation.} Incorporation occurs when the cost of maintaining exclusion exceeds the cost of extending consent. In the suffrage case, repression costs escalated as women's organizations grew more sophisticated and international pressure mounted---accommodating women's political participation became cheaper than suppressing it. In the labor case, codetermination emerged partly because postwar German reconstruction required labor cooperation, making accommodation cheaper than continued conflict \citep{mcgaughey2016}. In the LGBT case, criminalization became increasingly costly as visibility rose and public opinion shifted. By contrast, where repression remains cheap---prison populations, undocumented migrants, populations in geographically isolated authoritarian states---friction can be absorbed without alpha expansion. The framework predicts: when the ratio $\text{cost}(\text{repression}) / \text{cost}(\text{accommodation})$ exceeds a threshold, incorporation becomes the equilibrium response to friction.

\textbf{Factor 2: International pressure and norm diffusion.} Incorporation accelerates when international demonstration effects create legitimacy costs for non-adoption. \citet{ramirez1997} document this mechanism for women's suffrage: each country's adoption raised the normative cost for remaining holdouts. The same dynamic operated for LGBT marriage equality, with the Netherlands (2001) initiating a cascade that reached 35 countries by 2025. International pressure operates through multiple channels: diplomatic pressure, treaty obligations, international court rulings, media comparison, and diaspora advocacy. Where international isolation is possible---North Korea, Eritrea, pre-reform Myanmar---norm diffusion operates weakly and exclusion persists despite high internal friction.

\textbf{Factor 3: Coalition availability among enfranchised groups.} Excluded populations rarely achieve incorporation through their own efforts alone; they require allies with existing consent power. Abolition required white abolitionists with parliamentary voice. Women's suffrage required male legislators willing to extend the franchise. Civil rights required white allies in the federal government and judiciary. LGBT rights required heterosexual allies in legislatures and courts. \citet{chenoweth2011why} demonstrate that nonviolent resistance succeeds largely through the mechanism of coalition expansion---drawing sympathetic thirds into the movement's orbit. Where potential allies are absent, hostile, or indifferent, friction may persist without generating incorporation. Platform governance currently lacks a natural ally class: advertisers are potential allies but prioritize their own commercial interests, not user governance rights.

\textbf{Factor 4: Elite interest alignment with reform.} Incorporation becomes feasible when elite factions perceive benefits from consent expansion. \citet{acemoglu2000why} model franchise extension as a strategic response by elites facing revolutionary threat---extending the vote is cheaper than risking overthrow. In the corporate governance case, German industrialists accepted codetermination partly because it stabilized labor relations and reduced strike costs. In the climate case, renewable energy industries create an elite faction benefiting from climate policy---but their political influence remains insufficient to override fossil fuel incumbents in most jurisdictions. Where no elite faction benefits from reform, friction must overcome not only institutional inertia but active elite opposition.

These four factors interact multiplicatively rather than additively. Successful incorporation typically requires at least three of four to be favorable. The most compressed trajectories (LGBT rights in Western democracies, 1969--2015) occurred when all four aligned: repression became costly (rising visibility made criminalization politically embarrassing), international diffusion was rapid (demonstration effects cascaded after the Netherlands 2001), coalition allies were available (straight allies in positions of legislative and judicial power), and elite factions benefited (corporate diversity initiatives, pink tourism, cultural capital accumulation). The slowest trajectories---or outright failures---occur when none or few align.

We can formalize this interaction as a rough incorporation probability function:
\[
P(\text{incorporation} | F > \tau) = f(\text{cost ratio}, \text{int'l pressure}, \text{coalition}, \text{elite alignment})
\]
where friction above threshold $\tau$ is necessary but not sufficient, and the four factors determine the conditional probability of conversion. This formalization is deliberately loose---the factors resist quantification at this stage---but it captures the analytical structure: friction creates the pressure, and the four factors determine whether the system yields or absorbs it.

Table~\ref{tab:incorporation-factors} summarizes the four-factor profile across the eight case studies, illustrating why some trajectories succeeded and others stalled.

\begin{table}[htbp]
\centering
\caption{Incorporation Factor Profiles Across Case Studies}
\label{tab:incorporation-factors}
\small
\begin{tabular}{@{}lcccc@{}}
\toprule
Domain & Cost Ratio & Int'l Pressure & Coalition & Elite Align. \\
\midrule
Suffrage & Rising & High & Moderate & Moderate \\
Abolition & Rising (war) & High & High & Split \\
Labor & High (strikes) & Moderate & Low $\to$ High & Split \\
Civil Rights & Rising & High (Cold War) & Growing & Split \\
LGBT & Rising & High & High & Growing \\
Corporate & Moderate & Low & Low & Split \\
Platform & Low & Rising (EU) & Low & Low \\
Climate & Low & Rising & Growing & Split \\
\midrule
Indigenous & Low & Low & Low & Low \\
Stateless & Low & Moderate & Low & Low \\
Prisoners & Low & Low & Low & Negative \\
\bottomrule
\end{tabular}
\end{table}

The table reveals the pattern: successful incorporation cases (top six rows) have at least two ``High'' or ``Rising'' factors, while stalled cases (bottom three rows) have predominantly ``Low'' entries. Platform and climate governance---the contemporary cases---show intermediate profiles, suggesting that their trajectories depend on whether international pressure continues to rise and coalition availability expands.

\subsection{Cross-Case Comparative Table}
\label{subsec:comparative-table}

Table~\ref{tab:comparative} synthesizes the alpha-friction dynamics across all eight case studies, providing a structured basis for comparative analysis. Several patterns emerge from the comparison that are not visible from individual cases alone.

\begin{table*}[htbp]
\centering
\caption{Cross-Case Comparative Analysis of Consent Alignment Dynamics}
\label{tab:comparative}
\small
\begin{tabular}{@{}p{1.6cm}p{1.5cm}p{1.8cm}p{1.3cm}p{1.8cm}p{1.6cm}p{1.8cm}p{1.0cm}p{1.4cm}@{}}
\toprule
Domain & Time Span & $\alpha$ Proxy & $\alpha$ Range & $F$ Proxy & $F$ Pattern & Key Mechanism & H1--H5 & Incorp.\ Type \\
\midrule
Suffrage & 1848--1971 & Enfranchised / adult pop. & 0.50 $\to$ 1.0 & Petition, protest, civil disobedience & Escalating, then declining & International diffusion + elite concession & H1,H3, H4 & Direct: franchise extension \\
\addlinespace
Abolition & 1787--1865 & Free / total pop.\ in slavery domains & 0.0 $\to$ 1.0 & Rebellions, abolitionist campaigns, civil war & Extreme spikes, violent resolution & Proxy consent + moral friction & H1,H3, H4 & Proxy $\to$ direct: emancipation \\
\addlinespace
Labor & 1870s--1951 & Union density + board seats & 0.0 $\to$ 0.5 & Strike days, union drives & Cyclical, institutionalized & Negotiated codetermination & H1,H2, H4,H5 & Direct: works councils, board seats \\
\addlinespace
Civil Rights & 1865--1968 & Voter reg.\ + legal protections & 0.1 $\to$ 0.8 & Marches, sit-ins, litigation, riots & Escalating, punctuated by legislation & Litigation + mass mobilization & H1,H3, H4 & Direct: legislation + judicial \\
\addlinespace
LGBT & 1969--2015 & Legal recognition index (0--1) & 0.0 $\to$ 1.0 & Pride, litigation, ballot measures & Suppressed $\to$ rapid escalation & Strategic litigation + norm diffusion & H1,H3, H4 & Direct: judicial + legislative \\
\addlinespace
Corporate & 1920s--present & Stakeholder board seats / total & 0.0 $\to$ 0.5 & Strikes, activism, regulatory pressure & Declining in codetermined systems & Statutory codetermination & H1,H2, H5 & Direct: statutory board representation \\
\addlinespace
Platform & 2010s--present & User governance input index & $\approx$ 0.0 & Boycotts, migration, regulation & Rising, pre-institutional & External regulation (GDPR, DSA) & H1,H4, H5 & External: regulatory imposition \\
\addlinespace
Climate & 2015--present & Youth/future rep.\ in binding decisions & $\approx$ 0.0 & Strikes, litigation, direct action & Rising, sustained & Proxy consent (assemblies, litigation) & H1,H4, H5 & Proxy: constitutional + institutional \\
\bottomrule
\end{tabular}
\end{table*}

The comparative table reveals three structural patterns. First, \textit{incorporation type} clusters into three modes: direct incorporation (suffrage, labor, civil rights, LGBT, corporate), where the affected population gains its own voice; proxy incorporation (abolition, climate), where third parties represent absent or voiceless populations; and external imposition (platform governance), where regulatory bodies raise alpha without organic stakeholder demand. These modes have different durability: direct incorporation tends to be self-sustaining (enfranchised populations resist disenfranchisement), while proxy and externally imposed alpha is vulnerable to erosion when proxy institutions lose influence or regulatory regimes change.

Second, \textit{friction patterns} vary systematically with domain characteristics. Domains with clear group identity (suffrage, civil rights, LGBT) produce sustained escalating friction. Domains with diffuse stakeholders (platform, climate) produce episodic friction that spikes around focal events and declines between them. Domains with organized collective action capacity (labor) produce cyclical friction correlated with economic conditions. These pattern differences suggest that the framework should incorporate stakeholder organization capacity as a mediating variable between alpha deficits and friction expression.

Third, the \textit{alpha range} achieved through incorporation is bounded above by institutional constraints. No case achieves $\alpha = 1.0$ in the effective sense: suffrage grants formal equality but not effective equality (gender gaps in political representation persist), abolition grants formal freedom but not effective freedom (racial inequality persists), and marriage equality grants formal recognition but not effective inclusion. The maximum achievable effective alpha appears to be domain-specific and constrained by structural factors beyond institutional design.

\subsection{Ordinal Hypothesis Testing}
\label{subsec:hypothesis-testing}

Systematic evaluation across all eight case studies reveals the empirical standing of each hypothesis:

\textbf{H1 (Alpha-friction inverse relationship).} Supported in 8/8 cases. Every case study exhibits the predicted pattern: low alpha coexists with high or rising friction, and alpha increases are followed by friction reduction. The relationship is strongest in the suffrage, labor, and LGBT cases where longitudinal data allows temporal tracking. Platform and climate governance represent ongoing tests where low alpha and rising friction are observed but the predicted alpha increase has not yet (fully) materialized. The universal support for H1 across highly diverse institutional contexts---from eighteenth-century abolitionism to twenty-first-century platform governance---constitutes the framework's strongest empirical finding.

\textbf{H2 (Covariance increase reduces friction).} Supported in 3/8 cases with clear evidence: labor (codetermination explicitly increases the correlation between worker stakes and decision power), corporate governance (stakeholder board representation), and suffrage (franchise extension aligning electoral power with citizen stakes). The remaining cases do not provide clean tests because the covariance mechanism is less visible: abolition involved complete exclusion-to-inclusion rather than gradual covariance improvement; civil rights and LGBT cases involved rights-based rather than covariance-based alpha shifts. H2 is most naturally tested in domains with continuous consent measures (union density, board composition) rather than binary ones (enfranchised/disenfranchised).

\textbf{H3 (Threshold effects).} Supported in 5/8 cases with identifiable critical junctures \citep{capoccia2007study}:
\begin{itemize}
    \item Suffrage: Seneca Falls Convention (1848) and subsequent international demonstration effects
    \item Abolition: Haitian Revolution (1791) demonstrating that slave systems could be overthrown
    \item Civil Rights: \textit{Brown v.\ Board of Education} (1954) catalyzing the modern civil rights movement
    \item LGBT: Stonewall Riots (1969) converting suppressed into manifest friction
    \item Climate: Fridays for Future (2018) mobilizing a previously quiescent youth population
\end{itemize}
Labor, corporate, and platform governance exhibit more gradual dynamics without clear threshold moments, though the Wagner Act (1935) and GDPR (2018) might qualify as regulatory thresholds that shifted friction dynamics. The threshold mechanism appears most relevant in domains where friction is suppressed (by criminalization, social taboo, or cognitive distance) rather than continuously expressed.

\textbf{H4 (Friction predicts future alpha increase).} Supported in 6/8 cases: suffrage (decades of protest preceded franchise extension), abolition (abolitionist campaigns preceded emancipation), labor (strike waves preceded codetermination legislation), civil rights (mass mobilization preceded the Civil Rights Act), LGBT (Stonewall-to-Obergefell trajectory), and climate (youth mobilization is generating climate assemblies and litigation victories). Platform governance shows early-stage evidence (user revolt is generating regulatory responses). Corporate governance shows mixed evidence---shareholder activism has generated some governance reforms, but the relationship between stakeholder friction and alpha expansion is less direct.

The lag between sustained friction and alpha response varies enormously: 46 years for UK suffrage (1882 suffragist organizations to 1928 equal franchise), approximately 80 years for American abolition (1787 abolition society to 1865 Thirteenth Amendment), approximately 50 years for US LGBT rights (1969 Stonewall to 2015 Obergefell), and approximately 100 years for civil rights (1865 Thirteenth Amendment to 1965 Voting Rights Act). These lags suggest that the friction-to-alpha conversion is mediated by the four incorporation factors identified in Section~\ref{subsec:friction-incorporation} rather than operating mechanically.

The variation in lag length is itself analytically informative. The shortest successful lag (LGBT, approximately 50 years from Stonewall to Obergefell) occurred when all four incorporation factors were favorable and the movement inherited organizational infrastructure from the civil rights movement. The longest lag (civil rights, approximately 100 years from formal emancipation to effective enfranchisement) occurred when elite interest alignment was actively hostile (Jim Crow as organized resistance to incorporation) and coalition availability was constrained by regional political dynamics. If the four-factor model is correct, we should expect platform governance and climate governance trajectories to depend on how quickly their factor profiles evolve---which is an empirically testable prediction rather than a retrospective rationalization.

\textbf{H5 (Performance interactions).} Supported in 3/8 cases with clear evidence: corporate governance (German codetermination maintains competitive performance, validating the alpha-performance combination), platform governance (high platform performance partially compensates for zero alpha), and climate governance (the framework predicts that high emissions-reduction performance would reduce legitimacy deficits, but this counterfactual has not materialized). The labor case provides partial support---high-performing firms with codetermination experience less friction---but the evidence is complicated by selection effects. H5 is the hardest hypothesis to test because it requires independent variation in both alpha and performance, which is difficult to observe historically.

\subsection{Limitations and Non-Cases}
\label{subsec:limitations}

The framework's empirical validity depends not only on explaining successful incorporation but on accounting for cases where high friction has \textit{not} generated alpha expansion:

\textbf{Indigenous sovereignty.} Indigenous populations in settler-colonial states exhibit maximal stakes (land, culture, self-determination, survival) combined with minimal consent power. Sustained friction has continued for centuries---from the Lakota/Dakota resistance (1850s--present) to Aboriginal land rights struggles in Australia to M\=aori sovereignty movements in New Zealand. Incorporation has been partial at best: treaty processes, land claims settlements, and recognition programs have raised alpha incrementally but nowhere near proportional to stakes. The four-factor analysis explains the persistence of exclusion: repression costs have historically been low (geographic isolation, military asymmetry), international pressure has been limited (settler-colonial states dominate international institutions), coalition availability has been constrained (small population share limits political leverage), and elite interest alignment has been weak (land and resource interests oppose indigenous claims).

\textbf{Stateless populations.} Rohingya, Palestinians, Kurds, and other stateless groups face existential stakes with zero formal consent power in any sovereign jurisdiction. Their friction (refugee crises, armed conflict, diplomatic campaigns) has generated international attention but minimal alpha expansion. The framework correctly predicts persistent high friction, but the pathway to incorporation is blocked by a structural feature absent from successful cases: the absence of a governing authority willing or able to extend consent. In the suffrage case, the state that excluded women was also the state that could enfranchise them. In the stateless case, the states that exclude the population deny responsibility for its governance, creating a structural gap between the locus of friction and the locus of potential incorporation. This suggests a boundary condition: the friction-to-alpha pathway requires that the friction targets---the decision-makers whose policies generate misalignment---have the institutional capacity and sovereign authority to extend consent. Where sovereignty is contested, fragmented, or absent, friction may generate international sympathy without generating institutional change.

\textbf{Prisoners.} Incarcerated populations hold direct stakes in criminal justice policy---sentencing, conditions, rehabilitation, reentry---but possess negligible voice. Prisoner disenfranchisement is explicit policy in most US states. The cost-of-repression factor explains the persistence: incarcerated populations are already physically controlled, making friction suppression structurally inexpensive. Coalition availability is limited by social stigma. Elite interest alignment is negative---``tough on crime'' politics creates electoral incentives to maintain exclusion.

\textbf{Authoritarian repression.} The framework's predictions assume that friction can be observed
and that governance systems have some mechanism for responding to it.
Under authoritarian repression, friction may be generated but systematically suppressed
before it can trigger alpha responses.
North Korea, Turkmenistan, and Eritrea maintain near-zero alpha
across most governance domains
with high latent friction that never converts to manifest form
because the cost of friction expression---imprisonment, death, family punishment---exceeds
nearly any threshold.
This represents a boundary condition for the framework:
when repression is total, the friction-to-alpha conversion mechanism is blocked entirely.

The theory correctly predicts that these systems are characterized by extreme misalignment,
but it cannot predict when (or whether) the repression barrier will eventually break.
Historical evidence from the fall of Eastern European communism (1989) and the Arab Spring (2011)
suggests that repression barriers can collapse suddenly
when external shocks or internal contradictions reduce the state's capacity for suppression---but
the timing remains unpredictable.
\citet{granovetter1978threshold} threshold models may apply here:
individual willingness to protest depends on how many others are already protesting,
creating a coordination problem where latent friction remains suppressed
until a small perturbation pushes the system past a tipping point,
triggering cascading mobilization.
The framework interprets authoritarian collapse as a threshold event where
$F_{\text{observed}}$ jumps discontinuously from near-zero to maximum
as the repression barrier fails---precisely the H3 dynamic,
but with the threshold located in repression capacity rather than alpha level.

These non-cases do not constitute failures of the framework. The theory predicts friction when alpha is low relative to stakes---and friction is indeed observed in all cases where it can be measured. What the theory does not predict is \textit{guaranteed incorporation}: friction is a necessary but not sufficient condition for alpha expansion. The four-factor model specified above explains the variance between successful and unsuccessful incorporation trajectories. The theory predicts the pressure; the scope conditions predict the response.

This distinction is analytically important.
A theory that predicted guaranteed incorporation from sustained friction
would be empirically falsified by the cases above.
A theory that predicts friction from misalignment (which is confirmed)
and identifies the conditions under which friction converts to alpha expansion
(which the four-factor model provides) is both more modest and more useful.
The framework's contribution is not to promise that justice will prevail
but to identify the structural conditions under which it is more or less likely to---and
to provide measurement tools for tracking progress.

This scope-conditioned prediction also addresses a common objection
to legitimacy frameworks: that they serve merely as normative critique
without analytical purchase on actual governance dynamics.
The consent-holding framework, supplemented by the four-factor incorporation model,
generates falsifiable predictions:
it predicts friction levels from alpha-stakes gaps,
predicts incorporation likelihood from factor profiles,
and predicts friction trajectory shapes from domain characteristics.
These predictions are testable against the historical record
(as Part III demonstrates) and against ongoing developments
in platform and climate governance (as the next decade will reveal).

\subsection{Toward Quantified Panel Analysis}
\label{subsec:panel-analysis}

The ordinal alpha and friction proxies constructed across the eight case studies prepare the ground for systematic econometric testing. The panel regression specification from Section~\ref{sec:operationalization}:
\begin{equation}
F_{d,t} = \beta_0 + \beta_1 \cdot \alpha_{d,t} + \beta_2 \cdot P_{d,t} + \gamma \cdot X_{d,t} + \mu_d + \lambda_t + \varepsilon_{d,t}
\tag{\ref{eq:regression}}
\end{equation}
could be estimated using the proxies constructed here, supplemented by panel data from several established sources.

\textbf{Alpha measurement.} V-Dem (Varieties of Democracy) provides annual data on electoral democracy, liberal democracy, participatory democracy, deliberative democracy, and egalitarian democracy indices for 202 countries from 1789 to the present. These indices can be mapped onto domain-specific alpha measures: the participatory component captures direct citizen voice ($\alpha_{\text{participation}}$), the egalitarian component captures stakes-consent covariance ($\text{Cov}(s_i, C_i)$), and the liberal component captures minority rights protections (thresholds against extreme alpha deficits). OECD governance indicators provide additional data on regulatory quality, rule of law, and government effectiveness that can serve as performance ($P$) measures.

\textbf{Friction measurement.} The Cross-National Time-Series Data Archive (Banks) provides annual counts of protests, strikes, riots, revolutions, and government crises from 1815 to the present. The Armed Conflict Location and Event Data (ACLED) provides geocoded event-level data on protests, riots, and political violence from 2018 onward. The Global Database of Events, Language, and Tone (GDELT) provides automated event coding from news sources. Combining these sources enables construction of domain-specific friction series.

\textbf{Identification strategy.} Causal identification of $\beta_1$ (the alpha-friction relationship) faces endogeneity concerns: alpha and friction are jointly determined, with friction driving alpha changes (H4) while alpha affects friction levels (H1). Instrumental variable strategies could exploit exogenous variation in consent structures from several sources: (a) colonial institutional legacies providing quasi-random variation in initial alpha levels \citep{acemoglu2012why}; (b) international diffusion shocks from neighboring countries' reforms; (c) constitutional changes driven by factors orthogonal to friction (succession crises, natural disasters triggering constitutional conventions). \citet{capoccia2007study} identify critical junctures that produce exogenous institutional variation usable as instruments.

\textbf{Expected results.} Based on the ordinal evidence assembled across eight case studies, we would expect: $\beta_1 < 0$ (higher alpha reduces friction), $\beta_2 < 0$ (better performance reduces friction), interaction effects ($\beta_1 \times \beta_2$) indicating that performance partially compensates for low alpha (H5), and nonlinear specifications revealing threshold effects (H3) at low alpha values. Domain fixed effects ($\mu_d$) would capture structural differences in friction propensity across governance domains, while time fixed effects ($\lambda_t$) would control for global trends in mobilization capacity.

\textbf{Cross-domain estimation.} A key advantage of the consent-holding framework is its applicability across governance domains. Rather than treating suffrage, labor rights, platform governance, and climate policy as unrelated phenomena, the framework enables cross-domain panel estimation where domain-specific alpha and friction measures enter a common regression framework. This pooled estimation increases statistical power and enables testing of whether the alpha-friction relationship is stable across domains or varies systematically with domain characteristics (e.g., whether the relationship is stronger in domains with existential stakes than in domains with economic stakes).

The cross-domain approach also enables counterfactual analysis. By estimating the alpha-friction relationship from historical cases where trajectories are known, the model can generate predictions for contemporary cases (platform governance, climate) where trajectories are still unfolding. If the estimated $\beta_1$ from suffrage, abolition, labor, and civil rights data predicts the friction levels observed in platform governance and climate given their current alpha values, this constitutes out-of-sample validation of the framework's core hypothesis.

\textbf{Limitations of quantification.} Several caveats attend the quantification agenda. First, alpha and friction are multi-dimensional concepts that resist reduction to single indices without information loss. Second, the causal mechanisms connecting alpha to friction operate through institutional channels that vary across contexts, making a single $\beta_1$ estimate potentially misleading. Third, the longest historical series (suffrage, abolition) involve institutional contexts so different from contemporary governance that temporal comparability is uncertain. These limitations suggest that quantified panel analysis should complement rather than replace the case study approach---providing robustness checks on patterns identified qualitatively rather than serving as the sole basis for theoretical evaluation.

The transition from ordinal case studies to quantified panel analysis represents the framework's research frontier. The proxies constructed here are necessarily crude---legal recognition indices, board composition ratios, protest counts---but they provide the conceptual infrastructure for increasingly precise measurement. As \citet{fariss2014respect} demonstrates for human rights measurement, ordinal proxies can be refined into continuous latent measures through Bayesian item-response models, offering a pathway from the illustrative analysis presented here to rigorous empirical testing. The eight case studies examined in Part III---spanning two centuries of governance transformation across political, economic, social, and technological domains---demonstrate that the consent-holding framework provides a unified analytical language for understanding when and why governance systems incorporate excluded populations, and when they do not.

% ============================================================================
\part{Computational Validation}
\label{part:computational}
\thispagestyle{partpage}
% ============================================================================

\section{Computational Mechanism Comparison: Adaptive Learning Dynamics}
\label{sec:monte-carlo}

Beyond historical validation, we compare consent allocation mechanisms through computational simulation. Monte Carlo experiments \citep{robert2004} vary mechanisms across diverse preference distributions under adaptive learning dynamics to assess relative performance, using repeated random sampling to obtain numerical results about mechanism performance.

\textbf{Methodological Transparency}: The Bayesian learning model implements preference adaptation toward observed outcomes, which by construction reduces friction over time as agents' ideal points converge toward policy decisions. This is not a test of whether friction \textit{can} reduce—that follows definitionally from preference convergence—but a \textbf{comparative test} of which consent allocation mechanism produces superior alignment trajectories when agents adapt to institutional performance. The simulation addresses: given plausible behavioral assumptions (agents learn from outcomes), which mechanism best matches stakeholder interests with institutional decisions?

This computational exercise demonstrates mechanism rankings under specific auxiliary assumptions (Bayesian learning with stakes-weighted attention) rather than validating the framework's core theoretical claims. The framework's definitional structure (legitimacy as stakes-weighted alignment, friction as preference deviation) means the Monte Carlo provides illustrative comparison rather than empirical proof. Agent-based modeling \citep{epstein2006} enables bottom-up simulation of complex social systems through individual agent interactions, providing a natural methodology for exploring consent-holding dynamics computationally. Recent advances in evolutionary stability theory \citep{porter2026evolutionary} provide formal conditions under which institutional configurations resist invasion by alternative arrangements, complementing the simulation approach with analytical stability guarantees.

\subsection{Simulation Design}

We simulate 1000 societies, each with $N = 100$ agents making decisions in $M = 10$ domains over $T = 50$ time periods. Agent stakes $s_i(d)$ are drawn from heterogeneous distributions: some domains exhibit concentrated stakes (e.g., environmental policy affecting coastal residents heavily, inland minimally), others show uniform stakes (e.g., monetary policy affecting all). Preferences $x^*_{i,d}$ are initially drawn from various distributions (normal, bimodal, skewed) to test robustness.

We compare five consent allocation mechanisms:

1. \textbf{Equal voice} (pure democracy): $C_{i,d} = 1/N$ for all $i,d$
2. \textbf{Stakes-weighted}: $C_{i,d} = s_i(d) / \sum_j s_j(d)$
3. \textbf{Random} (sortition): $C_{i,d} = 1$ for randomly selected $i$, 0 otherwise
4. \textbf{Expert} (technocracy): $C_{i,d} = 1/|E|$ for top-$k$ performers on competence metric
5. \textbf{Plutocratic}: $C_{i,d} \propto$ wealth$_i$ independent of stakes

For each mechanism, we measure friction $F(d,t) = \sum_i s_i(d) \cdot |x_d(t) - x^*_{i,d}(t)|$ and consent alignment $\alpha(d,t)$ at each timestep.

\subsection{Bayesian Preference Learning Dynamics}

To test whether mechanism rankings reflect genuine convergence properties rather than cross-sectional snapshots, we implement temporal dynamics where agents update preferences based on observed policy outcomes. This addresses a critical methodological concern: static evaluation measures societies at fixed points but cannot justify claims about convergence or temporal stability.

\textbf{Learning Mechanism}: Each period, agents observe the institutional decision $d(t)$ with noise and update beliefs via Bayesian inference \citep{savage1954}, which provides the normative framework for belief updating given new evidence:

\begin{equation}
x_i^*(t+1) = \frac{\tau_0 x_i^*(t) + \tau_{obs,i} y(t)}{\tau_0 + \tau_{obs,i}}
\label{eq:bayesian-update}
\end{equation}

where $y(t) = d(t) + \epsilon$ with $\epsilon \sim \mathcal{N}(0, 0.1)$ represents noisy outcome observation, prior precision $\tau_0 = 1.0$ reflects initial belief strength, and observation precision $\tau_{obs,i} = s_i^*$ implements stakes-weighted attention.

High-stakes agents learn faster because observation precision scales with stakes, reflecting greater attention to outcomes that affect them more. This micro-foundation provides theoretical grounding for convergence dynamics: agents with strong interests in policy domains allocate cognitive resources proportionally to their exposure, producing faster belief updating when outcomes diverge from priors.

\textbf{Implementation}: Each of 1000 Monte Carlo runs evolves 50 periods with endogenous preference updating. Agents begin with heterogeneous preferences $x_i^*(0)$ drawn from empirical distributions. At each timestep $t$: (1) the institutional mechanism aggregates current preferences into decision $d(t)$, (2) agents observe outcome with noise, (3) Bayesian updating produces new preferences $x_i^*(t+1)$ serving as priors for period $t+1$, (4) metrics $\alpha(d,t)$ and $F(d,t)$ are recorded. This generates 50,000 observations per mechanism (1000 runs $\times$ 50 timesteps), enabling statistical tests of convergence properties.

\subsection{Results}

\subsubsection{Static Baseline Comparison}

Initial comparative statics establish baseline mechanism performance. Stakes-weighted mechanisms achieve significantly higher consent alignment ($\alpha = 0.6274$, 95\% CI: [0.6186, 0.6362]) compared to equal voice ($\alpha = 0.6042$, 95\% CI: [0.5962, 0.6122]), with plutocracy ($\alpha = 0.5962$) and expert rule ($\alpha = 0.5919$) performing worse. Random assignment establishes lower bound ($\alpha = 0.4884$), confirming structured mechanisms outperform chance. These cross-sectional differences demonstrate stakes-weighting advantage when heterogeneous exposure exists, but cannot justify convergence claims absent temporal dynamics.

\subsubsection{Bayesian Learning Dynamics: Genuine Convergence}

Under Bayesian preference updating, all mechanisms exhibit genuine temporal evolution with monotonic consent alignment increases and friction collapse. Stakes-weighted DoCS achieves final alignment $\alpha = 0.872$ (95\% CI: [0.858, 0.886]), representing 39\% improvement over static baseline ($0.627 \to 0.872$). Equal voice reaches $\alpha = 0.870$ (+44\% over static), while plutocracy converges to $\alpha = 0.860$ (+44\%). Expert rule attains $\alpha = 0.842$ (+42\%), and even random assignment improves to $\alpha = 0.761$ (+56\%), though remaining lowest overall.

\textbf{Friction Reduction}: All mechanisms dramatically reduce friction under learning dynamics. Stakes-weighted DoCS friction collapses 94.9\% from initial $F = 30.3$ to final $F = 1.5$, achieving lowest terminal friction. Equal voice reduces friction 94.2\% ($F = 30.2 \to 1.8$), plutocracy 93.5\% ($F = 32.0 \to 2.1$), expert rule 88.2\% ($F = 31.7 \to 3.7$), and random assignment 80.0\% ($F = 41.7 \to 8.3$). Friction collapse validates the theoretical prediction that preference alignment toward observed outcomes reduces stakes-weighted deviation.

\textbf{Initial Alignment Advantage}: Stakes-weighted mechanisms begin with higher consent alignment (mean initial $\alpha = 0.823$) compared to random assignment ($\alpha = 0.765$), reflecting that stakes-weighting produces better initial matches between consent power and stakeholder preferences. This superior starting position translates into lower friction throughout the learning process.

\textbf{Monotonic Convergence Validation}: Linear regression of $\alpha$ on time yields positive slope in 87.1\% of DoCS runs (mean $\beta_1 = 0.0048$, $p < 0.001$), confirming genuine convergence rather than random fluctuation. Equal voice exhibits monotonic increase in 71.9\% of runs, plutocracy in 65.0\%. Expert rule and random assignment show lower monotonicity rates (0\% for both due to measurement noise and random shocks), but mean trajectories still increase.

\textbf{Plutocracy Convergence}: Under learning dynamics, plutocracy converges nearly as high as DoCS ($\alpha = 0.86$ versus $0.87$, only 1.4\% gap), suggesting wealthy elites can adapt to align with stakeholder interests even when initially misaligned. However, plutocracy maintains higher friction throughout the learning process ($F = 2.1$ final versus DoCS $F = 1.6$). The normative implication: DoCS advantage lies in \textit{immediate alignment}—better initial matching produces consistently lower friction. Plutocracy's eventual convergence reflects co-option (elites learning to mimic stakeholder preferences) rather than initial legitimacy.

Figure \ref{fig:alpha-trajectories-sim} shows consent alignment trajectories under learning dynamics. Stakes-weighted mechanisms converge monotonically to highest equilibrium $\alpha$, while random assignment exhibits high variance and low mean throughout. Equal voice converges to near-DoCS levels, but slower initial alignment produces higher friction during transition periods. Plutocracy and expert rule converge to similar moderate levels, both eventually tracking stakeholder preferences through Bayesian updating despite opposing initial logics (wealth versus competence).

Cross-mechanism comparisons validate Postulate 1's legitimacy function $L = w_1 \cdot \alpha + w_2 \cdot P$: optimal mechanism depends on domain-specific weights. Technical domains with objective performance metrics (infrastructure engineering, public health interventions) rationally assign high $w_2$, favoring expert mechanisms despite consent costs. Value-laden domains (immigration policy, cultural regulations, distributive justice) assign high $w_1$, favoring stakes-weighted or equal voice mechanisms where stakeholder alignment outweighs technical optimization. The framework provides tools for domain-appropriate matching rather than universal prescriptions—no single mechanism dominates across all contexts, but stakes-weighting achieves superior consent alignment when heterogeneous stakes are empirically measured.

These results establish computational validity for the framework's core claim: consent power allocation should track stakes distribution to minimize friction and maximize legitimacy. When high-stakes minorities exist (environmental justice communities facing pollution, workers facing automation, indigenous groups facing resource extraction), equal voice systematically under-represents their interests. Stakes-weighting corrects this democratic deficit not through paternalism but through preference-weighted aggregation—those who bear consequences gain proportional voice in decisions.

\begin{figure*}[htbp]
\centering
\includegraphics[width=0.95\textwidth]{alpha_convergence_learning.pdf}
\caption{Consent alignment convergence under Bayesian learning over 50 time periods across five mechanisms (1000 Monte Carlo runs, 100 agents per society). Agents update preferences via Bayesian inference with stakes-weighted observation precision. Solid lines show mean $\alpha$ across runs; shaded regions indicate 95\% confidence intervals. Stakes-weighted DoCS (blue) achieves highest final alignment ($\alpha = 0.872$) with lowest terminal friction. Equal voice (orange) converges nearly as high ($\alpha = 0.870$). Plutocracy (red) and expert rule (green) reach moderate levels ($\alpha = 0.86$, $0.84$) despite opposing initial logics. Random assignment (purple) exhibits high variance and lowest convergence ($\alpha = 0.76$). Friction collapses 80-95\% across all mechanisms as preferences align with observed outcomes.}
\label{fig:alpha-trajectories-sim}
\end{figure*}

\begin{figure*}[htbp]
\centering
\includegraphics[width=0.95\textwidth]{friction_reduction_learning.pdf}
\caption{Friction reduction under Bayesian learning dynamics. All mechanisms exhibit dramatic friction collapse as agents update preferences toward observed outcomes. Stakes-weighted DoCS (blue) achieves lowest terminal friction ($F = 1.5$, 94.9\% reduction from initial $F = 30.3$). Equal voice (orange) reduces friction 94.2\% ($30.2 \to 1.8$), plutocracy (red) 93.5\% ($32.0 \to 2.1$). Solid lines show mean across 1000 runs; shaded regions indicate 95\% confidence intervals. Friction collapse validates theoretical prediction that preference alignment toward policy outcomes reduces stakes-weighted deviation.}
\label{fig:friction-reduction}
\end{figure*}

\section{Dynamic Validation and Robustness}
\label{sec:dynamic-validation}

The Bayesian learning dynamics implementation addresses a critical methodological concern: static evaluation cannot justify claims about convergence or institutional stability. This section demonstrates that mechanism rankings reflect genuine convergence properties, validates robustness across alternative dynamic modes, and interprets plutocracy's surprising performance.

\subsection{Convergence Statistics}

Across 50,000 observations per mechanism (1000 runs $\times$ 50 timesteps), Bayesian learning produces monotonic consent alignment increase in 87.1\% of DoCS runs. Linear regression of $\alpha$ on time yields mean slope $\beta_1 = 0.0048$ ($p < 0.001$), confirming genuine temporal dynamics rather than random fluctuation. Equal voice exhibits monotonic increase in 71.9\% of runs, plutocracy in 65.0\%, validating convergence across mechanisms.

Ljung-Box tests reject white noise hypothesis for friction trajectories (DoCS: $Q = 1847.3$, $p < 0.001$), confirming genuine autocorrelation from learning dynamics rather than independent draws. Friedman test shows mechanism rankings differ significantly across runs ($\chi^2 = 3842.7$, df = 4, $p < 0.001$). Post-hoc Nemenyi test establishes pairwise ranking: DoCS $>$ Equal Voice $>$ Plutocracy $>$ Expert $>$ Random (all $p < 0.01$).

Convergence speed varies systematically: DoCS reaches 90\% of final $\alpha$ by period 18, equal voice by period 20, plutocracy by period 22, expert rule by period 25, and random assignment by period 35. Stakes-weighting advantage manifests not only in terminal alignment but also in transition dynamics—agents experience preferred outcomes immediately, requiring less belief updating to reach equilibrium.

\subsection{Robustness Across Dynamic Mechanisms}

To test whether results depend on Bayesian learning assumptions, we implemented three alternative temporal dynamics: (1) \textbf{social mode} implementing DeGroot opinion dynamics via random network ($10\%$ connection probability), (2) \textbf{stakes mode} with endogenous stakes evolution where winners accumulate power proportional to alignment, (3) \textbf{static mode} as baseline comparative statics.

Stakes-weighted DoCS ranks first across \textit{all} modes: static ($\alpha = 0.627$), learning ($\alpha = 0.872$), social ($\alpha = 0.738$), stakes ($\alpha = 0.893$). This 0.627-0.893 range demonstrates robustness—superiority does not depend on temporal mechanism choice. Equal voice consistently ranks second (range: 0.604-0.873), plutocracy third (0.596-0.874), expert rule fourth (0.592-0.831), and random assignment fifth (0.488-0.661).

Stakes mode produces highest terminal $\alpha$ (0.893) but lowest friction reduction (72\% versus 80-95\% for learning/social modes), reflecting winner-take-all dynamics: agents whose preferences align with decisions gain stakes, creating self-reinforcing alignment through power concentration rather than preference convergence. This path-dependent outcome raises entrenchment concerns requiring institutional safeguards (term limits, redistribution, mandatory rotation).

Social mode demonstrates DoCS superiority persists even under pure opinion dynamics without outcome-based learning. Network diffusion produces slower convergence (35-40 periods to 90\% final $\alpha$) but ultimate rankings remain consistent. This validates that stakes-weighting advantage is not artifact of Bayesian assumptions.

\subsection{Plutocracy Convergence: Co-option Versus Legitimacy}

A surprising finding: plutocracy converges nearly as high as DoCS under learning dynamics ($\alpha = 0.86$ versus $0.87$, only 1.4\% gap), suggesting wealthy elites can adapt to align with stakeholder interests even when initially misaligned. However, three critical distinctions remain:

\textbf{First, convergence speed differs}: Plutocracy requires 22 periods to reach 90\% final $\alpha$ versus DoCS's 18 periods, imposing 4 additional periods of transition costs. During this learning lag, friction remains approximately 20\% higher ($F \approx 3.8\text{-}4.4$ versus $3.1\text{-}3.7$), generating observable instability.

\textbf{Second, initial alignment diverges}: At $t=0$, DoCS achieves $\alpha = 0.823$ while plutocracy starts at $\alpha = 0.811$, reflecting wealth-stakes misalignment. Stakes-weighting provides immediate consent alignment; plutocracy requires learning to discover stakeholder preferences.

\textbf{Third, normative interpretation differs}: Plutocracy's convergence reflects \textit{co-option}—elites learning to mimic stakeholder preferences to reduce friction—rather than \textit{initial legitimacy}. Wealthy agents update beliefs toward high-stakes populations' ideal points because Bayesian inference reveals those outcomes reduce system-wide friction, benefiting elite interests indirectly. This is strategic adaptation, not principled consent allocation.

The framework's prescription remains: DoCS minimizes transition costs through immediate alignment. Relying on plutocratic learning imposes friction costs during adjustment periods, creates path dependencies where early-period elite preferences shape outcomes before convergence, and substitutes strategic mimicry for structural consent alignment. Even if wealthy elites eventually learn to govern well, their authority lacks consent-based legitimacy throughout the learning process.

\subsection{Robustness to Parameter Variations}

Mechanism rankings remain stable across population sizes ($N \in \{50, 100, 200\}$), time horizons ($T \in \{25, 50, 100\}$), and stakes distributions (Gini coefficients 0.03-0.85). Stakes-weighting advantage increases with stakes heterogeneity: at high inequality (Gini = 0.78), DoCS outperforms equal voice by 4.2\% ($L = 0.644$ versus $0.618$). At low inequality (Gini = 0.03), advantage shrinks to 2.8\% ($L = 0.589$ versus $0.573$). At very low heterogeneity (Pareto $\alpha = 4.0$, Gini = 0.42), equal voice slightly outperforms stakes-weighting ($L = 0.594$ versus $0.584$), validating the theoretical claim that equal voice is optimal when stakes distribute uniformly.

This pattern confirms domain-appropriate mechanism selection: equal voice excels when exposure distributes homogeneously (monetary policy affecting all similarly, national defense providing public goods), while stakes-weighting excels when heterogeneous exposure exists (environmental justice, disability accommodations, minority rights).

% ============================================================================
\part{Implications and Extensions}
\label{part:implications}
\thispagestyle{partpage}
% ============================================================================

\section{Objections and Replies}
\label{sec:objections}

We address nine major objections to the framework, including two new objections on preference endogeneity and measurement impossibility.

\subsection{Objection 1: Infinite Regress}

\textit{``Who consents to the consent-holding rules? This generates infinite regress.''}

\textbf{Reply}: The regress is virtuous, not vicious. Each meta-level $n$ has its own $H_t(d^n)$: object-level policy ($d^0$) → constitutional rules ($d^1$) → amendment procedures ($d^2$) → founding acts ($d^3$). \citet{arendt1963} analyzes how constituent power creates constitutional order through founding acts outside existing legal frameworks, showing that the chain terminates pragmatically through revolution, convention, or ongoing practice—this \textit{is} politics. Demanding foundations outside consent-holding commits a category error like asking ``what causes causation?''

\subsection{Objection 2: Stakes Manipulation (Plutocracy)}

\textit{``If consent power follows stakes, agents will falsely claim high stakes to capture authority.''}

\textbf{Reply}: Measure stakes through revealed preference and behavioral proxies, not self-reports. Tax exposure comes from records; health outcomes from medical data; property threats from geographic location. A billionaire cannot falsely claim housing insecurity—consumption patterns contradict it. Additionally, friction $F(d)$ provides empirical falsification: if claimed high $\alpha$ still generates high observed friction, stakes were misweighted.

\subsection{Objection 3: Competence Sacrifice}

\textit{``Giving voice to high-stakes populations sacrifices expert competence on technical domains.''}

\textbf{Reply}: Postulate 1 addresses this directly through the legitimacy function $L = w_1 \cdot \alpha + w_2 \cdot P$. Different domains rationally weight these differently. Nuclear safety may set $w_2 >> w_1$ (prioritize competence); constitutional values set $w_1 >> w_2$ (prioritize consent). The framework doesn't prescribe universal voice maximization—it provides tools for domain-appropriate balance.

\subsection{Objection 4: Unresponsive Minorities}

\textit{``Small groups with extreme stakes can hold majorities hostage through veto threats.''}

\textbf{Reply}: This describes the tyranny of the minority—legitimate in some contexts, problematic in others. When stakes truly concentrate extremely (existential threats to minorities), veto rights may be justified. When stakes are fabricated or strategic, friction dynamics expose false claims. The framework makes these trade-offs explicit through stakes measurement rather than resolving them algorithmically.

\subsection{Objection 5: Future Generations}

\textit{``Future generations have stakes in climate policy but zero consent power—permanent $\alpha \approx 0$.''}

\textbf{Reply}: Proxy representation through guardianship institutions can raise effective $\alpha$. \citet{beckerman2001} articulate the principle that current generations hold Earth in trust for future generations, establishing fiduciary duties that constrain present choices even absent direct representation. Climate assemblies with youth quotas, constitutional provisions for sustainability, and fiduciary duties to future interests all operationalize this. The framework prescribes measuring whether such institutions actually incorporate future stakes or merely perform symbolic inclusion.

\subsection{Objection 6: Collective Action Problems}
\label{subsec:objection-collective-action}

\textit{``High-stakes diffuse populations (consumers, taxpayers) face coordination costs preventing mobilization—friction $F$ understates true misalignment.''}

\textbf{Reply}: Correct. Observed friction reflects both alignment and mobilization capacity. The framework acknowledges this: $\text{eff\_voice}_i$ includes capacity constraints. When diffuse populations cannot organize, institutional designers should proactively ensure voice through representatives, advocates, or procedural rights rather than waiting for friction to manifest.

\subsection{Objection 7: Cultural Relativism}

\textit{``Different cultures weight consent versus competence differently—this undermines universal applicability.''}

\textbf{Reply}: Theorem 3 addresses this. Content-level value relativism (different cultures prefer different $w_1/w_2$ weights) doesn't undermine structural analysis. The framework doesn't prescribe universal weights—it provides measurement tools applicable regardless of normative commitments. Cross-cultural variation in legitimacy functions becomes empirically testable rather than philosophically irresolvable.

\subsection{Objection 8: Preference Endogeneity}
\label{subsec:objection-endogeneity}

\textit{``Stakes and preferences are not exogenous---institutions shape what people want. If consent power allocation influences the very preferences it's meant to reflect, the framework is circular.''}

\textbf{Reply}: This is a serious objection connecting to the adaptive preferences literature in political philosophy and behavioral economics. \citet{anderson2006} demonstrates that democratic deliberation \textit{shapes} rather than merely aggregates preferences---citizens who participate in structured deliberation develop different policy views than those who do not. \citet{gerver2024} examine nudging against consent, showing how institutional design can manipulate preferences while appearing to respect autonomy. \citet{thaler2008} further establish that choice architecture effects are pervasive: the framing of options systematically influences which options are chosen, raising the question of whether any ``revealed preference'' can be treated as exogenous to institutional context.

The framework's response is threefold. First, the Bayesian learning dynamics in the Monte Carlo simulation (Section~\ref{sec:monte-carlo}) already model preference endogeneity: agents update beliefs based on institutional performance, and their stake assessments evolve with experience. Endogeneity is thus not an external threat to the model but a feature it incorporates. Second, the distinction between meta-level consent (constitutional) and object-level consent (policy) partially addresses the circularity charge. Meta-level preferences---about the \textit{rules} governing consent allocation---are more stable than object-level preferences about specific policies. Constitutional consent structures can be evaluated for alignment even if the preferences they shape at the policy level are endogenous, because the meta-level question (``do those with stakes have voice in setting the rules?'') is prior to the object-level question (``do current policies match current preferences?'').

Third, and most fundamentally, friction $F(d,t)$ provides an external check that does not depend on exogenous preferences. Friction is measured through \textit{behavioral} indicators---protests, strikes, litigation, exit---that are costly to produce and therefore resistant to preference manipulation. A regime can shape what people \textit{say} they want (survey responses, voting patterns under constraint), but it cannot easily suppress the behavioral manifestations of misalignment without incurring costs that are themselves observable. Even if preferences are entirely endogenous, persistent friction indicates misalignment that is not resolved by preference shaping. An institution that suppresses friction through preference manipulation rather than genuine consent alignment faces a specific diagnostic signature: low measured friction combined with high friction \textit{volatility} when the preference-shaping mechanism weakens. The framework can detect this: institutions with genuinely high $\alpha$ show low friction robustly across conditions, while institutions with manipulated preferences show low friction only under maintained manipulation---a distinction visible in time-series analysis of friction dynamics. Persistent regimes that collapse rapidly upon losing control of information or education channels (as documented in post-Soviet transitions and Arab Spring dynamics) exhibit precisely this pattern.

\subsection{Objection 9: Measurement Impossibility}
\label{subsec:objection-measurement}

\textit{``The framework's variables---stakes, effective voice, tolerance thresholds---are too complex to measure reliably. Without practical measurement, the framework is unfalsifiable.''}

\textbf{Reply}: Measurement difficulty is real but not unique to this framework---utility, social welfare, and democratic quality all face similar challenges. GDP purports to measure economic output through heroic aggregation of heterogeneous goods; the Human Development Index combines life expectancy, education, and income into a single scalar; the V-Dem project constructs over 450 indices of democratic quality from expert codings. None of these are measured with the precision of physical constants, yet all generate productive empirical research programs. The consent-holding framework's measurement challenges are of the same kind, not a different kind.

The response has three parts. First, Section~\ref{sec:historical-methodology}'s proxy construction demonstrates that $\alpha$ and $F$ \textit{can} be approximated using available data. Voter registration rates, union density, collective bargaining coverage, petition counts, strike days lost, litigation rates, and protest event frequencies all serve as observable proxies for the framework's theoretical variables. The proxies are imperfect, but imperfect measurement is not non-measurement. Appendix~\ref{sec:appendix-data} documents the data sources and coding protocols in detail.

Second, the framework makes ordinal predictions testable even without cardinal measurement. The claim that $\alpha_{\text{suffrage},1920} > \alpha_{\text{suffrage},1910}$ (consent alignment in suffrage domains increased after franchise extension) does not require precise measurement of either value---only their ordering. Similarly, the prediction that friction should decrease after franchise extensions (H1: higher $\alpha$ predicts lower future $F$) is testable through directional change in proxy indicators, not absolute magnitudes. Many of the framework's hypotheses concern comparative statics and directional dynamics rather than point estimates, making them robust to measurement noise.

Third, proxy validity can itself be assessed empirically through convergent validity testing. If the constructed $\alpha$ and $F$ proxies predict each other in the directions specified by H1--H4, this provides evidence both for the framework \textit{and} for the proxy validity. If they do not, either the measurement is inadequate or the theory is wrong---both are falsifiable outcomes. The inter-proxy correlations reported in the historical case studies provide precisely this kind of convergent validation.

More broadly, the measurement impossibility objection proves too much. If the standard for admissible frameworks is that all variables must be perfectly measurable, then no normative framework in political philosophy survives---not welfare economics (utility is unobservable), not democratic theory (``the will of the people'' is a construct), not justice theory (``fairness'' requires contested interpersonal comparisons). The consent-holding framework's measurement program is more transparent and more testable than most: it specifies exactly which observable proxies correspond to which theoretical variables, and it makes falsifiable predictions about the relationships between them. The measurement challenge is practical, not conceptual: it calls for better data and more sophisticated proxies, not abandonment of the measurement enterprise. Indeed, the very act of attempting measurement---constructing proxies, testing hypotheses, refining indicators---advances understanding even when individual measurements are imprecise. The V-Dem project's 450+ democracy indicators began as crude expert codings and have been progressively refined through methodological innovation over two decades. The consent-holding framework's measurement program is at an earlier stage of the same trajectory: the proxies constructed here are initial approximations that future empirical work will refine, challenge, and improve. The alternative---treating legitimacy as unmeasurable and therefore outside the domain of empirical research---abandons the field to pure normative speculation, which has proved insufficient for guiding institutional design.

\section{Weight Determination as Endogenous Constitutional Problem}
\label{sec:weight-determination}

The meta-legitimacy challenge---determining the weights $w_1$ and $w_2$ in the legitimacy function $L(d,t) = w_1 \cdot \alpha(d,t) + w_2 \cdot P(d,t)$ without presupposing answers to the legitimacy question---requires extending the framework to treat weight-determination itself as a domain subject to consent-holding analysis. This section develops the argument that weight determination is not an embarrassing free parameter but an endogenous constitutional problem with its own institutional dynamics.

\subsection{The Problem of Weights}
\label{subsec:problem-of-weights}

Every normative framework that balances competing values faces the weighting problem. Utilitarian calculations require interpersonal utility comparison. Rawlsian maximin requires a metric for ``worst off.'' Capabilities approaches require a list and relative weighting of capabilities. The consent-holding framework's version---how much should consent alignment matter relative to performance?---is structurally identical. What distinguishes the present approach is treating the weighting problem as itself amenable to consent-holding analysis rather than resolving it through external philosophical commitment. \citet{beinhocker2025fair} argues that dimensions of social contracts exhibit non-substitutability, suggesting that the legitimacy function $L = f(\alpha, P)$ may require min-operator structure rather than purely additive composition---a point with significant implications for the weight determination problem, since non-substitutability implies that deficiency in one dimension cannot be compensated by surplus in another.

The connection to cooperative game theory is direct. \citet{shapley1953value} establishes a unique value function for cooperative games satisfying symmetry, efficiency, dummy player, and additivity axioms. The Shapley value determines each player's expected marginal contribution to every possible coalition---analogous to determining each legitimacy dimension's marginal contribution to overall institutional stability. Applied to weight determination, the Shapley framework suggests that $w_1$ and $w_2$ should reflect the marginal contribution of consent alignment and performance, respectively, to friction minimization across the space of possible institutional configurations. Formally, the Shapley-inspired weight for consent alignment would be:
\begin{equation}
w_1^{\text{Shapley}} = \sum_{S \subseteq \{P\}} \frac{|S|!(1 - |S|)!}{2!} \left[ v(S \cup \{\alpha\}) - v(S) \right]
\label{eq:shapley-weight}
\end{equation}
where $v(S)$ is the friction-reducing value of the legitimacy dimensions in coalition $S$. This reduces to a simple calculation: $w_1$ reflects the average marginal contribution of $\alpha$ to friction reduction, both alone and in combination with $P$. The Shapley approach transforms the weight question from a normative choice into a functional one: weights should reflect each dimension's causal efficacy in producing legitimate outcomes.

\citet{binmore1989outside} demonstrate experimentally that outside options---the alternatives available to bargaining parties if negotiation fails---strongly constrain bargaining outcomes. Applied to constitutional weight determination, this means the weights that emerge from institutional bargaining depend on exit threats: populations capable of rebellion, secession, or non-compliance shift weights toward $w_1$ (consent alignment), while populations dependent on state capacity for survival shift weights toward $w_2$ (performance). The empirical observation that democratic transitions often follow periods where the cost of suppression exceeds the cost of inclusion \citep{acemoglu2000why} is consistent with this interpretation.

\citet{compte2010coalitional} extend Nash bargaining to multi-party settings where coalitions can form. In the weight-determination context, this captures the reality that constitutional moments involve multiple stakeholder groups with different weight preferences negotiating simultaneously. The resulting weights reflect not just bilateral bargaining but the entire coalition structure---which groups can credibly threaten exit, which can form blocking coalitions, and which have overlapping interests that enable cooperation.

Constitutional conventions, from Philadelphia to post-apartheid South Africa, exhibit precisely these coalition dynamics. The South African case is instructive: the ANC's bargaining position (massive popular support, capacity for sustained resistance) shifted weights toward $w_1$ (consent alignment), while the National Party's position (control of state apparatus, economic infrastructure) preserved significant $w_2$ (performance capacity). The resulting constitutional settlement---strong rights protections with independent judiciary and reserve bank independence---represents a negotiated weight configuration reflecting the coalition structure at the constitutional moment. The consent-holding framework predicts that this configuration generates stable legitimacy only insofar as the negotiated weights continue to reflect the evolving stakeholder structure; as the coalition dynamics shift (through demographic change, economic transformation, or institutional erosion), the weight configuration faces renegotiation pressure manifesting as friction.

\subsection{Four-Layer Architecture}
\label{subsec:four-layer-architecture}

Weight determination occurs through a four-layer architecture that generates finite recursion rather than infinite regress:

\textbf{Layer 1 (Constitutional Foundation)}: Weight determination occurs at the constitutional level, governed by the same legitimacy calculus but with astronomically high stakes (affecting all future decisions in all domains). This follows \citeauthor{brennan1985}'s \citeyearpar{brennan1985} distinction between constitutional and post-constitutional choice, creating finite recursion: weights at the constitutional level are determined by a higher-order consent process that terminates in founding acts, revolutionary moments, or ongoing constitutional practice. Constitutional-level friction $F(d_w, t)$ for weight-determination decisions becomes observable through reform pressure, constitutional amendment campaigns, and regime-change movements.

\textbf{Layer 2 (Empirical Calibration)}: Historical constitutional reforms reveal weight preferences through friction minimization. The optimization problem:
\begin{equation}
\arg\min_{w_1, w_2} \mathbb{E}[F(d,t; w_1, w_2)]
\label{eq:weight-calibration-empirical}
\end{equation}
estimates weights from observed institutional stability patterns. Franchise expansions reveal upward pressure on $w_1$ (populations demanded more consent alignment). Technocratic delegations---central bank independence, public health authority---reveal contexts where $w_2$ was elevated (populations accepted reduced consent alignment in exchange for superior performance). Codetermination mandates, participatory governance reforms, and citizens' assemblies provide quasi-experimental variation in weight configurations with measurable friction outcomes.

\textbf{Layer 3 (Axiomatic Constraints)}: Rather than arbitrary weight assignment, theoretical bounds can be derived from stability requirements. Any society avoiding persistent friction must satisfy:
\begin{equation}
\frac{w_2}{w_1} > f(\text{Var}[s_i(d)])
\label{eq:axiomatic-bound}
\end{equation}
where $f$ captures the minimum competence-weighting required for technical domains with high stakes variance. Symmetrically, societies avoiding legitimacy collapse must satisfy a floor on $w_1$ in domains where affected populations can organize resistance:
\begin{equation}
w_1 > g(\text{capacity}(S_d))
\label{eq:consent-floor}
\end{equation}
where $g$ captures the minimum consent-weighting required given the organizing capacity of the affected population. Populations with high organizing capacity (dense social networks, shared grievances, low coordination costs) require higher $w_1$ to avoid generating unsustainable friction. These axiomatic constraints limit the empirical search space, preventing overfitting while ensuring sociologically plausible configurations.

\textbf{Layer 4 (Computational Validation)}: Dynamic Monte Carlo with evolutionary weight adjustment validates the unified architecture. Societies initialize with random weights, adjust based on friction feedback within axiomatic bounds, and converge to stable configurations. The computational experiments in Section~\ref{sec:monte-carlo} demonstrate that only weight distributions satisfying Layer 3's constraints produce long-run stability, while empirical calibration (Layer 2) reveals which specific values minimize historical friction. The convergence of computational and empirical weight estimates provides mutual validation.

\subsection{Weights as Institutional Outcomes}
\label{subsec:weights-as-outcomes}

This unified framework treats weight determination not as an external parameter requiring normative resolution but as an endogenous feature of consent-holding structures. The legitimacy function can evaluate its own parameters when framed at appropriate meta-levels---analogous to how G\"{o}del numbering allows arithmetic self-reference without circularity. The self-referential structure is benign: constitutional-level consent processes determine the weights that govern object-level legitimacy evaluation, and constitutional-level legitimacy is itself subject to friction dynamics observable through institutional reform and resistance.

Practically, societies determine weights through three mechanisms, each corresponding to a different correction rule:

\textbf{Constitutional conventions} represent explicit negotiation over the consent-performance trade-off. The Philadelphia Convention of 1787, the French National Assembly of 1789, the post-apartheid South African Constitutional Assembly of 1996---each involved explicit bargaining over how much authority should rest on popular consent versus institutional competence. In consent-holding terms, these are high-stakes moments where $(w_1, w_2)$ are set through a meta-level consent process with its own alignment dynamics. The framework predicts that conventions dominated by incumbent elites will produce weight configurations favoring $w_2$ (performance-legitimacy), while conventions with broad popular participation will produce configurations favoring $w_1$ (consent-alignment). The historical record is broadly consistent: elite-dominated constitutional processes (e.g., Meiji Japan, Bismarckian Germany) emphasized state capacity; mass-participatory processes (e.g., post-independence India) emphasized democratic inclusion.

\textbf{Revolutionary moments} represent forced renegotiation when accumulated friction exceeds tolerance. Here weight adjustment is not negotiated but imposed: the old weight configuration has generated unsustainable friction, and the new configuration reflects the balance of power among revolutionary actors. The framework predicts that revolutionary weight adjustments are discontinuous---large shifts in $(w_1, w_2)$ occurring over short periods---and that post-revolutionary configurations are initially unstable, requiring subsequent consolidation through the other two mechanisms. The French Revolution illustrates the pattern: the ancien r\'{e}gime's weight configuration (extremely high $w_2$, near-zero $w_1$ for non-aristocratic populations) generated unsustainable friction; the revolutionary reconfiguration swung to high $w_1$ (popular sovereignty, universal rights); the subsequent instability (Terror, Thermidor, Consulate, Empire) reflects the challenge of stabilizing a radically new weight configuration without the institutional infrastructure to sustain it.

\textbf{Gradual reform} represents incremental weight adjustment through legislative and judicial action. Franchise extensions, administrative agency creation, judicial review expansion, and regulatory reform all shift weights without constitutional rupture. This mechanism generates the smoothest friction trajectories and is most amenable to the empirical calibration strategy in Layer~2. The prediction is that gradual reforms produce weight adjustments proportional to accumulated friction: small friction generates small reforms, while persistent high friction eventually produces larger institutional restructuring. The British constitutional tradition exemplifies this mechanism: the Reform Acts of 1832, 1867, 1884, and 1918 each shifted $(w_1, w_2)$ incrementally toward greater consent alignment, driven by accumulated friction (Chartist agitation, Reform League pressure, suffragette campaigns) but channeled through existing institutional pathways rather than revolutionary rupture. The consent-holding framework predicts that this gradualist pattern should be associated with lower overall friction volatility and more stable long-run equilibria---a prediction testable through cross-polity comparison of reform trajectories.

Future empirical work implementing this architecture through quantified historical case studies can estimate $(w_1^*, w_2^*)$ from constitutional reform patterns across societies, testing whether the convergence predicted by the Monte Carlo simulations matches observed institutional evolution.

\section{Research Agenda}
\label{sec:research-agenda}

The consent-holding framework opens several research frontiers, each requiring different methodological approaches and offering distinct theoretical payoffs. This section maps the most promising directions.

\subsection{Quadratic Voting and Novel Consent Mechanisms}
\label{subsec:quadratic-voting}

Conventional voting mechanisms allocate consent power uniformly: one person, one vote, regardless of stakes. Quadratic voting (QV) offers an alternative that directly connects to the consent-holding framework's core insight that legitimacy improves when consent power tracks stakes.

\citet{lalley2018quadratic} show that QV---in which voters purchase votes at quadratic cost, paying $n^2$ tokens for $n$ votes---approximates efficient outcomes under mild conditions. The mechanism allows voters to express preference intensity, concentrating consent power on issues where their stakes are highest. In consent-holding terms, QV is an institutional mechanism for raising $\alpha(d)$ by enabling endogenous stakes revelation: voters who care more about an issue (higher $s_i(d)$) rationally purchase more votes, shifting consent power toward affected parties.

\citet{posner2017quadratic} extend this analysis to public goods provision, demonstrating that QV approximates first-best allocation where conventional voting fails. The consent-holding framework provides a natural evaluation metric: does QV implementation raise measured $\alpha(d)$ relative to uniform voting? Does it reduce friction $F(d,t)$ in domains where preference intensity varies widely? These are testable predictions.

Alternative mechanisms deserve parallel analysis. \textbf{Liquid democracy} allows vote delegation, enabling consent power to flow to trusted representatives---a mechanism for raising effective voice $\text{eff\_voice}_i$ through voluntary expertise concentration. The consent-holding framework predicts that liquid democracy should raise $\alpha$ in domains where expertise is concentrated but stakes are distributed (e.g., technical policy), while potentially \textit{lowering} $\alpha$ in domains where delegation chains concentrate power among a small number of super-delegates, replicating the Rousseauian failure mode of formal inclusion masking effective concentration.

\textbf{Conviction voting} weights votes by duration of commitment, approximating temporal stakes. In consent-holding terms, the time-weighting introduces a proxy for stakes intensity: agents who maintain positions over longer periods signal higher stakes through costly persistence. The framework predicts that conviction voting should reduce friction in domains with stable, long-duration stakeholder interests (infrastructure, environmental protection) but may fail in domains requiring rapid response where extended commitment periods delay necessary correction.

\textbf{Futarchy} separates value judgments (voted on) from implementation predictions (decided by prediction markets), corresponding to a domain-specific allocation where $w_1$ governs value domains and $w_2$ governs prediction domains. This is perhaps the most direct institutional instantiation of the consent-performance trade-off in Postulate~1: voters determine \textit{what} outcomes to pursue (consent alignment), while markets determine \textit{how} to achieve them (performance optimization). The consent-holding framework provides a natural evaluation: does the separation actually reduce friction, or does it generate new friction at the value-implementation boundary?

Each of these mechanisms represents a different approach to raising $\alpha$ and reducing $F$. The consent-holding framework provides the common evaluation language---and the historical case studies in Part~III provide the baseline against which innovations should be measured.

\subsection{Institutional Experiments}
\label{subsec:institutional-experiments}

Citizens' assemblies represent natural experiments in consent-alignment raising. The Irish Citizens' Assembly on constitutional reform \citep{farrell2019}, the French Convention Citoyenne pour le Climat \citep{courant2021}, and the UK Climate Assembly \citep{ukclimate2020} all implemented temporary high-$\alpha$ governance in specific domains. The consent-holding framework provides outcome metrics for evaluating these innovations: did $\alpha$ increase in the target domain? Did friction decrease? Were the effects persistent or transient?

\citet{bovens2014oxford} provide a comprehensive framework for public accountability that complements the consent-holding approach: accountability mechanisms serve as correction rules (Layer 3 of the doctrinal analysis), enabling institutional adjustment when friction signals misalignment. The Bovens typology distinguishes political, legal, administrative, professional, and social accountability---each corresponding to a different friction channel in the consent-holding framework. Political accountability operates through electoral correction (periodic consent reallocation); legal accountability through judicial contestation (Lockean revocability); social accountability through public deliberation and media scrutiny (Rousseauian co-authorship). The consent-holding framework provides the unified metric---did friction decrease?---for evaluating which accountability channel is most effective in which domain.

The research agenda here is to design randomized evaluations of participatory governance innovations using $\alpha$ and $F$ as primary outcome variables, moving beyond the current literature's focus on process measures (``did participation increase?'') toward outcome measures (``did consent alignment improve and friction decrease?'').

Three experimental designs are particularly promising. First, \textbf{within-polity comparison}: randomly assign deliberative processes to some domains while maintaining status quo governance in others, measuring differential friction trajectories. The Irish model---where citizens' assemblies addressed specific constitutional questions (marriage equality, abortion, climate) while other domains retained standard legislative process---provides a natural template, though with selection effects that a truly randomized design would eliminate.

Second, \textbf{cross-polity matching}: compare jurisdictions that adopted participatory innovations with matched controls, using synthetic control methods to estimate causal effects on $\alpha$ and $F$. Porto Alegre's participatory budgeting \citep{wampler2007participatory}, now replicated in thousands of municipalities worldwide, provides the largest quasi-experimental base. The consent-holding framework sharpens the evaluation question: not merely ``did participation increase?'' (a tautological measure for participatory programs) but ``did friction in budget-related domains decrease relative to matched controls?''

Third, \textbf{mechanism variation}: within a single institutional reform, randomize the specific consent allocation mechanism (QV vs. sortition vs. stakeholder panels) to identify which mechanisms most efficiently raise $\alpha$ in which domain types. This design addresses the framework's core empirical question: is there a universal optimal consent allocation mechanism, or does optimal allocation depend on domain characteristics (stakes distribution, expertise requirements, scale)?

\subsection{AI Governance Applications}
\label{subsec:ai-governance}

Algorithmic decision-making poses the consent-holding challenge in acute form. Automated systems make consequential decisions affecting millions---credit scoring, content moderation, criminal risk assessment, resource allocation---with consent power concentrated among system designers and operators. The affected populations (those scored, moderated, assessed, or allocated to) typically have near-zero effective voice in rule formation.

The consent-holding framework predicts that AI governance will follow the same dynamics as other low-$\alpha$ regimes: friction accumulation (public backlash against algorithmic decisions), threshold effects (regulatory intervention once friction exceeds tolerance), and institutional correction (new governance structures raising stakeholder voice). The EU AI Act, the US Executive Order on AI Safety, and platform governance reforms represent early instances of these predicted dynamics.

The AI governance domain is particularly valuable for the consent-holding framework because it allows observation of consent-alignment dynamics in real time, without the historiographic reconstruction required for the historical case studies. The timeline is compressed: platform governance has moved from near-zero $\alpha$ (unilateral content moderation policies) to emerging correction mechanisms (oversight boards, transparency reports, regulatory frameworks) in less than a decade. This compression provides a natural laboratory for testing whether the framework's predictions hold at accelerated timescales.

Three mechanisms for raising $\alpha$ in AI systems merit investigation. First, \textbf{participatory design}: involving affected communities in system specification, training data selection, and performance metric definition. This corresponds to raising $\alpha$ at the design stage---ensuring that those with stakes in system outcomes have voice in system construction. The framework predicts that participatory AI design should reduce downstream friction (fewer contested decisions, fewer regulatory interventions) relative to developer-unilateral design, at the cost of slower deployment and higher coordination costs.

Second, \textbf{algorithmic auditing}: external review as a correction rule, enabling friction signals to reach decision-makers who can modify system behavior. In the four-layer doctrinal analysis, auditing is a Lockean correction mechanism: it preserves designer authority but makes it conditional on ongoing accountability. The framework's prediction is that auditing reduces friction only when audit findings are actionable---when they trigger actual system modifications. Auditing without correction authority is the algorithmic equivalent of formal revocability without effective capacity: a Lockean failure mode.

Third, \textbf{contestability mechanisms}: structured pathways for affected individuals to challenge automated decisions and trigger human review---the algorithmic analogue of Lockean revocability. The key design question is granularity: at what level should contestation operate? Individual decision appeals (``this credit score is wrong'') raise $\alpha$ at the case level but leave system-level consent allocation unchanged. System-level contestation (``this algorithm systematically disadvantages group $X$'') raises $\alpha$ at the design level but requires collective organization and technical expertise---reintroducing the collective action problems of Objection~6. The consent-holding framework suggests that effective AI governance requires contestability at \textit{both} levels: individual appeals for case-level correction and institutional channels for system-level consent alignment.

\subsection{Cross-National Panel Studies}
\label{subsec:cross-national-panel}

The proxied $\alpha$ and $F$ time series constructed in Part~III for individual case studies can be extended to cross-national panel analysis. The V-Dem dataset provides democracy indices across 179 countries from 1789 to the present, with over 450 indicators that can serve as components of $\alpha$ and $F$ proxies. OECD and ILO data provide labor-market variables (union density, collective bargaining coverage, strike data) relevant to industrial-domain consent alignment. Pew Research Center and Gallup World Poll data provide attitudinal measures of governance satisfaction and institutional trust that can proxy for friction.

The panel regression specification from Section~\ref{sec:historical-methodology}:
\[
F_{d,t} = \beta_0 + \beta_1 \cdot \alpha_{d,t} + \beta_2 \cdot P_{d,t} + \gamma \cdot X_{d,t} + \mu_d + \lambda_t + \varepsilon_{d,t}
\]
could be estimated across countries and domains, testing H1 ($\beta_1 < 0$: higher $\alpha$ predicts lower $F$), H2 ($\beta_1$ magnitude increases with stakes concentration), and H5 ($\beta_2 < 0$: high $P$ partially compensates for low $\alpha$). Instrumental variable strategies exploiting franchise expansions, codetermination mandates, and exogenous governance shocks provide identification.

The most powerful design would construct a domain-by-country-by-year panel, enabling within-domain and within-country analysis that controls for unobserved heterogeneity. Franchise expansions provide sharp variation in political-domain $\alpha$; codetermination mandates provide sharp variation in economic-domain $\alpha$; platform governance reforms provide variation in digital-domain $\alpha$. Each represents a natural experiment in consent-alignment change with measurable friction consequences.

A specific research design for the cross-national panel: construct annual $\alpha$ and $F$ proxies for 30 OECD countries across three domains (political, economic, digital) from 1960 to present, using V-Dem data for political $\alpha$, OECD labor statistics for economic $\alpha$, and Freedom House internet freedom scores for digital $\alpha$. Friction proxies would combine protest event data (GDELT/ACLED), strike data (ILO), and internet censorship circumvention rates. The resulting 30-country $\times$ 3-domain $\times$ 60-year panel ($\sim$5,400 observations) would have sufficient power to test H1--H5 while controlling for country and domain fixed effects. Instrumental variable identification could exploit the timing of EU directives (which impose consent-alignment requirements on member states at exogenously determined dates) to address endogeneity concerns.

\subsection{Extensions to the Formal Framework}
\label{subsec:formal-extensions}

Several directions extend the formal apparatus. \textbf{Mechanism design for consent allocation}: can we characterize the class of consent-allocation mechanisms satisfying incentive compatibility (no agent benefits from misrepresenting stakes) while maximizing $\alpha$? The revelation principle suggests that direct mechanisms---asking agents to report stakes and allocating consent power accordingly---should be sufficient, but the incentive-compatible mechanism may sacrifice efficiency relative to the first-best.

\textbf{Welfare theorems for optimal $H_t(d)$}: under what conditions does there exist a consent-holder mapping that is Pareto optimal (no alternative mapping reduces friction for all agents)? Are competitive equilibria in consent allocation efficient? The analogy to the fundamental welfare theorems of economics is suggestive but not straightforward, because consent power is not a private good.

\textbf{Network effects in coalition formation}: when agents form coalitions to concentrate consent power, network structure constrains which coalitions are feasible. The framework could incorporate network topology to predict which stakeholder groups can effectively organize and which face structural barriers to collective action---connecting to the collective action problems raised in Objection~6 (Section~\ref{subsec:objection-collective-action}). The key insight from network science is that consent power is not merely allocated but \textit{activated}: formal authority ($C_{i,d} > 0$) translates to effective voice only when network connectivity enables coordination. This explains the paradox of formally empowered but practically voiceless populations---consumers with market choice but no organized voice, voters with ballots but no coordinated influence.

\textbf{Learning and institutional memory}: institutions accumulate knowledge about which consent allocations minimize friction---but they also forget, and personnel turnover disrupts institutional memory. Modeling institutional learning dynamics could explain why some societies cycle between high and low $\alpha$ regimes rather than converging monotonically. The Monte Carlo simulations in Section~\ref{sec:monte-carlo} model agent-level learning (Bayesian updating, Thompson sampling, Q-learning) but assume institutional memory is perfect---an assumption that historical experience contradicts. Introducing institutional memory decay, knowledge loss during regime transitions, and path-dependent learning trajectories could generate the non-monotonic democratization patterns documented in the empirical literature.

These extensions connect to companion formalizations in the broader research program. \citet{farzulla2025aoc} derives the consent-friction framework from a single axiom, establishing friction as the canonical obstruction to coordination in multi-agent systems---the formal foundation for the friction dynamics documented throughout this paper. \citet{farzulla2025rom} embeds these dynamics within a scale-relative formalism where legitimacy enters as survival probability in the replicator equation, providing the dynamical foundation for the persistence patterns observed in the historical case studies. \citet{farzulla2025stakes} further develops the stakeholder voice dimension, analyzing populations that bear stakes in governance outcomes but lack effective institutional channels for exercising consent power---the structural condition that generates the most persistent friction patterns in the historical case studies. Together, these companion papers and the extensions outlined here constitute a research program aimed at making political legitimacy as tractable an object of formal analysis as market equilibrium or strategic interaction.

The research agenda as a whole is unified by a single methodological commitment: treating legitimacy as a measurable structural property rather than an abstract normative ideal. The consent-holding framework provides the measurement apparatus; the historical case studies provide calibration data; the computational simulations provide parameter-space exploration; and the extensions outlined here provide the theoretical frontier. If the framework's core predictions survive empirical testing---if higher $\alpha$ does predict lower $F$, if threshold effects are real, if reform pressure follows persistent misalignment---then political legitimacy joins the growing list of social phenomena amenable to rigorous quantitative analysis without sacrificing normative depth.

\section{Conclusion}
\label{sec:conclusion}

This monograph developed consent-holding theory, an axiomatic framework for measuring political legitimacy across heterogeneous governance domains. By operationalizing legitimacy as stakes-weighted consent alignment $\alpha(d,t)$ and friction as $F(d,t)$, the framework bridges normative democratic theory and empirical prediction. Five theorems establish that consent-holding is structurally necessary, friction is inevitable under plural preferences, legitimacy is measurable through alignment, competence and consent trade off in domain-specific ways, and this structural analysis survives value relativism.

Historical validation across nine cases spanning two centuries demonstrates the framework's predictive power: persistent misalignment between stakes and voice generates escalating friction until institutional reform or suppression occurs. Suffrage expansion, abolition movements, labor organizing, civil rights struggles, LGBT inclusion, corporate governance evolution, platform governance rebellions, and climate governance all exhibit the predicted dynamics. Computational validation through Monte Carlo simulation confirms that stakes-weighted mechanisms minimize friction while maintaining performance across diverse preference distributions.

The framework's limitations warrant acknowledgment. Stakes measurement remains conceptually contested and practically difficult---material exposure, capability impact, and existential threat often diverge. Effective voice measurement requires rich capacity data often unavailable cross-nationally. Temporal dynamics and institutional memory complicate longitudinal analysis. Aggregation across domains raises normative questions about weighting. These challenges suggest complementary methodologies: qualitative case studies illuminating causal mechanisms, experimental studies isolating specific dynamics, and computational modeling exploring parameter spaces.

The framework's central contribution lies in making legitimacy measurable without prescribing universal institutions. Just as markets can be analyzed without presuming capitalism's moral superiority, consent-holding structures can be measured without presuming democracy's unique virtue. This analytical stance enables rigorous comparison: Which systems achieve high $\alpha$ efficiently? How do alignment-performance trade-offs vary across domains? What institutional innovations shift legitimacy frontiers outward?

Political legitimacy has remained philosophically contested yet empirically elusive for millennia. Consent-holding theory doesn't resolve normative disputes---it provides tools for measuring their institutional consequences. By operationalizing alignment, friction, and legitimacy through $\alpha(d,t)$, $F(d,t)$, and $L(d,t) = w_1 \cdot \alpha + w_2 \cdot P$, the framework transforms legitimacy from abstract ideal into measurable structural property. The resulting empirical agenda promises to ground political philosophy in institutional reality while informing governance design with rigorous theory.

\textbf{Code and Data Availability}: All simulation code, Monte Carlo experiment implementations, and computational validation scripts are archived on Zenodo (\href{https://doi.org/10.5281/zenodo.17684679}{10.5281/zenodo.17684679}) and available via GitHub (\url{https://github.com/studiofarzulla/consent-holding-theory}). Complete replication materials include Python implementations of all four dynamic mechanisms (Bayesian learning, Thompson sampling, Q-learning, gradient descent), convergence analysis scripts, and figure generation code.

% ============================================================================
% APPENDIX
% ============================================================================

\appendix

\section{Appendix A: Robustness Checks}
\label{sec:robustness}

Monte Carlo results remain stable across parameter variations and stakes distribution specifications. Table \ref{tab:robustness_parameters} shows mechanism performance across nine combinations of population size ($N \in \{50, 100, 200\}$) and time periods ($T \in \{25, 50, 100\}$). Stakes-weighted mechanisms outperform equal voice in 8 of 9 parameter combinations (88.9\% rank consistency), with mean legitimacy advantage of 0.020 (95\% CI: [0.009, 0.030]). Statistical significance holds across specifications: one-sided t-test comparing stakes-weighted versus equal voice yields $p < 0.0044$ with Cohen's $d = 1.30$ (large effect size).

Table \ref{tab:robustness_distributions} demonstrates that mechanism performance tracks stakes heterogeneity as predicted theoretically. At high inequality (Gini = 0.78), stakes-weighted mechanisms achieve $L = 0.644$ versus equal voice $L = 0.618$ (4.2\% advantage). At low inequality (Gini = 0.03), this advantage shrinks to 2.8\% ($L = 0.589$ vs $L = 0.573$). Extreme Pareto distributions ($\alpha = 1.2$, Gini = 0.85) show stakes-weighting's largest advantage (6.3\%: $L = 0.658$ vs $L = 0.619$). Notably, at very low heterogeneity (Pareto $\alpha = 4.0$, Gini = 0.42), equal voice slightly outperforms stakes-weighting ($L = 0.594$ vs $L = 0.584$)—validating the framework's claim that equal voice is optimal when stakes distribute uniformly.

Figure \ref{fig:robustness-heatmap} visualizes legitimacy across the $(N, T)$ parameter space for three representative mechanisms. Stakes-weighted performance improves with larger populations and longer time horizons, while random assignment shows minimal sensitivity to parameters, confirming convergence validity.

\begin{figure*}[htbp]
\centering
\includegraphics[width=0.95\textwidth]{robustness_parameter_heatmap.pdf}
\caption{Robustness check: Final legitimacy across population size ($N$) and time periods ($T$) for three mechanisms. Stakes-weighted (left) shows consistent performance across parameters. Equal voice (center) improves with larger populations but remains below stakes-weighted. Random assignment (right) performs poorly universally, establishing baseline. Color intensity indicates legitimacy $L$ (darker = higher).}
\label{fig:robustness-heatmap}
\end{figure*}

\begin{table*}[t]
\centering
\caption{Robustness Check: Parameter Sensitivity}
\label{tab:robustness_parameters}
\begin{tabular}{ccccccc}
\toprule
N & T & Equal Voice & Stakes-Weighted & Plutocracy & Random & Expert \\
\midrule
50 & 25 & 0.579 & 0.609 & 0.570 & 0.487 & 0.575 \\
50 & 50 & 0.582 & 0.620 & 0.563 & 0.477 & 0.560 \\
50 & 100 & 0.582 & 0.623 & 0.584 & 0.483 & 0.563 \\
100 & 25 & 0.601 & 0.622 & 0.604 & 0.493 & 0.584 \\
100 & 50 & 0.602 & 0.639 & 0.604 & 0.492 & 0.612 \\
100 & 100 & 0.611 & 0.615 & 0.607 & 0.519 & 0.622 \\
200 & 25 & 0.619 & 0.622 & 0.619 & 0.528 & 0.622 \\
200 & 50 & 0.637 & 0.636 & 0.614 & 0.551 & 0.611 \\
200 & 100 & 0.624 & 0.629 & 0.625 & 0.507 & 0.617 \\
\bottomrule
\end{tabular}
\end{table*}

% Appendix B (Extended Literature Synthesis) REMOVED — content promoted to
% main body Sections 2.9--2.11 and expanded by lit-expander agent.

\section{Appendix B: Extended Dynamic Comparison Tables}
\label{sec:appendix-dynamics}

This appendix consolidates results from the dynamic comparison report into publication-ready tables. It preserves details that may be excessive for the main paper but useful for readers evaluating model behavior under alternative temporal assumptions.

\begin{table*}[t]
\centering
\caption{Mechanism Rankings Across Dynamic Modes (Final Alignment $\alpha$)}
\label{tab:appendix-mode-rankings}
\begin{tabular}{lccccc}
\toprule
Mode & DoCS (Stakes) & Equal Voice & Plutocracy & Expert Rule & Random \\
\midrule
Static (baseline) & \textbf{0.6274} & 0.6042 & 0.5962 & 0.5919 & 0.4884 \\
Learning (Bayesian) & \textbf{0.8722} & 0.8695 & 0.8600 & 0.8418 & 0.7612 \\
Social (DeGroot) & \textbf{0.7377} & 0.7278 & 0.7127 & 0.7082 & 0.5976 \\
Stakes (endogenous) & \textbf{0.8932} & 0.8729 & 0.8742 & 0.8307 & 0.6614 \\
\bottomrule
\end{tabular}
\end{table*}

\begin{table*}[t]
\centering
\caption{Convergence Speed and Friction Reduction by Dynamic Mode}
\label{tab:appendix-convergence-speed}
\begin{tabular}{lcccc}
\toprule
Metric & DoCS & Equal Voice & Plutocracy & Random \\
\midrule
Time to 90\% final $\alpha$ (Learning) & $\sim$18 & $\sim$20 & $\sim$22 & $\sim$35 \\
Time to 90\% final $\alpha$ (Social) & $\sim$35 & $\sim$38 & $\sim$40 & N/A \\
Time to 90\% final $\alpha$ (Stakes) & $\sim$12 & $\sim$15 & $\sim$16 & $\sim$30 \\
\midrule
Final friction, Static & 105.5 & 113.8 & 115.9 & --- \\
Final friction, Learning & \textbf{1.55} & 1.76 & 2.08 & --- \\
Final friction, Social & \textbf{1.00} & 1.03 & 1.11 & --- \\
Final friction, Stakes & \textbf{29.9} & 34.3 & 35.0 & --- \\
\bottomrule
\end{tabular}
\end{table*}

\paragraph{Interpretive note.}
Two distinctions are important for inference. First, ranking robustness (DoCS first across modes) is stronger than any single-mode effect size claim. Second, high terminal $\alpha$ under stakes evolution reflects path-dependent reinforcement and can coexist with slower friction collapse than learning/social modes.

% Appendix D (Methodological Claim Boundaries) REMOVED — content promoted
% to Section 17 (Dynamic Validation) subsection in Phase 2 integration.

% Appendix E (Additional Historical Cases) REMOVED — civil rights and climate
% content promoted to main body Sections 10 and 14.

% Appendix F (Extended Social Contract Architecture) REMOVED — content promoted
% to main body Section 5 via normative-writer agent.

% Appendix G (Hobbes-Locke-Rousseau Text-to-Formal Mapping) REMOVED — content
% promoted to main body Section 5 via normative-writer agent.

\section{Appendix C: Data Sources and Coding Protocols}
\label{sec:appendix-data}

This appendix documents the data sources, variable construction, and coding protocols used to construct the $\alpha$ and $F$ proxies in the historical case studies (Part~III). Transparency in proxy construction is essential given the measurement challenges discussed in Objection~9 (Section~\ref{subsec:objection-measurement}).

\subsection{Primary Datasets}

\textbf{Varieties of Democracy (V-Dem), v14.} The V-Dem project provides over 450 indicators of democratic governance across 179 countries from 1789 to 2023. Variables used in this paper include: electoral democracy index (v2x\_polyarchy), participatory component (v2x\_partip), deliberative component (v2x\_delibdem), suffrage share (v2x\_suffr), freedom of expression (v2x\_freexp), and civil society participation (v2x\_cspart). These variables serve as components of $\alpha$ proxies in the suffrage, civil rights, and cross-national analyses. V-Dem data are expert-coded with documented inter-coder reliability (median pairwise agreement typically exceeding 0.85) and are freely available at \url{https://www.v-dem.net}.

\textbf{OECD/ILO Labour Statistics.} Union density (trade union members as percentage of wage earners), collective bargaining coverage (share of workers covered by collective agreements), and days lost to industrial action are drawn from the OECD Labour Force Statistics database and ILO ILOSTAT. These variables proxy $\alpha$ in industrial and economic domains: higher union density corresponds to higher stakeholder voice in workplace governance, while strike activity proxies friction $F$ in labor-management relations. Coverage spans OECD member states from approximately 1960 to present, with historical extensions for major economies from ILO archives.

\textbf{Pew Research Center Global Attitudes Surveys.} Cross-national survey data on LGBT acceptance, political attitudes, and institutional trust are drawn from Pew's Global Attitudes Project (2002--present). The LGBT acceptance index serves as an $\alpha$ proxy in the LGBT rights case study (Section~\ref{sec:lgbt}): societal acceptance levels condition the effective voice that LGBT populations can exercise. Political attitude data proxy both stakeholder satisfaction (inversely related to $F$) and institutional legitimacy perceptions.

\textbf{Gallup World Poll.} Annual surveys across 160+ countries measuring governance satisfaction, institutional trust, and civic engagement. Variables used include: confidence in national government, satisfaction with freedom, and perception of corruption. These serve as cross-validation proxies for $F$: low satisfaction and high perceived corruption should correlate with high friction in the consent-holding framework.

\textbf{Freedom House Freedom on the Net.} Annual assessments of internet freedom across 70 countries, covering obstacles to access, limits on content, and violations of user rights. These variables are used in the platform governance case study (Section~\ref{sec:platform-governance}) to construct digital-domain $\alpha$ proxies: internet freedom scores capture the degree to which digital stakeholders can exercise effective voice in online governance domains.

\textbf{World Values Survey (WVS).} Longitudinal cross-national survey covering political culture, social values, and institutional attitudes across 120 societies from 1981 to present. WVS emancipative values indices and institutional confidence measures provide attitudinal proxies for both $\alpha$ (citizen expectations about voice and participation) and $F$ (discrepancy between expected and actual governance quality). These supplement the behavioral proxies (protests, strikes, litigation) with attitudinal measures capturing latent friction---misalignment that has not yet manifested in observable collective action.

\subsection{Historical Sources}

\textbf{Parliamentary Records.} Hansard (UK), Congressional Record (US), and equivalent parliamentary archives for other case-study countries provide direct evidence of legislative deliberation, petition reception, and franchise debate. Parliamentary petition counts serve as friction proxies in the suffrage and abolition case studies: rising petition volumes indicate organized stakeholder demand for consent-alignment reform.

\textbf{Protest Event Databases.} The Global Database of Events, Language, and Tone (GDELT), the Armed Conflict Location and Event Data Project (ACLED), and the Mass Mobilization in Autocracies Database (MMAD) provide event-level data on protests, demonstrations, riots, and strikes. These databases enable construction of friction time series with daily resolution for recent periods and annual resolution for historical periods. Protest event frequency and participant counts serve as direct proxies for $F(d,t)$ in their respective domains.

\textbf{Litigation Records.} Court filing data from national judicial statistics agencies provide friction proxies in domains where legal contestation is the primary correction mechanism: civil rights litigation rates, labor-management arbitration cases, and environmental enforcement actions. For the US, the Administrative Office of the United States Courts provides annual filings data by case type. For the UK, the Ministry of Justice publishes civil and family court statistics. Litigation as a friction proxy captures a specific correction mechanism---Lockean contestation through legal channels---and should be interpreted alongside other friction indicators rather than as a standalone measure.

\textbf{Trade Union Archives and Labor Statistics.} For the labor governance case study (Section~\ref{sec:labor}), historical union membership records, collective bargaining agreements, and strike statistics are drawn from national labor departments and union archives. UK data are sourced from the Department for Business and Trade (formerly BEIS); US data from the Bureau of Labor Statistics; German data from the Hans B\"{o}ckler Foundation and IAB. These provide both $\alpha$ proxies (union density, bargaining coverage) and $F$ proxies (strike frequency, days lost, wildcat action rates) at annual resolution from approximately 1890 to present for the UK and US, and from 1950 for Germany.

\subsection{Ordinal Coding Protocols}

For the three detailed case studies (abolition, LGBT rights, platform governance), ordinal alpha and friction indices are constructed on 5-point scales:

\textbf{Consent alignment ($\alpha$ proxy):}
\begin{enumerate}[leftmargin=2em]
\item[1.] No formal voice: affected population legally excluded from governance of the domain.
\item[2.] Symbolic voice: formal rights exist but enforcement and capacity are minimal.
\item[3.] Partial voice: some institutional channels exist; effective voice varies by subgroup.
\item[4.] Substantial voice: most affected parties have functional channels; remaining gaps are identified.
\item[5.] Near-full alignment: stakes and consent power are closely matched; friction is low.
\end{enumerate}

\textbf{Friction ($F$ proxy):}
\begin{enumerate}[leftmargin=2em]
\item[1.] Quiescent: no observable organized dissent or contestation.
\item[2.] Low friction: occasional petitions, isolated protests, or individual litigation.
\item[3.] Moderate friction: sustained organized campaigns, regular protests, growing media attention.
\item[4.] High friction: mass mobilization, strikes, civil disobedience, or legal crises.
\item[5.] Crisis: regime-threatening contestation, violence, or institutional breakdown.
\end{enumerate}

These ordinal scales are designed to be comparable across case studies: a ``3'' in the suffrage domain (partial voice: some women or racial minorities enfranchised, others excluded) is intended to capture the same structural relationship between stakes and consent power as a ``3'' in the labor domain (partial voice: some industries covered by collective bargaining, others not) or the platform domain (partial voice: some content moderation decisions appealable, others not). The comparability rests on the framework's domain-general definition of consent alignment rather than domain-specific institutional features.

Coding is performed by the author based on the historical sources cited in each case study. For the cross-national panel extensions proposed in Section~\ref{subsec:cross-national-panel}, independent coding by multiple researchers with formal inter-coder reliability assessment would be required. A pilot inter-coder study using the ordinal protocol across the suffrage and labor case studies would establish baseline reliability before extending to the full domain set.

\subsection{Inter-Coder Reliability Considerations}

The ordinal coding protocol is designed to minimize subjective judgment while acknowledging that any historical coding involves interpretation. Three strategies mitigate reliability concerns:

First, \textit{anchoring}: each ordinal level is defined with reference to observable institutional and behavioral indicators (legal status, protest frequency, petition counts) rather than subjective assessments of ``legitimacy'' or ``dissent intensity.'' Second, \textit{triangulation}: multiple proxy measures are constructed for each theoretical variable, and convergence across proxies provides evidence of construct validity. Third, \textit{transparency}: all coding decisions and their justifications are documented in the case study narratives, enabling replication and critique. For the present monograph, single-coder reliability is partially validated by the convergence between the ordinal codings and the quantitative proxy measures (V-Dem, OECD, protest event data) where both are available.

\subsection{Proxy Construction Methodology}

The mapping from observable variables to theoretical constructs ($\alpha$ and $F$) follows a consistent protocol across case studies:

\textbf{Step 1: Domain definition.} Identify the governance domain $d$, the relevant stakeholder population $S_d$, and the authority-holder set. This step determines which data sources are relevant.

\textbf{Step 2: Alpha proxy selection.} Identify observable variables that capture the degree to which stakeholder stakes $s_i(d)$ covary with consent power $C_{i,d}$. Preferred proxies are institutional measures (suffrage share, union density, board composition) rather than attitudinal measures, because institutional measures capture \textit{effective} voice rather than perceived voice. Where institutional measures are unavailable, attitudinal proxies (WVS, Gallup) are used with appropriate caveats.

\textbf{Step 3: Friction proxy selection.} Identify observable variables that capture misalignment between outcomes and stakeholder preferences. Behavioral proxies (protests, strikes, litigation, petitions) are preferred over attitudinal proxies (satisfaction surveys) because they capture \textit{revealed} friction---misalignment costly enough to motivate collective action. Attitudinal measures supplement behavioral proxies by capturing latent friction that has not yet manifested in observable action.

\textbf{Step 4: Time series construction.} Assemble annual (or finer-grained) time series for each proxy variable. Where multiple proxies are available for the same construct, construct composite indices using principal component analysis or simple averaging, with sensitivity analysis to ensure results are robust to proxy choice.

\textbf{Step 5: Validation.} Test proxy validity through convergent and discriminant validity assessment. Convergent validity: do multiple proxies for the same construct correlate positively? Discriminant validity: do $\alpha$ proxies and $F$ proxies correlate in the direction predicted by H1 ($\alpha \uparrow \implies F \downarrow$)?

This protocol is designed to be replicable across domains and countries, enabling the cross-national panel studies proposed in Section~\ref{subsec:cross-national-panel}. The protocol is deliberately conservative: it prefers institutional and behavioral measures over attitudinal ones, requires multiple proxy convergence before drawing conclusions, and documents all coding decisions for replication. The cost of this conservatism is reduced sensitivity (some real misalignment may go undetected), but the benefit is reduced false positive rate (detected misalignment is more likely to reflect genuine friction).

\subsection{Limitations of the Data}

Several limitations of the data sources warrant acknowledgment. First, \textbf{temporal coverage is uneven}: V-Dem provides consistent coverage from 1789, but many indicators are more reliable for the post-1900 period. OECD and ILO data begin systematically only in the 1960s, limiting the historical depth of cross-national labor analyses. Pre-twentieth-century proxy construction relies more heavily on qualitative historical sources and ordinal coding, with correspondingly lower measurement precision.

Second, \textbf{geographic coverage is Western-biased}: the historical case studies draw primarily on British, American, and Western European examples, reflecting both data availability and the author's expertise. The cross-national panel extension (Section~\ref{subsec:cross-national-panel}) would partially address this limitation, but data quality for non-OECD countries remains a constraint. The consent-holding framework's theoretical apparatus is domain- and culture-general, but its empirical validation is currently anchored in Western institutional histories.

Third, \textbf{proxy validity is assumed rather than formally tested} in the present monograph. The convergence between ordinal codings and quantitative indicators provides informal validation, but systematic construct validation---confirmatory factor analysis, known-groups validity, criterion validity against established democracy indices---remains a task for future empirical work.

\subsection{Data Availability Statement}

All V-Dem, OECD, ILO, Pew, and Gallup data used in this paper are publicly available through their respective institutional repositories. Parliamentary records are available through national archives (Hansard Online for UK, Congress.gov for US). Protest event data from GDELT and ACLED are freely accessible. The ordinal codings constructed for this paper, together with the R and Python scripts used for proxy construction and analysis, are archived on Zenodo (\href{https://doi.org/10.5281/zenodo.17684679}{10.5281/zenodo.17684679}) and available via GitHub (\url{https://github.com/studiofarzulla/consent-holding-theory}).

\section{Appendix D: Formal Model Specification}
\label{sec:appendix-model}

% PLACEHOLDER — formal model specification for computational validation
% Consolidates simulation parameters, algorithm pseudocode, convergence criteria

This appendix provides complete formal specifications for the computational models described in Part~IV, including algorithm pseudocode, parameter ranges, and convergence diagnostics.

% NOTE: The following table floated from Appendix A during LaTeX compilation.
% It will be properly placed in Phase 2 integration.

\begin{table*}[t]
\centering
\caption{Robustness Check: Stakes Distribution Heterogeneity}
\label{tab:robustness_distributions}
\begin{tabular}{lcccccc}
\toprule
Distribution (Gini) & Equal Voice & Stakes-Weighted & Plutocracy & Random & Expert \\
\midrule
High Gini (0.78) & 0.618 & 0.644 & 0.617 & 0.522 & 0.618 \\
Low Gini (0.03) & 0.573 & 0.589 & 0.572 & 0.461 & 0.570 \\
Medium Gini (0.26) & 0.584 & 0.596 & 0.585 & 0.486 & 0.556 \\
Pareto $\alpha$=1.2 (0.85) & 0.619 & 0.658 & 0.613 & 0.514 & 0.608 \\
Pareto $\alpha$=2.0 (0.53) & 0.610 & 0.605 & 0.596 & 0.480 & 0.596 \\
Pareto $\alpha$=4.0 (0.42) & 0.594 & 0.584 & 0.581 & 0.487 & 0.582 \\
\bottomrule
\end{tabular}
\end{table*}

% ============================================================================
% BACK MATTER
% ============================================================================
\clearpage

% ============================================================================
% ACKNOWLEDGEMENTS
% ============================================================================

\section*{Acknowledgements}

The author acknowledges the intellectual contributions of scholars in
constitutional political economy, social choice theory, and legitimacy
studies whose foundational work made this synthesis possible.

This paper benefited from extended collaboration with Claude (Anthropic),
whose contributions to literature synthesis, computational validation
design, and iterative refinement were substantive. The author gratefully
acknowledges this assistance while taking full responsibility for all
claims, errors, and interpretive choices.

This work is part of the Adversarial Systems Research program at
Dissensus AI, a broader investigation into stability, alignment, and
friction dynamics across political, financial, cognitive, and multi-agent
systems. Related papers in the series are available through the
Adversarial Systems \& Complexity Research Initiative
(\href{https://systems.ac}{ASCRI; systems.ac}).

All computational analysis was conducted at Resurrexi Lab, a distributed
computing cluster built from consumer-grade hardware, demonstrating that
rigorous political economy research is accessible without institutional
supercomputing infrastructure. Code and data are available at
\url{https://github.com/studiofarzulla/consent-holding-theory}.

The author welcomes feedback, criticism, and collaboration.
Correspondence should be directed to
\href{mailto:murad@dissensus.ai}{murad@dissensus.ai}.

% ============================================================================
% DECLARATIONS
% ============================================================================

\section*{Declarations}

\paragraph{Conflict of Interest.} The author declares no competing interests.

\paragraph{Funding.} This research received no external funding.

\paragraph{Data Availability.} All simulation code and generated data are
available at \url{https://github.com/studiofarzulla/consent-holding-theory}.

\paragraph{AI Assistance.} Claude (Anthropic) was used as a research
collaborator for literature synthesis, computational validation design,
LaTeX preparation, and iterative refinement of mathematical arguments.
All intellectual claims and errors remain the author's responsibility.

% ============================================================================
% BIBLIOGRAPHY (SINGLE-COLUMN)
% ============================================================================
\bibliography{../references}

\end{document}
