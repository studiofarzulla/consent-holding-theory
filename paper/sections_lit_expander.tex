% Literature Review Expansions — lit-expander agent
% Sections 2.4 (expand), 2.5 (expand), 2.9 (NEW), 2.10 (NEW), 2.11 (NEW)
% Target: ~8 pages (~320 lines of LaTeX)

%% ============================================================================
%% 2.4 COMMON-POOL RESOURCE GOVERNANCE (EXPANDED)
%% ============================================================================

\subsection{Common-Pool Resource Governance}
\label{subsec:cpr-governance}

Ostrom's \citeyearpar{ostrom1990} groundbreaking work on common-pool resources challenges both ``tragedy of the commons'' pessimism and top-down state solutions. Through field studies of fisheries, forests, irrigation systems, and groundwater basins across continents, she demonstrates that resource users frequently develop effective self-governance without privatization or centralized authority. Her eight design principles for successful commons management include particularly relevant insights for consent-holding theory. Design Principle 3 requires that ``most individuals affected by the operational rules can participate in modifying the operational rules''---essentially mandating high $\alpha(d_{rules})$ for those with high $s_i(d_{resources})$. Design Principle 8 specifies nested enterprises for larger systems, enabling polycentric governance with consent-holding at multiple scales. This builds on earlier insights from \citet{ostrom1961} on polycentric systems, demonstrating how multiple governing authorities at different scales can achieve better outcomes than monocentric alternatives.

The framework formalizes Ostrom's intuitions. Her ``collective choice arrangements'' represent $H_t(d)$ mappings where users participate in rule modification. Her design principles can be reinterpreted as conditions enabling high $\alpha(d)$: clear boundaries (defining who holds $s_i$), local monitoring (ensuring $C_i$ holders possess information), graduated sanctions (responses to low-$\alpha$ violations), and conflict resolution mechanisms (managing $F(d)$ when it arises). Successful commons maintain high consent alignment; failed commons exhibit persistent misalignment between stakes and voice.

Recent empirical work validates this interpretation quantitatively. \citet{cox2010} conduct a meta-analysis showing that Ostrom's design principles predict commons sustainability across diverse contexts, providing systematic evidence that consent alignment mechanisms enable effective resource governance. \citet{yadav2021institutional} analyze 83 Amazonian communities managing arapaima fisheries, showing that Ostrom's design principles predict ecological outcomes systematically. Communities exhibiting collective choice arrangements (high $\alpha$) maintain sustainable fish stocks; those lacking such arrangements experience depletion.

However, the relationship between community governance and resource outcomes is more complex than Ostrom's initial formulation suggests. \citet{agrawal1999enchantment} caution against romanticizing community-based conservation, arguing that ``community'' itself is a contested category whose boundaries, internal power dynamics, and relationship to external markets require careful specification. In consent-holding terms, defining $S_d = \{i | s_i(d) > 0\}$ for a natural resource domain is itself a political act: who counts as a ``community member'' determines whose stakes register in $\alpha(d)$ calculations. Agrawal and Gibson demonstrate that communities are internally differentiated by wealth, gender, caste, and political connections---differences that produce unequal $C_i$ distributions even within nominally participatory arrangements. A village forestry committee may formally include all households, but if wealthier households dominate agenda-setting while poorer households depend most heavily on forest products, the resulting $\alpha(d_{forest})$ remains low despite institutional inclusiveness.

\citet{dietz2003drama} synthesize the broader challenges facing commons governance, framing environmental problems not as a simple ``tragedy'' but as a complex ``drama'' involving multiple actors with heterogeneous interests, information asymmetries, and cross-scale interactions. Their analysis identifies three fundamental requirements for adaptive commons governance: providing information (enabling informed consent through transparency about resource states), managing conflict (developing institutions that channel $F(d)$ into productive reform rather than destructive competition), and inducing rule compliance (maintaining institutional stability when individual incentives favor defection). Each requirement maps directly onto consent-holding mechanisms. Information provision raises $\text{eff\_voice}_i$ by enabling stakeholders to assess whether $H_t(d)$ serves their interests. Conflict management channels friction $F(d)$ toward institutional adaptation. Rule compliance addresses the gap between formal consent power $C_i > 0$ and the temptation to free-ride on others' governance contributions.

The question of enforcement reveals a crucial insight for legitimacy theory. \citet{gibson2005local} demonstrate that local enforcement produces systematically better forest outcomes than external enforcement, even controlling for rule quality and forest type. Communities that monitor and sanction their own members---where enforcement authority maps onto the affected set $S_d$---maintain forest cover more effectively than those relying on distant government agencies. This finding validates a core prediction of consent-holding theory: governance arrangements where those bearing consequences also hold enforcement authority (high $\alpha(d_{enforcement})$) outperform arrangements where enforcement is externally imposed. Local monitors possess better information about rule violations, stronger social incentives for fair application, and greater legitimacy among those subject to sanctions.

Yet local governance can also fail catastrophically when consent-holding structures are disrupted by external forces. \citet{baland1996halting} document how colonial and post-colonial state interventions systematically undermined indigenous resource management institutions across the developing world. By appropriating forests, fisheries, and rangelands into state ownership, governments simultaneously dismantled existing $H_t(d)$ mappings (removing community consent authority) and failed to replace them with effective alternatives. The result was a legitimacy vacuum: communities retained high $s_i(d_{resources})$ but lost all $C_i$, while distant state agencies held formal authority but lacked local knowledge, monitoring capacity, and social embeddedness. Friction mounted not as organized protest but as quiet defiance---illegal logging, poaching, encroachment---reflecting the theory's prediction that persistent misalignment between stakes and voice produces friction even when formal institutional channels are absent.

\citet{berkes2006globalization} identify an analogous dynamic at the global scale through the concept of ``roving bandits''---mobile actors who exploit resources across jurisdictions without bearing long-term consequences. Industrial fishing fleets operating in developing nations' waters exemplify this pattern: their $s_i(d_{fishery})$ is entirely short-term extractive, they hold effective $C_i$ through economic power and flag-state protection, while local fishing communities possess high long-term $s_i$ but negligible $C_i$ over marine resource decisions. The roving bandit phenomenon thus represents a systematic failure of consent alignment: actors with the lowest legitimate stakes capture decision authority, while those with the highest stakes---communities whose livelihoods, food security, and cultural practices depend on resource health---are excluded from $H_t(d)$. This is domain-specific alpha optimization in reverse: governance arrangements that systematically minimize $\alpha(d)$ by disconnecting stakes from voice.

\citet{weeratunge2014smallscale} deepen this analysis by documenting how small-scale fisheries governance affects multidimensional wellbeing beyond economic income. Their wellbeing framework encompasses material conditions, relational dynamics, and subjective experience---dimensions that map onto our stakes measurement. When $s_i(d_{fishery})$ is measured solely as income share, the case for community governance rests on efficiency grounds alone. But when stakes include nutritional security, cultural identity, intergenerational knowledge transmission, and social cohesion, the affected set $S_d$ broadens substantially, and the stakes-weighted case for high $\alpha(d)$ strengthens correspondingly. Communities' comprehensive stakes in fishery governance exceed their economic stakes, implying that legitimacy assessments based on narrow economic measures systematically understate the consent alignment deficit.

Finally, \citet{ostrom2010} extends the polycentric governance framework to global environmental challenges, arguing that monocentric approaches to climate change and biodiversity loss are both politically infeasible and theoretically suboptimal. Polycentric systems---multiple overlapping governance authorities operating at different scales---allow consent-holding to be domain-specific and scale-appropriate. Municipal recycling programs, regional air quality management, national emissions standards, and international climate treaties each represent distinct $H_t(d)$ mappings for overlapping environmental domains. This nested structure enables what we might call \textit{domain-specific alpha optimization}: each governance level can achieve higher $\alpha(d)$ for the scale at which its affected set $S_d$ is defined than any single-level arrangement could achieve across all scales simultaneously. The insight connects directly to our framework's emphasis on domain-specific analysis: there is no single ``optimal'' governance structure, but rather an ecology of institutional arrangements whose legitimacy must be assessed domain by domain.


%% ============================================================================
%% 2.5 DELIBERATIVE DEMOCRACY AND MINI-PUBLICS (EXPANDED)
%% ============================================================================

\subsection{Deliberative Democracy and Mini-Publics}
\label{subsec:deliberative-democracy}

Building on \citet{dahl1971} polyarchy framework of participation and opposition and \citet{mill1861} considerations on representative government balancing participation and competence, Habermas's \citeyearpar{habermas1984,habermas1990} communicative action theory distinguishes strategic action (oriented toward achieving one's goals) from communicative action (oriented toward mutual understanding through reasoned argument). Legitimate norms are those acceptable to all affected parties through rational discourse free from coercion. His discourse principle holds that ``only those norms can claim validity that could meet with the acceptance of all concerned in their capacity as participants in a practical discourse.'' This maps onto consent-holding directly: ``all concerned'' represents our affected set $S_d = \{i | s_i(d) > 0\}$, while ``acceptance'' requires $C_i > 0$ in decision procedures $H_t(d)$.

Fishkin's \citeyearpar{fishkin2009when,fishkin2018} deliberative polling research operationalizes these theoretical commitments. By convening randomly selected representative samples, providing balanced information, facilitating structured deliberation, and measuring preference changes, deliberative polls demonstrate that informed public judgment shifts significantly through discourse. \citet{fishkin2009when} documents cases across multiple countries where deliberative polls produced measurably different policy preferences compared to raw survey data, with opinion shifts averaging 10--15 percentage points on salient issues. The mechanism is straightforward in consent-holding terms: information provision and structured deliberation raise $\text{eff\_voice}_i$ by transforming uninformed preferences into considered judgments, while random selection approximates equal $C_i$ across the affected population.

Citizens' assemblies extend deliberative innovation to consequential policy domains. The Irish Citizens' Assembly (2016--2018) addressed abortion and climate change through 99 randomly selected citizens deliberating after expert input, demonstrating how sortition combined with deliberation can shift preferences systematically \citep{farrell2019}. \citet{courant2021} analyze the French Citizens' Convention on Climate (2019--2020), which generated 149 policy proposals from 150 randomly selected participants through sortition and deliberation, with many subsequently adopted into legislation.

The framework interprets these innovations as institutional experiments raising $\alpha(d)$ through sortition and deliberation. Random selection approximates equal $C_i$ for participants; demographic stratification can approximate stakes-weighting if groups correlate with $s_i(d)$. Learning phases improve $\text{eff\_voice}_i$ through information provision; deliberation structures enable preference refinement.

Participatory budgeting represents perhaps the most direct institutional mechanism for raising $\alpha(d)$ in municipal governance domains. \citet{wampler2007participatory} analyzes the Brazilian experience, where Porto Alegre pioneered citizen-directed allocation of municipal investment budgets beginning in 1989. The process grants residents direct $C_i$ over spending priorities---infrastructure, health, education, housing---in proportion to neighborhood attendance and need indices. Wampler demonstrates that participatory budgeting systematically redirects resources toward poorer districts, precisely the pattern predicted by consent-holding theory: when $C_i$ is distributed more evenly while $s_i(d_{infrastructure})$ concentrates among underserved populations, the resulting $\alpha(d_{budget})$ better reflects genuine stakes distribution, reducing friction and improving perceived legitimacy. The Brazilian experience has since spread to over 7,000 municipalities worldwide \citep{participatory2023}, constituting a natural experiment in alpha-raising across diverse institutional contexts. Crucially, outcome variation across implementations provides testable predictions: cities where participatory budgeting grants meaningful decision authority (high $C_i$ for participants) should exhibit lower governance friction than those where participation is merely consultative.

\citet{smith2009democratic} provides a systematic comparative analysis of democratic innovations including mini-publics, participatory budgeting, direct legislation, and e-democracy initiatives. Smith evaluates each innovation against criteria that map directly onto our framework: inclusiveness (breadth of the affected set included in $H_t(d)$), popular control (degree of genuine $C_i$ granted to participants), considered judgment (quality of $\text{eff\_voice}_i$ exercised), and transparency (enabling external assessment of $\alpha(d)$). His analysis reveals a persistent tension between depth and breadth: deliberative mini-publics achieve high $\text{eff\_voice}_i$ for participants through intensive information and discussion, but include only a small fraction of the affected population. Referenda include the entire electorate but sacrifice deliberation quality. This depth-breadth trade-off is structural, not merely practical, and consent-holding theory illuminates why: achieving simultaneously high $C_i$ for all members of $S_d$ and high $\text{eff\_voice}_i$ for each participant requires institutional resources that scale with population size.

\citet{bouricius2013democracy} proposes a radical solution to this tension through multi-body sortition, drawing explicitly on Athenian democratic practice. Rather than concentrating deliberative authority in a single citizens' assembly, Bouricius designs a system of six functionally differentiated bodies---agenda council, interest panels, review panels, policy jury, rules council, and oversight council---each populated by sortition with different selection criteria and rotation schedules. The architecture distributes consent-holding across multiple specialized $H_t(d)$ mappings, each optimized for different governance functions. Agenda-setting requires broad representation (maximizing $|S_d|$ coverage); policy deliberation requires depth (maximizing $\text{eff\_voice}_i$ through sustained engagement); oversight requires independence (minimizing capture by interested parties). This multi-body approach represents systematic consent allocation: rather than asking a single institution to optimize all dimensions simultaneously, it decomposes governance into functional domains and assigns consent-holding structures appropriate to each.

The capacity-building dimension of deliberation has received increasing attention. \citet{niemeyer2011emancipatory} provides empirical evidence for what he terms the ``emancipatory effect'' of deliberation: participants in mini-publics develop not only better-informed preferences but enhanced civic capacities that persist after the deliberative event concludes. Through pre- and post-deliberation surveys, Niemeyer demonstrates increases in political efficacy, complexity of political reasoning, and willingness to engage in subsequent civic action. In consent-holding terms, deliberation raises $\text{eff\_voice}_i$ not merely for the issue under discussion but as a durable capacity enhancement. If this emancipatory effect generalizes---and \citet{schneiderhan2008realizing} provide supporting evidence from community organizing contexts, showing that meaningful participatory experiences transform participants' understanding of their own citizenship capacity---then deliberative institutions generate positive externalities beyond their immediate policy domain. Each deliberative episode raises participants' $\text{eff\_voice}_i$ across multiple governance domains, creating spillover effects that improve $\alpha(d)$ systemically rather than domain-specifically.

\citet{schneiderhan2008realizing} examine the relationship between deliberative authority and citizenship formation, documenting how the experience of exercising genuine decision authority---as opposed to merely being consulted---transforms participants' self-understanding as political agents. Their distinction between ``thin'' participation (surveys, public comment periods) and ``thick'' participation (binding deliberation with real consequences) maps onto our framework's emphasis on effective voice: formal inclusion in $H_t(d)$ without meaningful $C_i$ does not constitute genuine consent alignment. The implication is that institutional arrangements granting merely consultative roles---advisory committees, non-binding referenda, stakeholder ``engagement'' processes---may actually reduce $\alpha(d)$ by creating an appearance of consent that masks continued concentration of decision authority.

However, scaling deliberative democracy encounters significant complexity challenges. \citet{hendriks2014emergent} analyzes what she terms ``emergent complexity'' in deliberative systems---the unintended interactions, feedback loops, and power asymmetries that arise when multiple deliberative institutions operate simultaneously within a governance ecosystem. Mini-publics do not function in isolation; they interact with elected legislatures, courts, media, lobbying organizations, and each other. Hendriks demonstrates that these interactions can produce perverse outcomes: a well-designed citizens' assembly may generate high-quality recommendations that are subsequently captured or distorted by existing power structures during implementation. In consent-holding terms, achieving high $\alpha(d_{deliberation})$ within the assembly does not guarantee high $\alpha(d_{policy})$ if the translation mechanism from deliberative output to binding policy is itself characterized by low consent alignment. The deliberative system as a whole---not merely its individual components---must be evaluated for legitimacy.

Our stakes-weighted consent framework confronts democratic equality arguments directly. Building on \citet{mill1859} foundations regarding individual liberty, consent, and limits of state power, \citet{christiano2008} defends equal political voice on dignity grounds: each person possesses equal moral status, entitling them to equal say in collective decisions regardless of stakes or competence. \citet{waldron1999} argues that persistent disagreement about what justice requires makes equal voice procedurally fair even if some possess superior judgment. \citet{brighouse2010} examine whether proportional influence could improve democratic outcomes but conclude that equal voice better respects equality of persons.

We acknowledge this tension while distinguishing \textit{political} domains from \textit{governance} domains generally. In constitutional fundamentals and citizenship rights, equal voice may be intrinsically required by equal moral status---each person gets one vote precisely because they are persons, not because they possess equal stakes. But many governance domains are not \textit{political} in this sense: corporate boards allocating firm resources, technical committees setting safety standards, platform algorithms moderating speech, common-pool resource users managing fisheries. In these contexts, stakes-weighting may be both more efficient (reducing friction, improving outcomes) and more legitimate (those bearing consequences should influence decisions proportionally). The framework enables empirical testing: do equal-voice or stakes-weighted mechanisms generate higher measured legitimacy $L(d,t)$ in different domain types?


%% ============================================================================
%% 2.9 POLITICAL ANTHROPOLOGY (NEW)
%% ============================================================================

\subsection{Political Anthropology}
\label{subsec:political-anthropology}

Before modern political philosophy articulated theories of consent, legitimacy, and sovereignty, human societies experimented with radically diverse governance structures across millennia. Recent scholarship in political anthropology challenges linear progress narratives---from band to tribe to chiefdom to state---and reveals that the dynamics formalized by consent-holding theory have been contested for as long as humans have lived in groups.

\citet{graeber2021} present the most comprehensive challenge to conventional narratives of political development. Drawing on archaeological and ethnographic evidence from across the globe, Graeber and Wengrow demonstrate that early human societies were not locked into simple egalitarian bands that inevitably evolved toward hierarchical states. Instead, societies deliberately experimented with different governance forms, often oscillating seasonally between hierarchical and egalitarian arrangements. The Nambikwara of Brazil concentrated authority in a chief during rainy-season village life but dispersed into autonomous bands during the dry season. Pacific Northwest societies maintained rigid hierarchies during winter ceremonial seasons but operated as egalitarian fishing camps in summer. These oscillations were not transitions between developmental stages but conscious institutional choices---societies actively constructing and deconstructing $H_t(d)$ mappings in response to changing ecological, social, and ritual contexts.

This finding carries profound implications for consent-holding theory. If governance structures are \textit{chosen} rather than determined by material conditions or evolutionary stage, then the framework's emphasis on consent as a structural feature rather than a modern invention receives anthropological validation. Societies that oscillate between governance forms demonstrate awareness that different domains and seasons of collective life may require different $H_t(d)$ mappings---precisely the domain-specific analysis our framework insists upon. Graeber and Wengrow further document cases where societies explicitly rejected governance models practiced by their neighbors, suggesting that consent alignment (or its absence) was recognized and acted upon long before political philosophers formalized the concept.

\citet{boehm1999} provides the evolutionary foundation for understanding why consent-holding has always been contested. Studying egalitarian societies across multiple continents, Boehm identifies a universal mechanism he terms ``reverse dominance hierarchy'': coalitions of subordinates actively suppressing would-be dominators through ridicule, ostracism, and in extreme cases, assassination. This is not the absence of hierarchy but its active prevention---a costly institutional arrangement requiring continuous collective action. In consent-holding terms, reverse dominance hierarchies represent friction mechanisms $F(d)$ directed specifically against consent concentration. When any individual attempts to accumulate $C_i \gg \bar{C}$ (disproportionate decision authority), the coalition enforces redistribution. Boehm's evidence suggests that the disposition toward such enforcement is ancient and cross-cultural, implying that resistance to low-$\alpha$ governance is not a modern democratic value but a deep feature of human social organization.

The mechanisms Boehm documents---gossip, public ridicule, refusal to follow, coalition formation against aspirant leaders---constitute a repertoire of friction expression remarkably consistent across unrelated societies. A would-be ``big man'' in a Hadza camp, a domineering chief among the !Kung San, and an overreaching leader in an Amazonian village all face structurally similar collective responses. This cross-cultural convergence suggests that humans possess evolved psychological dispositions that detect consent misalignment (perceived $\alpha$ below some acceptable threshold) and generate friction responses calibrated to restore alignment. The framework captures this dynamic formally: when $\alpha(d) < \tau$ (some culturally variable but psychologically grounded threshold), friction $F(d)$ increases discontinuously, consistent with our Hypothesis H3 on threshold effects.

\citet{scott2017} examines the question from the opposite direction: how did states first impose centralized $H_t(d)$ on populations that actively resisted? Scott argues that early states in Mesopotamia, China, and the Indus Valley succeeded not through legitimacy or superior governance but through the coincidence of grain agriculture (which produces storable, taxable surplus) and population concentration (which enables surveillance and control). Early state formation was, in Scott's analysis, primarily a project of consent extraction rather than consent alignment: concentrating $C_i$ in ruling elites while maximizing $s_i(d)$ for subject populations through sedentarization, walls (which kept people in as much as enemies out), and the replacement of diverse subsistence strategies with legible monocultures.

Scott documents extensive evidence of populations fleeing early states into upland regions, swamps, and frontier zones---what he terms ``escape agriculture'' and ``escape social structures'' designed specifically to resist state incorporation. These populations were not ``primitive'' societies awaiting state formation; they were communities that had experienced state governance and rejected it. In consent-holding terms, they had observed low-$\alpha$ regimes and chosen exit over voice. The ``barbarian'' periphery of every early civilization represented not developmental lag but active friction response: populations voting with their feet when formal voice mechanisms within $H_t(d)$ were absent.

These anthropological findings ground the consent-holding framework in deep historical evidence. Three implications deserve emphasis. First, $H_t(d)$ has always been contested: the question of who decides is not a modern political invention but a perennial structural problem of collective life. Second, friction $F(d)$ is the universal response to consent misalignment---its expression varies from Boehm's reverse dominance hierarchies to Scott's escape agriculture, but the underlying dynamic (perceived misalignment between stakes and voice triggering costly resistance) is constant across millennia and continents, as formalized in the friction dynamics of multi-agent coordination \citep{farzulla2025aoc}. Third, the framework's domain-specific approach is validated by anthropological evidence: societies that separated governance by domain (seasonal oscillation, domain-specific authority structures) demonstrated greater institutional resilience than those imposing uniform $H_t(d)$ across all collective decisions. The consent-holding framework, far from imposing modern liberal categories on pre-modern societies, formalizes dynamics that human communities have navigated for at least twelve thousand years.


%% ============================================================================
%% 2.10 SOCIAL MOVEMENTS AND COLLECTIVE ACTION (NEW)
%% ============================================================================

\subsection{Social Movements and Collective Action}
\label{subsec:social-movements}

Social movements are the primary mechanism through which friction $F(d)$ manifests at scale and generates pressure for institutional reform. Where Section~\ref{subsec:political-anthropology} established that consent contestation is anthropologically universal, this section examines how organized collective action translates individual-level misalignment perceptions into systemic friction that restructures $H_t(d)$.

\citet{olson1965logic} established the foundational paradox of collective action: rational self-interested individuals will free-ride on others' contributions to public goods, implying that large groups should be unable to organize for collective benefit. This result carries a crucial implication for consent-holding measurement: observed friction $F(d)$ systematically underestimates true consent misalignment. If organizing collective action is costly and individual contributions to large-scale movements are negligible, then many individuals who perceive $\alpha(d) < \tau$ will nevertheless fail to express friction through protest, petition, or political mobilization. The gap between experienced misalignment and expressed friction introduces measurement error into any empirical assessment of $\alpha(d)$ based on observed conflict indicators. Olson's theory predicts that smaller groups with concentrated stakes will mobilize more effectively than larger groups with diffuse stakes---precisely why industry lobbies outperform consumer movements, and why minority rights organizations often generate friction disproportionate to their population share. In consent-holding terms, the covariance between $s_i(d)$ and mobilization capacity introduces systematic bias: groups whose stakes are intense but narrowly distributed produce more observable friction per unit of misalignment than groups whose stakes are broadly distributed but individually dilute.

\citet{tarrow1998power} extends collective action theory by identifying the structural conditions under which social movements overcome Olsonian barriers. His concept of ``political opportunity structures'' describes the institutional openings---elite divisions, electoral realignments, international pressures, regime transitions---that reduce the costs of collective action and increase the expected returns from friction expression. In consent-holding terms, political opportunity structures are moments when the institutional translation function from $F(d)$ to $\Delta H_t(d)$ becomes more favorable: the same level of friction produces larger consent-holding adjustments during periods of elite vulnerability than during periods of consolidated authority.

Tarrow's analysis also highlights the role of framing and cultural resources in movement mobilization. Movements do not simply respond to objective misalignment; they construct interpretive frameworks that make misalignment visible, urgent, and actionable. The civil rights movement's framing of racial segregation as un-American---contradicting national commitments to equality and freedom---was more effective than purely interest-based appeals, precisely because it identified consent misalignment not merely as harmful to Black Americans but as a violation of the polity's own legitimacy principles. This framing dynamic connects to our measurement framework: $\alpha(d)$ is partly socially constructed. The ``objective'' distribution of stakes and voice provides a structural foundation, but collective interpretation of whether that distribution constitutes legitimate governance depends on cultural frames, historical narratives, and political entrepreneurs who articulate misalignment in compelling terms.

\citet{tilly2007} provides the historical foundation for understanding how social movements relate to democratization---the long-term process of raising $\alpha(d)$ across political domains. Tilly argues that democracy is not a fixed institutional form but a moving target, defined by the breadth, equality, binding consultation, and protection of political participation. Democratization occurs when these dimensions expand; de-democratization when they contract. This processual view aligns precisely with the consent-holding framework's dynamic emphasis: $\alpha(d,t)$ is a continuously varying quantity whose trajectory depends on the interaction between institutional design and social mobilization. Tilly's evidence from European and Latin American cases demonstrates that democratization rarely occurs through top-down institutional design alone; it requires sustained bottom-up friction from social movements that force elite concessions, combined with institutional mechanisms that lock in consent-holding gains.

\citet{tilly2008contentious} further specifies the repertoire of friction expression available to social movements. ``Contentious performances''---demonstrations, strikes, petitions, blockades, occupations---are not spontaneous eruptions but culturally learned routines that evolve over time. The petition was invented; the demonstration has a history; the sit-in was consciously innovated. Each performance type represents a technology for converting individual-level misalignment perceptions into collective friction visible to $H_t(d)$ incumbents. Tilly's insight is that the available repertoire constrains friction expression: movements can only deploy performances that are culturally recognized as legitimate protest within their historical context. A medieval grain riot and a modern social media campaign both express friction, but through different culturally available channels. This implies that our friction measure $F(d)$ is not a simple aggregate of individual dissatisfaction but is mediated by the protest technologies and cultural repertoires available in a given historical-institutional context.

At the individual level, \citet{schussman2005process} investigate why some individuals participate in protest while others with similar grievances do not. Their analysis identifies biographical availability (having time and resources), prior activism experience, and social network embeddedness as key predictors of participation, controlling for policy disagreement. These findings connect directly to $\text{eff\_voice}_i$: formal grievance (perceiving low $\alpha(d)$ in a salient domain) is necessary but not sufficient for friction expression. Individuals must also possess the capacity to act---resources, social connections, organizational infrastructure---that transforms dissatisfaction into observable friction. This individual-level analysis reinforces Olson's structural insight: the gap between experienced misalignment and expressed friction is not random but systematically correlated with socioeconomic position. Those with the highest stakes and lowest voice are often precisely those with the least capacity to express friction, creating a measurement paradox where the most severe consent misalignment is the least visible---a dynamic analyzed in detail by \citet{farzulla2025stakes}.

\citet{stekelenburg2010social} synthesize the social psychology of protest participation, identifying perceived injustice, efficacy beliefs, and social identity as the three primary motivational pathways. Perceived injustice maps onto our misalignment detection: individuals who perceive that $\alpha(d)$ falls below acceptable thresholds experience moral outrage that motivates action. Efficacy beliefs correspond to expectations about the translation function from individual friction contributions to institutional change: will protesting actually shift $H_t(d)$? Social identity determines whether individuals conceptualize their stakes as individual or collective---whether misalignment in domain $d$ is experienced as ``my problem'' or ``our problem.'' This three-pathway model explains why identical objective conditions produce different levels of friction across populations and time periods: the same $\alpha(d)$ can generate mass mobilization or quiescent acceptance depending on injustice framing, efficacy perceptions, and identity politicization.

\citet{singh2020globalization} extends the analysis to contemporary globalization, demonstrating that economic integration intensifies cross-domain friction through interconnected stakes. When global supply chains link workers in Bangladesh to consumers in Europe, environmental degradation in the Amazon to commodity prices in Chicago, and financial instability in one country to pension values in another, the affected set $S_d$ for each governance domain expands beyond national boundaries. But $H_t(d)$ remains largely bounded by nation-states, creating a structural gap between transnational stakes and national consent-holding arrangements. Singh's evidence shows that this gap manifests as increased social instability: anti-globalization protests, nationalist backlash, populist movements, and transnational advocacy campaigns all represent friction responses to a global misalignment between who bears consequences and who holds decision authority. The consent-holding framework predicts this pattern: as globalization expands $S_d$ beyond the boundaries of existing $H_t(d)$ arrangements, $\alpha(d)$ declines mechanically, and friction increases as predicted by H1.

These collective action dynamics connect directly to the framework's hypotheses. H1 (higher $\alpha$ predicts lower future $F$) is validated by the inverse: social movements arise precisely in domains where consent misalignment is perceived as severe. H4 (persistent $F$ predicts future $\alpha$ increases) captures the core mechanism of democratic reform through movement pressure. The collective action literature adds crucial nuance: the translation from misalignment to friction is neither automatic nor proportional, but mediated by organizational capacity, political opportunity, cultural repertoires, and individual-level resources. Empirical testing of the framework must account for these mediating factors or risk confounding low friction with high consent alignment when in fact it may reflect high barriers to collective action expression.


%% ============================================================================
%% 2.11 INSTITUTIONAL THEORY AND CRITICAL JUNCTURES (NEW)
%% ============================================================================

\subsection{Institutional Theory and Critical Junctures}
\label{subsec:institutional-theory}

Institutions---the formal rules, informal norms, and enforcement mechanisms that structure collective life---determine how consent-holding evolves over time. While the preceding sections examined how stakeholders contest $H_t(d)$ from below (social movements) and how communities self-govern specific domains (common-pool resources), this section addresses the macro-institutional structures that constrain and enable consent alignment across entire polities.

\citet{acemoglu2012why} provide the most influential contemporary account of how institutional configurations determine long-run development trajectories. Their distinction between \textit{inclusive} institutions (those distributing political and economic power broadly) and \textit{extractive} institutions (those concentrating power among a narrow elite) maps directly onto consent-holding theory's $\alpha(d)$ metric. Inclusive institutions, by definition, extend $C_i > 0$ to broad populations across multiple governance domains---political voice through elections, economic opportunity through property rights and contract enforcement, social mobility through education access. Extractive institutions concentrate $C_i$ among elites while imposing high $s_i(d)$ on excluded populations through taxation, forced labor, and restricted economic participation. In our notation, inclusive institutions maintain high $\alpha(d)$ across multiple domains simultaneously; extractive institutions maintain low $\alpha(d)$ sustained by coercive capacity that suppresses friction expression below the threshold at which institutional change becomes inevitable.

\citet{acemoglu2019narrow} refine this framework through the metaphor of the ``narrow corridor''---the constrained institutional space within which both state capacity and societal mobilization are sufficiently strong to maintain what they call ``Shackled Leviathan'' governance. Too much state capacity without societal counterweight produces despotism ($C_i$ concentrated in state apparatus, $\alpha \approx 0$ for civil domains); too much societal mobilization without state capacity produces anarchy ($H_t(d)$ undefined or contested across all domains, friction $F(d)$ unconstrained). The narrow corridor represents the dynamic equilibrium where state and society check each other: institutional alpha is maintained not by benevolent design but by continuous tension between centralizing and decentralizing forces.

This framework has two important implications for consent-holding theory. First, legitimacy is not a static achievement but a dynamic balance requiring continuous recalibration. Even polities that achieve high $\alpha(d)$ can de-democratize if the balance between state capacity and societal mobilization shifts---a process visible in contemporary democratic backsliding. Second, the narrow corridor metaphor implies that moderate friction is not pathological but structurally necessary: societal mobilization capacity (the potential for friction expression) constrains state overreach, while state enforcement capacity constrains societal fragmentation. A polity with zero observed friction may be despotic (friction suppressed) rather than legitimate (friction absent because $\alpha$ is high). Empirical measurement must distinguish these cases---a distinction the consent-holding framework enables through independent measurement of $\alpha(d)$ and $F(d)$ rather than inferring one from the other.

Critical junctures---moments of radical institutional restructuring---represent the most dramatic changes in $H_t(d)$. \citet{soifer2012causal} clarifies the causal logic of critical junctures, distinguishing between permissive conditions (structural factors that make change possible) and productive conditions (specific events or decisions that determine the direction of change). In consent-holding terms, permissive conditions arise when existing $H_t(d)$ arrangements become unsustainable---accumulated friction exceeds institutional absorptive capacity, external shocks destabilize enforcement mechanisms, or elite coalitions fracture. Productive conditions determine which new $H_t(d)$ emerges from the juncture: revolutionary leadership, constitutional conventions, foreign intervention, or civil war each produce different post-juncture consent configurations.

The critical juncture framework, complemented by \citet{capoccia2007study} on theory, narrative, and counterfactual analysis, implies that $\alpha(d)$ trajectories are path-dependent: early institutional choices create self-reinforcing feedback loops that make subsequent reform either easier or harder. Colonial institutional legacies, for instance, locked in extractive $H_t(d)$ configurations that persisted centuries after colonial rule ended \citep{acemoglu2012why}. The framework's dynamic emphasis---tracking $H_t(d)$ as a time-varying function rather than a static institutional description---captures this path dependence formally: the consent-holding configuration at time $t$ constrains the feasible set of configurations at $t+1$, with critical junctures representing discontinuous jumps outside the normal evolutionary path.

\citet{linz1996problems} bring this theoretical apparatus to bear on the specific challenge of democratic transition and consolidation---the process of raising $\alpha(d)$ from authoritarian to democratic levels and sustaining those gains. Their comparative analysis of transitions in Southern Europe, South America, and post-communist Europe reveals that democratic consolidation requires not merely formal institutional change (new constitutions, elections) but normative commitment: democracy becomes ``the only game in town'' when all significant political actors accept democratic $H_t(d)$ as the exclusive framework for pursuing their interests. In consent-holding terms, consolidation occurs when high $\alpha(d_{political})$ becomes self-reinforcing: the benefits of operating within democratic consent structures exceed the expected gains from extra-institutional power seizure for all relevant actors. Linz and Stepan's conditions for consolidation---an autonomous civil society, a relatively autonomous political society, rule of law, a usable state bureaucracy, and an institutionalized economic society---can be reinterpreted as the infrastructure requirements for sustaining high $\alpha(d)$ across multiple governance domains simultaneously. A democracy that achieves high $\alpha(d_{elections})$ but fails to extend consent alignment to economic governance, judicial independence, or bureaucratic accountability remains unconsolidated and vulnerable to regression.

\citet{sartori1994comparative} examines the constitutional engineering dimension of consent-holding design. Electoral systems, executive-legislative relations, federalism, and judicial review represent institutional choices that structure how $C_i$ is distributed across populations and governance domains. Sartori demonstrates that these choices have systematic consequences: proportional representation raises $\alpha(d_{legislative})$ by granting representation to minority parties but may reduce governability; presidentialism concentrates executive $C_i$ but provides electoral accountability; federalism enables domain-specific $H_t(d)$ optimization but creates coordination challenges. Constitutional design, in this light, is the engineering of consent-holding structures---an optimization problem that the consent-holding framework renders explicit and, in principle, empirically evaluable.

\citet{gerring2015institutional} evaluates one specific institutional arrangement---direct democracy---through a comparative lens. Their institutional theory of direct democracy identifies the conditions under which referenda and initiatives raise or lower governance quality: direct democracy performs well when issues are salient, information is accessible, and outcomes are reversible, but poorly when issues are technical, information asymmetries are severe, and decisions have irreversible consequences. In consent-holding terms, direct democracy maximizes $C_i$ breadth (all citizens vote) but does not guarantee high $\text{eff\_voice}_i$ (informed, considered judgment). The performance conditions Gerring and colleagues identify correspond to domain characteristics where equal-voice mechanisms are likely to produce high or low $L(d,t)$---providing empirical support for our distinction between domains where equal voice and stakes-weighting produce different legitimacy outcomes.

\citet{evans1995embedded} introduces the concept of ``embedded autonomy'' to explain how certain developmental states---South Korea, Taiwan, Singapore---achieved sustained economic growth through technocratic governance that maintained moderate legitimacy despite limited democratic participation. Evans argues that these states succeeded because their bureaucracies were simultaneously autonomous from societal capture (enabling coherent policy) and embedded in dense networks of information exchange with the private sector (enabling responsiveness). In consent-holding terms, embedded autonomous states maintained moderate $\alpha(d_{economic})$ through competence rather than consent: high performance $P(d)$ partially compensated for low $\alpha(d)$ in our legitimacy equation $L(d,t) = w_1 \cdot \alpha(d,t) + w_2 \cdot P(d,t)$. This pattern validates Proposition 1 (competence-consent trade-off) and illuminates its boundary conditions: technocratic legitimacy is sustainable when economic performance remains high but vulnerable to legitimacy crises when performance falters, because no consent reservoir cushions performance shocks.

\citet{slater2010informative} address a methodological challenge facing institutional analysis---and by extension, consent-holding measurement. Their concept of ``informative regress'' highlights the problem of tracing causal chains backward from institutional outcomes to antecedent conditions: at what point does the causal chain become uninformative, and how do analysts distinguish genuinely explanatory critical antecedents from arbitrary starting points? For consent-holding theory, this methodological concern is directly relevant: when we trace $H_t(d)$ backward to explain current consent configurations, we must specify which historical antecedents are genuinely causal versus merely temporally prior. Slater and Simmons propose that informative antecedents are those that \textit{could have been otherwise}---moments of genuine contingency where different choices would have produced different institutional trajectories. This criterion complements our critical juncture analysis: the most informative moments for understanding consent-holding evolution are those where $H_t(d)$ was genuinely contested and could have settled into different configurations. Moreover, as \citet{nolan2015international} demonstrates, critical junctures in consent-holding frequently have international dimensions: wars, alliances, and geopolitical pressures reshape domestic institutional configurations by altering the balance of power between state elites and societal actors. The international system both constrains and enables domestic consent-holding reform, adding an external dimension to the dynamics formalized by our framework.

The institutional theory literature thus provides consent-holding theory with both macro-structural context and analytical tools. Institutions determine the feasible range of $\alpha(d)$ in any given polity; critical junctures explain how that range shifts discontinuously; constitutional engineering describes the design choices available for optimizing consent-holding structures; and embedded autonomy illustrates the competence-consent trade-off that our Proposition 1 formalizes. Together, these insights ground the framework's historical applications (Part III of this monograph) in established institutional analysis while demonstrating that consent-holding theory contributes something the institutional literature lacks: a unified metric ($\alpha(d,t)$) for comparing legitimacy across institutional configurations, governance domains, and historical periods.
